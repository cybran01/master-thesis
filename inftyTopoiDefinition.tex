In this section we will introduce the notions of local presentability, universality and descent for \inftycats/ as announced in \cref{sec:motivationInftyTopoi}.
We will introduce local presentability first, but since this will be the least interesting property for our purposes, we will give a very brief introduction.

After this we will define universality and descent via cartesian natural transformations and prove that checking universality and descent for a given \inftycat/ requires only checking the properties for pushouts and coproducts.
\begin{definition}\label{def:locallyPresentable}
    An \inftycat/ $C$ is \emph{locally presentable} if there exists a combinatorial simplicial model category $A$ and an equivalence of \inftycats/ $\Nhc(A^°)\to C$.
\end{definition}
\begin{corollary}
    A locally presentable \inftycat/ $C$ has small colimits and small limits.
    \begin{reference}
        \cite[Corollary 4.2.4.8]{HTT}
    \end{reference}
\end{corollary}
\begin{corollary}
    The \inftycat/ of spaces $\spaces$ is locally presentable.
    \begin{proof}\label{cor:spacesIsLocPres}
        This follows directly by \cref{prop:homotopyHypothesis} and \cref{cor:htpyCoherentNerveIsLoc}.
    \end{proof}
\end{corollary}
Note that $\Top$ as an ordinary category is not locally presentable. %TODO why
An overview for some equivalent characterizations for an \inftycat/ to be locally presentable can be found in \cite[Theorem 5.5.1.1 and Proposition A.3.7.6]{HTT}.

Next we introduce cartesian transformations which allow us to encode universality and descent concisely.
\begin{definition}[Cartesian Transformation]
    Let $C$ be an \inftycat/ and let $K$ be a simplicial set.
    Let $q,p\in\Fun(K,C)_0$ and let $\alpha\in\Fun(K,C)_1$ be a natural transformation $\alpha\colon p\to q$.
    Then we say $\alpha$ is \emph{cartesian} if for all maps $u\colon x\to y\in K_1$ the square 
    \begin{center}
        \begin{tikzcd} [sep= 4em]
            p(x) \arrow[r, "\alpha(x)"] \arrow [d, "p(u)"'] & q(x) \arrow [d, "q(u)"]\\
            p(y) \arrow[r, "\alpha(y)"] & q(y)
        \end{tikzcd}
    \end{center}
    represented by the map $\Delta^1\times\Delta^1\xrightarrow{\left(u,\id_{\Delta^1}\right)}K\times\Delta^1\xrightarrow{\alpha}C$ is a pullback square in $C$. 
\end{definition}
\begin{definition}[Universality]
    Let $C$ be an \inftycat/ that admits small colimits and pullbacks and let $K$ be a small simplicial set.
    We say that \emph{$C$ has $K$-shaped universal colimits} if for all natural transformations $\alpha\colon p\to q\in\Fun(K^{\rhd},C)_1$ where $q$ is a colimit cone and $\alpha$ is cartesian, $p$ is a also a colimit cone.
    
    We say that \emph{$C$ has universal colimits} if $C$ has $K$-shaped universal colimits for every small simplicial set $K$.
\end{definition}
\begin{definition}[Descent]
    Let $C$ be an \inftycat/ that admits small colimits and pullbacks and let $K$ be a small simplicial set.
    We say that \emph{$C$ has $K$-shaped descent} if for all natural transformations $\alpha\colon p\to q\in\Fun(K^{\rhd},C)_1$ where $p,q$ are colimit cones and $\alpha|_K$ is cartesian, $\alpha$ is also cartesian.
    
    We say that \emph{$C$ has descent} if $C$ has $K$-shaped descent for every small simplicial set $K$.
\end{definition}
\begin{remark}
    We want to describe universality and descent explicitly for pushouts ($\Lambda^2_0$-shaped diagrams), because these will be important cases as \cref{lem:univColimIffUnivPoAndCoprod} and \cref{lem:descentIffDescentPoAndCoprod} will show.
    A natural transformation $\alpha\colon\left(\Lambda_0^2\right)^{\rhd}\times\Delta^1\cong\Delta^1\times\Delta^1\times\Delta^1\to C$ is a cube
    \begin{center}
        \begin{tikzcd} [sep = .5 cm]
            \alpha|_{\set{0,0,0}} \arrow [dr] \arrow [rr] \arrow [dd] & & \alpha|_{\set{0,1,0}} \arrow [dr] \arrow [dd] \\
            & \alpha|_{\set{1,0,0}} \arrow [rr, crossing over] \arrow [dd] & & \alpha|_{\set{1,1,0}} \arrow [dd] & \\
            \alpha|_{\set{0,0,1}} \arrow [dr] \arrow [rr] & & \alpha|_{\set{0,1,1}} \arrow [dr] \\
            & \alpha|_{\set{1,0,1}} \arrow [from=uu, crossing over] \arrow [rr] & & \alpha|_{\set{1,1,1}}
        \end{tikzcd}\;.
    \end{center}
    Universality for pushouts means that whenever the vertical faces are pullbacks and the bottom face is a pushout, then the top face is a pushout.
    Descent for pushouts means that whenever the left and rear faces are pullbacks and top and bottom faces are pushouts, then the right and front faces are pullbacks as well.
    
    So this definition is exactly the one given in \cref{sec:motivationInftyTopoi}.
\end{remark}
\begin{definition}[\infty-Topos]
    Let $C$ be an \inftycat/ fulfilling the following properties:
    \begin{itemize}
        \item $C$ is locally presentable.
        \item $C$ has universal colimits.
        \item $C$ has descent.
    \end{itemize}
    Then we call $C$ an \emph{\infty-topos}.
\end{definition}
In order to prove that a given \inftycat/ has universality and descent we only need to check it for two types of diagrams: pushouts and small coproducts.
This is the content of \cref{lem:univColimIffUnivPoAndCoprod} and \cref{lem:descentIffDescentPoAndCoprod} and will be used later to prove that $\spaces$ is an \inftytop/.

We will require the following proposition for the proof.
\begin{prop}\label{prop:colimitDecompositionPushouts} 
    Let $I$ be a small category, $X\colon I\to\sSet$ a functor and let $K$ denote the colimit of this functor.
    We assume that one of the following conditions is satisfied:
    \begin{itemize}
        \item The category $I$ is discrete, hence $K\cong\coprod\limits_{i\in I}X(i)$.
        \item We have $\N(I)\cong\Lambda_0^2$ and the map $X(0)\xhookrightarrow{} X(2)$ is a monomorphism, hence $K$ is a pushout \begin{center}
                \begin{tikzcd} [sep = 4em]
                    X(0) \arrow[r] \arrow[d, hook] & X(1) \arrow[d, hook] \\
                    X(2) \arrow[r] & K
                \end{tikzcd}\;.
            \end{center}
        \item The category $I$ is filtered.
    \end{itemize}
    Then let $C$ be an cocomplete \inftycat/ and $\Phi\colon Y\to C$ a functor.
    For all $i\in I$ let $\Phi_i$ denote the induced map $\Phi_i\colon X(i)\to Y\to C$.

    Then the assignment $i\mapsto\colim\limits_{x\in X(i)} \Phi_i(x)$ extends naturally to a functor $\N(I)\to C$ such that there is a canonical invertible map of the form 
    \begin{equation*}
        \colim\limits_{y\in Y}\Phi(y)\simeq\colim\limits_{i\in\N(I)}\colim\limits_{x\in X(i)}\Phi_i(x)\;.
    \end{equation*}
    \begin{reference}
        \cite[Proposition 7.3.16]{cisinski_2019} (applied to $C^{\op}$)
    \end{reference}
\end{prop}
\begin{lemma}\label{lem:spineInclIsWCE}
    We view the linearly ordered set $\Nat$ as a category as in \cref{rmk:linOrderedAsCat}.
    Let $L\subset\N(\Nat)$ be the simplicial subset given by $L=\bigcup\limits_{i\in\Nat}\Delta^{\set{i,i+1}}$, hence $L$ is the simplicial subset consisting of all vertices and edges joining consecutive vertices.
    
    Then $L\subset\N(\Nat)$ is a weak categorical equivalence.
    \begin{proof}
        By \cite[Proposition 3.7.4]{cisinski_2019} the inclusions $\Sp^n=\bigcup\limits_{0\leq i<n}\Delta^{\set{n,n+1}}\xhookrightarrow{}\Delta^n$ are weak categorical equivalences.
        The canonical inclusions of categories $[n]\subset[n+1]$ for all $n\in\Nat$ then induce a map $\colim\limits_{n\in\Nat}\Sp^n\xhookrightarrow{}\colim\limits_{n\in\Nat}\Delta^n$.
        By \cite[Corollary 3.9.8]{cisinski_2019} weak categorical equivalences are stable under filtered colimits, hence this map is a weak categorical equivalence.
        We have $\colim\limits_{n\in\Nat}\Sp^n\cong L$ as colimits commute and $\colim\limits_{n\in\Nat}\Delta^n\cong\N(\Nat)$ as the nerve commutes with filtered colimits (\cite[Corollary 1.3.12]{cisinski_2019}), so this corresponds to the map $L\subset\N(\Nat)$.
    \end{proof}
\end{lemma}
\begin{lemma}\label{lem:univColimIffUnivPoAndCoprod}
    A \inftycat/ $C$ has universal colimits if and only if it has universal pushouts and universal coproducts.
    \begin{proof}
        Let $\alpha\colon p\to q\in\Fun(K^{\rhd},C)_1$ be a cartesian transformation and let $q$ be a colimit cone. 
        We prove the statement first for all finite dimensional $K$ by induction over the dimension of $K$.

        For $\dim K=-1$ the statement follows since $K=\emptyset$ is the coproduct over the empty indexing category.
        
        Let $\dim K=n+1$. 
        Then by the skeletal filtration
        \begin{equation*}
            K=K^{n+1}=K^n\bigcup\limits_{\coprod\limits_S \partial\Delta^{n+1}}\left(\coprod_S \Delta^{n+1}\right)
        \end{equation*}
        where $S$ is the set of non-degenerate simplices of dimension $n+1$.
        For ease of notation, we set $X_{0}=\coprod\limits_S \partial\Delta^{n+1}$, $X_1=K^n$ and $X_2=\coprod\limits_S \Delta^{n+1}$.

        Let $p_i,q_i\colon X_i^{\rhd}\to C$ be the maps obtained by precomposing $p$ or $q$ with the respective map to $K$ from the skeletal filtration and taking the conepoint of $q_i$ to be the colimit, and the conepoint of $p_i$ to be the pullback 
        \begin{center}
            \begin{tikzcd} [sep = 4em]
                p_i(\infty) \arrow[r] \arrow[d] & p(\infty) \arrow[d]\\
                q_i(\infty) \arrow[r] & q(\infty)
            \end{tikzcd}\;.
        \end{center}
        Then $\alpha$ induces natural transformations $\alpha_i\colon p_i\to q_i$.
        We claim these natural transformations are also cartesian:

        Let $f\colon x\to y$ be a map in $X_i^{\rhd}$. 
        If $f$ is in $X_i$, the claim follows immediately from the fact that $\alpha$ is cartesian, hence we assume $y=\infty$.
        In this case the outer and right square of the diagram
        \begin{center}
            \begin{tikzcd} [sep = 4em]
                p_i(x) \arrow[r] \arrow[d] & p_i(\infty) \arrow[r] \arrow[d] & p(\infty) \arrow[d] \\
                q_i(x) \arrow[r] & q_i(\infty) \arrow[r] & q(\infty)
            \end{tikzcd}
        \end{center}
        are pullbacks, hence by the pasting law the left square is a pullback as well.
        This proves our claim.

        By the induction hypothesis we therefore know that $p_0$ and $p_1$ are colimit cones. 
        We claim that $p_2$ is also a colimit cone:
        
        Since $\Delta^{n+1}$ has a terminal object, colimits over $\Delta^{n+1}$ can be computed by evaluating the terminal object.
        Let the family $(*_s)_{s\in S}$ be the family of terminal objects of the components in $X_2=\coprod\limits_{s\in S} \Delta^{n+1}$.
        Then we have $q_2(\infty)\cong\coprod\limits_{s\in S} q_2(*_s)$ by \cref{prop:colimitDecompositionPushouts}.

        Since for all $s\in S$ the square
        \begin{center}
            \begin{tikzcd} [sep = 4em]
                p_2(*_s) \arrow[r] \arrow[d] & p_2(\infty) \arrow[d]\\
                q_2(*_s) \arrow[r] & q_2(\infty)\cong\coprod\limits_{s\in S} q_2(*_s)
            \end{tikzcd}\;.
        \end{center}
        is a pullback since $\alpha_2$ is cartesian, by universality for coproducts it follows that $p_2(\infty)\cong\coprod\limits_{s\in S} p_2(*_s)$.
        This shows that $p_2$ is a colimit cone, proving the claim.
        
        By \cref{prop:colimitDecompositionPushouts} we know that 
        \begin{center}
            \begin{tikzcd} [sep = 4em]
                q_0(\infty) \arrow[r] \arrow[d] & q_1(\infty) \arrow[d] \\
                q_2(\infty) \arrow[r] & q(\infty)
            \end{tikzcd}
        \end{center}
        is a pushout square.
        Furthermore, the natural transformation $\beta\in\Fun\left(\left(\Lambda_0^2\right)^{\rhd},C\right)_1$ induced by $\alpha$ and the $\alpha_i$ which we represent by a cube
        \begin{center}
            \begin{tikzcd} [sep = .5 cm]
                p_0(\infty) \arrow [dr] \arrow [rr] \arrow [dd] & & p_1(\infty) \arrow [dr] \arrow [dd] \\
                & p_2(\infty) \arrow [rr, crossing over] \arrow [dd] & & p(\infty) \arrow [dd] & \\
                q_0(\infty) \arrow [dr] \arrow [rr] & & q_1(\infty) \arrow [dr] \\
                & q_2(\infty) \arrow [from=uu, crossing over] \arrow [rr] & & q(\infty) &
            \end{tikzcd}
        \end{center}
        is cartesian by the pasting law.
        By universality for pushouts (since the bottom square is pushout) we thus have that the top square is also a pushout.
        Again by \cref{prop:colimitDecompositionPushouts} it follows that $p$ is a colimit cone.
        It now remains to show that statement for general $K$.

        Let $L\subset\N(\Nat)$ as in \cref{lem:spineInclIsWCE}.
        Let $\alpha\colon p\to q\in\Fun(K^{\rhd},C)_1$ be a cartesian transformation, let $q$ be a colimit cone and let $\left(K^n\right)_{n\in\Nat}$ denote the skeletal filtration of $K$.
        By \cref{prop:colimitDecompositionPushouts} we can compute the colimit of a diagram indexed by $K$ by computing the colimit of the diagram $\N(\Nat)\to C$ that for each $0$-simplex $n\in\Nat$ is the colimit of $K^n\to C$.
        Since $L\subset\N(\Nat)$ is a weak categorical equivalence (which preserve and reflect (co)limits), it even suffices to compute the colimit of the induced diagram $L\to C$.


        For $q\colon K\to C$, let $F_q\colon L^{\rhd}\to C$ be the colimit diagram constructed as described above.
        Let $F_p\colon L^{\rhd}\to C$ be given by taking pullbacks
        \begin{center}
            \begin{tikzcd} [sep = 4em]
                F_p(n) \arrow[r] \arrow[d] & F_p(\infty)=p(\infty) \arrow[d] \\
                F_q(n) \arrow[r] & F_q(\infty)=q(\infty)
            \end{tikzcd}
        \end{center}
        which induce a cartesian natural transformation $\gamma\colon F_p\to F_q\in\Fun(L^{\rhd},C)_1$.
        Since $L$ is $1$-dimensional, $F_p$ is a colimit cone by universality for finite dimensional diagrams.
        Universality for finite dimensional diagrams also implies that $F_p(n)$ is the colimit of $p|_{K^n}$ for every $n\in\Nat$, hence $F_p(\infty)=p(\infty)$ is a colimit of $p|_K$ by \cref{prop:colimitDecompositionPushouts}.
        This proves that $p$ is a colimit cone and thus the proposition. 
    \end{proof}
\end{lemma}
\begin{lemma}\label{lem:descentIffDescentPoAndCoprod}
    A \inftycat/ $C$ that has universal colimits has descent if and only if it has descent for pushouts and coproducts.
    \begin{proof}
        Let $\alpha\colon p\to q\in\Fun(K^{\rhd},C)_1$ be natural transformation between colimit cones such that $\alpha|_K$ is cartesian.
        We first prove the statement for all finite dimensional $K$ by induction over the dimension of $K$.

        For $\dim K=-1$ the statement follows since $K=\emptyset$ is the coproduct over the empty indexing category.

        So let $\dim K=n+1$.
        Then by the skeletal filtration
        \begin{equation*}
            K=K^{n+1}=K^n\bigcup\limits_{\coprod\limits_S \partial\Delta^{n+1}}\left(\coprod_S \Delta^{n+1}\right)
        \end{equation*}
        where $S$ is the set of non-degenerate simplices of dimension $n+1$.
        For ease of notation, we again set $X_{0}=\coprod\limits_S \partial\Delta^{n+1}$, $X_1=K^n$ and $X_2=\coprod\limits_S \Delta^{n+1}$.
        
        Let $p_i,q_i\colon X_i^{\rhd}\to C$ be the maps obtained by precomposing $p$ and $q$ with the respective map to $K$ from the skeletal filtration and taking the conepoints to be the colimits.
        Then $\alpha$ induces natural transformations $\alpha_i\colon p_i\to q_i$.
        By the induction hypothesis we know that $\alpha_0$ and $\alpha_1$ are cartesian.
        We claim that the natural transformation $\alpha_2$ is cartesian as well: 

        For $i\in\set{0,\ldots,n+1}$ and $s\in S$ let $i_s\in X_2$ denote the $i$-th vertex of the $s$-th component in $X_2=\coprod\limits_{s\in S} \Delta^{n+1}$.
        Since the objects $(n+1)_s$ are terminal in their respective component $\Delta^{n+1}$, we have $q_2(\infty)\cong\coprod\limits_{s\in S} q_2((n+1)_s)$ as well as $p_2(\infty)\cong\coprod\limits_{s\in S} p_2((n+1)_s)$ by \cref{prop:colimitDecompositionPushouts}.
        For each $s\in S$ and $i\in\set{0,\ldots,n+1}$ the outer square of the diagram
        \begin{center}
            \begin{tikzcd} [sep = 4em]
                p_2(i_s) \arrow[r] \arrow[d] & p_2((i+1)_s) \arrow[r] \arrow[d] & \coprod\limits_{s\in S} p_2((n+1)_s)\cong p_2(\infty) \arrow[d] \\
                q_2(i_s) \arrow[r] & q_2((i+1)_s) \arrow[r] & \coprod\limits_{s\in S} q_2((n+1)_s)\cong q_2(\infty)
            \end{tikzcd}
        \end{center}
        is a pullback: The left square is a pullback because $\alpha_2|_{X_2}$ is cartesian and the right square is a pullback by descent for coproducts, so by the pasting law the outer square must be cartesian as well.
        This proves our claim that $\alpha_2$ is cartesian.

        By \cref{prop:colimitDecompositionPushouts} we know that both squares
        \begin{center}
            \begin{tikzcd} [sep = 4em]
                p_0(\infty) \arrow[d] \arrow[r] & p_1(\infty) \arrow[d] \\
                p_2(\infty) \arrow[r] & p(\infty)
            \end{tikzcd}
            \begin{tikzcd} [sep = 4em]
                q_0(\infty) \arrow[d] \arrow[r] & q_1(\infty) \arrow[d] \\
                q_2(\infty) \arrow[r] & q(\infty)
            \end{tikzcd}
        \end{center}
        are pushouts.
        We claim that the diagrams
        \begin{center}
            \begin{tikzcd} [sep = 4em]
                p_0(\infty) \arrow[d] \arrow[r] & p_1(\infty) \arrow[d] \\
                q_0(\infty) \arrow[r] & q_1(\infty)
            \end{tikzcd}
            \begin{tikzcd} [sep = 4em]
                p_0(\infty) \arrow[d] \arrow[r] & p_2(\infty) \arrow[d] \\
                q_0(\infty) \arrow[r] & q_2(\infty)
            \end{tikzcd}
        \end{center}
        are pullbacks: 

        For all objects $x$ in $X_0$ and $i\in\set{1,2}$ the left square of the diagram 
        \begin{center}
            \begin{tikzcd} [sep = 4em]
                p_0(x) \arrow[r] \arrow[d] & q_0(\infty)\times_{q_i(\infty)}p_i(\infty) \arrow[r] \arrow[d] & p_i(\infty) \arrow[d] \\
                q_0(x) \arrow[r] & q_0(\infty) \arrow[r] & q_i(\infty)\\
            \end{tikzcd}
        \end{center}
        is a pullback by the pasting law, since the outer square is a pullback as $\alpha_i$ is cartesian.
        By universality, this implies that $q_0(\infty)\times_{q_i(\infty)}p_i(\infty)$ is a colimit of $p_0|_{X_0}$ and therefore $q_0(\infty)\times_{q_i(\infty)}p_i(\infty)\cong p_0(\infty)$ which proves our claim.

        The natural transformations $\alpha$ and $\alpha_i$ induce a natural transformation $\beta\in\Fun\left(\left(\Lambda_0^2\right)^{\rhd}, C\right)_1$ which we represent by a cube
        \begin{center}
            \begin{tikzcd} [sep = .5 cm]
                p_0(\infty) \arrow [dr] \arrow [rr] \arrow [dd] & & p_1(\infty) \arrow [dr] \arrow [dd] \\
                & p_2(\infty) \arrow [rr, crossing over] \arrow [dd] & & p(\infty) \arrow [dd] & \\
                q_0(\infty) \arrow [dr] \arrow [rr] & & q_1(\infty) \arrow [dr] \\
                & q_2(\infty) \arrow [from=uu, crossing over] \arrow [rr] & & q(\infty) &
            \end{tikzcd}\;.
        \end{center}
        We know that $\beta$ cartesian because $C$ has descent for pushouts.
        We now show that that $\alpha$ is cartesian:

        Since every object $x\in K_0$ lies in the image of either $X_1$ or $X_2$, there is an $i\in\set{1,2}$ and $x_i\in X_i$ such that the maps $p(x)\to p(\infty)$ and $q(x)\to q(\infty)$ factor through $p_i(\infty)$ and $q_i(\infty)$ respectively.
        This implies that the diagram
        \begin{center}
            \begin{tikzcd} [sep = 4em]
                p_i(x_i)=p(x) \arrow[r] \arrow[d] & p_i(\infty) \arrow[r] \arrow[d] & p(\infty) \arrow[d] \\
                q_i(x_i)=q(x) \arrow[r] & q_i(\infty) \arrow[r] & q(\infty)\\
            \end{tikzcd}
        \end{center}
        is a pullback by the pasting lemma because the left square is cartesian as $\alpha_i$ is cartesian and the right square is cartesian due as $\beta$ is cartesian.

        The case for general $K$ follows from the following argument:
        Let $p,q\colon K^{\rhd}\to C$  be colimit cones and $\alpha\colon p\to q$ be a natural transformation such that $\alpha|_K$ is cartesian.
        Furthermore, let $L\subset\N(\Nat)$ be as in \cref{lem:spineInclIsWCE}.
        Let $K^n\subset K$ denote the $n$-th skeleton of $K$ and let $F_p\colon L^{\rhd}\to C$ and $F_q\colon L^{\rhd}\to C$ be the functors that 
        \begin{itemize}
            \item for each object $n\in L_0$ are the colimit of $p|_{K^n}$ and $q|_{K^n}$ respectively.
            \item have $p(\infty)$ and $q(\infty)$ as their respective conepoints.
        \end{itemize}
        In particular by \cref{prop:colimitDecompositionPushouts}, they are colimit cones.
        By descent for finite dimensions we know that for all $n\in\Nat$ and $x\in K^n$ the outer square of
        \begin{center}
            \begin{tikzcd} [sep = 4em]
                p|_{K^n}(x) \arrow[r] \arrow[d] & q|_{K^n}(\infty)\times_{q|_{K^{n+1}}(\infty)}p|_{K^{n+1}}(\infty) \arrow[r] \arrow[d] & p|_{K^{n+1}}(\infty) \arrow[d] \\
                q|_{K^n}(x) \arrow[r] & q|_{K^n}(\infty) \arrow[r] & q|_{K^{n+1}}(\infty)\\
            \end{tikzcd}
        \end{center}
        is a pullback and hence by the pasting lemma so is the left square.
        By universality this implies that $q|_{K^n}(\infty)\times_{q|_{K^{n+1}}(\infty)}p|_{K^{n+1}}(\infty)\cong p|_{K^n}(\infty)$ which shows that
        \begin{center}
            \begin{tikzcd} [sep = 4em]
                p|_{K^n}(\infty) \arrow[d] \arrow[r] & p|_{K^{n+1}}(\infty) \arrow[d] \\
                q|_{K^n}(\infty) \arrow[r] & q|_{K^{n+1}}(\infty)
            \end{tikzcd}
        \end{center}
        is a pullback, so the transformation $\gamma\colon F_p\to F_q$ induced by $\alpha$ is cartesian when restricted to $L$.
        Since $L$ is $1$-dimensional, finite dimensional descent implies that $\gamma$ is cartesian.
        This shows that for all $n\in\Nat$ the square
        \begin{center}
            \begin{tikzcd} [sep = 4em]
                p|_{K^n}(\infty) \arrow[d] \arrow[r] & p(\infty) \arrow[d] \\
                q|_{K^n}(\infty) \arrow[r] & q(\infty)
            \end{tikzcd}
        \end{center}
        is cartesian.

        We claim that $\alpha$ is cartesian, which means showing that
        \begin{center}
            \begin{tikzcd} [sep = 4em]
                p(x) \arrow[d] \arrow[r] & p(\infty) \arrow[d] \\
                q(x) \arrow[r] & q(\infty)
            \end{tikzcd}
        \end{center}
        is a pullback for all $x\in K_0$
        But every such $x$ is already in some $K^n$, which gives a factorization
        \begin{center}
            \begin{tikzcd} [sep = 4em]
                p(x)=p|_{K^n}(x) \arrow[d] \arrow[r] & p|_{K^n}(\infty) \arrow[d] \arrow[r] & p(\infty) \arrow [d]\\
                q(x)=q|_{K^n}(x) \arrow[r] & q|_{K^n}(\infty) \arrow[r] & q(\infty)
            \end{tikzcd}\;.
        \end{center}
        Since we have already proven that both left and right square of this diagram are cartesian, the outer square is also cartesian which proves the proposition.
    \end{proof}
\end{lemma}