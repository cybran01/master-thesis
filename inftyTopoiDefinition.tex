\begin{definition}[$\kappa$-filtered] %HTT Definition 5.3.1.7.
    Let $\kappa$ be a regular cardinal and $C$ be an \inftycat/.
    We say that $C$ is \emph{$\kappa$-filtered} if for every map $f:K\to C$ with $K$ a $\kappa$-small simplicial set there exists a map $\overline{f}:K^{\rhd}\to C$ such that $\overline{f}\vert_K=f$.
\end{definition}
%TODO include Proposition 5.3.1.18. without proof
% \begin{prop} %HTT Proposition 5.3.3.3.
%     Let $C$ be an \inftycat/ and $\kappa$ a regular cardinal. Then the following are equivalent:
%     \begin{itemize}
%         \item $C$ is $\kappa$-filtered
%         \item The colimit functor $\colim\colon\Fun(C,\Spaces)\to\Spaces$ preserves $\kappa$-small limits.
%     \end{itemize}
% \end{prop}
\begin{definition} %HTT Definition 5.4.2.1.
    Let $\kappa$ be a regular cardinal and let $C$ be a small $\infty$-category.
    We define $\Ind_{\kappa}(C)\subset P(C)$ to be the full subcategory of $P(C)$ with objects consisting of those maps $C^{\op}\to\spaces$ that classify right fibrations $X\to C$ such that $X$ is $\kappa$-filtered. %TODO define what it means to classify
\end{definition}
\begin{definition}
    Let $\kappa$ be a regular cardinal. An \inftycat/ $C$ is \emph{$\kappa$-accessible} if there exists a small \inftycat/ $C^0$ and a categorical equivalence
    \begin{equation*}
        \Ind_{\kappa}(C^0)\to C\;.
    \end{equation*}
    We say that $C$ is \emph{accessible} if it is $\kappa$-accessible for some cardinal $\kappa$.
\end{definition}
%TODO why is it then accessible for bigger kappa? Proposition HTT 5.4.2.9.
\begin{definition}
    An \inftycat/ is \emph{locally presentable} if it is accessible and admits small colimits.
\end{definition}
\begin{lemma} %TODO HTT Corollary 5.5.2.4., maybe also Corollary 5.3.5.4.
    A locally presentable \inftycat/ $C$ admits small limits.
\end{lemma}
\begin{prop} %HTT Proposition A.3.7.6., due to Simpson
    Let $C$ be an \inftycat/. Then the following are equivalent:
    \begin{itemize}
        \item $C$ is locally presentable.
        \item There exists a combinatorical simplicial model category $M$ and a categorical equivalence $\Nhc(M^°)\simeq C$.
    \end{itemize}
\end{prop}
% \begin{definition}[Universality of Colimits] %TODO how to elegantly define pullback functor?
%     Let $C$ be an \inftycat/ that admits small colimits and pullbacks.
%     If for any map $u\colon x\to y\in C_1$ the pullback functor $u^*\colon \faktor{C}{y}\to\faktor{C}{x}$ preserves colimits, then we say that \emph{$C$ has universal colimits}.
% \end{definition}
\begin{definition}[Cartesian Transformation]
    Let $C$ be an \inftycat/ and let $K$ be a simplicial set.
    Let $q,p\in\Fun(K,C)_0$ and let $\alpha\in\Fun(K,C)_1$ be a natural transformation $\alpha\colon p\to q$.
    Then we say $\alpha$ is \emph{cartesian} if for all maps $u\colon\Delta^1\to K$ the induced map $\Delta^1\times\Delta^1\xrightarrow{\left(u,\id_{\Delta^1}\right)}K\times\Delta^1\xrightarrow{\alpha}C$ is a pullback square.
\end{definition}
% \begin{definition}[Descent] %HTT Definition 6.1.3.1.
%     Let $C$ be an \inftycat/ that admits small colimits and pullbacks.
    
%     Its objects the same as $\Fun(\Delta^1,C)$, but we set the morphisms between them to be maps $\Delta^1\times\Delta^1\to C$ that are pullback squares. %TODO this needs pasting law for pullbacks. Also, why can we just compute in \Fun(\Delta^1,C)?
    
%     We say that \emph{$C$ has descent} if $\Cart(C)$ has small colimits and the inclusion $\Cart(C)\subset\Fun(\Delta^1,C)$ preserves them.
% \end{definition}
%TODO HTT Proposition 4.4.2.6.
\begin{definition}[Universality]
    Let $C$ be an \inftycat/ that admits small colimits and pullbacks.
    We say that \emph{$C$ has universal colimits} if for all natural transformations $\alpha\colon p\to q\in\Fun(K^{\rhd},C)_1$ such that $q$ is a  colimit cone and $\alpha$ is cartesian, $p$ is a colimit cone.
\end{definition}
\begin{definition}[Descent]
    Let $C$ be an \inftycat/ that admits small colimits and pullbacks.
    We say that \emph{$C$ has descent} if for all natural transformations $\alpha\colon p\to q\in\Fun(K^{\rhd},C)_1$ such that $p,q$ are colimit cones and $\alpha|_K$ is cartesian, $\alpha$ is cartesian.
\end{definition}
\begin{lemma}
    A \inftycat/ $C$ has universal colimits if and only if it has universal pushouts and universal coproducts.
\end{lemma}
\begin{lemma}
    A \inftycat/ $C$ has descent if and only if it has descent for pushouts and coproducts.
    \begin{proof}
        Let $\alpha\colon p\to q\in\Fun(K^{\rhd},C)_1$ be a map.
        We prove the statement first for all finite dimensional $K$ by induction over the dimension of $K$. %TODO define dimension

        For $\dim K=-1$ the statement is clear since $K$ is empty.

        For $\dim K=0$ we have that $K\cong\bigcup\limits_{i\in I}\Delta^0$ for some small set $I$. 
        In this case the statement is also clear since all maps are identity maps.

        So let $\dim K=n+1$.
        Then by the skeletal filtration, $K=K^{n+1}=K^n\bigcup\limits_{\cup_S \partial\Delta^{n+1}}\cup_S \Delta^{n+1}$ where $S$ is the set of non-degenerate simplicies of dimension $n+1$.
        For ease of notation, we set $X=\cup_S \partial\Delta^{n+1}$ and $Y=\cup_S \Delta^{n+1}$.

        By induction hypothesis and because $C$ admits small colimits, we obtain that $\alpha|_{\left(K^n\right)^{\rhd}}$, $\alpha|_{X^{\rhd}}$ are cartesian (where we abuse the notation of restriction, since it it not necessarily injective for $X$ and the conepoints which are the colimits are not necessarily in $K^{\rhd}$).

        The map $\alpha|_{Y^{\rhd}}$ is cartesian as well: Since $\Delta^{n+1}$ has a terminal object, the colimit can be computed by evaluating the final vertex. 
        In particular, the maps to the cone point are isomorphisms.
        Then one again uses descent for coproducts.

        By %HTT Proposition 4.4.2.2.
        we know that the diagrams in $C$
        \begin{center}
            \begin{tikzpicture}
                \matrix (m) [matrix of math nodes,row sep=3em,column sep=1em,minimum width=2em]
                {
                  p|_{X^{\rhd}}(\infty) & p|_{\left(K^n\right)^{\rhd}}(\infty)\\
                  p|_{Y^{\rhd}}(\infty) & p(\infty)\\};
                \path[-stealth]
    
                (m-1-1) edge (m-1-2)
                (m-1-1) edge (m-2-1)
                (m-2-1) edge (m-2-2)
                (m-1-2) edge (m-2-2);
            \end{tikzpicture}
            \begin{tikzpicture}
                \matrix (m) [matrix of math nodes,row sep=3em,column sep=1em,minimum width=2em]
                {
                  q|_{X^{\rhd}}(\infty) & q|_{\left(K^n\right)^{\rhd}}(\infty)\\
                  q|_{Y^{\rhd}}(\infty) & q(\infty)\\};
                \path[-stealth]
    
                (m-1-1) edge (m-1-2)
                (m-1-1) edge (m-2-1)
                (m-2-1) edge (m-2-2)
                (m-1-2) edge (m-2-2);
            \end{tikzpicture}
        \end{center}
        where $\infty$ denotes the respective cone points and the morphisms are the universal ones (which are unique up to contractible choice), are pushouts.
    \end{proof}
\end{lemma}
\begin{lemma}
    
\end{lemma}
\begin{definition}[\infty-Topos]
    Let $C$ be an \inftycat/ fulfilling the following properties:
    \begin{itemize}
        \item $C$ is locally presentable.
        \item $C$ has universal colimits.
        \item $C$ has descent.
    \end{itemize}
    Then we call $C$ an \emph{\infty-topos}.
\end{definition}