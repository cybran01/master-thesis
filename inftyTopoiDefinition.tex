Before giving a proper definition of universality and descent, we need to introduce local presentability for \inftycats/.
Since this will be the least interesting for our purposes, we will give a very brief introduction.
\begin{definition}\label{def:locallyPresentable}
    An \inftycat/ $C$ is \emph{locally presentable} if there exists a combinatorial simplicial model category $A$ and an equivalence of \inftycats/ $\Nhc(A^°)\to C$.
\end{definition}
\begin{corollary}
    A locally presentable \inftycat/ $C$ has small colimits and small limits.
    \begin{reference}
        \cite[Corollary 4.2.4.8]{HTT}
    \end{reference}
\end{corollary}
\begin{corollary}
    The \inftycat/ of spaces $\spaces$ is locally presentable.
    \begin{proof}\label{cor:spacesIsLocPres}
        This follows directly by \cref{prop:homotopyHypothesis} and \cref{cor:htpyCoherentNerveIsLoc}.
    \end{proof}
\end{corollary}
Note that $\Top$ as an ordinary category is not locally presentable. %TODO why
An overview for some equivalent characterizations for an \inftycat/ to be locally presentable can be found in \cite[Theorem 5.5.1.1 and Proposition A.3.7.6]{HTT}.

Next we introduce cartesian transformations which allow us to encode universality and descent concisely.
\begin{definition}[Cartesian Transformation]
    Let $C$ be an \inftycat/ and let $K$ be a simplicial set.
    Let $q,p\in\Fun(K,C)_0$ and let $\alpha\in\Fun(K,C)_1$ be a natural transformation $\alpha\colon p\to q$.
    Then we say $\alpha$ is \emph{cartesian} if for all maps $u\colon\Delta^1\to K$ the square 
    \begin{center}
        \begin{tikzcd} [sep= 4em]
            p(x) \arrow[r, "\alpha(x)"] \arrow [d, "p(f)"'] & q(x) \arrow [d, "q(f)"]\\
            p(y) \arrow[r, "\alpha(y)"] & q(y)
        \end{tikzcd}
    \end{center}
    represented by the map $\Delta^1\times\Delta^1\xrightarrow{\left(u,\id_{\Delta^1}\right)}K\times\Delta^1\xrightarrow{\alpha}C$ is a pullback square in $C$. 
\end{definition}
\begin{definition}[Universality]
    Let $C$ be an \inftycat/ that admits small colimits and pullbacks and let $K$ be a small simplicial set.
    We say that \emph{$C$ has $K$-shaped universal colimits} if for all natural transformations $\alpha\colon p\to q\in\Fun(K^{\rhd},C)_1$ where $q$ is a colimit cone and $\alpha$ is cartesian, $p$ is a also a colimit cone.
    
    We say that \emph{$C$ has universal colimits} if $C$ has $K$-shaped universal colimits for every small simplicial set $K$.
\end{definition}
\begin{definition}[Descent]
    Let $C$ be an \inftycat/ that admits small colimits and pullbacks and let $K$ be a small simplicial set.
    We say that \emph{$C$ has $K$-shaped descent} if for all natural transformations $\alpha\colon p\to q\in\Fun(K^{\rhd},C)_1$ where $p,q$ are colimit cones and $\alpha|_K$ is cartesian, $\alpha$ is also cartesian.
    
    We say that \emph{$C$ has descent} if $C$ has $K$-shaped descent for every small simplicial set $K$.
\end{definition}
\begin{remark}
    We want to describe universality and descent explicitly for pushouts ($\Lambda^2_0$-shaped diagrams), because these will be important cases as \cref{lem:univColimIffUnivPoAndCoprod} and \cref{lem:descentIffDescentPoAndCoprod} will show.
    A natural transformation $\alpha\colon\left(\Lambda_0^2\right)^{\rhd}\times\Delta^1\cong\Delta^1\times\Delta^1\times\Delta^1\to C$ is a cube
    \begin{center}
        \begin{tikzcd} [sep = .5 cm]
            \alpha|_{\set{0,0,0}} \arrow [dr] \arrow [rr] \arrow [dd] & & \alpha|_{\set{0,1,0}} \arrow [dr] \arrow [dd] \\
            & \alpha|_{\set{1,0,0}} \arrow [rr, crossing over] \arrow [dd] & & \alpha|_{\set{1,1,0}} \arrow [dd] & \\
            \alpha|_{\set{0,0,1}} \arrow [dr] \arrow [rr] & & \alpha|_{\set{0,1,1}} \arrow [dr] \\
            & \alpha|_{\set{1,0,1}} \arrow [from=uu, crossing over] \arrow [rr] & & \alpha|_{\set{1,1,1}}
        \end{tikzcd}\;.
    \end{center}
    Universality for pushouts means that whenever the vertical faces are pullbacks and the bottom face is a pushout, then the top face is a pushout.
    Descent for pushouts means that whenever the left and rear faces are pullbacks and top and bottom faces are pushouts, then the right and front faces are pullbacks as well.
    
    So this definition is exactly the one given in \cref{sec:motivationInftyTopoi}.
\end{remark}
\begin{definition}[\infty-Topos]
    Let $C$ be an \inftycat/ fulfilling the following properties:
    \begin{itemize}
        \item $C$ is locally presentable.
        \item $C$ has universal colimits.
        \item $C$ has descent.
    \end{itemize}
    Then we call $C$ an \emph{\infty-topos}.
\end{definition}
In order to prove that a given \inftycat/ has universality and descent we only need to check it for two types of diagrams: pushouts and small coproducts.
This is the content of the next two lemmas and will be used later to prove that $\spaces$ is an \inftytop/.

We will require the following proposition for the proof.
\begin{prop}\label{prop:colimitDecompositionPushouts}
    Let $C$ be a cocomplete \inftycat/, let $K$ be a simplicial set given by a pushout
    \begin{center}
        \begin{tikzcd} [sep = 4em]
            K_0 \arrow[r] \arrow[d, hook] & K_2 \arrow[d, hook] \\
            K_1 \arrow[r] & K
        \end{tikzcd}
    \end{center}
    where $K_0\xhookrightarrow{} K_1$ is a monomorphism, and let $p\colon K\to C$ be a map of simplicial sets.

    Set $X_i$ to be the colimit of $K_i\to K\to C$ for $i\in\set{0,1,2}$.
    Then the induced cone $\overline{p}\colon K^{\rhd}\to C$ given by $\overline{p}|_K=p$ and universal maps to the conepoint $\overline{p}(\infty)=X_1\cup_{X_0}X_2$ is a colimit cone in $C$.
    \begin{reference}
        \cite[Proposition 4.4.2.2]{HTT}
    \end{reference}
\end{prop}
\begin{lemma}\label{lem:univColimIffUnivPoAndCoprod}
    A \inftycat/ $C$ has universal colimits if and only if it has universal pushouts and universal coproducts.
    \begin{proof}
        Let $\alpha\colon p\to q\in\Fun(K^{\rhd},C)_1$ be a cartesian transformation and let $q$ be a colimit cone. 
        We prove the statement first for all finite dimensional $K$ by induction over the dimension of $K$.

        For $\dim K=-1$ the statement follows since $K=\emptyset$ is the coproduct over the empty indexing category.
        
        Let $\dim K=n+1$. 
        Then by the skeletal filtration
        \begin{equation*}
            K=K^{n+1}=K^n\coprod\limits_{\coprod\limits_S \partial\Delta^{n+1}}\left(\coprod_S \Delta^{n+1}\right)
        \end{equation*}
        where $S$ is the set of non-degenerate simplices of dimension $n+1$.
        For ease of notation, we set $X_{0}=\coprod\limits_S \partial\Delta^{n+1}$, $X_1=K^n$ and $X_2=\coprod\limits_S \Delta^{n+1}$.

        Let $p_i,q_i\colon X_i^{\rhd}\to C$ be the maps obtained by precomposing $p$ or $q$ with the respective map to $K$ from the skeletal filtration and taking the conepoint of $q_i$ to be the colimit, and the conepoint of $p_i$ to be the pullback 
        \begin{center}
            \begin{tikzpicture}
                \matrix (m) [matrix of math nodes,row sep=3em,column sep=4em,minimum width=2em]
                {
                  p_i(\infty) & p(\infty)\\
                  q_i(\infty) & q(\infty)\\};
                \path[-stealth]
    
                (m-1-1) edge (m-1-2)
                (m-1-1) edge (m-2-1)
                (m-2-1) edge (m-2-2)
                (m-1-2) edge (m-2-2);
            \end{tikzpicture}\;.
        \end{center}
        Then $\alpha$ induces maps $\alpha_i\colon p_i\to q_i$ which are cartesian by the pasting law.

        By the induction hypothesis we thus have that $p_0$ and $p_1$ are colimit cones.

        Since $\Delta^{n+1}$ has a terminal object, colimits over $\Delta^{n+1}$ can be computed by evaluating the terminal object.
        Let the family $(x_s)_{s\in S}$ be the images of the terminal objects of the components in $X_2=\coprod\limits_{s\in S} \Delta^{n+1}$ under the map $X_2\to K$.
        Then we have $q_2(\infty)\cong\coprod\limits_{s\in S} q(x_s)$ as colimits commute.
        So for every $s\in S$ the left square in the diagram
        \begin{center}
            \begin{tikzpicture}
                \matrix (m) [matrix of math nodes,row sep=3em,column sep=4em,minimum width=2em]
                {
                    p(x_s) & p_2(\infty) & p(\infty)\\
                    q(x_s) & q_2(\infty) & q(\infty)\\};
                \path[-stealth]
    
                (m-1-1) edge (m-1-2)
                (m-1-1) edge (m-2-1)
                (m-2-1) edge (m-2-2)
                (m-1-2) edge (m-2-2)
                (m-1-2) edge (m-1-3)
                (m-2-2) edge (m-2-3)
                (m-1-3) edge (m-2-3);
            \end{tikzpicture}
        \end{center}
        is a pullback by the pasting law, thus by universality for coproducts $p_2(\infty)\cong\coprod\limits_{s\in S} p(x_s)$ and in particular $p_2$ is a colimit cone.
        
        By \cref{prop:colimitDecompositionPushouts} we know that 
        \begin{center}
            \begin{tikzpicture}
                \matrix (m) [matrix of math nodes,row sep=3em,column sep=4em,minimum width=2em]
                {
                  q_0(\infty) & q_1(\infty)\\
                  q_2(\infty) & q(\infty)\\};
                \path[-stealth]
    
                (m-1-1) edge (m-1-2)
                (m-1-1) edge (m-2-1)
                (m-2-1) edge (m-2-2)
                (m-1-2) edge (m-2-2);
            \end{tikzpicture}
        \end{center}
        is a pushout square.
        Furthermore, the natural transformation $\beta\in\Fun\left(\left(\Lambda_0^2\right)^{\rhd},C\right)_1$ induced by $\alpha$ and the $\alpha_i$ which we represent by a cube
        \begin{center}
            \begin{tikzcd} [sep = .5 cm]
                p_0(\infty) \arrow [dr] \arrow [rr] \arrow [dd] & & p_1(\infty) \arrow [dr] \arrow [dd] \\
                & p_2(\infty) \arrow [rr, crossing over] \arrow [dd] & & p(\infty) \arrow [dd] & \\
                q_0(\infty) \arrow [dr] \arrow [rr] & & q_1(\infty) \arrow [dr] \\
                & q_2(\infty) \arrow [from=uu, crossing over] \arrow [rr] & & q(\infty) &
            \end{tikzcd}
        \end{center}
        is cartesian by the pasting law.
        By universality for pushouts (since the bottom square is pushout) we thus have that the top square is also a pushout.
        Again by \cref{prop:colimitDecompositionPushouts} it follows that $p$ is a colimit cone.

        It now remains to show that statement for general $K$.
        Let $L\subset\N(\Nat)$ (where we view $\Nat$ as a category via the ordering $0\to1\to2\to3\ldots$) be the simplicial subset containing all $0$-simplices and only the $1$-simplices which are between consecutive integers (with all higher simplices degenerate).
        It can be shown that the inclusion $L\subset\N(\Nat)$ is a weak categorical equivalence.

        Let $\alpha\colon p\to q\in\Fun(K^{\rhd},C)_1$ be a cartesian transformation, let $q$ be a colimit cone and let $\left(K^n\right)_{n\in\Nat}$ denote the skeletal filtration of $K$.
        We can compute colimits of any diagram indexed by $K$ by computing the colimit of the diagram $\N(\Nat)\to C$ that for each $0$-simplex $n\in\Nat$ is the colimit of $K^n\to C$ and all higher simplices are given by chains of the canonical maps. 

        Since $L\subset\N(\Nat)$ is a weak categorical equivalence (which preserve and reflect (co)limits), we can also compute the colimit of the induced diagram $L\to C$. %TODO maybe more detail, colimits can be computed as colimits over \N(\Nat) and skeletal filtration

        So let $F_q\colon L^{\rhd}\to C$ be the diagram induced by taking colimits of the skeleta as described above.
        Let $F_p\colon L^{\rhd}\to C$ be given by taking pullbacks
        \begin{center}
            \begin{tikzpicture}
                \matrix (m) [matrix of math nodes,row sep=3em,column sep=4em,minimum width=2em]
                {
                  F_p(n) & F_p(\infty)=p(\infty)\\
                  F_q(n) & F_q(\infty)=q(\infty)\\};
                \path[-stealth]
    
                (m-1-1) edge (m-1-2)
                (m-1-1) edge (m-2-1)
                (m-2-1) edge (m-2-2)
                (m-1-2) edge (m-2-2);
            \end{tikzpicture}
        \end{center}
        which induce a cartesian natural transformation $\gamma\colon F_p\to F_q\in\Fun(L^{\rhd},C)_1$.
        Since $L$ is $1$-dimensional, $F_p$ is a colimit cone by universality for finite dimensional diagrams.
        By universality for finite dimensional $K$ we also have that $F_p(n)$ is the colimit of $p|_{K^n}$ for every $n\in\Nat$.
        This proves that $p$ is a colimit cone and thus the proposition.
    \end{proof}
\end{lemma}
\begin{lemma}\label{lem:descentIffDescentPoAndCoprod}
    A \inftycat/ $C$ that has universal colimits has descent if and only if it has descent for pushouts and coproducts.
    \begin{proof}
        Let $\alpha\colon p\to q\in\Fun(K^{\rhd},C)_1$ be natural transformation between colimit cones such that $\alpha|_K$ is cartesian.
        We first prove the statement for all finite dimensional $K$ by induction over the dimension of $K$.

        For $\dim K=-1$ the statement follows since $K=\emptyset$ is the coproduct over the empty indexing category.

        So let $\dim K=n+1$.
        Then by the skeletal filtration        
        \begin{equation*}
            K=K^{n+1}=K^n\coprod\limits_{\bigcup\limits_S \partial\Delta^{n+1}}\bigcup_S \Delta^{n+1}
        \end{equation*}
        where $S$ is the set of non-degenerate simplices of dimension $n+1$.
        For ease of notation, we again set $X_{0}=\bigcup\limits_S \partial\Delta^{n+1}$, $X_1=K^n$ and $X_2=\bigcup\limits_S \Delta^{n+1}$.
        
        Let $p_i,q_i\colon X_i^{\rhd}\to C$ be the maps obtained by precomposing $p$ and $q$ with the respective map to $K$ from the skeletal filtration and taking the conepoints to be the colimits.
        Then $\alpha$ induces maps $\alpha_i\colon p_i\to q_i$.
        By the induction hypothesis we obtain that $\alpha_0$ and $\alpha_1$ are cartesian.

        The map $\alpha_2$ is cartesian as well: 
        Since $\Delta^{n+1}$ has a terminal object, a colimit of a diagram indexed over $\Delta^{n+1}$ can be computed by evaluating the final vertex. 
        Let the family $(x_s^i)_{i\in\set{0,\ldots,n+1},s\in S}$ be family of images consisting of the $i$-th vertex of the components in $X_2=\bigcup\limits_{s\in S} \Delta^{n+1}$ under the map $X_2\to K$.
        We have $q_2(\infty)\cong\bigcup\limits_{s\in S} q(x_s^{n+1})$ and $p_2(\infty)\cong\bigcup\limits_{s\in S} p(x_s^{n+1})$.
        For each $s\in S$ and $i\in\set{0,\ldots,n+1}$ the outer square of the diagram
        \begin{center}
            \begin{tikzcd} [sep = 4em]
                p_2(x_s^i) \arrow[r] \arrow[d] & p_2(x_s^{n+1}) \arrow[r] \arrow[d] & p_2(\infty) \arrow[d] \\
                q_2(x_s^i) \arrow[r] & q_2(x_s^{n+1}) \arrow[r] & q_2(\infty)\\
            \end{tikzcd}
        \end{center}
        is a pullback since the left square is a pullback because $\alpha_2|_{X_2}$ is cartesian and the right square is a pullback by descent for coproducts.
        Thus $\alpha_2$ is cartesian.

        By \cref{prop:colimitDecompositionPushouts} we know that
        \begin{center}
            \begin{tikzcd} [sep = 4em]
                p_0(\infty) \arrow[d] \arrow[r] & p_1(\infty) \arrow[d] \\
                p_2(\infty) \arrow[r] & p(\infty)
            \end{tikzcd}
            \begin{tikzcd} [sep = 4em]
                q_0(\infty) \arrow[d] \arrow[r] & q_1(\infty) \arrow[d] \\
                q_2(\infty) \arrow[r] & q(\infty)
            \end{tikzcd}
        \end{center}
        are pushouts.
        Furthermore, we know that the diagrams 
        \begin{center}
            \begin{tikzcd} [sep = 4em]
                p_0(\infty) \arrow[d] \arrow[r] & p_1(\infty) \arrow[d] \\
                q_0(\infty) \arrow[r] & q_1(\infty)
            \end{tikzcd}
            \begin{tikzcd} [sep = 4em]
                p_0(\infty) \arrow[d] \arrow[r] & p_2(\infty) \arrow[d] \\
                q_0(\infty) \arrow[r] & q_2(\infty)
            \end{tikzcd}
        \end{center}
        are pullbacks: 

        For all objects $x$ in $X_0$ and $i\in\set{1,2}$ the left square of the diagram 
        \begin{center}
            \begin{tikzcd} [sep = 4em]
                p_0(x) \arrow[r] \arrow[d] & q_0(\infty)\times_{q_i(\infty)}p_i(\infty) \arrow[r] \arrow[d] & p_i(\infty) \arrow[d] \\
                q_0(x) \arrow[r] & q_0(\infty) \arrow[r] & q_i(\infty)\\
            \end{tikzcd}
        \end{center}
        is a pullback by the pasting law since the outer square is a pullback as $\alpha_i$ is cartesian.
        By universality, this implies that $q_0(\infty)\times_{q_i(\infty)}p_i(\infty)$ is a colimit of $p_0|_{X_0}$ and thus equivalent to $p_0(\infty)$.

        Thus the natural transformation $\beta\in\Fun\left(\left(\Lambda_0^2\right)^{\rhd}, C\right)_1$ induced by $\alpha$ and the $\alpha_i$ which we again represent by a cube
        \begin{center}
            \begin{tikzcd} [sep = .5 cm]
                p_0(\infty) \arrow [dr] \arrow [rr] \arrow [dd] & & p_1(\infty) \arrow [dr] \arrow [dd] \\
                & p_2(\infty) \arrow [rr, crossing over] \arrow [dd] & & p(\infty) \arrow [dd] & \\
                q_0(\infty) \arrow [dr] \arrow [rr] & & q_1(\infty) \arrow [dr] \\
                & q_2(\infty) \arrow [from=uu, crossing over] \arrow [rr] & & q(\infty) &
            \end{tikzcd}
        \end{center}
        is cartesian because $C$ has descent for pushouts.

        Since every object $x\in K_0$ already lies in the image of either $X_1$ or $X_2$, the maps $p(x)\to p(\infty)$ and $q(x)\to q(\infty)$ always factor through $p_1(\infty)$ or $p_2(\infty)$ respectively.
        Because for $i\in\set{1,2}$ the diagram
        \begin{center}
            \begin{tikzcd} [sep = 4em]
                p_i(x) \arrow[r] \arrow[d] & p_i(\infty) \arrow[r] \arrow[d] & p(\infty) \arrow[d] \\
                q_i(x) \arrow[r] & q_i(\infty) \arrow[r] & q(\infty)\\
            \end{tikzcd}
        \end{center}
        is a pullback by the pasting lemma due to the above cube and since $\alpha_i$ is cartesian, it follows that $\alpha$ is cartesian.

        The case for general $K$ follows from the following argument:

        Let again $L\subset\N(\Nat)$ as in the previous lemma and let $\alpha\colon p\to q$ be a natural transformation between colimit cones $p,q\colon K^{\rhd}\to C$ such that $\alpha|_K$ is cartesian.
        Let $K^n\subset K$ denote the $n$-th skeleton and let $F_p\colon L^{\rhd}\to C$ an $F_q\colon L^{\rhd}\to C$ be functors that 
        \begin{itemize}
            \item for each object $n\in L_0$ are the colimit of $p|_{K^n}$ and $q|_{K^n}$ respectively 
            \item have $p(\infty)$ and $q(\infty)$ as their respective conepoints
        \end{itemize}
        so in particular, they are colimit cones.
        By descent for finite dimensions we know that for all $n\in\Nat$ and $x\in K^n$ the outer square of
        \begin{center}
            \begin{tikzcd} [sep = 4em]
                p|_{K^n}(x) \arrow[r] \arrow[d] & q|_{K^n}(\infty)\times_{q|_{K^{n+1}}(\infty)}p|_{K^{n+1}}(\infty) \arrow[r] \arrow[d] & p|_{K^{n+1}}(\infty) \arrow[d] \\
                q|_{K^n}(x) \arrow[r] & q|_{K^n}(\infty) \arrow[r] & q|_{K^{n+1}}(\infty)\\
            \end{tikzcd}
        \end{center}
        is a pullback and hence by the pasting lemma so is the left square.
        By universality this implies that $q|_{K^n}(\infty)\times_{q|_{K^{n+1}}(\infty)}p|_{K^{n+1}}(\infty)$ is equivalent to $p|_{K^n}(\infty)$ which shows that
        \begin{center}
            \begin{tikzpicture}
                \matrix (m) [matrix of math nodes,row sep=3em,column sep=4em,minimum width=2em]
                {
                  p|_{K^n}(\infty) & p|_{K^{n+1}}(\infty)\\
                  q|_{K^n}(\infty) & q|_{K^{n+1}}(\infty)\\};
                \path[-stealth]
    
                (m-1-1) edge (m-1-2)
                (m-1-1) edge (m-2-1)
                (m-2-1) edge (m-2-2)
                (m-1-2) edge (m-2-2);
            \end{tikzpicture}
        \end{center}
        is a pullback, so the transformation $\gamma\colon F_p\to F_q$ induced by $\alpha$ is cartesian when restricted to $L$.
        Since $L$ is $1$-dimensional, again from finite dimensional descent we have that $\gamma$ is cartesian.
        This shows that for all $n\in\Nat$ the square
        \begin{center}
            \begin{tikzpicture}
                \matrix (m) [matrix of math nodes,row sep=3em,column sep=4em,minimum width=2em]
                {
                  p|_{K^n}(\infty) & p(\infty)\\
                  q|_{K^n}(\infty) & q(\infty)\\};
                \path[-stealth]
    
                (m-1-1) edge (m-1-2)
                (m-1-1) edge (m-2-1)
                (m-2-1) edge (m-2-2)
                (m-1-2) edge (m-2-2);
            \end{tikzpicture}
        \end{center}
        is cartesian.

        We want to show that 
        \begin{center}
            \begin{tikzpicture}
                \matrix (m) [matrix of math nodes,row sep=3em,column sep=4em,minimum width=2em]
                {
                  p(x) & p(\infty)\\
                  q(x) & q(\infty)\\};
                \path[-stealth]
    
                (m-1-1) edge (m-1-2)
                (m-1-1) edge (m-2-1)
                (m-2-1) edge (m-2-2)
                (m-1-2) edge (m-2-2);
            \end{tikzpicture}
        \end{center}
        is a pullback for all $x\in K_0$. 
        But every such $x$ already is in some $K^n$, so we obtain a factorization
        \begin{center}
            \begin{tikzpicture}
                \matrix (m) [matrix of math nodes,row sep=3em,column sep=4em,minimum width=2em]
                {
                    p(x)=p|_{K^n}(x) & p|_{K^n}(\infty) & p(\infty)\\
                    q(x)=q|_{K^n}(x) & q|_{K^n}(\infty) & q(\infty)\\};
                \path[-stealth]
    
                (m-1-1) edge (m-1-2)
                (m-1-1) edge (m-2-1)
                (m-2-1) edge (m-2-2)
                (m-1-2) edge (m-2-2)
                (m-1-2) edge (m-1-3)
                (m-2-2) edge (m-2-3)
                (m-1-3) edge (m-2-3);
            \end{tikzpicture}
        \end{center}
        where the left square is pullback by descent for finite dimensions. 
        Thus the outer square cartesian, proving the proposition.
    \end{proof}
\end{lemma}