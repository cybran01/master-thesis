Before giving a proper definition of universality and descent, we need to introduce another more subtle property that we want an \inftytop/ to fulfill.
Since this we be the least interesting for our purposes, we will give a very pragmatic introduction.
\begin{definition}
    An \inftycat/ $C$ is \emph{locally presentable} if there exists a combinatorial simplicial model category $A$ and an equivalence of \inftycats/ $\Nhc(A^°)\to C$.
\end{definition}
\begin{corollary}
    A locally presentable \inftycat/ $C$ has small colimits and small limits.
    \begin{proof}
        These are computed in some combinatorial simplicial model category $A$ where they exist.
    \end{proof}
\end{corollary}
\begin{corollary}
    The \inftycat/ of spaces $\spaces$ is locally presentable.
    \begin{proof}\label{cor:spacesIsLocPres}
        This follows directly by \cref{prop:homotopyHypothesis} and \cref{cor:htpyCoherentNerveIsLoc}.
    \end{proof}
\end{corollary}
An overview for some equivalent characterizations for an \inftycat/ to be locally presentable can be found in \cite[Theorem 5.5.1.1 and Proposition A.3.7.6]{HTT}.
In essence, an \inftycat/ being locally presentable means it is being determined by a ``small'' amount of data, despite the category in question not needing to be small itself.

Next we introduce cartesian transformations which allow us to encode universality and descent in a concise manner.
\begin{definition}[Cartesian Transformation]
    Let $C$ be an \inftycat/ and let $K$ be a simplicial set.
    Let $q,p\in\Fun(K,C)_0$ and let $\alpha\in\Fun(K,C)_1$ be a natural transformation $\alpha\colon p\to q$.
    Then we say $\alpha$ is \emph{cartesian} if for all maps $u\colon\Delta^1\to K$ the induced map $\Delta^1\times\Delta^1\xrightarrow{\left(u,\id_{\Delta^1}\right)}K\times\Delta^1\xrightarrow{\alpha}C$ is a pullback square. %TODO add comm sq
\end{definition}
\begin{definition}[Universality]
    Let $C$ be an \inftycat/ that admits small colimits and pullbacks and let $K$ be a simplicial set.
    We say that \emph{$C$ has $K$-shaped universal colimits} if for all natural transformations $\alpha\colon p\to q\in\Fun(K^{\rhd},C)_1$ such that $q$ is a colimit cone and $\alpha$ is cartesian, $p$ is a colimit cone.
    
    We say that \emph{$C$ has universal colimits} if $C$ has $K$-shaped universal colimits for every small simplicial set $K$.
\end{definition}
\begin{definition}[Descent]
    Let $C$ be an \inftycat/ that admits small colimits and pullbacks.
    We say that \emph{$C$ has $K$-shaped descent} if for all natural transformations $\alpha\colon p\to q\in\Fun(K^{\rhd},C)_1$ such that $p,q$ are colimit cones and $\alpha|_K$ is cartesian, $\alpha$ is cartesian.
    
    We say that \emph{$C$ has desent} if $C$ has $K$-shaped descent for every small simplicial set $K$.
\end{definition}
\begin{remark}
    We want to describe universality and descent explicitly for pushouts ($\Lambda^2_0$-shaped diagrams), because these will be important cases as \cref{lem:univColimIffUnivPoAndCoprod} and \cref{lem:descentIffDescentPoAndCoprod} will show.
    A natural transformation $\alpha\colon\left(\Lambda_0^2\right)^{\rhd}\times\Delta^1\cong\Delta^1\times\Delta^1\times\Delta^1\to C$ is a cube
    \begin{center}
        \begin{tikzcd} [sep = .5 cm]
            \alpha|_{\set{0,0,0}} \arrow [dr] \arrow [rr] \arrow [dd] & & \alpha|_{\set{0,1,0}} \arrow [dr] \arrow [dd] \\
            & \alpha|_{\set{1,0,0}} \arrow [rr, crossing over] \arrow [dd] & & \alpha|_{\set{1,1,0}} \arrow [dd] & \\
            \alpha|_{\set{0,0,1}} \arrow [dr] \arrow [rr] & & \alpha|_{\set{0,1,1}} \arrow [dr] \\
            & \alpha|_{\set{1,0,1}} \arrow [from=uu, crossing over] \arrow [rr] & & \alpha|_{\set{1,1,1}} &
        \end{tikzcd}\;.
    \end{center}
    Universality for pushouts means that whenever the vertical faces are pullbacks and the bottom face is a pushout, then the top face is a pushout.
    Descent for pushouts means that whenever left and rear face are pullbacks and top and bottom faces are pushouts, then the right and front face are pullbacks as well.
    
    So this definition is exactly the one given in \cref{sec:motivationInftyTopoi}.
\end{remark}
\begin{definition}[\infty-Topos]
    Let $C$ be an \inftycat/ fulfilling the following properties:
    \begin{itemize}
        \item $C$ is locally presentable.
        \item $C$ has universal colimits.
        \item $C$ has descent.
    \end{itemize}
    Then we call $C$ an \emph{\infty-topos}.
\end{definition}
In order to prove that a given \inftycat/ has universality and descent we only need to check it for two types of diagrams: pushouts and small coproducts.
This is the content of the next two lemmas and will be used later to prove that $\spaces$ is an \inftytop/.

We will require the following proposition for the proof.
\begin{prop}\label{prop:colimitDecompositionPushouts}
    Let $C$ be a cocomplete \inftycat/ and let $p\colon K\to C$ be a map of simplicial sets. 
    Let $K$ be the pushout of
    \begin{center}
        \begin{tikzcd} [sep = 4em]
            K_0 \arrow[r] \arrow[d, hook] & K_2 \arrow[d, hook] \\
            K_1 \arrow[r] & K
        \end{tikzcd}
    \end{center}
    where $K_0\xhookrightarrow{} K_1$ is a monomorphism.

    Let $X$ be a colimit of $K_0\to K\to C$, $Y$ colimit of $K_1\to K\to C$ and $Z$ a colimit of $K_2\to K\to C$.
    Then colimits of $p$ can be identified with pushouts $K_1\cup_{K_0}K_2$ in $C$.
    \begin{reference}
        \cite[Proposition 4.4.2.2]{HTT}
    \end{reference}
\end{prop}
\begin{lemma}\label{lem:univColimIffUnivPoAndCoprod}
    A \inftycat/ $C$ has universal colimits if and only if it has universal pushouts and universal coproducts.
    \begin{proof}
        Let $\alpha\colon p\to q\in\Fun(K^{\rhd},C)_1$ be a cartesian transformation and let $q$ be a colimit cone. 
        We prove the statement first for all finite dimensional $K$ by induction over the dimension of $K$.

        For $\dim K=-1$ the statement follows since $K$ must be empty and thus is the coproduct over the empty indexing category.
        
        Let $\dim K=n+1$. 
        Then by the skeletal filtration
        \begin{equation*}
            K=K^{n+1}=K^n\coprod\limits_{\bigcup\limits_S \partial\Delta^{n+1}}\bigcup_S \Delta^{n+1}
        \end{equation*}
        where $S$ is the set of non-degenerate simplicies of dimension $n+1$.
        For ease of notation, we set $X_{0}=\bigcup\limits_S \partial\Delta^{n+1}$, $X_1=K^n$ and $X_2=\bigcup\limits_S \Delta^{n+1}$.

        Let $p_i,q_i\colon X_i^{\rhd}\to C$ be the maps obtained by precomposing $p$ or $q$ with the respective map to $K$ from the skeletal filtration and taking the conepoint of $q_i$ to be the colimit, and the conepoint of $p_i$ to be the pullback %TODO how does that even define a cone?
        \begin{center}
            \begin{tikzpicture}
                \matrix (m) [matrix of math nodes,row sep=3em,column sep=4em,minimum width=2em]
                {
                  p_i(\infty) & q(\infty)\\
                  q_i(\infty) & p(\infty)\\};
                \path[-stealth]
    
                (m-1-1) edge (m-1-2)
                (m-1-1) edge (m-2-1)
                (m-2-1) edge (m-2-2)
                (m-1-2) edge (m-2-2);
            \end{tikzpicture}\;.
        \end{center}
        Then $\alpha$ induces maps $\alpha_i\colon p_i\to q_i$ which are cartesian. %TODO how? (where does map to cone come from?)

        By the pasting law and the induction hypothesis we thus have that $p_0$ and $p_1$ are colimit cones.

        Since $\Delta^{n+1}$ has a terminal object, colimits over $\Delta^{n+1}$ can be computed by evalutaing the terminal object.
        In particular it is given by evaluation of some element $x\in K_0$.

        Thus we have $q_2(\infty)\cong\bigcup\limits_{s\in S} q(x_s)$ (where $x_s\in K_0$ is the final vertex of the corresponding $\Delta^{n+1}$ in $X_2=\bigcup\limits_{s\in S} \Delta^{n+1}$).
        So for every $s\in S$ the left square in the diagram
        \begin{center}
            \begin{tikzpicture}
                \matrix (m) [matrix of math nodes,row sep=3em,column sep=4em,minimum width=2em]
                {
                    p(x_s) & p_2(\infty) & p(\infty)\\
                    q(x_s) & q_2(\infty) & q(\infty)\\};
                \path[-stealth]
    
                (m-1-1) edge (m-1-2)
                (m-1-1) edge (m-2-1)
                (m-2-1) edge (m-2-2)
                (m-1-2) edge (m-2-2)
                (m-1-2) edge (m-1-3)
                (m-2-2) edge (m-2-3)
                (m-1-3) edge (m-2-3);
            \end{tikzpicture}
        \end{center}
        is pullback by the pasting law, thus by universality for coproducts $p_2(\infty)\cong\bigcup\limits_{s\in S} p(x_s)$ and in particular $p_2$ is a colimit cone.
        
        By \cref{prop:colimitDecompositionPushouts} we know that 
        \begin{center}
            \begin{tikzpicture}
                \matrix (m) [matrix of math nodes,row sep=3em,column sep=4em,minimum width=2em]
                {
                  q_0(\infty) & q_1(\infty)\\
                  q_2(\infty) & q(\infty)\\};
                \path[-stealth]
    
                (m-1-1) edge (m-1-2)
                (m-1-1) edge (m-2-1)
                (m-2-1) edge (m-2-2)
                (m-1-2) edge (m-2-2);
            \end{tikzpicture}
        \end{center}
        is a pushout square.
        Furthermore, the cube induced by $\alpha$ and the $\alpha_i$
        \begin{center}
            \begin{tikzcd} [sep = .5 cm]
                p_0(\infty) \arrow [dr] \arrow [rr] \arrow [dd] & & p_1(\infty) \arrow [dr] \arrow [dd] \\
                & p_2(\infty) \arrow [rr, crossing over] \arrow [dd] & & p(\infty) \arrow [dd] & \\
                q_0(\infty) \arrow [dr] \arrow [rr] & & q_1(\infty) \arrow [dr] \\
                & q_2(\infty) \arrow [from=uu, crossing over] \arrow [rr] & & q(\infty) &
            \end{tikzcd}
        \end{center}
        which determines a natural transformation $\beta\in\Fun\left(\left(\Lambda_0^2\right)^{\rhd},C\right)_1$ is cartesian by the pasting law.
        By universality for pushouts (since the bottom square is pushout) we thus have that the top square is also a pushout.
        Again by \cref{prop:colimitDecompositionPushouts} it follows that $p(\infty)$ is a colimit cone.

        It now remains to show that statement for general $K$.
        Let $L\subset\N(\Nat)$ (where we view $\Nat$ as a category via the ordering $0\to1\to2\to3\ldots$) be the simplicial subset containing all $0$-simplices and only the $1$-simplices which are between consecutive integers (and all higher simplices are degenerate).
        It can be shown that the inclusion $L\subset\N(\Nat)$ is a categorical equivalence. %TODO maybe more detail, colimits can be computed as colimits over \N(\Nat) and skeletal filtration

        Let $\alpha\colon p\to q\in\Fun(K^{\rhd},C)_1$ be a cartesian transformation and let $q$ be a colimit cone and let $\left(K^n\right)_{n\in\Nat}$ denote the skeletal filtration of $K$.
        We can compute colimits of any diagram indexed by $K$ by computing the colimit of the diagram $\N(\Nat)\to C$ that for each $0$-simplex is the colimit of the skeleton of corresponding degree and all higher simplicies are given by chains of the canonical maps. 

        Since $L\subset\N(\Nat)$ is a categorical equivalence (which preserve and reflect (co)limits), we can also compute the colimit of the induced diagram $L\to C$. %TODO maybe more detail, colimits can be computed as colimits over \N(\Nat) and skeletal filtration

        So let $F_q\colon L^{\rhd}\to C$ be the diagram induced by taking colimits of the skeleta as above.
        Let $F_p\colon L^{\rhd}\to C$ be given on the $0$-simplices by taking pullbacks
        \begin{center}
            \begin{tikzpicture}
                \matrix (m) [matrix of math nodes,row sep=3em,column sep=4em,minimum width=2em]
                {
                  F_p(n) & F_q(n)\\
                  p(\infty)=F_p(\infty) & F_q(\infty)=q(\infty)\\};
                \path[-stealth]
    
                (m-1-1) edge (m-1-2)
                (m-1-1) edge (m-2-1)
                (m-2-1) edge (m-2-2)
                (m-1-2) edge (m-2-2);
            \end{tikzpicture}
        \end{center}
        and on $1$-simplices by taking canonical maps.
        Thus we get a cartesian natural transformation $\beta\colon F_p\to F_q$.
        By universality for finite dimensional $K$ we have that for every $n\in\Nat$, $F_p(n)$ is the colimit of $p|_{K^n}$.
        Since $L$ is $1$-dimensional, $F_p$ is a colimit cone and thus $p(\infty)$ is the colimit of its skeleta.
        This proves that $p$ is a colimit cone and thus the proposition.
    \end{proof}
\end{lemma}
\begin{lemma}\label{lem:descentIffDescentPoAndCoprod}
    A \inftycat/ $C$ that has universal colimits has descent if and only if it has descent for pushouts and coproducts.
    \begin{proof} %TODO maybe mention why it suffices to prove things for arrows x\to\infty
        Let $\alpha\colon p\to q\in\Fun(K^{\rhd},C)_1$ be a map.
        We prove the statement first for all finite dimensional $K$ by induction over the dimension of $K$. %TODO define dimension

        For $\dim K=-1$ the statement follows since $K$ must be empty and thus is the coproduct over the empty indexing category.

        So let $\dim K=n+1$.
        Then by the skeletal filtration        
        \begin{equation*}
            K=K^{n+1}=K^n\coprod\limits_{\bigcup\limits_S \partial\Delta^{n+1}}\bigcup_S \Delta^{n+1}
        \end{equation*}
        where $S$ is the set of non-degenerate simplicies of dimension $n+1$.
        For ease of notation, we again set $X_{0}=\bigcup\limits_S \partial\Delta^{n+1}$, $X_1=K^n$ and $X_2=\bigcup\limits_S \Delta^{n+1}$.
        
        Let $p_i,q_i\colon X_i^{\rhd}\to C$ be the maps obtained by precomposing $p$ or $q$ with the respective map to $K$ from the skeletal filtration and taking the conepoints to be the colimits.
        Then $\alpha$ induces maps $\alpha_i\colon p_i\to q_i$. %TODO how?
        By induction hypothesis we obtain that $\alpha_0$ and $\alpha_1$ are cartesian.

        The map $\alpha_2$ is cartesian as well: 
        Since $\Delta^{n+1}$ has a terminal object, the colimit can be computed by evaluating the final vertex. 
        In particular it is given by evaluation of some element $x\in K_0$.
        Therefore we have $q_2(\infty)\cong\bigcup\limits_{s\in S} q(x_s^{n+1})$ and $p_2(\infty)\cong\bigcup\limits_{s\in S} p(x_s^{n+1})$ (where $x_s^i\in K_0$ is the $i$-th vertex of the corresponding $\Delta^{n+1}$ in $X_2=\bigcup\limits_{s\in S} \Delta^{n+1}$).
        For each $s\in S$ and $i\in\set{0,\ldots,n+1}$ the outer square of the diagram
        \begin{center}
            \begin{tikzcd} [sep = 4em]
                p(x_s^i) \arrow[r] \arrow[d] & p_2(x_s^{n+1}) \arrow[r] \arrow[d] & p_2(\infty) \arrow[d] \\
                q(x_s^i) \arrow[r] & q_2(x_s^{n+1}) \arrow[r] & q_2(\infty)\\
            \end{tikzcd}
        \end{center}
        is a pullback since the left square is pullback as $\alpha_2|_{X_2}$ is cartesian and the right square is pullback by descent for coproducts.
        Thus $\alpha_2$ is cartesian.

        By \cref{prop:colimitDecompositionPushouts} we know that
        \begin{center}
            \begin{tikzcd} [sep = 4em]
                p_0(\infty) \arrow[d] \arrow[r] & p_1(\infty) \arrow[d] \\
                p_2(\infty) \arrow[r] & p(\infty)
            \end{tikzcd}
            \begin{tikzcd} [sep = 4em]
                q_0(\infty) \arrow[d] \arrow[r] & q_1(\infty) \arrow[d] \\
                q_2(\infty) \arrow[r] & q(\infty)
            \end{tikzcd}
        \end{center}
        are pushouts.
        Furthermore, we know that the diagrams 
        \begin{center}
            \begin{tikzcd} [sep = 4em]
                p_0(\infty) \arrow[d] \arrow[r] & p_1(\infty) \arrow[d] \\
                q_0(\infty) \arrow[r] & q_1(\infty)
            \end{tikzcd}
            \begin{tikzcd} [sep = 4em]
                p_0(\infty) \arrow[d] \arrow[r] & p_2(\infty) \arrow[d] \\
                q_0(\infty) \arrow[r] & q_2(\infty)
            \end{tikzcd}
        \end{center}
        are pullbacks: 

        For all $x\in \left(X_0\right)_0$ and $i\in\set{1,2}$ the left square of the diagram 
        \begin{center}
            \begin{tikzcd} [sep = 4em]
                p_0(x) \arrow[r] \arrow[d] & q_0(\infty)\times_{q_i(\infty)}p_i(\infty) \arrow[r] \arrow[d] & p_i(\infty) \arrow[d] \\
                q_0(x) \arrow[r] & q_0(\infty) \arrow[r] & q_i(\infty)\\
            \end{tikzcd}
        \end{center}
        is a pullback by the pasting law since the outer square is a pullback as $\alpha_i$ is cartesian.
        By universality, this implies that $q_0(\infty)\times_{q_i(\infty)}p_i(\infty)$ is a colimit of $p_0|_{X_0}$ and thus equivalent to $p_0(\infty)$.

        Thus the natural transformation $\beta\in\Fun\left(\left(\Lambda_0^2\right)^{\rhd}, C\right)_1$ induced by $\alpha$ and the $\alpha_i$ which we again represent by a cube
        \begin{center}
            \begin{tikzcd} [sep = .5 cm]
                p_0(\infty) \arrow [dr] \arrow [rr] \arrow [dd] & & p_1(\infty) \arrow [dr] \arrow [dd] \\
                & p_2(\infty) \arrow [rr, crossing over] \arrow [dd] & & p(\infty) \arrow [dd] & \\
                q_0(\infty) \arrow [dr] \arrow [rr] & & q_1(\infty) \arrow [dr] \\
                & q_2(\infty) \arrow [from=uu, crossing over] \arrow [rr] & & q(\infty) &
            \end{tikzcd}
        \end{center}
        is cartesian because $C$ has descent for pushouts.

        Since every object $x\in K_0$ already lies in the image of either $X_1$ or $X_2$, the map $p(x)\to p(\infty)$ always factors as $p_1(x)\to p_1(\infty)\to p(\infty)$ or $p_2(x)\to p_2(\infty)\to p(\infty)$.
        Descent for $\alpha$ therefore follows by the pasting lemma since both $\alpha_1$ and $\alpha_2$ are cartesian and the front and right side of the cube are pullbacks.

        The case for general $K$ follows from the following argument:

        Let again $L\subset\N(\Nat)$ as in the previous lemma and let $\alpha\colon p\to q$ be a natural transformation between colimit cones $p,q\colon K^{\rhd}\to C$ such that $\alpha|_K$ is cartesian.
        Let $K^n\subset K$ denote the $n$-th skeleton and let $F_p\colon L^{\rhd}\to C$ an $F_q\colon L^{\rhd}\to C$ be functors that 
        \begin{itemize}
            \item for each object $n\in L_0$ are the colimit of $p|_{K^n}$ and $q|_{K^n}$ respectively 
            \item have $p(\infty)$ and $q(\infty)$ for their respective conepoints
            \item have choices for canonical maps as their $1$-simplices.
        \end{itemize}
        By descent for finite dimensions and universality we know that for all $n\in\Nat$ and $x\in K^n$ the outer square of
        \begin{center}
            \begin{tikzcd} [sep = 4em]
                p|_{K^n}(x) \arrow[r] \arrow[d] & q|_{K^n}(\infty)\times_{q|_{K^{n+1}}(\infty)}p|_{K^{n+1}}(\infty) \arrow[r] \arrow[d] & p|_{K^{n+1}}(\infty) \arrow[d] \\
                q|_{K^n}(x) \arrow[r] & q|_{K^n}(\infty) \arrow[r] & q|_{K^{n+1}}(\infty)\\
            \end{tikzcd}
        \end{center}
        is a pullback and hence by the pasting lemma so is the left square.
        By universality this implies that $q|_{K^n}(\infty)\times_{q|_{K^{n+1}}(\infty)}p|_{K^{n+1}}(\infty)$ is equivalent to $p|_{K^n}$ which shows that
        \begin{center}
            \begin{tikzpicture}
                \matrix (m) [matrix of math nodes,row sep=3em,column sep=4em,minimum width=2em]
                {
                  p|_{K^n}(\infty) & p|_{K^{n+1}}(\infty)\\
                  q|_{K^n}(\infty) & q|_{K^{n+1}}(\infty)\\};
                \path[-stealth]
    
                (m-1-1) edge (m-1-2)
                (m-1-1) edge (m-2-1)
                (m-2-1) edge (m-2-2)
                (m-1-2) edge (m-2-2);
            \end{tikzpicture}
        \end{center}
        is a pullback, so the transformation $\beta\colon F_p\to F_q$ induced by $\alpha$ is cartesian when restricted to $L$.
        Since $L$ is $1$-dimensional, again from finite dimensional descent we have that $\beta$ is cartesian.
        This shows that for all $n\in\Nat$ the square
        \begin{center}
            \begin{tikzpicture}
                \matrix (m) [matrix of math nodes,row sep=3em,column sep=4em,minimum width=2em]
                {
                  p|_{K^n}(\infty) & p(\infty)\\
                  q|_{K^n}(\infty) & q(\infty)\\};
                \path[-stealth]
    
                (m-1-1) edge (m-1-2)
                (m-1-1) edge (m-2-1)
                (m-2-1) edge (m-2-2)
                (m-1-2) edge (m-2-2);
            \end{tikzpicture}
        \end{center}
        is cartesian.

        We want to show that 
        \begin{center}
            \begin{tikzpicture}
                \matrix (m) [matrix of math nodes,row sep=3em,column sep=4em,minimum width=2em]
                {
                  p(x) & p(\infty)\\
                  q(x) & q(\infty)\\};
                \path[-stealth]
    
                (m-1-1) edge (m-1-2)
                (m-1-1) edge (m-2-1)
                (m-2-1) edge (m-2-2)
                (m-1-2) edge (m-2-2);
            \end{tikzpicture}
        \end{center}
        is a pullback for all $x\in K_0$. 
        But every such $x$ already is in some $K^n$, so we obtain a factorization
        \begin{center}
            \begin{tikzpicture}
                \matrix (m) [matrix of math nodes,row sep=3em,column sep=4em,minimum width=2em]
                {
                    p(x)=p|_{K^n}(x) & p|_{K^n}(\infty) & p(\infty)\\
                    q(x)=q|_{K^n}(x) & q|_{K^n}(\infty) & q(\infty)\\};
                \path[-stealth]
    
                (m-1-1) edge (m-1-2)
                (m-1-1) edge (m-2-1)
                (m-2-1) edge (m-2-2)
                (m-1-2) edge (m-2-2)
                (m-1-2) edge (m-1-3)
                (m-2-2) edge (m-2-3)
                (m-1-3) edge (m-2-3);
            \end{tikzpicture}
        \end{center}
        where the left square is pullback by descent for finite dimensions. 
        Thus the outer square cartesian, proving the propsition.
    \end{proof}
\end{lemma}