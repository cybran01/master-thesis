\documentclass[a4paper,12pt,twoside,bibliography=totoc]{scrartcl}

\usepackage{polyglossia}
\usepackage{csquotes}
\setmainlanguage{english}
\usepackage{abstract}
\usepackage{pdfpages}
\usepackage{amsmath}
\usepackage{amsthm}
\usepackage{amssymb}
\usepackage{amsfonts}
\usepackage{mathtools}
\usepackage{stackengine}
\usepackage{faktor}
\usepackage{adjustbox}
\usepackage{circledsteps}
\usepackage{tikz}
\usetikzlibrary{matrix}
\usetikzlibrary{arrows}
\usetikzlibrary{decorations.pathmorphing}
\usetikzlibrary{cd}
\usepackage[automark, headsepline]{scrlayer-scrpage}
\ihead{\leftmark}
\ohead{\rightmark}
\usepackage{enumitem}
\usepackage[backend=biber,style=alphabetic]{biblatex}
\addbibresource{sources.bib}
\usepackage[pdfencoding=auto, psdextra]{hyperref}
\usepackage{cleveref}
\usepackage{subfiles}

\setkomafont{section}{\normalfont\Large}
\setkomafont{subsection}{\normalfont\large}

\theoremstyle{plain}

\newtheorem{definition}{Definition}[section]
\newtheorem{prop}[definition]{Proposition}
\newtheorem{thm}[definition]{Theorem}
\newtheorem{notation}[definition]{Notation}
\newtheorem{lemma}[definition]{Lemma}
\newtheorem{corollary}[definition]{Corollary}
\newtheorem{remark}[definition]{Remark}
\newtheorem{construction}[definition]{Construction}

\newenvironment{reference}{\renewcommand{\qedsymbol}{}\begin{proof}[Reference]}{\end{proof}}

\newcommand{\spaces}{\mathcal{S}}
\newcommand{\Nat}{\mathbb{N}}
\newcommand{\id}{\text{id}}
\newcommand{\set}[1]{\left\lbrace#1\right\rbrace}
\newcommand{\inftycat}{}
\def\inftycat/{$\infty$-category}
\newcommand{\inftyCat}{}
\def\inftyCat/{$\infty$-Category}
\newcommand{\inftycats}{}
\def\inftycats/{$\infty$-categories}
\newcommand{\inftyCats}{}
\def\inftyCats/{$\infty$-Categories}
\newcommand{\inftytop}{}
\def\inftytop/{$\infty$-topos}
\newcommand{\inftytops}{}
\def\inftytops/{$\infty$-topoi}
\newcommand{\inftyTops}{}
\def\inftyTops/{$\infty$-Topoi}
\newcommand{\Cech}{}
\def\Cech/{\v{C}ech}
\newcommand{\Strom}{}
\def\Strom/{Str{\o}m}

\DeclareMathOperator{\Z}{\mathbb{Z}}
\DeclareMathOperator{\R}{R}
\DeclareMathOperator{\C}{C}
\DeclareMathOperator{\CN}{\text{\v{C}}}
\DeclareMathOperator{\N}{N}
\DeclareMathOperator{\Sp}{Sp}
\DeclareMathOperator{\M}{M}
\DeclareMathOperator{\D}{D}
\DeclareMathOperator{\Nhc}{N^{hc}}
\DeclareMathOperator{\Hom}{Hom}
\DeclareMathOperator{\Map}{Map}
\DeclareMathOperator{\Sing}{Sing}
\DeclareMathOperator{\Cof}{Cof}
\DeclareMathOperator{\Fib}{Fib}
\DeclareMathOperator{\W}{W}
\DeclareMathOperator{\Fun}{Fun}
\DeclareMathOperator{\Cat}{Cat}
\DeclareMathOperator{\Cart}{Cart}
\DeclareMathOperator{\op}{op}
\DeclareMathOperator{\Set}{Set}
\DeclareMathOperator{\sSet}{sSet}
\DeclareMathOperator{\Top}{Top}
\DeclareMathOperator{\Gpd}{Gpd}
\DeclareMathOperator{\Grp}{Grp}
\DeclareMathOperator{\Conn}{Conn}
\DeclareMathOperator*{\colim}{colim}
\DeclareMathOperator{\hcolim}{hcolim}
\DeclareMathOperator{\ho}{ho}
\DeclareMathOperator{\Ho}{Ho}
\DeclareMathOperator{\ev}{ev}
\DeclareMathOperator{\const}{const}
\DeclareMathOperator{\im}{Img}
\DeclareMathOperator{\essim}{essImg}
\DeclareMathOperator{\rel}{rel}
\DeclareMathOperator{\Disc}{Disc}

\newcommand\xtailrightarrow[2][]{\ensurestackMath{\mathrel{%
  \stackengine{1pt}{%
    \stackengine{0pt}{\rightarrowtail}{\scriptstyle#2}{O}{c}{F}{F}{S}%
  }{\scriptstyle#1}{U}{c}{F}{F}{S}%
}}}

\newcommand\xtwoheadrightarrow[2][]{\ensurestackMath{\mathrel{%
  \stackengine{1pt}{%
    \stackengine{0pt}{\twoheadrightarrow}{\scriptstyle#2}{O}{c}{F}{F}{S}%
  }{\scriptstyle#1}{U}{c}{F}{F}{S}%
}}}

\DeclareDocumentCommand{\newfaktor}{s m O{0.5} m O{-0.5}}{% 
  \setbox0=\hbox{\ensuremath{#2}}% 
  \setbox1=\hbox{\ensuremath{\diagup}}%
  \setbox2=\hbox{\ensuremath{#4}}%
  \raisebox{#3\ht1}{\usebox0}% 
  \mkern-5mu\ifthenelse{\equal{#1}{\BooleanTrue}}%
    {\diagup}% 
    {\rotatebox{-44}{\rule[#5\ht2]{0.4pt}{-#5\ht2+#3\ht0+\ht0}}}% 
  \mkern-4mu%
  \raisebox{#5\ht2}{\usebox2}%
}

\makeatletter
\newcounter{blankpages}
\def\cleardoubleoddstandardpage{%
\clearpage
\if@twoside \ifodd \c@page \else
 \stepcounter{blankpages}%
  \hbox {}\thispagestyle{empty}\newpage \if@twocolumn \hbox {}\thispagestyle{empty}\newpage \fi 
\fi \fi }
\makeatother

\makeatletter
\def\cleardoubleevenstandardpage{%
\clearpage
\if@twoside \ifodd \c@page
 \stepcounter{blankpages}%
  \hbox {}\thispagestyle{empty}\newpage \if@twocolumn \hbox {}\thispagestyle{empty}\newpage \fi 
\fi \fi }
\makeatother

\def\startpagenumbering{%
  \renewcommand\thepage{\the\numexpr\value{page}-\value{blankpages}\relax}
  \setcounter{page}{1}
  \setcounter{blankpages}{1}
}

\begin{document}
    \pagenumbering{gobble}
    \includepdf{cover.pdf}
    \cleardoubleoddstandardpage
    \thispagestyle{empty}
    \begin{abstract}\noindent
      The main goal of this thesis is to prove that the \inftycat/ of spaces $\spaces$, which we take to be the simplicial localization of the category of topological spaces at the weak equivalences $\N(\Top)[W^{-1}]$, is an \inftytop/.
      Concretely, we will show that $\spaces$ is a locally presentable \inftycat/ that has universal colimits and descent.
      Jacob Lurie attributes this characterization of \inftytops/ to Charles Rezk in \cite[\S 6.1.3]{HTT}, who studied these properties in terms of model topoi (e.g. \cite[\S 6]{toposes_and_htpy_toposes}).
      
      We will achieve this by using the interaction of the Quillen and \Strom/ model structures on $\Top$, in particular the fact that pushouts along cofibrations of the \Strom/ model structure are homotopy pushouts in the Quillen model structure.
      The most difficult part of the proof will be showing descent for homotopy pushouts; 
      we will give two related proofs of this, both building on a subdivision argument from \cite{may1990weak}.
      The second proof serves to highlight the relationship of descent and locality of quasifibrations.
  
      In the second part of this thesis we will highlight some consequences of universality and descent:
      As an application of descent, we will prove the May Recognition Theorem \cite{may_geometry_loop} for $n=1$ in general \inftytops/, which allows one to detect when a pointed object is deloopable.
      Lastly, we will use universality of colimits to prove the James Splitting Theorem for general \inftytops/ as in \cite{splittings_21} and derive the reduced ordinary homology of loop spaces $\Omega S^n$ with $n\geq 2$ in $\spaces$ from it.
    \end{abstract}
    \tableofcontents
    \startpagenumbering
    \section{Preliminaries and Notation}
    \subfile{basics.tex}
    \cleardoubleevenstandardpage
    \section{The $\infty$-Category of Spaces $\spaces$}
    \subfile{inftyCatOfSpaces.tex}
    \cleardoubleevenstandardpage
    \section{Motivation for $\infty$-Topoi}\label{sec:motivationInftyTopoi}
    \subfile{inftyTopoiMotivation.tex}
    \cleardoubleevenstandardpage
    \section{The Definition of an $\infty$-Topos}\label{sec:inftyTopDef}
    \subfile{inftyTopoiDefinition.tex}
    \cleardoubleevenstandardpage
    \section{$\spaces$ is an $\infty$-Topos}\label{sec:spacesIsInftyTop}
    \subfile{spacesIsInftyTopos.tex}
    \cleardoubleevenstandardpage
    \section{Delooping of Pointed Objects}\label{sec:deloopingPtdConn}
    \subfile{mayRecognitionTheorem.tex}
    \cleardoubleevenstandardpage
    \section{The James Splitting Theorem}\label{sec:jamesSplittingTheorem}
    \subfile{jamesSplittingTheorem.tex}
    \cleardoubleevenstandardpage
    \printbibliography
    \includepdf{declaration.pdf}
\end{document}