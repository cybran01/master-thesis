\documentclass[a4paper,12pt,twoside,bibliography=totoc]{scrartcl}

\usepackage{polyglossia}
\usepackage{csquotes}
\setmainlanguage{english}
\usepackage{abstract}
\usepackage{pdfpages}
\usepackage{amsmath}
\usepackage{amsthm}
\usepackage{amssymb}
\usepackage{amsfonts}
\usepackage{mathtools}
\usepackage{stackengine}
\usepackage{faktor}
\usepackage{tikz}
\usetikzlibrary{matrix}
\usetikzlibrary{arrows}
\usetikzlibrary{cd}
\usepackage[automark, headsepline]{scrlayer-scrpage}
\ihead{\leftmark}
\ohead{\rightmark}
\usepackage{enumitem}
\usepackage[backend=biber,style=alphabetic]{biblatex}
\addbibresource{sources.bib}
\usepackage[pdfencoding=auto, psdextra]{hyperref}
\usepackage{cleveref}
\usepackage{subfiles}

\setkomafont{section}{\normalfont\Large}
\setkomafont{subsection}{\normalfont\large}

\theoremstyle{plain}

\newtheorem{definition}{Definition}[section]
\newtheorem*{definition*}{Definition}
\newtheorem{prop}[definition]{Proposition}
\newtheorem{convention}[definition]{Convention}
\newtheorem{thm}[definition]{Theorem}
\newtheorem{notation}[definition]{Notation}
\newtheorem*{prop*}{Proposition}
\newtheorem{lemma}[definition]{Lemma}
\newtheorem{corollary}[definition]{Corollary}
\newtheorem{remark}[definition]{Remark}
\newtheorem{construction}[definition]{Construction}

\newenvironment{reference}{\renewcommand{\qedsymbol}{}\begin{proof}[Reference]}{\end{proof}}

\newcommand{\Kan}{\textrm{Kan}} %TODO better font
\newcommand{\spaces}{\mathcal{S}}
\newcommand{\spacesPtd}{\mathcal{S_*}}
\newcommand{\Nat}{\mathbb{N}}
\newcommand{\id}{\text{id}}
\newcommand{\set}[1]{\left\lbrace#1\right\rbrace}
\newcommand{\inftycat}{}
\def\inftycat/{$\infty$-category}
\newcommand{\inftycats}{}
\def\inftycats/{$\infty$-categories}
\newcommand{\inftytop}{}
\def\inftytop/{$\infty$-topos}
\newcommand{\inftytops}{}
\def\inftytops/{$\infty$-topoi}
\newcommand{\Cech}{}
\def\Cech/{\v{C}ech}
\newcommand{\Strom}{}
\def\Strom/{Str{\o}m}

\DeclareMathOperator{\R}{R}
\DeclareMathOperator{\C}{C}
\DeclareMathOperator{\CN}{\text{\v{C}}}
\DeclareMathOperator{\N}{N}
\DeclareMathOperator{\M}{M}
\DeclareMathOperator{\Nhc}{N^{hc}}
\DeclareMathOperator{\Hom}{Hom}
\DeclareMathOperator{\Map}{Map}
\DeclareMathOperator{\Sing}{Sing}
\DeclareMathOperator{\Cof}{Cof}
\DeclareMathOperator{\Fib}{Fib}
\DeclareMathOperator{\W}{W}
\DeclareMathOperator{\Fun}{Fun}
\DeclareMathOperator{\Cat}{Cat}
\DeclareMathOperator{\Cart}{Cart}
\DeclareMathOperator{\op}{op}
\DeclareMathOperator{\Set}{Set}
\DeclareMathOperator{\sSet}{sSet}
\DeclareMathOperator{\Top}{Top}
\DeclareMathOperator{\Gpd}{Gpd}
\DeclareMathOperator{\Grp}{Grp}
\DeclareMathOperator{\Conn}{Conn}
\DeclareMathOperator{\colim}{colim}
\DeclareMathOperator{\hcolim}{hcolim}
\DeclareMathOperator{\ho}{ho}
\DeclareMathOperator{\ev}{ev}
\DeclareMathOperator{\const}{const}
\DeclareMathOperator{\im}{Img}
\DeclareMathOperator{\Disc}{Disc}

\newcommand\xtailrightarrow[2][]{\ensurestackMath{\mathrel{%
  \stackengine{1pt}{%
    \stackengine{0pt}{\rightarrowtail}{\scriptstyle#2}{O}{c}{F}{F}{S}%
  }{\scriptstyle#1}{U}{c}{F}{F}{S}%
}}}

\newcommand\xtwoheadrightarrow[2][]{\ensurestackMath{\mathrel{%
  \stackengine{1pt}{%
    \stackengine{0pt}{\twoheadrightarrow}{\scriptstyle#2}{O}{c}{F}{F}{S}%
  }{\scriptstyle#1}{U}{c}{F}{F}{S}%
}}}

\makeatletter
\newcounter{blankpages}
\def\cleardoubleoddstandardpage{%
\clearpage
\if@twoside \ifodd \c@page \else
 \stepcounter{blankpages}%
  \hbox {}\thispagestyle{empty}\newpage \if@twocolumn \hbox {}\thispagestyle{empty}\newpage \fi 
\fi \fi }
\makeatother

\def\startpagenumbering{%
  \renewcommand\thepage{\the\numexpr\value{page}-\value{blankpages}\relax}
  \setcounter{page}{1}
  \setcounter{blankpages}{0}
}


\begin{document}
    \pagenumbering{gobble}
    \includepdf{cover.pdf}
    \cleardoubleoddstandardpage
    \thispagestyle{empty}
    \begin{abstract}\noindent
      The main goal of this thesis is to prove that the \inftytop/ of spaces $\spaces$, which we take to be the simplicial localization of the category of topological spaces at the weak equivalences $\N(\Top)[W^{-1}]$, is an \inftytop/ in the sense of Charles Rezk.
      This means we will show it is a locally presentable \inftycat/ that has universal colimits and descent.
      We will achieve this by using the interaction of the Quillen and \Strom/ model structures on $\Top$, in particular the fact that pushouts along cofibrations of the \Strom/ model structure are homotopy pushouts in the Quillen model structure.

      \noindent The most difficult part of the proof will be showing descent for homotopy pushouts; 
      we will give two related proofs of this, both building on a subdivision argument from \cite{may1990weak}.
      The second proof serves to highlight the relationship of descent and locality of quasifibrations.
  
      \noindent In the second part of this thesis we will see some consequences of universality and descent:
      As an application of descent, we will prove the May Recognition Theorem (\cite{may2006geometry}) for $n=1$ in general \inftytops/, which allows one to detect when a pointed connected object is deloopable.
      Lastly, we will use universality of colimits to prove the James Splitting Theorem for general \inftytops/ as in \cite{splittings_21} and derive the reduced ordinary homology of loop spaces $\Omega S^n$ with $n\geq 2$ in $\spaces$ from it.
    \end{abstract}
    \tableofcontents
    \cleardoubleoddstandardpage
    \startpagenumbering
    \section{Preliminaries and Notation}%TODO add some intermittent text to explain why we are introducing things
    \subfile{basics.tex}
    \cleardoubleoddstandardpage
    \section{The $\infty$-Category of Spaces $\spaces$}
    \subfile{inftyCatOfSpaces.tex}
    \cleardoubleoddstandardpage
    \section{Motivation for $\infty$-Topoi}\label{sec:motivationInftyTopoi}
    \subfile{inftyTopoiMotivation.tex}
    \cleardoubleoddstandardpage
    \section{Definition of an $\infty$-Topos}\label{sec:inftyTopDef}
    \subfile{inftyTopoiDefinition.tex}
    \cleardoubleoddstandardpage
    \section{$\spaces$ is an $\infty$-Topos}
    \subfile{spacesIsInftyTopos.tex}
    \cleardoubleoddstandardpage
    \section{Delooping of Pointed Connected Objects}
    \subfile{mayRecognitionTheorem.tex}
    \cleardoubleoddstandardpage
    \section{The James Splitting Theorem}
    \subfile{jamesSplittingTheorem.tex}
    \cleardoubleoddstandardpage
    \printbibliography
    \includepdf{declaration.pdf}
\end{document}