In this section we will give a proof on the May Recognition Theorem in the abstract setting of an \inftytop/.
This theorem allows one to recognize which pointed spaces are (equivalent to) loop spaces.
We will see that descent will play a major role in this.
\begin{definition}[(Augmented) Simplicial Objects] %TODO maybe define for all finite totally ordered sets
    Let $C$ be an \inftycat/. 
    We call a map $U\colon\N(\Delta)^{\op}\to C$ a \emph{simplicial object of the \inftycat/ $C$}.
    We call a map $U^+\colon\N(\Delta_+)^{\op}\to C$ an \emph{augmented simplicial object of the \inftycat/ $C$}.
    %
    %We refer to the \inftycats/ $\Fun(\N(\Delta)^{\op},C)$ and $\Fun(\N(\Delta_+)^{\op},C)$ as the \emph{\inftycat/ of simplicial objects} and the \emph{\inftycat/ of augmented simplicial objects} respectively.
\end{definition}
\begin{definition}[Groupoid Object]
    Let $C$ be an \inftycat/ and let $U$ be a simplicial object such that for every $n\in\Nat$ and partition $[n]=S\cup S'$ with $S\cap S'$ consisting of a single element $\set{s}$ the square
    \begin{center}
        \begin{tikzpicture} %TODO make this nice, explain usage of S and S'
            \matrix (m) [matrix of math nodes,row sep=3em,column sep=4em,minimum width=2em]
            {
              U_n & U(S)\\
              U(S') &  U(\set{s})\\};
            \path[-stealth]

            (m-1-1) edge (m-1-2)
            (m-1-1) edge (m-2-1)
            (m-2-1) edge (m-2-2)
            (m-1-2) edge (m-2-2);
        \end{tikzpicture}
    \end{center}
    is a pullback square in $C$.
    Then we call $U$ a \emph{groupoid object of $C$}.
    We let $\Gpd(C)$ denote the full subcategory of $\Fun(\N(\Delta)^{\op},C)$ consisting of groupoid objects.
\end{definition}
\begin{definition}[Group Object]
    Let $C$ be an \inftycat/ and let $U$ be a groupoid object.
    If $U_0\cong *$ where $*$ is a terminal object of $C$, then we call $U$ a \emph{group object of $C$}.
\end{definition}
\begin{definition}[\Cech/ Nerve]
    Let $C$ be an \inftycat/ with small limits and let $\Delta^1\subset\N(\Delta_+)^{\op}$ be the inclusion corresponding to the full subcategory $[0]\to[-1]$.
    
    Then by \cref{prop:exKanExt} there exists a functor $C\colon\Fun(\Delta^1,C)\to\Fun(\N(\Delta_+)^{\op},C)$ that is right adjoint to the restriction functor $F\colon\Fun(\N(\Delta_+)^{\op},C)\to\Fun(\Delta^1,C)$ and for a map $f\colon x\to y\in\Fun(\Delta^1,C)_0$ is the right Kan-extension of $f$ along $\Delta^1\subset\N(\Delta_+)^{\op}$.
    An object $U^+\in\Fun(\N(\Delta_+)^{\op},C)_0$ in the essential image of $C$ is called a \emph{\Cech/ nerve}.
\end{definition}
\begin{remark}
    Computing the \Cech/ nerve for a map $f\colon x\to y$ explicitly, one can see that up to equivalence it is given by
    \begin{center}
        \begin{tikzcd}
            & C(f)_3 & C(f)_2 & C(f)_1 & C(f)_0 &  C(f)_{-1} \\ 
            \ldots \arrow[r] \arrow[r, yshift=-1em] \arrow[r, yshift=1em] \arrow[r, yshift=-2em] \arrow[r, yshift=2em] 
            & x\times_y x \times_y x \times_y x \arrow[r, yshift=.5em] \arrow[r, yshift=-.5em] \arrow[r, yshift=1.5em] \arrow[r, yshift=-1.5em] \arrow[l, yshift=.5em] \arrow[l, yshift=-.5em] \arrow[l, yshift=1.5em] \arrow[l, yshift=-1.5em]
            & x\times_y x \times_y x\arrow[r] \arrow[r, yshift=-1em] \arrow[r, yshift=1em] \arrow[l] \arrow[l, yshift=-1em] \arrow[l, yshift=1em]
            & x\times_y x \arrow[r, yshift=.5em] \arrow[r, yshift=-.5em] \arrow[l, yshift=.5em] \arrow[l, yshift=-.5em]
            & x \arrow[l] \arrow[r, "f"] & y
        \end{tikzcd}
    \end{center}
    where
    \begin{itemize}
        \item the unnamed arrows going from left to right are the maps induced by the face maps and are projections 
        \item the unnamed arrows going from right to left are the maps induced by the degeneracy maps and are diagonal maps. 
    \end{itemize}
    Here we have only shown a selection of the $1$-skeleton of $C(f)$.
    The remaining maps of $\Delta_+$ are however compositions of the shown face and degeneracy maps, which determines the $1$-skeleton; for all higher simplices we make choices via the universal property of the iterated pullbacks, which all yield equivalent objects $C(f)$. %TODO confirm
\end{remark}
\begin{definition}[Effective Groupoid Object]
    Let $C$ be an \inftycat/ with small colimits and let $U$ be a groupoid object.
    Then we call $U$ \emph{effective} if the diagram 
    \begin{center}
        \begin{tikzpicture}
            \matrix (m) [matrix of math nodes,row sep=3em,column sep=4em,minimum width=2em]
            {
            U_1 & U_0\\
            U_0 & \colim U\\};
            \path[-stealth]

            (m-1-1) edge (m-1-2)
            (m-1-1) edge (m-2-1)
            (m-2-1) edge (m-2-2)
            (m-1-2) edge (m-2-2);
        \end{tikzpicture}
    \end{center}
    is a pullback square.
\end{definition}
We will need the following proposition.
\begin{prop}\label{prop:grpdEffectiveIfPullback} 
    Let $U^+$ be an augmented simplicial object in an \inftycat/ $C$.
    Then it is a \Cech/ nerve if and only if $U^+|_{\N(\Delta)^{\op}}$ is a groupoid object and the square
    \begin{center} %TODO add description to top and left map
        \begin{tikzcd} [sep = 4em]
            U_1^+ \arrow[r] \arrow[d] & U_0^+ \arrow[d] \\
            U_0^+ \arrow[r] & U_{-1}^+ \\
        \end{tikzcd}
    \end{center}
    is a pullback square.
    \begin{reference}
        \cite[Proposition 6.1.2.11]{HTT}
    \end{reference}
\end{prop}
\begin{corollary}\label{cor:groupoidEffectiveIffColimCechNerve}
    Let $C$ be an \inftycat/ with small colimits. 
    Then a groupoid object $U$ is effective if and only if its colimit cone $\N(\Delta)^{\op}\star\Delta^0\cong\N(\Delta_+)^{\op}\to C$ is a \Cech/ nerve.
    \begin{proof}
        This follows directly from the equivalent characterization of an effective groupoid from \cref{prop:grpdEffectiveIfPullback}.
    \end{proof}
\end{corollary}
The following proposition is at the heart of the proof of the May Recognition Theorem. 
The proof follows \cite[pp. 29-30]{toposes_and_htpy_toposes}.
\begin{prop}\label{prop:groupoidObjInToposAreEffective}
    Let $C$ be an \inftytop/.
    Then every groupoid object $U$ is effective.
    \begin{proof}
        Let $\alpha\colon\N\left(\Delta_+\right)\to\N\left(\Delta\right)$ be the map induced by the functor sending a morphism $f\colon [m]\to[n]\in\Map(\Delta_+)$ to 
        \begin{align*}
            \alpha(f)\colon [m+1]&\to[n+1]\\
            i&\mapsto
            \begin{cases}
                f(i) & 0\leq i\leq m\\
                n+1 & i=m+1\;.
            \end{cases}
        \end{align*}
        Then $\overline{U}\coloneqq U\circ\alpha^{\op}$ is an augmented simplicial object with $U_{n+1}=\overline{U}_n$.
        Let $f\colon [m]\to[n]\in\Map(\Delta)$ be some map. Then the diagram
        \begin{center}
            \begin{tikzcd} [sep = 4em]
                \overline{U}_n=U_{n+1} \arrow[r] \arrow[d,"\overline{U}(f)"] & U_n \arrow[d, "U(f)"] \\
                \overline{U}_m=U_{m+1} \arrow[r] \arrow[d] & U_m \arrow[d] \\
                U_{\set{m,m+1}} \arrow[r] & U_{\set{m}} \\
            \end{tikzcd}
        \end{center}
        where the horizontal maps are given by the inclusions $[n]\xhookrightarrow{}[n+1]$, is commutative. 
        As the outer and lower square are pullbacks because $U$ is a groupoid object, the upper square is a pullback by the pasting law.
        Thus the natural transformation $\overline{U}|_{\N(\Delta)^{\op}}\to U$ induced by the inclusions $[n]\xhookrightarrow{}[n+1]$ is cartesian.

        Descent now implies that 
        \begin{center}
            \begin{tikzcd} [sep = 4em]
                \overline{U}_0 \arrow[r] \arrow[d] & \colim\overline{U}|_{\N(\Delta)^{\op}} \arrow[d] \\
                U_0 \arrow[r] & \colim U \\
            \end{tikzcd}
        \end{center}
        is a pullback diagram.
        Since by e.g. \cite[Lemma 6.1.3.17]{HTT} we have that $\colim\overline{U}|_{\N(\Delta)^{\op}}\cong\overline{U}_{-1} = U_0$ and $\overline{U}_0=U_1$ this proves the proposition.
    \end{proof}
\end{prop}
\begin{definition}[Effective Epimorphism],
    Let $C$ be an \inftycat/ with small limits and let $f\colon x\to y\in C_1$ be a map.
    We say that $f$ is an \emph{effective epimorphism} if the \Cech/ nerve $C(f)$ is a colimit diagram.
\end{definition}
\begin{prop}\label{prop:mayRecognitionTheoremGroupoid}
    Let $C$ be an \inftytop/ and let $\Fun(\Delta^1,C)_{\text{eff}}\subset\Fun(\Delta^1,C)$ be the full subcategory of effective epimorphisms.  
    Then the functors
    \begin{equation*}
        F\colon\Gpd(C)\subset\Fun(\N(\Delta)^{\op},C)\xrightarrow{\colim}\Fun(\N(\Delta_+)^{\op},C)
    \end{equation*}
    and 
    \begin{equation*}
        G\colon\Fun(\Delta^1,C)_{\text{eff}}\subset\Fun(\Delta^1,C)\xrightarrow{C}\Fun(\N(\Delta_+)^{\op},C)
    \end{equation*}
    are fully faithful and have the same essential image in $\Fun(\N(\Delta_+)^{\op},C)$.
    
    Thus they induce an equivalence of \inftycats/ $\Gpd(C)\simeq\Fun(\Delta^1,C)_{\text{eff}}$.
    \begin{proof}
        Both functors are pointwise Kan-extensions along fully faithful functors:
        The functor $F$ is a left Kan-extension along $\N(\Delta)^{\op}\subset\N(\Delta_+)^{\op}$, and $G$ a right Kan-extension along $\Delta^1\subset\N(\Delta_+)^{\op}$.
        Thus they are both fully faithful.

        First, let $U$ be an element in the essential image of $G$. 
        Then by definition of an effective epimorphism it is already a colimit diagram, and since $U|_{\N(\Delta)^{\op}}$ is a groupoid, it is in the essential image of $F$.

        Now let $V$ be an element in the essential image of $F$. 
        Then by \cref{prop:groupoidObjInToposAreEffective} and \cref{cor:groupoidEffectiveIffColimCechNerve} we know that it is a \Cech/ nerve and therefore by definition in the essential image of $G$.
    \end{proof}
\end{prop}
\begin{definition}
    Let $C$ be an \inftycat/ with a terminal object.
    Then write $C_*\subset\Fun(\Delta^1,C)$ for the full subcategory of maps with domain a fixed terminal object $*\in C$.
    We call $C_*$ the \emph{\inftycat/ of pointed objects}.
\end{definition}
%TODO remark that choice of * does not matter
\begin{definition}
    Let $C$ be an \inftycat/ with terminal object and colimits. 
    Then let $\Conn(C_*)\subset C_*$ be the full subcategory of effective morphisms.
    We call $\Conn(C_*)$ the \emph{\inftycat/ of pointed connected objects}.
\end{definition}
\begin{corollary}[May Recognition Theorem]\label{prop:mayRecognitionTheoremGroup}
    Let $C$ be an \inftytop/. 
    Then there is an equivalence of \inftycats/ $\Grp(C)\simeq\Conn(C)$.
    \begin{proof}
        This follows immediately from \cref{prop:mayRecognitionTheoremGroupoid} and the definition of group objects.
    \end{proof}
\end{corollary}
\subsection*{The May Recognition Theorem in $\spaces$}
We now want to check that our formulation of the May Recognition Theorem really gives the expected formulation in $\Top$.
\begin{definition}
    Let $C$ be an \inftytop/.
    With the convention that the $(-2)$-truncated maps are the equivalences, we recursively define that a map $f\colon X\to Y$ in $C$ is \emph{$k$-truncated} for $k\geq -1$ if the diagonal map $X\to X\times_YX$ is $(k-1)$-truncated.
    
    We consider an object $X$ to be \emph{$k$-truncated} if the map $X\to *$ is $k$-truncated.
\end{definition}
%TODO make this a little cleaner to read, check that is really works in all degrees
\begin{remark}
    Note that in $\spaces$ this definition agrees with the following:
    \begin{definition*}
        A map of topological spaces $f\colon X\to Y$ is $k$-truncated if and only if for all its homotopy fibers $F$, $x\in F$ and $n>k$ we have $\pi_n(F,x)\cong 0$.
        We set $(-2)$-truncated to mean that it is non-empty and contractible, and $(-1)$-truncated to mean that it is non-empty or contractible.
    \end{definition*}
    This is because for a pullback square in $\spaces$ 
    \begin{center}
        \begin{tikzcd} [sep = 4em]
            X\times_{Y} X \arrow[r, "g"] \arrow[d, "g"] & X \arrow[d, "f"] \\
            X \arrow[r, "f"] & Y \\
        \end{tikzcd}
    \end{center}
    by the fiberwise characterization of homotopy pullbacks we have that $g$ is $k$-truncated if and only if $f$ is $k$-truncated.
    
    Furthermore a map $f\colon X\to Y$ equipped with a section $s\colon Y\to X$ is $n$-truncted if and only if the section $s$ is $(n-1)$-truncated (as can be proven by inspection of the long exact sequences of homotopy groups).
    
    Since $X\to X\times_YX$ is the canonical section of $X\times_{Y} X\to Y$, this allows us to recursively reduce to the case of $f\colon X\to Y$ being $(-2)$-truncated, which in both definitions means that it is an equivalence.
\end{remark}
\begin{definition}
    Let $C$ be an \inftytop/ and let $\Disc(C)\subset C$ be the full subcategory of $0$-truncated objects.
    We call $\Disc(C)$ the \emph{\inftycat/ of discrete objects of $C$}.
\end{definition}
\begin{prop}\label{prop:effInSpaceIffSurj}
    In map $f\colon x\to y$ in $\spaces$ is an effective epimorphism if and only if it induces a surjection on connected components.
    \begin{proof}
        First, by \cite[Proposition 7.8.3]{cisinski_2019} there is a fully faithful functor $i\colon\N(\Set)\to\spaces$ whose essential image consists precisely of the discrete objects and has $\pi_0\colon\spaces\to\N(\Set)$ as a left adjoint.
        By \cite[Proposition 7.2.1.14]{HTT} the statement of the lemma is thus equivalent to checking whether $\pi_0(f)$ is an effective epimorphism in $\Set$.
        In $\Set$ the effective epimorphisms are exactly the surjective morphisms, so we have that a map $f$ in $\spaces$ is an effective epimorphism precisely if $\pi_0(f)$ is surjective.
    \end{proof}
\end{prop}
\begin{corollary}
    Let $x\colon*\to X\in\spaces_1$ be a map.
    Then $x$ is an effective epimorphism if and only if $X$ is connected as a topological space.
    \begin{proof}
        This follows from \cref{prop:effInSpaceIffSurj} and the fact that $*$ has exactly one connected component.
    \end{proof}
\end{corollary}
\begin{remark}
    By definition of loop spaces and explicit computation, the \Cech/ nerve of an effective map $x\colon*\to X$ in $\Top$ (so $X$ is a pointed connected space) is given by
    \begin{center}
        \begin{tikzcd}
            & C(x)_3 & C(x)_2 & C(x)_1 & C(x)_0 &  C(x)_{-1} \\ 
            \ldots \arrow[r] \arrow[r, yshift=-1em] \arrow[r, yshift=1em] \arrow[r, yshift=-2em] \arrow[r, yshift=2em] 
            & \Omega X\times\Omega X\times\Omega X \arrow[r, yshift=.5em] \arrow[r, yshift=-.5em] \arrow[r, yshift=1.5em] \arrow[r, yshift=-1.5em] \arrow[l, yshift=.5em] \arrow[l, yshift=-.5em] \arrow[l, yshift=1.5em] \arrow[l, yshift=-1.5em]
            & \Omega X\times\Omega X\arrow[r] \arrow[r, yshift=-1em] \arrow[r, yshift=1em] \arrow[l] \arrow[l, yshift=-1em] \arrow[l, yshift=1em]
            & \Omega X \arrow[r, yshift=.5em] \arrow[r, yshift=-.5em] \arrow[l, yshift=.5em] \arrow[l, yshift=-.5em]
            & * \arrow[l] \arrow[r, "x"] & X
        \end{tikzcd}
    \end{center}
    where
    \begin{itemize}
        \item the map induced by a face map 
        \begin{itemize}
            \item $d_0^n$ is $(\gamma_1,\gamma_2,\ldots,\gamma_n)\mapsto(\gamma_2,\ldots,\gamma_n)$
            \item $d_n^n$ is $(\gamma_1,\ldots,\gamma_{n-1},\gamma_n)\mapsto(\gamma_1,\ldots,\gamma_{n-1})$
            \item $d_k^n$ for $0<k<n$ is $(\gamma_1,\ldots,\gamma_k,\gamma_{k+1},\ldots,\gamma_n)\mapsto(\gamma_1,\ldots,\gamma_k\star\gamma_{k+1},\ldots,\gamma_n)$ where $\star$ denotes concatenation of loops
        \end{itemize}
        \item the map induced by a degeneracy map $s_k^n$ is injecting the constant loop $c_x$ at the $k$-th position.
    \end{itemize}
    We can thus view the map $\Conn(\spaces)\to\Grp(\spaces)$ as being a canonically enriched version of the loop space functor.
    The statement of \cref{prop:mayRecognitionTheoremGroup} therefore allows us to recognize whether a pointed connected space $X$ is deloopable:

    It is deloopable if and only if there exists a group object $U\colon\N\left(\Delta^{\op}\right)\to\spaces$ where $U_n= X\times\ldots\times X$ with $X$ appearing $n$ times and $U_0=*$. %TODO is this true or does it need additional hypothesis (like certain maps being projections, etc)?
    The delooping is then given by taking the colimit of this diagram and taking the canonical map $A\colon*\to\colim U$ to be a pointed space, as this fulfills $\Omega A\simeq X$.
\end{remark}

