\begin{definition}[(Augmented) Simplicial Objects] %TODO maybe define for all finite totally ordered sets
    Let $C$ be an \inftycat/. 
    We call a map $U\colon\N(\Delta)^{\op}\to C$ a \emph{simplicial object of the \inftycat/ $C$}.
    We call a map $U^+\colon\N(\Delta_+)^{\op}\to C$ an \emph{augmented simplicial object of the \inftycat/ $C$}.
    %
    %We refer to the \inftycats/ $\Fun(\N(\Delta)^{\op},C)$ and $\Fun(\N(\Delta_+)^{\op},C)$ as the \emph{\inftycat/ of simplicial objects} and the \emph{\inftycat/ of augmented simplicial objects} respectively.
\end{definition}
\begin{definition}[Groupoid Object]
    Let $C$ be an \inftycat/ and let $U$ be a simplicial object such that for every $n\in\Nat$ and partition $[n]=S\cup S'$ with $S\cap S'$ consisting of a single element $\set{s}$ the square
    \begin{center}
        \begin{tikzpicture} %TODO make this nice, explain usage of S and S'
            \matrix (m) [matrix of math nodes,row sep=3em,column sep=4em,minimum width=2em]
            {
              U_n & U(S)\\
              U(S') &  U(\set{s})\\};
            \path[-stealth]

            (m-1-1) edge (m-1-2)
            (m-1-1) edge (m-2-1)
            (m-2-1) edge (m-2-2)
            (m-1-2) edge (m-2-2);
        \end{tikzpicture}
    \end{center}
    is a pullback square in $C$.
    Then we call $U$ a \emph{groupoid object of $C$}.
    We let $\Gpd(C)$ denote the full subcategory of $\Fun(\N(\Delta)^{\op},C)$ consisting of groupoid objects.
\end{definition}
\begin{definition}[Group Object]
    Let $C$ be an \inftycat/ and let $U$ be a groupoid object.
    If $U_0\cong *$ where $*$ is a terminal object of $C$, then we call $U$ a \emph{group object of $C$}.
\end{definition}
\begin{definition}
    Let $C$ be an \inftycat/ with small limits and let $\Delta^1\subset\N(\Delta_+)^{\op}$ be the inclusion corresponding to the full subcategory $[0]\to[-1]$.
    
    Then there exists a functor $C\colon\Fun(\Delta^1,C)\to\Fun(\N(\Delta_+)^{\op},C)$ that is right adjoint to the restriction functor $F\colon\Fun(\N(\Delta_+)^{\op},C)\to\Fun(\Delta^1,C)$ and for a map $f\colon x\to y\in\Fun(\Delta^1,C)_0$ is the right Kan-extension of $f$ along $\Delta^1\subset\N(\Delta_+)^{\op}$.
    An object $U^+\in\Fun(\N(\Delta_+)^{\op},C)_0$ in the essential image of $C$ is called a \emph{\Cech/ nerve}.
\end{definition}
\begin{remark}
    Computing the \Cech/ nerve for a map $f\colon x\to y$ explicitly, one can see that up to equivalence it is given by
    \begin{center}
        \begin{tikzcd}
            & C(f)_3 & C(f)_2 & C(f)_1 & C(f)_0 &  C(f)_{-1} \\ 
            \ldots \arrow[r] \arrow[r, yshift=-1em] \arrow[r, yshift=1em] \arrow[r, yshift=-2em] \arrow[r, yshift=2em] 
            & x\times_y x \times_y x \times_y x \arrow[r, yshift=.5em] \arrow[r, yshift=-.5em] \arrow[r, yshift=1.5em] \arrow[r, yshift=-1.5em] \arrow[l, yshift=.5em] \arrow[l, yshift=-.5em] \arrow[l, yshift=1.5em] \arrow[l, yshift=-1.5em]
            & x\times_y x \times_y x\arrow[r] \arrow[r, yshift=-1em] \arrow[r, yshift=1em] \arrow[l] \arrow[l, yshift=-1em] \arrow[l, yshift=1em]
            & x\times_y x \arrow[r, yshift=.5em] \arrow[r, yshift=-.5em] \arrow[l, yshift=.5em] \arrow[l, yshift=-.5em]
            & x \arrow[l] \arrow[r, "f"] & y
        \end{tikzcd}
    \end{center}
    where
    \begin{itemize}
        \item the unnamed arrows going from left to right are the maps induced by the face maps and are projections 
        \item the unnamed arrows going from right to left are the maps induced by the degeneracy maps and are diagonal maps. 
    \end{itemize}
    Here we have only shown a selection of the $1$-skeleton of $C(f)$.
    The remaining maps of $\Delta_+$ are however compositions of the shown face and degeneracy maps, which determines the $1$-skeleton; for all higher simplices we make choices via the universal property of the iterated pullbacks, which all yield equivalent objects $C(f)$. %TODO confirm
\end{remark}
%TODO Add note: while kan ext is not unique, all kan ext have same essential image
\begin{definition}
    Let $C$ be an \inftycat/ with small limits. 
    For a map $f\in C_1$ we call $C(f)$ the \Cech/ nerve of $f$.
\end{definition}
\begin{lemma}
    A \Cech/ nerve is determined up to equivalence by the map $u\colon U_0\to U_{-1}$. %TODO is this needed?
\end{lemma}
\begin{prop}\label{prop:grpdEffectiveIfPullback} %TODO HTT Proposition 6.1.2.11.
    Let $U^+$ be an augmented simplicial object in an \inftycat/ $C$.
    Then it is a \Cech/ nerve if and only if $U^+|_{\N(\Delta)^{\op}}$ is a groupoid object and the square
    \begin{center}
        \begin{tikzpicture}
            \matrix (m) [matrix of math nodes,row sep=3em,column sep=4em,minimum width=2em]
            {
            U_1 & U_0\\
            U_0 & U_{-1}\\};
            \path[-stealth]

            (m-1-1) edge (m-1-2)
            (m-1-1) edge (m-2-1)
            (m-2-1) edge (m-2-2)
            (m-1-2) edge (m-2-2);
        \end{tikzpicture}
    \end{center}
    is a pullback square.
\end{prop}
\begin{definition}[Effective Groupoid Object]
    Let $C$ be an \inftycat/ with small colimits and let $U$ be a groupoid object.
    Then we call $U$ \emph{effective} if the diagram 
    \begin{center}
        \begin{tikzpicture}
            \matrix (m) [matrix of math nodes,row sep=3em,column sep=4em,minimum width=2em]
            {
            U_1 & U_0\\
            U_0 & \colim U\\};
            \path[-stealth]

            (m-1-1) edge (m-1-2)
            (m-1-1) edge (m-2-1)
            (m-2-1) edge (m-2-2)
            (m-1-2) edge (m-2-2);
        \end{tikzpicture}
    \end{center}
    is a pullback square.
\end{definition}
%TODO explain how cech nerve is computed and how it looks
\begin{corollary}\label{cor:groupoidEffectiveIffColimCechNerve}
    Let $C$ be an \inftycat/ with small colimits. 
    Then a groupoid $U$ is effective if and only if its colimit cone $\N(\Delta)^{op}\star\Delta^0\cong\N(\Delta_+)^{op}\to C$ is a \Cech/ nerve.
    \begin{proof}
        This follows directly from the equivalent characterization of an effective groupoid from \cref{prop:grpdEffectiveIfPullback}.
    \end{proof}
\end{corollary}
\begin{prop}\label{prop:groupoidObjInToposAreEffective} % Charles Rezk lecture notes 29pp
    Let $C$ be an \inftytop/.
    Then every groupoid object $U$ is effective.
    \begin{proof}
        Let $\alpha\colon\N\left(\Delta_+\right)\to\N\left(\Delta\right)$ the map induced by the functor sending a morphism $f\colon [m]\to[n]\in\Delta_+$ to 
        \begin{align*}
            \alpha(f)\colon [m+1]&\to[n+1]\\
            i&\mapsto
            \begin{cases}
                f(i) & 0\leq i\leq m\\
                n+1 & i=m+1\;.
            \end{cases}
        \end{align*}
        Then $U\circ\alpha^{\op}$ is an augmented simplicial object with $U_{n+1}=(U\circ\alpha^{\op})_n$.
        Let $f\colon [m]\to[n]$ be some map. Then the diagram
        \begin{center}
            \begin{tikzpicture}
                \matrix (m) [matrix of math nodes,row sep=3em,column sep=4em,minimum width=2em]
                {
                U_{n+1} & U_n\\
                U_{m+1} & U_m\\
                U_{\set{m,m+1}} & U_{\set{m}}\\};
                \path[-stealth]
    
                (m-1-1) edge (m-1-2)
                (m-1-1) edge (m-2-1)
                (m-2-1) edge (m-2-2)
                (m-1-2) edge (m-2-2)
                (m-2-1) edge (m-3-1)
                (m-2-2) edge (m-3-2)
                (m-3-1) edge (m-3-2);
            \end{tikzpicture}
        \end{center}
        where the horizontal maps are given by the inclusions $[n]\xhookrightarrow{}[n+1]$, is commutative. 
        As the outer and lower square are pullbacks because $U$ is a groupoid object, the upper square is a pullback by the pasting law.
        Thus the natural transformation $(U\circ\alpha^{\op})|_{\N(\Delta)^{\op}}\to U$ is cartesian.

        Descent now implies that 
        \begin{center}
            \begin{tikzpicture}
                \matrix (m) [matrix of math nodes,row sep=3em,column sep=4em,minimum width=2em]
                {
                  (U\circ\alpha^{\op})_0 & \colim (U\circ\alpha^{\op})\\
                  U_0 &  \colim U\\};
                \path[-stealth]
    
                (m-1-1) edge (m-1-2)
                (m-1-1) edge (m-2-1)
                (m-2-1) edge (m-2-2)
                (m-1-2) edge (m-2-2);
            \end{tikzpicture}
        \end{center}
        is a pullback diagram.
        Since by %TODO HTT Lemma 6.1.3.17 
        we have that $\colim (U\circ\alpha^{\op})\cong(U\circ\alpha^{\op})_{-1} = U_0$ and $(U\circ\alpha^{\op})_0=U_1$ this proves the proposition.
    \end{proof}
\end{prop}
\begin{definition}
    Let $C$ be an \inftycat/ with colimits and let $f\colon x\to y\in C_1$ be a map.
    We say that $f$ is an \emph{effective epimorphism} if the \Cech/ nerve $C(f)$ is a colimit diagram.
\end{definition}
\begin{prop}\label{prop:mayRecognitionTheoremGroupoid}
    Let $C$ be an \inftytop/. 
    Then the functors
    \begin{equation*}
        F\colon\Gpd(C)\subset\Fun(\N(\Delta)^{\op},C)\xrightarrow{\colim}\Fun(\N(\Delta_+)^{\op},C)
    \end{equation*}
    and 
    \begin{equation*}
        G\colon\Fun(\Delta^1,C)_{\text{eff}}\subset\Fun(\Delta^1,C)\xrightarrow{C}\Fun(\N(\Delta_+)^{\op},C)
    \end{equation*}
    are fully faithful and have the same essential image in $\Fun(\N(\Delta_+)^{\op},C)$.
    
    Thus they induce an equivalence of \inftycats/ $\Gpd(C)\simeq\Fun(\Delta^1,C)_{\text{eff}}$.
    \begin{proof}
        Both functors are pointwise Kan-extensions along fully faithful functors:
        The functor $F$ is a left Kan-extension along $\N(\Delta)^{\op}\subset\N(\Delta_+)^{\op}$, and $G$ a right Kan-extension along $\Delta^1\subset\N(\Delta_+)^{\op}$.
        Thus they are both fully faithful.

        First, let $U$ be an element in the essential image of $G$. 
        Then by definition of an effective epimorphism it is already a colimit diagram, and since $U|_{\N(\Delta)^{\op}}$ is a groupoid, it is in the essential image of $F$.

        Now let $V$ be an element in the essential image of $F$. 
        Then by \cref{prop:groupoidObjInToposAreEffective} and \cref{cor:groupoidEffectiveIffColimCechNerve} we know that it is a \Cech/ nerve and therefore by definition in the essential image of $G$.
    \end{proof}
\end{prop}
\begin{definition}
    Let $C$ be an \inftycat/ with a terminal object.
    Then write $C_*\subset\Fun(\Delta^1,C)$ for the full subcategory of maps with domain a fixed terminal object $*\in C$.
    We call $C_*$ the \emph{\inftycat/ of pointed objects}.
\end{definition}
%TODO remark that choice of * does not matter
\begin{definition}
    Let $C$ be an \inftycat/ with terminal object and colimits. 
    Then let $\Conn(C_*)\subset C_*$ be the full subcategory of effective morphisms.
    We call $\Conn(C_*)$ the \emph{\inftycat/ of pointed connected objects}.
\end{definition}
\begin{corollary}\label{prop:mayRecognitionTheoremGroup}
    Let $C$ be an \inftytop/. 
    Then there is an equivalence of \inftycats/ $\Grp(C)\simeq\Conn(C)$.
    \begin{proof}
        This follows immediately from \cref{prop:mayRecognitionTheoremGroupoid} and the definition of group objects.
    \end{proof}
\end{corollary}
%TODO explain looping/delooping and do it all in Top
\subsection{The May Recognition Theorem in $\spaces$}
%TODO define truncated obj/morphism HTT Definition 5.5.6.1. 
\begin{definition}
    Let $C$ be an \inftytop/ and let $\Disc(C)\subset C$ be the full subcategory of $0$-truncated objects.
    We call $\Disc(C)$ the \emph{\inftycat/ of discrete objects of $C$}.
\end{definition}
%TODO HTT Proposition 5.5.6.18. for the fact that Disc has an accessible left adjoint
%TODO Proposition 7.2.1.14. for being able to check eff epi in htpy cat
\begin{lemma}
    Let $x\colon*\to X\in\spaces_1$ be a map.
    Then $x$ is an effective epimorphism if and only if $X$ is connected as a topological space.
    \begin{proof}
        By %HTT 7.2.1.14.
        this is equivalent to checking that $L(x)$ is an effective epimorphism in the ordinary category $\ho(\Disc(C))$. %Wwhat is L? (adjoint to incl!)
        However $\ho(\Disc(C))$ is equivalent to the discrete topological spaces. %TODO why is this true?
        Here the effective epimorphisms are exactly the surjective morphisms, and since for a discrete space there is a bijection $X\to\pi_0(X)$ we have that a map $f$ is effective if and only $\pi_0(L(f))$ is a surjective map.
        Thus $x$ is effective if and only if $X$ has a single connected component, which proves the proposition.
    \end{proof}
\end{lemma}
\begin{remark}
    By definition of loop spaces and explicit computation, the \Cech/ nerve of an effective map $x\colon*\to X$ in $\Top$ (so $X$ is a pointed connected space) is given by 
    \begin{center}
        \begin{tikzcd}
            & C(x)_3 & C(x)_2 & C(x)_1 & C(x)_0 &  C(x)_{-1} \\ 
            \ldots \arrow[r] \arrow[r, yshift=-1em] \arrow[r, yshift=1em] \arrow[r, yshift=-2em] \arrow[r, yshift=2em] 
            & \Omega X\times\Omega X\times\Omega X \arrow[r, yshift=.5em] \arrow[r, yshift=-.5em] \arrow[r, yshift=1.5em] \arrow[r, yshift=-1.5em] \arrow[l, yshift=.5em] \arrow[l, yshift=-.5em] \arrow[l, yshift=1.5em] \arrow[l, yshift=-1.5em]
            & \Omega X\times\Omega X\arrow[r] \arrow[r, yshift=-1em] \arrow[r, yshift=1em] \arrow[l] \arrow[l, yshift=-1em] \arrow[l, yshift=1em]
            & \Omega X \arrow[r, yshift=.5em] \arrow[r, yshift=-.5em] \arrow[l, yshift=.5em] \arrow[l, yshift=-.5em]
            & * \arrow[l] \arrow[r, "x"] & X
        \end{tikzcd}
    \end{center}
    where
    \begin{itemize}
        \item the map induced by a face map 
        \begin{itemize}
            \item $d_0^n$ is $(\gamma_1,\gamma_2,\ldots,\gamma_n)\mapsto(\gamma_2,\ldots,\gamma_n)$
            \item $d_n^n$ is $(\gamma_1,\ldots,\gamma_{n-1},\gamma_n)\mapsto(\gamma_1,\ldots,\gamma_{n-1})$
            \item $d_k^n$ for $0<k<n$ is $(\gamma_1,\ldots,\gamma_k,\gamma_{k+1},\ldots,\gamma_n)\mapsto(\gamma_1,\ldots,\gamma_k\star\gamma_{k+1},\ldots,\gamma_n)$ where $\star$ denotes concatenation of loops
        \end{itemize}
        \item the map induced by a degeneracy map $s_k^n$ is injecting the constant loop $c_x$ at the $k$-th position.
    \end{itemize}
    We can thus view the map $\Conn(\spaces)\to\Grp(\spaces)$ as being a canonically enriched version of the loop space functor.
    The statement of \cref{prop:mayRecognitionTheoremGroup} therefore allows us to recognize whether a pointed connected space $X$ is deloopable:

    It is deloopable if and only if there exists a group object $U\colon\N\left(\Delta^{\op}\right)\to\spaces$ %TODO can do this in top using replacement lemma?
    where $U_n\cong X\times\ldots\times X$ with $X$ appearing $n$ times.
    The delooping is then given by taking the colimit of this diagram and taking the canonical map $A\colon*\to\colim U$ to be a pointed space, as this fulfills $\Omega A\cong X$.
\end{remark}

