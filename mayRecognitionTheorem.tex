A pointed object $X$ in an \inftytop/ is \emph{deloopable} if there exists a pointed connected object $A$ such that $X\simeq\Omega A$ (as pointed objects).

In this section we will give a proof of the fact that a pointed object $X$ is deloopable if and only if it admits additional algebraic structure, which we assemble into a \emph{group object} (see \cref{def:grpObj}).
This phenomenon was also studied by Stasheff in \cite{stasheff_hspaces} in the language of H-spaces.
The idea and presentation of the proof given here is due to \cite{segal_cat_and_cohom}.
We will see that descent plays a major role in this.

A generalization characterizing higher loop spaces was proven in \cite{may2006geometry} and is called the \emph{May Recognition Theorem}.
\begin{definition}[(Augmented) Simplicial Objects]
    Let $C$ be an \inftycat/. 
    We call a map $U\colon\N(\Delta)^{\op}\to C$ a \emph{simplicial object of the \inftycat/ $C$}.
    We call a map $U^+\colon\N(\Delta_+)^{\op}\to C$ an \emph{augmented simplicial object of the \inftycat/ $C$}.
    
    Given an (augmented) simplicial object 

    For any nonempty finite linearly ordered set $J$ we always have a canonical nondecreasing bijection $J\cong[n]$ for some $n\in\Nat$.
    Analogous to the definition of simplicial sets (\cref{def:simpSet}), given a simplicial object $U$ we will write $U_J$ for the evaluation $U_n$ under this canonical identification. 
    We also use the analogous convention for augmented simplicial objects.
\end{definition}
\begin{definition}[Groupoid Object]
    Let $C$ be an \inftycat/ and let $U\colon\N(\Delta)^{\op}\to C$ be a simplicial object such that for every $n\in\Nat$ and partition $[n]=S\cup T$ with $S\cap T=\set{s}$ consisting of a single element, the square 
    \begin{center}
        \begin{tikzcd}[sep = 4em]
            U_n \ar[r] \ar[d] & U_S \ar[d]\\
            U_T \ar[r] &  U_{\set{s}}
        \end{tikzcd}
    \end{center}
    is a pullback square in $C$.

    Then we call $U$ a \emph{groupoid object of $C$}.
    We let $\Gpd(C)$ denote the full subcategory of $\Fun(\N(\Delta)^{\op},C)$ consisting of groupoid objects.
\end{definition}
\begin{definition}[Group Object]\label{def:grpObj}
    Let $C$ be an \inftycat/ and let $U\colon\N(\Delta)^{\op}\to C$ be a groupoid object.
    If $U_0\cong *$ where $*$ is a terminal object of $C$, then we call $U$ a \emph{group object of $C$}.
    We let $\Grp(C)$ denote the full subcategory of $\Fun(\N(\Delta)^{\op},C)$ consisting of group objects.
\end{definition}
The \Cech/ nerve construction is a way to associate a groupoid object to a given map (see \cref{prop:grpdEffectiveIfPullback}).
\begin{definition}[\Cech/ Nerve]
    Let $C$ be an \inftycat/ with finite limits and let $\Delta^1\subset\N(\Delta_+)^{\op}$ be the inclusion corresponding to the full subcategory $[0]\to[-1]$.
    
    Then by \cref{prop:exKanExt} there exists a functor $\CN\colon\Fun(\Delta^1,C)\to\Fun(\N(\Delta_+)^{\op},C)$ that is right adjoint to the restriction functor $F\colon\Fun(\N(\Delta_+)^{\op},C)\to\Fun(\Delta^1,C)$ and for a map $f\colon x\to y\in\Fun(\Delta^1,C)_0$ is the right Kan-extension of $f$ along $\Delta^1\subset\N(\Delta_+)^{\op}$.
    An object $U^+\in\Fun(\N(\Delta_+)^{\op},C)_0$ in the essential image of $\CN$ is called a \emph{\Cech/ nerve}.

    Note that 
\end{definition}
\begin{remark}\label{rmk:cechNerveExplicit}
    Computing the \Cech/ nerve for a map $f\colon x\to y$ explicitly via \cref{def:ptwiseKanExt}, one can see that up to equivalence it is given by
    \begin{center}
        \begin{tikzcd}
            & \CN(f)_3 & \CN(f)_2 & \CN(f)_1 & \CN(f)_0 & \CN(f)_{-1} \\ 
            \ldots \arrow[r] \arrow[r, yshift=-1em] \arrow[r, yshift=1em] \arrow[r, yshift=-2em] \arrow[r, yshift=2em] 
            & x\times_y x \times_y x \times_y x \arrow[r, yshift=.5em] \arrow[r, yshift=-.5em] \arrow[r, yshift=1.5em] \arrow[r, yshift=-1.5em] \arrow[l, yshift=.5em] \arrow[l, yshift=-.5em] \arrow[l, yshift=1.5em] \arrow[l, yshift=-1.5em]
            & x\times_y x \times_y x\arrow[r] \arrow[r, yshift=-1em] \arrow[r, yshift=1em] \arrow[l] \arrow[l, yshift=-1em] \arrow[l, yshift=1em]
            & x\times_y x \arrow[r, yshift=.5em] \arrow[r, yshift=-.5em] \arrow[l, yshift=.5em] \arrow[l, yshift=-.5em]
            & x \arrow[l] \arrow[r, "f"] & y
        \end{tikzcd}
    \end{center}
    where
    \begin{itemize}
        \item the arrows going from left to right are the maps induced by the face maps and are projections 
        \item the arrows going from right to left are the maps induced by the degeneracy maps and are diagonal maps. 
    \end{itemize}
    Here we have only shown a selection of the $1$-skeleton of $\CN(f)$.
    The remaining maps and higher simplices are also determined by the universal property of the iterated pullbacks.
\end{remark}
\begin{definition}[Effective Groupoid Object]
    Let $C$ be an \inftycat/ with small colimits and let $U\colon\N(\Delta)^{\op}\to C$ be a groupoid object.
    Then we call $U$ \emph{effective} if the diagram 
    \begin{center} 
        \begin{tikzcd}[sep = 4em]
            U_1 \ar[r,"U(d_0^1)"] \ar[d,"U(d_1^1)"'] & U_0 \ar[d]\\
            U_0 \ar[r] & \colim\limits_{\N(\Delta)^{\op}} U
        \end{tikzcd}
    \end{center}
    is a pullback square.
\end{definition}
The following proposition provides an alternative characterization of a \Cech/ nerve.
\begin{prop}\label{prop:grpdEffectiveIfPullback} 
    Let $C$ be an \inftycat/ with finite limits and $U^+\colon\N(\Delta_+)^{\op}\to C$ an augmented simplicial object.
    Then it is a \Cech/ nerve if and only if $U^+|_{\N(\Delta)^{\op}}$ is a groupoid object and the square
    \begin{center}
        \begin{tikzcd} [sep = 4em] 
            U_1^+ \arrow[r, "U(d_0^1)"] \arrow[d, "U(d_1^1)"'] & U_0^+ \arrow[d] \\
            U_0^+ \arrow[r] & U_{-1}^+ \\
        \end{tikzcd}
    \end{center}
    is a pullback square.
    \begin{reference}
        \cite[Proposition 6.1.2.11]{HTT}
    \end{reference}
\end{prop}
\begin{corollary}\label{cor:groupoidEffectiveIffColimCechNerve}
    Let $C$ be an \inftycat/ with finite limits and small colimits. 
    Then a groupoid object $U\colon\N(\Delta)^{\op}\to C$ is effective if and only if its induced colimit cone $\N(\Delta)^{\op}\star\Delta^0\cong\N(\Delta_+)^{\op}\to C$ is a \Cech/ nerve.
    \begin{proof}
        This follows directly from the equivalent characterization of an effective groupoid from \cref{prop:grpdEffectiveIfPullback}.
    \end{proof}
\end{corollary}
The following proposition allows us to construct colimit diagrams out of simplicial objects by shifting the degrees.
\begin{prop}\label{prop:shiftIsColimDiam}
    Let $C$ be an \inftycat/ and $U\colon\N(\Delta)^{\op}\to C$ a simplicial object.
    Let $F\colon\Delta_+\to\Delta$ denote the functor sending 
    \begin{itemize}
        \item objects $[n]\in\Delta_+$ to $F([n])=[n+1]$
        \item morphisms $f\colon [m]\to[n]$ of $\Delta_+$ to 
        \begin{align*}
            F(f)\colon [m+1]&\to[n+1]\\
            i&\mapsto
            \begin{cases}
                f(i) & 0\leq i\leq m\\
                n+1 & i=m+1\;.
            \end{cases}
        \end{align*}
        
        Then the augmented simplicial object $\overline{U}=U\circ\N(F)^{\op}$ is a colimit diagram.
    \end{itemize}
    \begin{reference}
        \cite[Lemma 6.1.3.17 (1)]{HTT}
    \end{reference}
\end{prop}
The following proposition is at the heart of our proof of the delooping of pointed objects. 
The proof follows \cite[29-30]{toposes_and_htpy_toposes}.
\begin{prop}\label{prop:groupoidObjInToposAreEffective}
    Let $C$ be an \inftytop/.
    Then every groupoid object $U\colon\N(\Delta)^{\op}\to C$ is effective.
    \begin{proof}
        Let $\overline{U}$ be defined as in \cref{prop:shiftIsColimDiam}.
        Then $\overline{U}$ is an augmented simplicial object with $U_{n+1}=\overline{U}_n$.
        We claim that the natural transformation $\overline{U}|_{\N(\Delta)^{\op}}\to U$ induced by the inclusions $[n]\subset[n+1]$ is cartesian.
        
        Let $f\colon [m]\to[n]$ be a map in $\Delta$. 
        Then the diagram
        \begin{center}
            \begin{tikzcd} [sep = 4em]
                \overline{U}_n=U_{n+1} \arrow[r] \arrow[d,"\overline{U}(f)"'] & U_n \arrow[d, "U(f)"] \\
                \overline{U}_m=U_{m+1} \arrow[r] \arrow[d,"{\overline{U}\left(\set{m}\subset[m]\right)}"'] & U_m \arrow[d,"{U\left(\set{m}\subset[m]\right)}"] \\
                \overline{U}_{\set{m}}=U_{\set{m,m+1}} \arrow[r] & U_{\set{m}}
            \end{tikzcd}
        \end{center}
        where the horizontal maps are given by the inclusions $[n]\xhookrightarrow{}[n+1]$, is commutative. 
        The bottom square is a pullback since $U$ is a groupoid object.
        The outer square is equivalent to
        \begin{center}
            \begin{tikzcd} [sep = 4em]
                \overline{U}_n=U_{n+1} \arrow[r] \arrow[d,"{\overline{U}\left(\set{f(n)}\subset[n]\right)}"'] & U_n \arrow[d, "{U\left(\set{f(n)}\subset[n]\right)}"] \\
                \overline{U}_{\set{f(m)}}=U_{\set{f(m),n+1}} \arrow[r] & U_{\set{f(n)}}
            \end{tikzcd}
        \end{center}
        under the canonical identifications, thus it is also a pullback square since $U$ is a groupoid object.
        
        As the outer and bottom square are pullbacks, the upper square is a pullback by the pasting law.
        Thus the natural transformation $\overline{U}|_{\N(\Delta)^{\op}}\to U$ induced by the inclusions $[n]\xhookrightarrow{}[n+1]$ is cartesian.

        Descent now implies that 
        \begin{center}
            \begin{tikzcd} [sep = 4em]
                \overline{U}_0 \arrow[r] \arrow[d,"U(d_1^1)"'] & \colim\limits_{\N(\Delta)^{\op}}\overline{U}|_{\N(\Delta)^{\op}} \arrow[d] \\
                U_0 \arrow[r] & \colim\limits_{\N(\Delta)^{\op}} U \\
            \end{tikzcd}
        \end{center}
        is a pullback diagram.
        By \cref{prop:shiftIsColimDiam} we have that $\colim\limits_{\N(\Delta)^{\op}}\overline{U}|_{\N(\Delta)^{\op}}\cong\overline{U}_{-1} = U_0$, $\overline{U}_0=U_1$ and thus that $\overline{U}_{0}\to\colim\limits_{\N(\Delta)^{\op}}\overline{U}|_{\N(\Delta)^{\op}}$ corresponds to $U(d_0^1)$ which proves the proposition.
    \end{proof}
\end{prop}
\begin{definition}[Effective Epimorphism]
    Let $C$ be an \inftycat/ with finite limits and let $f\in C_1$ be a map.
    We say that $f$ is an \emph{effective epimorphism} if the \Cech/ nerve $\CN(f)$ is a colimit diagram.
\end{definition}
\begin{prop}\label{prop:mayRecognitionTheoremGroupoid}
    Let $C$ be an \inftytop/ and let $\Fun(\Delta^1,C)_{\text{eff}}\subset\Fun(\Delta^1,C)$ be the full subcategory of effective epimorphisms.  
    Then the functors
    \begin{equation*}
        F\colon\Gpd(C)\subset\Fun(\N(\Delta)^{\op},C)\xrightarrow{\colim}\Fun(\N(\Delta_+)^{\op},C)
    \end{equation*}
    and 
    \begin{equation*}
        G\colon\Fun(\Delta^1,C)_{\text{eff}}\subset\Fun(\Delta^1,C)\xrightarrow{\CN}\Fun(\N(\Delta_+)^{\op},C)
    \end{equation*}
    are fully faithful and have the same essential image in $\Fun(\N(\Delta_+)^{\op},C)$.
    
    Thus they induce an equivalence of \inftycats/ $\Gpd(C)\simeq\Fun(\Delta^1,C)_{\text{eff}}$.
    \begin{proof}
        The functors $\colim\colon\Fun(\N(\Delta)^{\op},C)\to\Fun(\N(\Delta_+)^{\op},C)$ and $\CN\colon\Fun(\Delta^1,C)\to\Fun(\N(\Delta_+)^{\op},C)$ are both pointwise Kan-extensions along fully faithful functors:
        The first is a left Kan-extension along $\N(\Delta)^{\op}\subset\N(\Delta_+)^{\op}$, and the latter a right Kan-extension along $\Delta^1\subset\N(\Delta_+)^{\op}$.
        Thus they are both fully faithful by \cref{prop:exKanExt} and hence $F$ and $G$ are as well.
        It remains to show that they have the same essential image.

        First, let $U$ be an element in the essential image of $G$. 
        Then by definition of an effective epimorphism it is already a colimit diagram, and since $U|_{\N(\Delta)^{\op}}$ is a groupoid, it is in the essential image of $F$.

        Now let $V$ be an element in the essential image of $F$. 
        Then by \cref{prop:groupoidObjInToposAreEffective} and \cref{cor:groupoidEffectiveIffColimCechNerve} we know that it is a \Cech/ nerve and since $\CN(f)|_{\Delta^1}\simeq f$ it is in the essential image of $G$.
    \end{proof}
\end{prop}
The above proposition is a more general version of the delooping of pointed objects.
We can restrict it appropriately to obtain the version for pointed objects.
\begin{definition}[Pointed Objects]
    Let $C$ be an \inftycat/ with a terminal object.
    Then write $C_*\subset\Fun(\Delta^1,C)$ for the full subcategory of maps with domain a fixed terminal object $*\in C_0$.
    We call $C_*$ the \emph{\inftycat/ of pointed objects}, and we call an object $*\to X$ in $C_*$ a \emph{pointed object in $C$}.

    We refer to maps in $C_*$ as \emph{maps of pointed objects}.
\end{definition}
\begin{remark}
    Since terminal objects are essentially unique, choosing different terminal objects induces equivalent models of \inftycats/ of pointed objects.
\end{remark}
\begin{definition}\label{def:catOfPtdConnObj}
    Let $C$ be an \inftycat/ with finite limits and let $\Conn(C_*)\subset C_*$ be the full subcategory of effective epimorphisms.
    We call $\Conn(C_*)$ the \emph{\inftycat/ of pointed connected objects}.
\end{definition}
\begin{remark}
    We will later introduce notions of higher connectedness. 
    It will follow from \cref{prop:conn} \ref{prop:connSection} that the notion of a connected pointed space $*\to X$ here means that the space $X$ is $0$-connected in the language of \cref{def:connected}, which is why we call these objects connected.
\end{remark}
\begin{corollary}[Delooping of Pointed Objects]\label{prop:mayRecognitionTheoremGroup}
    Let $C$ be an \inftytop/. 
    Then the equivalence of \inftycats/ from \cref{prop:mayRecognitionTheoremGroupoid} restricts to an equivalence $\Grp(C)\simeq\Conn(C_*)$.
    \begin{proof}
        This follows immediately from the definition of group objects.
    \end{proof}
\end{corollary}
\begin{definition}[Loop Object]\label{def:loopSpace}
    Let $C$ be an \inftytop/ and let $X\in\left(C_*\right)_0$ be a pointed object.
    Then we call a pullback
    \begin{center}
        \begin{tikzcd} [sep = 4em]
            \Omega X \arrow[r] \arrow[d] & * \arrow[d] \\
            * \arrow[r] & X \\
        \end{tikzcd}
    \end{center}
    a \emph{loop space of $X$}.
\end{definition}
\begin{remark}
    By \cref{rmk:cechNerveExplicit} the \Cech/ nerve of a map $x\colon*\to X$ in an \inftytop/ $C$ is given by
    \begin{center}
        \begin{tikzcd}
            & \CN(x)_3 & \CN(x)_2 & \CN(x)_1 & \CN(x)_0 &  \CN(x)_{-1} \\ 
            \ldots \arrow[r] \arrow[r, yshift=-1em] \arrow[r, yshift=1em] \arrow[r, yshift=-2em] \arrow[r, yshift=2em] 
            & \Omega X\times\Omega X\times\Omega X \arrow[r, yshift=.5em] \arrow[r, yshift=-.5em] \arrow[r, yshift=1.5em] \arrow[r, yshift=-1.5em] \arrow[l, yshift=.5em] \arrow[l, yshift=-.5em] \arrow[l, yshift=1.5em] \arrow[l, yshift=-1.5em]
            & \Omega X\times\Omega X\arrow[r] \arrow[r, yshift=-1em] \arrow[r, yshift=1em] \arrow[l] \arrow[l, yshift=-1em] \arrow[l, yshift=1em]
            & \Omega X \arrow[r, yshift=.5em] \arrow[r, yshift=-.5em] \arrow[l, yshift=.5em] \arrow[l, yshift=-.5em]
            & * \arrow[l] \arrow[r, "x"] & X
        \end{tikzcd}
    \end{center}
    We can thus view the map $\CN\colon C_*\to\Fun(\N(\Delta_+)^{\op},C)$ as being a canonically enriched version of the loop space functor.
    The statement of \cref{prop:mayRecognitionTheoremGroup} therefore allows us to recognize whether a pointed object $x\colon*\to X\in\left(C_*\right)_0$ is deloopable:

    It is deloopable if and only if there exists a group object $U\colon\N\left(\Delta\right)^{\op}\to C$ such that 
    \begin{itemize}
        \item $U_0\simeq*$
        \item $U_1\simeq X$
        \item for $s_0^0\colon[1]\to[0]$ the unique constant map in $\Delta$, we have that $U(s_0^0)$ is equivalent to the map $x\colon*\to X$.
    \end{itemize}
    (This implies in particular $U_n\simeq X\times\ldots\times X$ for $n\geq1$ with $X$ appearing $n$ times.)

    This is because when such a group object exists, a delooping of $X$ is given by taking the colimit $A=\colim\limits_{\N\left(\Delta\right)^{\op}} U$ of this group object, which gives a pointed connected space $a\colon*\to A$ that fulfills $\Omega A\simeq X$ by \cref{prop:mayRecognitionTheoremGroup}.

    Conversely, \cref{prop:mayRecognitionTheoremGroup} also implies that when given a delooping $X\simeq\Omega A$, a group object of the above form exists by taking the \Cech/ nerve of the pointed connected space $a\colon*\to A$ and restricting it to $\N(\Delta)^{\op}$.
\end{remark}
\begin{remark}
    Given a map $x_0\colon*\to X$ in $\Top$, we fix the ordinary pullback
    \begin{center}
        \begin{tikzcd} [sep = 4em]
            \Omega X \arrow[r] \arrow[d] & P_{x_0}X \arrow[->>,d, "h"] \\
            * \arrow[r] & X \\
        \end{tikzcd}
    \end{center}
    (which is also a homotopy pullback) as a model for the loop space.
    Hence the loop space consists of all paths $\gamma\colon [0,1]\to X$ that start and end at $x_0\in X$ and is equipped with the subspace topology of the compact open topology on $\hom_{\Top}\left([0,1],X\right)$ (see e.g. \cite[Definition 1.2.1]{cubical_htpy_theory}).
    
    Then by explicit computation and \cref{rmk:cechNerveExplicit}, one can show that the \Cech/ nerve of a map $x_0\colon*\to X$ in $\Top$ is given by
    \begin{center}
        \begin{tikzcd}
            & \CN(x)_3 & \CN(x)_2 & \CN(x)_1 & \CN(x)_0 &  \CN(x)_{-1} \\ 
            \ldots \arrow[r] \arrow[r, yshift=-1em] \arrow[r, yshift=1em] \arrow[r, yshift=-2em] \arrow[r, yshift=2em] 
            & \Omega X\times\Omega X\times\Omega X \arrow[r, yshift=.5em] \arrow[r, yshift=-.5em] \arrow[r, yshift=1.5em] \arrow[r, yshift=-1.5em] \arrow[l, yshift=.5em] \arrow[l, yshift=-.5em] \arrow[l, yshift=1.5em] \arrow[l, yshift=-1.5em]
            & \Omega X\times\Omega X\arrow[r] \arrow[r, yshift=-1em] \arrow[r, yshift=1em] \arrow[l] \arrow[l, yshift=-1em] \arrow[l, yshift=1em]
            & \Omega X \arrow[r, yshift=.5em] \arrow[r, yshift=-.5em] \arrow[l, yshift=.5em] \arrow[l, yshift=-.5em]
            & * \arrow[l] \arrow[r, "x_0"] & X
        \end{tikzcd}
    \end{center}
    where
    \begin{itemize}
        \item the map induced by a face map 
        \begin{itemize}
            \item $d_0^n$ is $(\gamma_1,\gamma_2,\ldots,\gamma_n)\mapsto(\gamma_2,\ldots,\gamma_n)$
            \item $d_n^n$ is $(\gamma_1,\ldots,\gamma_{n-1},\gamma_n)\mapsto(\gamma_1,\ldots,\gamma_{n-1})$
            \item $d_k^n$ for $0<k<n$ is $(\gamma_1,\ldots,\gamma_k,\gamma_{k+1},\ldots,\gamma_n)\mapsto(\gamma_1,\ldots,\gamma_k*\gamma_{k+1},\ldots,\gamma_n)$ where $*$ denotes concatenation of loops
        \end{itemize}
        \item the map induced by a degeneracy map $s_k^n$ is injecting the constant loop $c_{x_0}$ at the $k$-th position.
    \end{itemize}
\end{remark}
\addcontentsline{toc}{subsection}{Delooping of Pointed Objects in $\spaces$}
\subsection*{Delooping of Pointed Objects in $\spaces$}
We now want to check that the notion of pointed connected objects of general \inftytops/ coincides with pointed connected spaces in the \inftytop/ $\spaces$.
Thus we recover the statement of the May Recognition Theorem for $n=1$, which says that a pointed space is equivalent to a loop space of a pointed connected object if and only if it admits the structure of a group object in $\Top$. 
\begin{definition}
    Let $C$ be an \inftytop/.
    With the convention that the $(-2)$-truncated maps are the equivalences, we recursively define that a map $f\colon X\to Y$ in $C$ is \emph{$k$-truncated} for $k\geq -1$ if the diagonal map $X\to X\times_YX$ is $(k-1)$-truncated.
    
    We consider an object $X$ to be \emph{$k$-truncated} if the map $X\to *$ is $k$-truncated.
\end{definition}
\begin{remark}
    There is ``another'' definition of truncatedness in $\Top$:

    A space $X$ in $\Top$ is 
    \begin{itemize}
        \item \emph{topologically $(-2)$-truncated} if it is contractible 
        \item \emph{topologically $n$-truncated} for $n\in\Nat\cup\set{-1}$ if for all $x\in X$ and $k>n$, we have $\pi_k(X,x)\cong 0$.
    \end{itemize}
    A map $f\colon X\to Y$ in $\Top$ is \emph{topologically $n$-truncated} for $n\in\Nat\cup\set{-2,-1}$ if all its homotopy fibers are $n$-truncated.
    
    This definition agrees with the definition of truncatedness in $\spaces$:
    \begin{itemize}
        \item A topologically $(-2)$-truncated map is precisely a weak equivalence in $\Top$, hence $(-2)$-truncated in $\spaces$.
        \item For $n\geq -1$, the pullback square
        \begin{center}
            \begin{tikzcd} [sep = 4em]
                X\times_{Y} X \arrow[r, "g"] \arrow[d, "g"] & X \arrow[d, "f"] \\
                X \arrow[r, "f"] & Y \\
            \end{tikzcd}
        \end{center}
        in $\spaces$ shows that $f$ is $n$-truncated if and only if $g$ is $n$-truncated by the fiberwise characterization of homotopy pullbacks \cref{prop:fiberwiseCharOfHtpyPb}.
        (Note that a map is $(-1)$-truncated if and only if its homotopy fibers are contractible or empty.)
        Furthermore, the universal map $s\colon X\to X\times_Y X$ is a section of $g$ in $\spaces$.
        Inspection of the long exact sequences of homotopy groups shows that $g$ is topologically $n$-truncted if and only if the section $s$ is topologically $(n-1)$-truncated.

        This allows us to recursively reduce to the case of $f\colon X\to Y$ being $(-2)$-truncated, in which we have already seen that both definitions coincide.
    \end{itemize}
\end{remark}
\begin{definition}
    Let $C$ be an \inftytop/ and let $\Disc(C)\subset C$ be the full subcategory of $0$-truncated objects.
    We call $\Disc(C)$ the \emph{\inftycat/ of discrete objects of $C$}.
\end{definition}
\begin{prop}\label{prop:inclOfSetIsFF}
    There is a canonical fully faithful functor $i\colon\N(\Set)\to\spaces$ whose essential image consists precisely of $\Disc(\spaces)$.
    Furthermore, this functor $i$ has $\pi_0\colon\spaces\to\N(\Set)$ (induced by the path-connected component functor $\pi_0\colon\Top\to\Set$) as a left adjoint.
    \begin{reference}
        \cite[Proposition 7.8.3]{cisinski_2019}
    \end{reference}
\end{prop}
\begin{prop}\label{prop:testEffEpiInDisc}
    Let $C$ be an \inftytop/ and let $\tau\colon C\to\Disc(C)$ be a left adjoint to the inclusion $\Disc(C)\subset C$.
    Then a map $f$ in $C$ is an effective epimorphism if and only if $\tau(f)$ is an effective epimorphism in the ordinary category $\ho(\Disc(C))$ (via \cref{rmk:boardmanVogt}).
    \begin{reference}
        \cite[Proposition 7.2.1.14]{HTT}
    \end{reference}
\end{prop}
\begin{remark}
    It follows from the argument given in \cite[Notation 5.5.6.2]{HTT} that $\Disc(C)$ is in fact equivalent to $\N(\ho(\Disc(C)))$.
\end{remark}
\begin{prop}\label{prop:effInSpaceIffSurj}
    A map $f\colon x\to y$ in $\spaces$ is an effective epimorphism if and only if it induces a surjection on connected components.
    \begin{proof}
        First, by \cref{prop:inclOfSetIsFF} there is a fully faithful functor $i\colon\N(\Set)\to\spaces$ whose essential image consists precisely of the discrete objects and has $\pi_0\colon\spaces\to\N(\Set)$ as a left adjoint.
        By \cref{prop:testEffEpiInDisc} the statement of the lemma is thus equivalent to checking whether $\pi_0(f)$ is an effective epimorphism in $\Set$.
        In $\Set$, computation shows that the effective epimorphisms are exactly the surjective morphisms, hence a map $f$ in $\spaces$ is an effective epimorphism precisely if $\pi_0(f)$ is surjective. %TODO rezk top and htpy top Example 1.14
    \end{proof}
\end{prop}
\begin{corollary}
    Let $*\to X$ be a pointed space in $\spaces$.
    Then it is an effective epimorphism if and only if $X$ is path-connected as a topological space.
    \begin{proof}
        This follows from \cref{prop:effInSpaceIffSurj} and the fact that $*$ has exactly one connected component.
    \end{proof}
\end{corollary}