%TODO motivate presentability (perhaps its necessary for categorical homotopy groups)
In this section, we want to motivate the definition of an \inftytop/.
Very vaguely speaking, an \inftytop/ is an \inftycat/ which behaves much like the \inftycat/ of spaces $\spaces$.
There are different equivalent characterizations of an \inftytop/; for an overview and the proofs of some of these equivalent characterizations, see e.g. \cite[\S 6.1]{HTT}.
We will say that an \inftycat/ $C$ is an \inftytop/ if it 
\begin{itemize} 
    \item is ``locally presentable''
    \item has ``universal colimits'' (also just called ``universality'')
    \item has ``descent'' 
\end{itemize}
which corresponds to \cite[Theorem 6.1.0.6 (2)]{HTT}.
This characterization of \inftytops/ is due to Charles Rezk.
We will give precise meaning of all of these conditions in \cref{sec:inftyTopDef} and will only give some intuition here.

First, the condition that an \inftycat/ is locally presentable means that it is determined by a ``small'' amount of data, despite it not needing to be small itself.
We will deal with this condition in a very pragmatic way (see \cref{def:locallyPresentable}).

The other two conditions are somewhat related to each other as we will see when proving them for $\spaces$.
They are both supposed to ensure homotopy fibers of maps behave nicely under various glueings.

To motivate universal colimits, let us restrict to the case of pushouts.
For a cube 
\begin{center}
    \begin{tikzcd} [sep = .5 cm]
        Y_{0} \arrow [dr] \arrow [rr] \arrow [dd] & & Y_1 \arrow [dr] \arrow [dd] \\
        & Y_2 \arrow [rr, crossing over] \arrow [dd] & & Y \arrow [dd] & \\
        X_{0} \arrow [dr] \arrow [rr] & & X_1 \arrow [dr] \\
        & X_2 \arrow [from=uu, crossing over] \arrow [rr] & & X &
    \end{tikzcd}
\end{center}
whose side faces are homotopy pullbacks and whose bottom face is a homotopy pushout, the top face is also a homotopy pushout.
This is called ``universality of pushouts'' and is also known as Mathers Second Cube Theorem in $\Top$.

The relation to homotopy fibers can be seen in the following:
Let 
\begin{center}
    \begin{tikzcd} [sep= 4em]
        X_{0} \arrow[r] \arrow [d] & X_1 \arrow [d]\\
        X_2 \arrow[r] & X
    \end{tikzcd}
\end{center}
be a homotopy pushout square and let $X\xtwoheadrightarrow{} B$ be a fibration such that the induced maps $X_i\xtwoheadrightarrow{} B$ for $i\in\set{0,1,2}$ are also fibrations.
We can view this situation as the fibration $X\xtwoheadrightarrow{} B$ being ``glued together'' over a fixed base space by the fibrations $X_i\xtwoheadrightarrow{} B$.
% \begin{center}
%     \begin{tikzcd} [column sep = 1em, row sep = 3em]
%         X_{0} \arrow[>->, rr] \arrow [>->, d] && X_1 \arrow [>->, d]\\
%         X_2 \arrow[>->, rr] & {} \arrow [->>, d] & X\\
%         & B &
%     \end{tikzcd}
% \end{center}
% over $B$, where for $X_i\xtwoheadrightarrow{} B$ for all $i\in\set{0,1,2}$ are also fibrations.
For a given point $b\in B$ we can then form a cube 
\begin{center}
    \begin{tikzcd} [sep = .5 cm]
        F_{0} \arrow [dr] \arrow [rr] \arrow [dd] & & F_1 \arrow [dr] \arrow [dd] \\
        & F_2 \arrow [rr, crossing over] \arrow [dd] & & F \arrow [dd] & \\
        X_{0} \arrow [dr] \arrow [rr] & & X_1 \arrow [dr] \\
        & X_2 \arrow [from=uu, crossing over] \arrow [rr] & & X &
    \end{tikzcd}
\end{center}
where the top face consists of the fibers over $b$ of the fibrations to $B$.
By the pasting law we know that the side faces are all homotopy pullbacks.
Thus we can apply Mathers Second Cube Theorem which shows that the homotopy fiber $F$ is the homotopy pushout of the homotopy fibers of $X_i\xtwoheadrightarrow{} B$ for $i\in\set{0,1,2}$.
% We may view this situation as the fibration $X\xtwoheadrightarrow{} B$ being ``glued together'' by the fibrations $X_1\xtwoheadrightarrow{} B$ and $X_2\xtwoheadrightarrow{} B$ which are compatible along the intersection.
% Now let $b\colon*\to B$ be a point. 
% Pulling the square back along this point gives a diagram
% \begin{center}
%     \begin{tikzcd} [column sep = 1em, row sep = 3em]
%         F_{0} \arrow[>->, rr] \arrow [>->, d] && F_1 \arrow [>->, d] & X_{0} \arrow[>->, rr] \arrow [>->, d] && X_1 \arrow [>->, d]\\
%         F_2 \arrow[>->, rr] & {} \arrow [->>, d] & F & X_2 \arrow[>->, rr] & {} \arrow [->>, d] & X\\
%         & * \arrow[rrr, "b"] & & & B &
%     \end{tikzcd}
% \end{center}
% where we omit the maps $F_i\to X_i$ for $i\in\set{0,1,2}$ and $F\to X$ for clarity.
% Since the right square is a homotopy pushout, we also want the map of homotopy fibers to be a homotopy pushout as this is true in $\Top$.

% Asking for this to be true implies a more general statement:
% Let $f\colon A\to B$ be a map and $a\colon *\to A$ be a point.
% Then by pulling back we get a diagram 
% \begin{center}
%     \begin{tikzcd} [column sep = 1em, row sep = 3em]
%         F_{0} \arrow[>->, rr] \arrow [>->, d] && F_1 \arrow [>->, d] & Y_{0} \arrow[>->, rr] \arrow [>->, d] && Y_1 \arrow [>->, d] & X_{0} \arrow[>->, rr] \arrow [>->, d] && X_1 \arrow [>->, d]\\
%         F_2 \arrow[>->, rr] & {} \arrow [->>, d] & F & Y_2 \arrow[>->, rr] & {} \arrow [->>, d] & Y & X_2 \arrow[>->, rr] & {} \arrow [->>, d] & X\\
%         & * \arrow[rrr, "a"] & & & A \arrow[rrr, "f"] & & & B &
%     \end{tikzcd}
% \end{center}
% and $F$ is the homotopy fiber of $X\twoheadrightarrow B$ at $f(a)$ and $Y\twoheadrightarrow A$ at $a$.
% Then the the middle square is a homotopy pushout:
% Note that the pushout $Y_1\cup_{Y_0}Y_2$ is a homotopy pushout. 
% By replacing the canonical map $Y_1\cup_{Y_0}Y_2\to A$ with a fibration via a factorization $Y_1\cup_{Y_0}Y_2\xtailrightarrow{\sim}\overline{Y}\twoheadrightarrow A$ (note that $\overline{Y}$ is also a homotopy pushout) and the canonical map $Y_1\cup_{Y_0}Y_2\to Y$ by a lift $\alpha\colon\overline{Y}\to Y$, we know that $F$ is also a homotopy fiber of $\overline{Y}\twoheadrightarrow A$.
% But this means that $\alpha$ induces weak equivalences between all homotopy fibers of the fibrations $\overline{Y}\twoheadrightarrow A$ and $Y\twoheadrightarrow A$, hence $\alpha$ is a weak equivalence (e.g. by the long exact sequence of homotopy groups) and therefore $Y$ is a homotopy pushout.

% Using the pasting lemma, asking for the middle square to be a homotopy pushout gives the following statement: 
We can also ask for this to be true not only for pushouts but for general small colimits; 
this also  holds in $\Top$ as well, but it turns it that is suffices to check it for pushouts and small coproducts as we will see in \cref{lem:univColimIffUnivPoAndCoprod}.
This more general property is then called ``universality''.

We next take a look at descent.
When given a diagram 
\begin{center}
    \begin{tikzcd} [sep = 4em]
        X_1 \arrow[d] & X_{0} \arrow[l] \arrow[r] \arrow[d] & X_2 \arrow[d] \\
        B_1 & B_{0} \arrow[l] \arrow[r] & B_2 \\
    \end{tikzcd}
\end{center}
where the vertical maps have equivalent homotopy fibers, then the induced map of homotopy pushouts $X\to B$ should have equivalent fibers as well.
``Having equivalent fibers'' is supposed to mean that the induced maps between fibers of the vertical maps are weak equivalences.
By the fiberwise characterization of homotopy pullbacks this is equivalent to saying that in a cube 
\begin{center}
    \begin{tikzcd} [sep = .5 cm]
        X_{0} \arrow [dr] \arrow [rr] \arrow [dd] & & X_1 \arrow [dr] \arrow [dd] \\
        & X_2 \arrow [rr, crossing over] \arrow [dd] & & X \arrow [dd] & \\
        B_{0} \arrow [dr] \arrow [rr] & & B_1 \arrow [dr] \\
        & B_2 \arrow [from=uu, crossing over] \arrow [rr] & & B
    \end{tikzcd}
\end{center}
where the left and rear faces are homotopy pullbacks and the top and bottom faces are homotopy pushouts, the front and right faces are homotopy pullbacks as well.
This property will be called ``descent for pushouts'' and is also known as Mathers First Cube Theorem in $\Top$.
Again a natural generalization to consider is to replace pushouts by arbitrary small diagrams, and this property will later be called ``descent''.
It will again turn out that we can reduce to showing descent for only pushouts and small coproducts, which we will see in \cref{lem:descentIffDescentPoAndCoprod}.