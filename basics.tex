In this section we would like to fix the general notation on \inftycats/ and model categories used throughout the rest of the text and remind the reader of some concepts which will be important later on.
Some familiarity with \inftycats/ and Kan-complexes is assumed, in particular familiarity with limits/colimits in \inftycats/.
We also assume familiarity with (proper) model categories and the computation of homotopy limits/colimits (in particular homotopy pullbacks/pushouts and homotopy products/coproducts).
We will not give proofs for any statements in this section; we will however point to the relevant literature.
\begin{definition}
    For every $n\in\Nat$, let $[n]$ denote the linearly ordered set $\set{0<\ldots<n}\subset\Nat$ and let $[-1]=\emptyset$, which we also consider to be linearly ordered.
\end{definition}
\begin{definition}
    Let $X,Y$ be linearly ordered sets and $f\colon X\to Y$ a function of the underlying sets.
    Then we call $f$ \emph{nondecreasing} if for all $a,b\in X$ with $a\leq b$, we have $f(a)\leq f(b)$.
\end{definition}
\begin{definition}[(Augmented) Simplex Category]
    Let $\Delta$ be the category with
    \begin{itemize}
        \item objects the linearly ordered sets $[n]$ for all $n\in\Nat$
        \item morphisms the \emph{nondecreasing} functions.
    \end{itemize}
    We call this category the \emph{simplex category $\Delta$}.

    Similarly, let $\Delta_+$ be the category with
    \begin{itemize}
        \item objects the linearly ordered sets for all $n\in\Nat\cup\set{-1}$ 
        \item morphisms again the \emph{nondecreasing} functions.
    \end{itemize}
    We call this category the \emph{augmented simplex category $\Delta_+$}.
\end{definition}
\begin{definition}[Simplicial Set]
    We set $\sSet=\Fun(\Delta^{\op},\Set)$ to be the \emph{category of simplicial sets}.
    Given a simplicial set $X\in\sSet$ and an object $[n]\in\Delta$ we denote evaluation by $X([n])=X_n$.

    Note that for any given nonempty finite linearly ordered set $A$ with $n\in\Nat$ elements, there exists a unique nondecreasing bijection to $[n]$.
    We write $X(A)$ for the evaluation $X_n$ under this canonical identification. 
\end{definition}
\begin{remark}\label{rmk:linOrderedAsCat}
    Any linearly ordered set $X$ can be viewed as a category by setting
    \begin{equation*}
        \hom_{X}(a,b)=\begin{cases}
            * & a\leq b\\
            \emptyset & otherwise
        \end{cases}
    \end{equation*}
    for $a,b\in X$.
    When viewing linearly ordered sets $X,Y$ as categories in this way, the nondecreasing functions $f\colon X\to Y$ correspond precisely to functors.

    This gives fully faithful inclusions $\Delta\subset\Cat$ and $\Delta_+\subset\Cat$, where $\Cat$ denotes the \emph{category of small categories}.
\end{remark}
\begin{definition}[Nerve of an Ordinary Category] %smallness problematic when applied to sSet and Top?
    Let
    \begin{align*}
        Y\colon\Cat&\to\Fun(\Cat^{\op},\Set)\\
        C&\mapsto\hom_{\Cat}(-,C)
    \end{align*}
    denote the \emph{Yoneda embedding}.

    Then composing the Yoneda embedding with the functor $\Fun(\Cat^{\op},\Set)\to\sSet$ induced by the inclusion $\Delta\subset\Cat$ from the previous remark gives a functor $\N\colon\Cat\to\sSet$ called the \emph{nerve functor}.
    For a given small category $C$ we refer to the simplicial set $\N(C)$ as the \emph{nerve of the category $C$}.
\end{definition}
\begin{definition}[Standard Simplex]
    For every $n\in\Nat$ we write $\N([n])=\Delta^n$ and refer to $\Delta^n$ as \emph{the standard $n$-simplex}.
    Similarly, for any nonempty finite linearly ordered set $A$ we write $\N(A)=\Delta^A$.

    We also define 
    \begin{equation*}
        \partial\Delta^n=\bigcup\limits_{A\subsetneq[n]}\Delta^A\subset\Delta^n
    \end{equation*}
    to be the \emph{boundary of $\Delta^n$} and 
    \begin{equation*}
        \Lambda_k^n=\bigcup\limits_{k\in A\subsetneq[n]}\Delta^A\subset\Delta^n
    \end{equation*}
    to be the \emph{$k$-th horn of $\Delta^n$} for $0\leq k\leq n$.
    (The unions are the degreewise unions in $\Set$.) %TODO is this true?
\end{definition}
\begin{lemma}[Homotopy Category]\label{lem:nerveFF}
    The nerve functor $\N\colon\Cat\to\sSet$ is fully faithful and has a left adjoint $\ho\colon\sSet\to\Cat$ which we call the \emph{homotopy category} of a simplicial set.
    \begin{reference}
        \cite[Proposition 1.2.2.1]{kerodon}
    \end{reference}
\end{lemma}
\begin{definition}[Simplicial Category]
    We call a category enriched over $\sSet$ a \emph{simplicial category}.
    For a simplicial category $C$, we let $\Hom_C(-,-)\colon C^{\op}\times C\to\sSet$ denote the enriched $\Hom$-functor.
   
    Let $\Cat_{\Delta}$ denote the \emph{category of small simplicial categories} with morphisms the \emph{$\sSet$-enriched functors}.%TODO maybe explain what that is/cite
    
    We call a simplicial category $C$ \emph{tensored over $\sSet$} if for each $X\in C$ and $K\in\sSet$ there is an object $X\otimes K$ such that there is a natural isomorphism
    \begin{equation*}
        \Hom_C(X\otimes K,-)\cong\Fun(K,\Hom_C(X,-))\;.
    \end{equation*}

    We call a simplicial category $C$ \emph{powered over $\sSet$} if for each $Y\in C$ and $K\in\sSet$ there is an object $Y^K$ such that there is a natural isomorphism
    \begin{equation*}
        \Hom_C(-,Y^K)\cong\Fun(K,\Hom_C(-,Y))\;.
    \end{equation*}
    Note that due to the functoriality of the right side, these extend to functors $(-)\otimes(-):C\times\sSet\to C$ and $(-)^{(-)}:C\times\sSet^{\op}\to C$.
\end{definition}
\begin{definition}[LLP/RLP]
    Let $C$ be a category and let $S$ be a class of maps in $C$.
    We say that a morphism $f\colon A\to B$ has the \emph{left lifting property (LLP) with respect to $S$} if for every map $g\colon X\to Y\in S$ and all choices of maps $\lambda\colon A\to X$ and $\mu\colon B\to Y$ that make the square
    \begin{center}
        \begin{tikzcd} [sep = 4em]
            A \arrow[r,"\lambda"] \arrow[d, "f"] & X \arrow[d, "g"] \\
            B \arrow[ur, dashed, "\exists"] \arrow[r, "\mu"] & Y
        \end{tikzcd}
    \end{center}
    commute, there exists a map $B\to X$ as indicated making the diagram commute.

    Dually, we say that a morphism $g\colon X\to Y$ has the \emph{right lifting property (RLP) with respect to $S$} if for every map $f\colon A\to B\in S$ and all choices of maps $\lambda\colon A\to X$ and $\mu\colon B\to Y$ that make the square
    \begin{center}
        \begin{tikzcd} [sep = 4em]
            A \arrow[r,"\lambda"] \arrow[d, "f"] & X \arrow[d, "g"] \\
            B \arrow[ur, dashed, "\exists"] \arrow[r, "\mu"] & Y
        \end{tikzcd}
    \end{center}
    commute, there exists a map $B\to X$ as indicated making the diagram commute.

    We denote the class of maps having the LLP with respect to $S$ with $l(S)$, and the class of maps having the RLP with respect to $S$ with $r(S)$.
\end{definition}
\begin{definition}[Weak Factorization System]
    Let $C$ be a category and let $\Map(C)$ denote the class of all maps of $C$.
    Let $L,R\subset\Map(C)$ be subclasses.
    If 
    \begin{itemize}
        \item every morphism $f$ in $C$ can be factorized as $f=p\circ i$ with $i\in L$ and $p\in R$
        \item $L=l(R)$
        \item $R=r(L)$
    \end{itemize}
    then the pair $(L,R)$ is a \emph{weak factorization system (WFS) on $C$}.
\end{definition}
\begin{definition}[Functorial WFS]
    Let $C$ be a category and let $(L,R)$ be a weak factorization system on $C$.
    Let $i,p\colon\Fun\left([2],C\right)\to\Fun\left([1],C\right)$ be induced by the maps
    \begin{align*}
        [1]&\to[2]\\
        i&\mapsto i
    \end{align*}
    \begin{align*}
        [1]&\to[2]\\
        i&\mapsto i+1
    \end{align*}
    respectively.

    Then we call $(L,R)$ \emph{functorial} if there exists a functor $\Gamma\colon\Fun\left([1],C\right)\to\Fun\left([2],C\right)$ such that
    \begin{itemize}
        \item for all maps $f$ in $C$, $(i\circ\Gamma)(f)\in L$
        \item for all maps $f$ in $C$, $(p\circ\Gamma)(f)\in R$
        \item and $\Gamma$ is a section of the canonical functor $\Fun\left([2],C\right)\to\Fun\left([1],C\right)$ induced by composition.
    \end{itemize}
\end{definition}
\begin{definition}[Model Category]
    Let $C$ be a complete and cocomplete category and let $\Map(C)$ denote the class of all maps of $C$.
    Let $\Cof,\Fib,\W\subset\Map(C)$ be subclasses which we will call \emph{cofibrations}, \emph{fibrations} and \emph{weak equivalences} respectively.
    Then we call $C$ together with this data a \emph{model category} if the following hold:
    \begin{itemize}
        \item $\W$ contains all isomorphisms and has the two-out-of-three property.
        \item $(\Cof\cap\W,\Fib)$ is a weak factorization system.
        \item $(\Cof,\Fib\cap\W)$ is a weak factorization system.
    \end{itemize}
\end{definition}
\begin{definition}[$\kappa$-Compact Object]
    Let $C$ be a category, $X\in C$ an object and $\kappa$ a regular cardinal.
    We say that $C$ is \emph{$\kappa$-compact} if the functor $\hom_C(X,-)\colon C\to\Set$ preserves all $\kappa$-filtered colimits. %TODO introduce kappa filtered colim, see cisinsky Proposition 2.1.9
\end{definition}
\begin{definition}[Locally Presentable Category] %TODO http://www.ub.edu/topologia/seminars/Locally_presentable.pdf
    Let $C$ be a category.
    We call $C$ \emph{locally presentable} if it has small colimits and there exists a regular cardinal $\kappa$ such that
    \begin{itemize}
        \item the isomorphism classes of $\kappa$-compact objects form a set
        \item and every object in $C$ is a $\kappa$-filtered colimit of $\kappa$-compact objects.
    \end{itemize}
\end{definition}
\begin{definition}[$\lambda$-Sequence] %hovey model cats Definition 2.1.1
    Let $C$ be a category with small colimits and $\lambda$ an ordinal.
    A \emph{$\lambda$-sequence} is a functor $X\colon\lambda\to C$ (where we view $\lambda$ as a category via \cref{rmk:linOrderedAsCat}) that preserves colimits.

    We call the induced morphism $X_0\to\colim\limits_{\beta<\lambda} X(\beta)$ the \emph{transfinite composition of $X$}.
\end{definition}
\begin{definition}[Saturated Class]
    Let $C$ be a category with small colimits and $I$ a set of morphisms of $C$.
    Then we call $I$ \emph{saturated} if it is closed under retracts, pushouts and transfinite composition. %TODO perhaps find way to avoid transfinite composition
\end{definition}
\begin{definition}[Relative $\kappa$-Small]
    Let $C$ be a category with small colimits, let $S$ be a class of morphisms in $C$ and let $\kappa$ be a cardinal.
    
    We say that an object $A\in C$ is \emph{$\kappa$-small relative to $S$} if the functor $\hom_C(A,-)\colon C\to\Set$ preserves colimits indexed by $\lambda$-sequences $X$ such that $\lambda$ is a $\kappa$-filtered ordinal and $X(\beta)\to X(\beta+1)$ is in $D$ for $\beta+1<\lambda$.
\end{definition}
\begin{thm}[Small Object Argument] %TODO Proposition 10.5.16 Model Categories and Their Localizations 
    Let $C$ be a category with small colimits and $I$ a set of morphisms of $C$.
    If there exists a cardinal $\kappa$ such that the domains of all morphisms in $I$ are $\kappa$-small relative to transfinite compositions of pushouts of maps in $I$, we say that \emph{$I$ admits the small object argument}.
    
    In this case $(l(r(I)),r(I))$ is a functorial weak factorization system and $l(r(I))$ is the smallest (with respect to inclusion) saturated class containing $I$.
\end{thm}
\begin{reference}
    \cite[Theorem 2.1.14]{hovey2007model} and \cite[Proposition 2.1.5]{cisinski_2019}
\end{reference}
\begin{definition}[Cofibrantly Generated Model Structure]
    Let $C$ be a model category with $\Cof,\Fib$ and $\W$ the cofibrations, fibrations and weak equivalences respectively.
    Then we say that $C$ is \emph{cofibrantly generated} if there exist small sets of morphisms $I,J$ such that
    \begin{itemize}
        \item $I$ and $J$ admit the small object argument
        \item $(\Cof,\Fib\cap\W)=(l(r(I)),r(I))$
        \item $(\Cof\cap\W,\Fib)=(l(r(J)),r(J))$.
    \end{itemize} 
\end{definition}
\begin{definition}[Combinatorial Model Category]
    We say that a model category $C$ is a \emph{combinatorial model category} if $C$ is locally presentable and the model structure is cofibrantly generated.
\end{definition} %TODO maybe also put here that Top is not locally presentable
We will need the standard model structure on $\sSet$, the Quillen model structure. 
Whenever we speak of cofibrations, fibrations or weak equivalences between simplicial sets, we are referring to this model structure.
\begin{thm}[Quillen Model Structure on $\sSet$] %TODO probably best to just prove all model cat stuff about \sSet here already
    There is a unique model structure on $\sSet$ where 
    \begin{itemize}
        \item the cofibrations are the monomorphisms, which also admit the following description: 
        They are the smallest (with respect to inclusion) saturated class of maps containing the inclusions $\set{\partial\Delta^n\xhookrightarrow{}\Delta^n\mid n\in\Nat}$.
        \item the fibrations are the maps having the right lifting property with respect to the inclusions $\set{\Lambda_k^n\xhookrightarrow{}\Delta^n\mid 0\leq k\leq n, n\in\Nat}$ and are called \emph{Kan-fibrations}.
    \end{itemize}
    The fibrant objects are the \emph{Kan-complexes}.
    The weak equivalences are the maps which are weak equivalences in $\Top$ after geometric realization.

    We call this model structure the \emph{Quillen Model Structure on $\sSet$}.
    \begin{reference}
        \cite[Theorem 3.1.8 and Theorem 3.1.29]{cisinski_2019} and \cite[Chap. II, \S 3, Theorem 1]{Quillen1967}
    \end{reference}
\end{thm}
\begin{definition}[Simplicial Model Category]
    A \emph{simplicial model category} is a simplicial category $C$ that is 
    \begin{itemize}
        \item tensored and powered
        \item equipped with a model structure
    \end{itemize} 
    such that for a cofibration $i:A\rightarrowtail B$ and a fibration $p:X\twoheadrightarrow Y$ in the model structure the induced map of simplicial sets
    \begin{equation*}
        \Hom_C(B,X)\to\Hom_C(A,X)\times_{\Hom_C(A,Y)}\Hom_C(B,Y)
    \end{equation*}
    is a fibration that additionally is a weak equivalence whenever $i$ or $p$ is a weak equivalence.
    
    We say that a simplicial model category is a \emph{combinatorial simplicial model category} if its underlying model category is a combinatorial model category.
\end{definition}
\begin{construction}[{\cite[Notation 2.4.3.1]{kerodon}}]
    For $[n]\in\Delta$ we define a simplicial category $\C[n]$ as follows:
    \begin{itemize}
        \item the objects of $\C[n]$ are the elements of the set $\set{0,\ldots,n}$
        \item for $i,j\in\set{0,\ldots,n}$, the simplicial $\Hom$-objects are given by 
            \begin{equation*}
                \Hom_{\C[n]}(i,j)=\N(P_{i,j})
            \end{equation*}
            where $P_{i,j}$ is the partially ordered set of subsets of $\set{0,\ldots,n}$ which contain both $i$ and $j$ with the relation given by reverse inclusion
        \item for $i,j,k\in\set{0,\ldots,n}$ the composition is given by applying the nerve $\N(-)$ to the map
        \begin{align*}
                P_{i,j}\times P_{j,k}&\to P_{i,k}\\
                (I,J)&\mapsto I\cup J\;.
        \end{align*}
    \end{itemize}
    Given a morphism $[n]\to[m]\in\Delta$ we define
    \begin{equation*}
        \C[f]\colon\C[n]\to C[m]
    \end{equation*}
    which
    \begin{itemize}
        \item sends an object $i\in\C[n]$ to $f(i)\in\C[m]$
        \item for $i,j\in\C[n]$ induces a map between $\Hom$-objects by applying $\N(-)$ to the map
            \begin{align*}
                P_{i,j}&\to P_{f(i),f(j)}\\
                I&\mapsto f(I)
            \end{align*}
    \end{itemize}
    so we obtain a functor $\C[-]\colon\Delta\to\Cat_{\Delta}$.
\end{construction}
\begin{definition}[Homotopy Coherent Nerve] %smallness problematic when applied to sSet?
    Let
    \begin{align*}
        Y\colon\Cat_{\Delta}&\to\Fun(\Cat_{\Delta}^{\op},\Set)\\
        C&\mapsto\hom_{\Cat}(-,C)
    \end{align*}
    denote the \emph{Yoneda embedding}.
    Then composing the Yoneda embedding with the functor $\Fun(\Cat_{\Delta}^{\op},\Set)\to\sSet$ induced by the map $C[-]$ from the previous construction gives a functor $\Nhc\colon\Cat_{\Delta}\to\sSet$.
    For a given small simplicial category $A$ we refer to $\Nhc(A)$ as the \emph{simplicial nerve} or \emph{homotopy coherent nerve of $A$}.
\end{definition}
For completeness we will also describe the Joyal model structure on $\sSet$, as it plays a similar role for \inftycats/ as the Quillen model structure does for Kan-complexes.
\begin{thm}[Joyal Model Structure on $\sSet$]
    The following two classes of maps endow $\sSet$ with a unique model structure:
    \begin{itemize}
        \item The cofibrations are the monomorphisms.
        \item The weak equivalences are the maps $A\to B$ such that for every \inftycat/ $C$ the induced map $\ho(\Fun(B,C))\to\ho(\Fun(A,C))$ is an equivalence of ordinary categories.
    \end{itemize}
    The fibrant objects of this model structure are called \inftycats/ and are precisely the simplicial sets whose map to the point has the RLP with respect to $\set{\Lambda_k^n\xhookrightarrow{}\Delta^n\mid 0< k< n, n\in\Nat}$.
    We refer to the weak equivalences of this model structure as \emph{weak categorical equivalences}.
    We refer to weak categorical equivalences whose domain and target are both \inftycats/ as \emph{equivalences of \inftycats/}.
    We call this model structure the \emph{Joyal Model Structure on $\sSet$}.
    \begin{reference}
        \cite[Definition 3.3.7, Theorem 3.6.8 and Theorem 3.6.1]{cisinski_2019}
    \end{reference}
\end{thm}
\begin{definition}[Equivalences in an \inftyCat/]
    Let $C$ be an \inftycat/ and $f\colon x\to y\in C_1$ a map.
    Then we say that $f$ is an \emph{equivalence} if there exists a map $g\colon y\to x\in C_1$ and $2$-simplices of the form
    \begin{center}
        \begin{tikzcd} [sep = 3em]
            & x \ar[dr,"f"]&\\
            y \ar[ur,"g"] \ar[rr,"\id_y"]&  & y
        \end{tikzcd}
        \begin{tikzcd} [sep = 3em]
            & y \ar[dr,"g"]&\\
            x \ar[ur,"f"] \ar[rr,"\id_x"]&  & x
        \end{tikzcd}\;.
    \end{center}
    We say that objects $x,y\in C_0$ are \emph{equivalent} if there exists an equivalence $f\colon x\to y$.
\end{definition}
\begin{definition}[Natural Transformations]
    Let $f,g\colon C\to D$ be maps between \inftycats/.
    Then a \emph{natural transformation from $f$ to $g$} is a map if simplicial sets $h\colon C\times\Delta^1\to D\in\Fun(C,D)_1$ such that $h|_{C\times\set{0}}=f$ and $h|_{C\times\set{1}}=g$.
    
    We say that a natural transformation is \emph{invertible} if for all $c\in C_0$ the map $h|_{\set{c}\times\Delta^1}\in D_1$ is an equivalence.
\end{definition}
\begin{definition}[Essentially Surjective]
    Let $F\colon C\to D$ be a map between \inftycats/.
    Then we call $F$ \emph{essentially surjective} if for all $d\in D_0$ there exists a $c\in C_0$ such that $d$ and $f(c)$ are equivalent in $D$.
\end{definition}
\begin{definition}[Mapping Space]\label{def:mappingSpace}
    Let $C$ be an \inftycat/ and let $x,y\in C_0$ and let $\Map_C(x,y)$ denote the pullback 
    \begin{center}
        \begin{tikzcd} [sep = 3em]
            \Map_C(x,y) \ar[d] \ar[r] & \Fun(\Delta^1,C) \ar[d, "{(\ev_0,\ev_1)}"] \\
            \Delta^0 \ar[r, "{(x,y)}"] & C\times C
        \end{tikzcd}
    \end{center}
    which we refer to as the \emph{mapping space from $x$ to $y$}.
\end{definition}
\begin{remark}\label{rmk:functorialMappingSpace}
    By \cite[116]{cisinski_2019} this space is always a Kan-complex.
    By \cite[248]{cisinski_2019} for every (locally small) \inftycat/ $C$, there is an essentially unique $\hom$-functor $\hom_C(-,-)\colon C^{\op}\times C\to\spaces$, where $\spaces$ is the (yet to be defined) \inftycat/ of spaces, such that for all objects $x,y\in C_0$ we have a canonical equivalence $\hom_C(x,y)\cong\Map_C(x,y)$ in $\spaces$ (the objects $\spaces_0$ can be identified with Kan-complexes, see \cref{cor:htpyCoherentNerveIsLoc}). %TODO maybe say what is essentially unique
\end{remark}
\begin{definition}[Fully Faithful]
    Let $F\colon C\to D$ be a map between \inftycats/.
    Then call $F$ \emph{fully faithful} if for all $x,y\in C_0$ the map $\Map_C(x,y)\to\Map_D(f(x),f(y))$ induced by the commutative cube
    \begin{center}
        \begin{tikzcd} [sep = 1em]
            \Map_C(x,y) \arrow [dr] \arrow [rr] \arrow [dd] & & \Fun(\Delta^1,C) \arrow [dr, "F\circ -"] \arrow [dd] \\
            & \Map_D(f(x),f(y)) \arrow [rr, crossing over] \arrow [dd] & &[1em] \Fun(\Delta^1,D) \arrow [dd] & \\
            \Delta^0 \arrow [dr] \arrow [rr, near start, "{(x,y)}"] & & C\times C \arrow [dr,"{(F,F)}"] \\
            & \Delta^0 \arrow [from=uu, crossing over] \arrow [rr, "{(f(x),f(y))}"] & & D\times D
        \end{tikzcd}
    \end{center}
    is a weak equivalence of Kan-complexes.
\end{definition}
\begin{thm}\label{thm:eqCharEqOfInftycats}
    Let $F\colon C\to D$ be a map between \inftycats/.
    Then the following are equivalent:
    \begin{itemize}
        \item $F$ is a weak categorical equivalence (thus an equivalence of \inftycats/)
        \item $F$ is fully faithful and essentially surjective %TODO not defined yet
        \item There exists a map $G\colon D\to C$ and invertible natural transformations $FG\simeq \id_D$ and $GF\simeq \id_C$.
    \end{itemize}
    \begin{reference}
        \cite[Corollary 3.6.6 and Theorem 3.9.7]{cisinski_2019}
    \end{reference}
\end{thm}
\begin{prop}
    Let $A$ be a simplicial category such that for every pair of objects $X,Y\in A$ the simplicial set $\Hom_A(X,Y)$ is a Kan-complex.
    Then $\Nhc(A)$ is an \inftycat/.
    \begin{reference}
        \cite[Theorem 2.4.5.1]{kerodon}
    \end{reference}
\end{prop}
\begin{corollary}
    Let $M$ be a simplicial model category.
    Then the simplicial set $\Nhc(M^°)$ where $M^°$ denotes the restriction to the full subcategory of cofibrant-fibrant objects is an \inftycat/.
    \begin{proof}
        The axioms of a simplicial model category imply that for every pair of objects $X,Y\in M^°$ the simplicial set $\Hom_{M^°}(X,Y)$ is a Kan-complex.
    \end{proof}
\end{corollary}
\begin{definition}[Simplicial Localization]
    Let $C$ be a simplicial set and $W\subset C$ be a simplicial subset.

    For a given \inftycat/ $D$, let $\Fun_W(C,D)\subset\Fun(C,D)$ be the full subcategory of functors $f\colon C\to D$ such that for every map in $u\in W_1$, $f(u)$ is invertible.
    Then a \emph{localization of $C$ by $W\subset C$} is an \inftycat/ $C[W^{-1}]$ together with a functor $\gamma\in\Fun_W(C, C[W^{-1}])_0$ such that for every \inftycat/ $D$, $\gamma$ induces an equivalence of \inftycats/
    \begin{equation*}
        \Fun(C[W^{-1}],D)\xrightarrow{\sim}\Fun_W(C,D)\;.
    \end{equation*}
\end{definition}
\begin{lemma}\label{prop:simpLocEssSurj}
    Let $C$ be a simplicial set and $W\subset C$ be a simplicial subset.
    Then a simplicial localization exists, is essentially unique and the localization functor $\gamma\colon C\to C[W^{-1}]$ is essentially surjective.

    Furthermore, we can always assume $\gamma$ to be bijective on objects.
    \begin{reference}
        \cite[Proposition 7.1.3 and Remark 7.1.4]{cisinski_2019}
    \end{reference}
\end{lemma}
\begin{definition}[Join]
    For simplicial sets $X,Y\in\sSet$ let $X\star Y$ denote the \emph{join} which on simplices is given by $(X\star Y)_n=X_n\cup Y_n\cup\bigcup\limits_{i+j=n-1}X_i\times X_j$.
    The join is functorial in both arguments.
    
    For a simplicial set $X$, we further define $X^{\lhd}=\Delta^0\star X$ and $X^{\rhd}=X\star\Delta^0$.
\end{definition}
\begin{definition}[(Co)Slice] %TODO Construction 4.3.5.1
    Let $f\colon K\to C$ be a map of simplicial sets.
    Setting 
    \begin{equation*}
        \left(C_{/f}\right)_n=\set{\sigma\colon \Delta^n\star K\to X \mid \sigma|_K=f}
    \end{equation*}
    defines a simplicial set $C_{/f}$ which we refer to as the \emph{slice simplicial set of $C$ over $f$}.

    Dually, setting 
    \begin{equation*}
        \left(C_{f/}\right)_n=\set{\sigma\colon K\star \Delta^n \to X \mid \sigma|_K=f}
    \end{equation*}
    defines a simplicial set $C_{f/}$ which we refer to as the \emph{coslice simplicial set of $C$ over $f$}.
\end{definition}
\begin{construction}[(Co)Slice Contraction]%TODO kerodon 4.3.5.12.
    Let $f\colon X\to Y$ be a morphism of simplicial sets.
    Then let $c\colon Y_{/f}\star X\to Y$ be defined as follows:
    \begin{itemize}
        \item the restriction to $Y_{/f}$ is the slice projection $Y_{/f}\to Y$
        \item the restriction to $X$ is $f$
        \item a simplex $\sigma\colon\Delta^n\to Y_{/f}\star X$ not covered by the above cases factors uniquely as $\sigma=\sigma^+\star\sigma^-$ where $\sigma^+\colon\colon\Delta^p\to Y_{/f}$ and $\sigma^-\colon\Delta^q\to X$.
              We can thus identify $\sigma^+\colon\Delta^p\to Y_{/f}$ with a morphism $\overline{f}\colon\Delta^p\star X\to Y$ such that $\overline{f}|_{X}=f$.
              We then set $c(\sigma)=\Delta^n\cong\Delta^p\star\Delta^q\xrightarrow{\id_{\Delta^p}\star\sigma^-}\Delta^p\star X\xrightarrow{\overline{f}} Y$.
    \end{itemize}
    We call $c$ the \emph{slice contraction morphism}.
    The analogous construction $c\colon X\star Y_{f/}\to Y$ is called the \emph{coslice contraction morphism}.
\end{construction}
\begin{definition}[Adjoints]
    Let $F\colon C\to D$ and $G\colon D\to C$ be functors between \inftycats/.
    Then we say that \emph{$F$ is left adjoint to $G$} or equivalently, that \emph{$G$ is right adjoint to $F$}
    if there exist natural transformations $\varepsilon\colon FG\to \id_D$ and $\eta\colon\id_C\to GF$ (called \emph{counit} and \emph{unit} respectively) and $2$-simplices 
    \begin{center}
        \begin{tikzcd} [sep = 3em]
            & FGF \ar[dr,"\varepsilon F"]&\\
            F \ar[ur,"F\eta"] \ar[rr,"\id_F"]&  & F
        \end{tikzcd}
        \begin{tikzcd} [sep = 3em]
            & GFG \ar[dr,"G\varepsilon"]&\\
            G \ar[ur,"\eta G"] \ar[rr,"\id_G"]&  & G
        \end{tikzcd}
    \end{center}
    in $\Fun(C,D)$ and $\Fun(D,C)$ respectively.
\end{definition}
\begin{remark}\label{rmk:homAdjunction}
    By \cite[Theorem 6.1.23]{cisinski_2019} there is another equivalent characterization of adjoints via the functorial version of the mapping space functor (see \cref{rmk:functorialMappingSpace}):
    
    A functor $F\colon C\to D$ is left adjoint to a functor $G\colon D\to C$ if and only if there exists an invertible natural transformation $c\colon (C^{\op}\times D)\times\Delta^1 \to\spaces$ (where $\spaces$ is the yet to be defined \inftycat/ of spaces) between the functors $\hom_D(F(-),-)\cong\hom_C(-,G(-))$.

    Adjoints are also ``essentially unique'' in the sense of \cite[Proposition 6.1.9]{cisinski_2019}.
\end{remark}
\begin{definition}[Kan-Extension]
    Let $u\colon A\to B$ be a functor between \inftycats/ and let $C$ be an \inftycat/.
    
    Then if the functor $u^*\colon\Fun(B,C)\to\Fun(A,C)$ has a right adjoint $u_*\colon\Fun(A,C)\to\Fun(B,C)$, we say that \emph{$u_*$ is the right Kan-extension of $u$ in $C$}.
    For a map $f\in\Fun(A,C)_0$, we say that a map $g\in\Fun(B,C)_0$ is a \emph{right Kan extension of $f$ along $u$} if $g$ is equivalent to $u_*(f)$.

    Dually, if $u^*$ has a left adjoint $u_!\colon\Fun(A,C)\to\Fun(B,C)$, we say that \emph{$u_!$ is the left Kan-extension of $u$ in $C$}.
    For a map $f\in\Fun(A,C)_0$, we say that a map $g\in\Fun(B,C)_0$ is a \emph{left Kan extension of $f$ along $u$} if $g$ is equivalent to $u_!(f)$.
\end{definition}
We will only deal with Kan-extensions where $u$ is an inclusion and $C$ is complete and cocomplete.
In this case, we can give a more explicit description.
\begin{definition}[Pointwise Kan-Extension] %kerodon Definition 7.3.2.1
    Let $F\colon C\to D$ be a functor between \inftycats/ and let $i\colon C_0\subset C$ be a full subcategory.
    Then $F$ is a \emph{pointwise right Kan-extension along $i$} if for all objects $c\in C$ the diagram
    \begin{equation*}
        \left(C_{c/}\times_C C_0\right)^{\lhd}\to \Delta^0\star C_{c/}\xrightarrow{c} C\xrightarrow{F} D
    \end{equation*}
    where the map $c\colon\Delta^0\star C_{c/}\to C$ is the coslice contraction map, is a limit cone.

    Dually, $F$ is a \emph{pointwise left Kan-extension along $i$} if for all objects $c\in C$ the diagram
    \begin{equation*}
        \left(C_{/c}\times_C C_0\right)^{\rhd}\to C_{/c}\star\Delta^0 \xrightarrow{c} C\xrightarrow{F} D
    \end{equation*}
    where the map $c\colon C_{/c}\star\Delta^0\to C$ is the slice contraction map, is a colimit cone.
\end{definition}
\begin{prop}[Existence of Pointwise Kan-Extensions]\label{prop:exKanExt}
    Let $C,D$ be \inftycats/ and $i\colon C_0\subset C$ a full subcategory.
    
    If $D$ is complete, then there exists a right Kan-extension $i_*\colon\Fun(C_0,D)\to\Fun(C,D)$ of $i$ such that for every map $f\colon C_0\to D$ the induced map $i_*(f)$ is equivalent to a pointwise right Kan-extension along $i$.

    Dually if $D$ is cocomplete, then there exists a left Kan-extension $i_!\colon\Fun(C_0,D)\to\Fun(C,D)$ of $i$ such that for every map $f\colon C_0\to D$ the induced map $i_!(f)$ is equivalent to a pointwise left Kan-extension along $i$.

    Furthermore, the functors $i_*$ and $i_!$ are fully faithful.
    \begin{reference}
        The existence and computation of the Kan-extensions via the pointwise description is shown by \cite[Proposition 6.4.9]{cisinski_2019} (and its dual).
        
        For the full faithfulness, it is implied by \cite[Corollary 7.3.1.16]{kerodon} that counit/unit respectively of the adjunction is an equivalence.
        One can then use \cref{rmk:homAdjunction} to prove that this implies full faithfulness by essentially the same argument as in ordinary category theory.
    \end{reference}
\end{prop}
