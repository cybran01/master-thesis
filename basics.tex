In this section we would like to fix some of the notation used througout the rest of the text and remind the reader of some concepts which we will be important later on.
We will not give proofs for any statements; however we will point to the relevant literature in case one is unfamiliar with them.
\begin{definition}[Simplex Category]
    Let $\Delta$ be the category with
    \begin{itemize}
        \item objects the categories $[n]=\{0\to1\to\ldots\to n\}\subset\Nat$ for all $n\in\Nat$
        \item morphisms functors between these categories (which are exactly monotone increasing functions).
    \end{itemize}
    We call this category the \emph{simplex category $\Delta$}.
\end{definition}
\begin{definition}[Augmented Simplex Category]
    Let $\Delta_+$ be the category with
    \begin{itemize}
        \item objects the categories $[n]$ for all $n\in\Nat$ and $[-1]=\emptyset$
        \item morphisms functors between these categories (which again are exactly the monotone increasing functions).
    \end{itemize}
    We call this category the \emph{augmented simplex category $\Delta_+$}.
\end{definition}
\begin{definition}[Nerve of an Ordinary Category]
    For a category $C$, we denote the composition $\Hom_{\Cat}(-,C)\circ i$ as $\Hom_{\Delta}(-,C)$, where $i:\Delta\xhookrightarrow{}\Cat$ is the inclusion as a full subcategory.
    Let $\N\colon\Cat\to\sSet$ be the functor that sends
    \begin{itemize}
        \item a category $C$ to the simplicial set $\Hom_{\Delta}(-,C)$
        \item a functor $F:C\to D$ to $\N(F)\colon\Hom_{\Delta}(-,C)\xrightarrow{F\circ -}\Hom_{\Delta}(-,D)$\;.
    \end{itemize}
    We then call $\N(C)$ the \emph{nerve of the category $C$}.
\end{definition}
\begin{lemma} %Kerodon Proposition 1.2.2.1
    The nerve functor $\N\colon\Cat\to\sSet$ is fully faithful and has a left adjoint $\ho\colon\sSet\to\Cat$ which we call the \emph{homotopy category} of a simplicial set.
\end{lemma}
\begin{definition}
    We let $\Cat_{\Delta}$ denote the category of small categories enriched over $\sSet$.
    We call a category enriched over $\sSet$ a \emph{simplicial category}.
\end{definition}
\begin{definition}[Simplicial Model Category]
    A \emph{simplicial model category} is a simplicial category $C$ that is equipped with a model structure such that for a cofibration $i:A\rightarrowtail B$ and a fibration $p:X\twoheadrightarrow Y$ the induced map of simplicial sets
    \begin{equation*}
        \Hom_C(B,X)\to\Hom_C(A,X)\times_{\Hom_C(A,Y)}\Hom_C(B,Y)
    \end{equation*}
    is a fibration that additionally is a weak equivalence whenever $i$ or $p$ is a weak equivalence.
\end{definition}
\begin{construction}
    For $[n]\in\Delta$ we define a simplicial category $\C[n]$ as follows:
    \begin{itemize}
        \item the objects of $\C[n]$ are the elements of the set $\set{0,\ldots,n}$
        \item for $i,j\in\set{0,\ldots,n}$, the simplicial $\Hom$-objects are given by 
            \begin{equation*}
                \Hom_{\C[n]}(i,j)=\N(P_{i,j})
            \end{equation*}
            where $P_{i,j}$ is the partially ordered set of subsets of $\set{i,\ldots,j}$ which contain both $i$ and $j$ with the relation given by reverse inclusion
        \item for $i,j,k\in\set{0,\ldots,n}$ the composition is given by $\N(-)$ applied to the map
        \begin{align*}
                P_{i,j}\times P_{j,k}&\to P_{i,k}\\
                (I,J)&\mapsto I\cup J\;.
        \end{align*}
    \end{itemize}
    For a morphism $[n]\to[m]\in\Delta$ we can define a functor
    \begin{equation*}
        \C[f]\colon\C[n]\to C[m]
    \end{equation*}
    which
    \begin{itemize}
        \item sends an object $i\in\C[n]$ to $f(i)\in\C[m]$
        \item for $i,j\in\C[n]$ induces a map between $\Hom$-objects by applying $\N(-)$ to the map
            \begin{align*}
                P_{i,j}&\to P_{f(i),f(j)}\\
                I&\mapsto f(I)
            \end{align*}
    \end{itemize}
    so we obtain a functor $\C[-]\colon\Delta\to\Cat_{\Delta}$.
\end{construction}
\begin{definition}[Homotopy Coherent Nerve] %HTT Definition 1.1.5.1. 
    For a simplicial category $A$ we define the \emph{simplicial nerve (or homotopy coherent nerve) $\Nhc(A)$ of $A$} as $\Nhc(A)=\Hom_{\Cat_{\Delta}}(\C[-],A)$ and thus we obtain a functor $\Nhc\colon\Cat_{\Delta}\to\sSet$.
\end{definition}
We will also need the standard model structure on $\sSet$, the Quillen model structure. 
Whenever we speak of cofibrations, fibrations or weak equivalences between simplicial sets, we are referring to this model structure.
\begin{prop}[Quillen Model Structure on $\sSet$]
    The following three classes of maps endow $\sSet$ with a model structure:
    \begin{itemize}
        \item Cofibrations are the monomorphisms.
        \item Weak equivalences are the maps that are weak equivalences under geometric realization. %TODO define geometric realization
        \item Fibrations are the maps with right lifting property against the maps $\set{\Lambda_k^n\xhookrightarrow{}\Delta^n\mid 0\leq k\leq n}$.
    \end{itemize}
    We call this model structure the \emph{Quillen Model Structure on $\sSet$}.
\end{prop}
For completess sake we will also describe the Joyal model structure on $\sSet$, as it plays a similar role for \inftycats/ as the Quillen model structre does for Kan-complexes.
\begin{prop}[Quillen Model Structure on $\sSet$]
    The following three classes of maps endow $\sSet$ with a model structure:
    \begin{itemize}
        \item Cofibrations are the monomorphisms.
        \item The trivial cofibrations are generated by the class $J=\set{\Lambda_k^n\hookrightarrow\Delta^n\mid 0< k< n}$.
    \end{itemize}
    The model structure is already uniquely determined by these choices.
    We call this model structure the \emph{Joyal Model Structure on $\sSet$}.
\end{prop}
\begin{remark}
    When we say that a class of maps is generated by another class $I$, we mean that it is the smallest saturated class of morphisms containing $I$.
    Being saturated means that it is closed under
    \begin{itemize}
        \item transfinite composition
        \item pushouts
        \item and retracts.
    \end{itemize}
\end{remark}
\begin{prop} %TODO CatLR Theorem 3.9.7. Corollary 3.6.6.
    Let $F\colon C\to D$ be a map between \inftycats/.
    Then the following are equivalent:
    \begin{itemize}
        \item $F$ is a weak equivalence in the Joyal model structre
        \item $F$ is fully faithful and essentially surjective
        \item There exists a map $G\colon D\to C$ and invertible natural transformations $FG\simeq \id_D$ and $GF\simeq \id_C$.
    \end{itemize}
\end{prop}
\begin{prop} %HTT Theorem 2.4.5.1.
    Let $A$ be a simplicial category such that for every pair of objects $X,Y\in A$ the simplicial set $\Hom_A(X,Y)$ is a Kan-complex.
    Then $\Nhc(A)$ is an \inftycat/.
\end{prop}
\begin{corollary}
    Let $M$ be a simplicial model category.
    Then the simplicial set $\Nhc(M^°)$ where $M^°$ denotes the restriction to cofibrant-fibrant objects is an \inftycat/.
    \begin{proof}
        The axioms of a simplicial model category imply that for every pair of objects $X,Y\in M^°$ the simplicial set $\Hom_{M^°}(X,Y)$ is a Kan-complex.
    \end{proof}
\end{corollary}
\begin{definition}[Simplicial Localization] %TODO define localization CatLR Definition 7.1.2. and add existence
    Let $C$ be a simplicial set and $W\subset C$ be a simplicial subset.

    For a given \inftycat/ $D$, let $\Fun_W(C,D)\subset\Fun(C,D)$ be the full subcategory of functors $f\colon C\to D$ such that for every map in $u\in W_1$, $f(u)$ is invertible.
    Then a \emph{localization of $C$ by $W\subset C$} is an \inftycat/ $C[W^{-1}]$ together with a functor $\gamma\in\Fun_W(C, C[W^{-1}])_0$ such that for every \inftycat/ $D$, $\gamma$ induces an equivalence of \inftycats/
    \begin{equation*}
        \Fun(C[W^{-1}],D)\xrightarrow{\sim}\Fun_W(C,D)\;.
    \end{equation*}
\end{definition}
\begin{lemma}
    Let $C$ be a simplicial set and $W\subset C$ be a simplicial subset.
    The any localization functor $\gamma\colon C\to C[W^{-1}]$ is essentially surjective.
    \begin{proof}
        A simplicial localization can be constructed in such a way that it is bijective on the objects, thus in particular essentially surjective. %TODO cite CatLR Remark 7.1.4. 
        By essential uniqueness we thus have that all localization functors are essentially surjective.

        One can also prove this from ordinary category theory, since $\ho(C)\to \ho(C)[W_1^{-1}]\cong\ho(C[W^{-1}])$ is a localization and thus essentially surjective. %TODO CatLR Remark 7.1.6
    \end{proof}
\end{lemma}
\begin{prop} %TODO CatLR Proposition 7.1.3
    Let $C$ be a simplicial set and $W\subset C$ be a simplicial subset.
    Then a localization $\gamma\colon C\to C[W^{-1}]$ exists and is essentially unique.
\end{prop}
%TODO define (co)limits and prove/state that they are computed pointwise in Functorcats HTT Proposition 5.1.2.2., HTT Corollary 5.1.2.3.
\begin{construction}[(Co)Slice Contraction]%TODO kerodon 4.3.5.12.
    Let $f\colon X\to Y$ be a morphism of simplicial sets.
    Then let $c\colon Y_{/f}\star X\to Y$ be defined as follows:
    \begin{itemize}
        \item the restriction to $Y_{/f}$ is the slice projection $Y_{/f}\to Y$
        \item the restriction to $X$ is $f$
        \item a simplex $\sigma\colon\Delta^n\to Y_{/f}\star X$ not covered by the above cases factors uniquely as $\sigma=\sigma^+\star\sigma^-$ where $\sigma^+\colon\colon\Delta^p\to Y_{/f}$ and $\sigma^-\colon\Delta^q\to X$.
              We can thus identify $\sigma^+\colon\Delta^p\to Y_{/f}$ with a morphism $\overline{f}\colon\Delta^p\star X\to Y$ such that $\overline{f}|_{X}=f$.
              We then set $c(\sigma)=\Delta^n\cong\Delta^p\star\Delta^q\xrightarrow{\id_{\Delta^p}\star\sigma^-}\Delta^p\star X\xrightarrow{\overline{f}} Y$.
    \end{itemize}
    we call $c$ the \emph{slice contraction morphism}.
    The analogous construction $c\colon c\colon X\star Y_{f/}\to Y$ is called the \emph{coslice contraction morphism}.
\end{construction}
\begin{definition} %kerodon Definition 7.3.2.1
    Let $F\colon C\to D$ be a functor between \inftycats/ and let $C_0\subset C$ be a full subcategory.
    Then $F$ is \emph{right Kan-extended from $C_0$} if for all objects $c\in C$ the diagram
    \begin{equation*}
        \left(C_{c/}\times_C C_0\right)^{\lhd}\to \Delta^0\star C_{c/}\xrightarrow{c} C\to D
    \end{equation*}
    where the map $c\colon\Delta^0\star C_{c/}\to C$ is the coslice contraction map, is a limit cone.

    Dually, $F$ is \emph{left Kan-extended from $C_0$} if for all objects $c\in C$ the diagram
    \begin{equation*}
        \left(C_{/c}\times_C C_0\right)^{\rhd}\to C_{/c}\star\Delta^0 \xrightarrow{c} C\to D
    \end{equation*}
    where the map $c\colon C_{/c}\star\Delta^0\to C$ is the slice contraction map, is a colimit cone.
\end{definition}
\begin{remark}
    There is also a notion of Kan-extensions along general functors. 
    We will not need this, but it can be found e.g. %TODO Kerodon 7.3.1
    .
\end{remark}
%TODO Kan-ext induce adjoints
\begin{prop} %TODO kerodon Proposition 7.3.5.1
    Let $C$ be an \inftycat/, $C_0\subset C$ a full subcategory and $F\colon C_0\to D$ a functor between \inftycats/.
    If $D$ is complete, then there exists a functor $F\colon C\to D$ that is right Kan-extended from $C_0$.

    Dually if $C$ is cocomplete, then then there exists a functor $F\colon C\to D$ that is left Kan-extended from $C_0$.
    Furthermore, both Kan-extensions are fully faithful.
    \begin{proof}
        %TODO reference fully faithfulness, TODO kerodon Proposition 7.3.5.1. for existence of Kan ext
    \end{proof}
\end{prop}
