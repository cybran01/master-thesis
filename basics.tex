In this section we would like to fix the general notation on \inftycats/ and model categories used throughout the rest of the text and remind the reader of some concepts which we will be important later on.
Some familiarity with \inftycats/ and Kan-complexes is assumed, in particular familiarity with limits/colimits in \inftycats/.
We also assume familiarity with (proper) model categories and the computation of homotopy limits/colimits (in particular homotopy pullbacks/pushouts and homotopy products/coproducts).
We will not give proofs for any statements in this section; we will however point to the relevant literature.
\begin{definition}[Simplex Category]
    Let $\Delta$ be the category with
    \begin{itemize}
        \item objects the categories $[n]=\{0\to1\to\ldots\to n\}\subset\Nat$ for all $n\in\Nat$
        \item morphisms functors between these categories (which are precisely the monotone increasing functions).
    \end{itemize}
    We call this category the \emph{simplex category $\Delta$}.
\end{definition}
\begin{definition}[Augmented Simplex Category]
    Let $\Delta_+$ be the category with
    \begin{itemize}
        \item objects the categories $[n]$ for all $n\in\Nat\cup\set{-1}$ where $[-1]=\emptyset$ is the empty category
        \item morphisms functors between these categories (which again are precisely the monotone increasing functions).
    \end{itemize}
    We call this category the \emph{augmented simplex category $\Delta_+$}.
\end{definition}
\begin{definition}[Simplicial Set]
    We set $\sSet=\Fun(\Delta^{\op},\Set)$ to be the \emph{category of simplicial sets}.
\end{definition}
\begin{definition}[Nerve of an Ordinary Category]
    Let $\Cat$ denote the category of small categories and let $i:\Delta\xhookrightarrow{}\Cat$ be the inclusion as a full subcategory.
    For a small category $C$, we set composition $\N(C)=\hom_{\Cat}(-,C)\circ i^{\op}$.
    Since this construction is functorial in $C$, we get a functor $\N\colon\Cat\to\sSet$. 
    We then call $\N(C)$ the \emph{nerve of the category $C$}.
\end{definition}
\begin{lemma}[Homotopy Category]
    The nerve functor $\N\colon\Cat\to\sSet$ is fully faithful and has a left adjoint $\ho\colon\sSet\to\Cat$ which we call the \emph{homotopy category} of a simplicial set.
    \begin{proof}[Reference]
        \cite[Proposition 1.2.2.1]{kerodon}
    \end{proof}
\end{lemma}
\begin{definition}[Simplicial Category]
    We call a category enriched over $\sSet$ a \emph{simplicial category}.
    Let $\Cat_{\Delta}$ denote the \emph{category of small categories enriched over $\sSet$} with morphisms $\sSet$-enriched functors.
    
    We call a simplicial category $C$ \emph{tensored} if for each $X\in C$ and $K\in\sSet$ there is an object $X\otimes K$ such that there is a natural isomorphism
    \begin{equation*}
        \Hom_C(X\otimes K,-)\cong\Hom_{\sSet}(K,\Hom_C(X,-))\;.
    \end{equation*}

    We call a simplicial category $C$ \emph{powered} if for each $Y\in C$ and $K\in\sSet$ there is an object $Y^K$ such that there is a natural isomorphism
    \begin{equation*}
        \Hom_C(-,Y^K)\cong\Hom_{\sSet}(K,\Hom_C(-,Y))\;.
    \end{equation*}
    Note that due to the functoriality of the right side, these extend to functors $-\otimes -:C\times\sSet\to C$ and $-^-:C\times\sSet^{\op}\to C$.
\end{definition}
%TODO perhaps define RLP/LLP
\begin{definition}[Weak Factorization System]
    Let $C$ be a category and let $\Map(C)$ denote the class of all maps of $C$.
    Let $L,R\subset\Map(C)$ be subclasses.
    If 
    \begin{itemize}
        \item every morphism $f$ in $C$ can be factorized as $f=p\circ i$ with $i\in L$, $p\in R$
        \item $L$ consists precisely of the morphisms having the left lifting property against $R$
        \item $R$ consists precisely of the morphisms having the right lifting property against $L$
    \end{itemize}
    then the pair $(L,R)$ is a \emph{weak factorization system on $C$}.
\end{definition}
\begin{definition}[Model Category]
    Let $C$ be a complete and cocomplete category and let $\Map(C)$ denote the class of all maps of $C$.
    Let $\Cof,\Fib,\W\subset\Map(C)$ be subclasses which we will call \emph{cofibrations}, \emph{fibrations} and \emph{weak equivalences} respectively.
    Then we call $C$ together with this data a \emph{model category} if the following hold:
    \begin{itemize}
        \item $\W$ contains all isomorphisms and has the two-out-of-three property.
        \item $(\Cof\cap\W,\Fib)$ is a weak factorization system.
        \item $(\Cof,\Fib\cap\W)$ is a weak factorization system.
    \end{itemize}
    We say that $C$ together with $(\Cof,\Fib,\W)$ is a \emph{combinatorial model structure} if $C$ is locally presentable and the model structure is cofibrantly generated.
\end{definition}
We will need the standard model structure on $\sSet$, the Quillen model structure. 
Whenever we speak of cofibrations, fibrations or weak equivalences between simplicial sets, we are referring to this model structure.
\begin{prop}[Quillen Model Structure on $\sSet$]
    There is a unique model structure on $\sSet$ where 
    \begin{itemize}
        \item the cofibrations are the maps having the left lifting property against the maps $\set{\partial\Delta^n\xhookrightarrow{}\Delta^n\mid n\in\Nat}$ which are exactly the monomorphisms
        \item the fibrations are the maps having the right lifting property against the maps $\set{\Lambda_k^n\xhookrightarrow{}\Delta^n\mid 0\leq k\leq n, n\in\Nat}$.
    \end{itemize}
    The fibrant objects are the \emph{Kan-complexes}.
    The weak equivalences are the maps which are weak equivalences in $\Top$ after geometric realization.

    We call this model structure the \emph{Quillen Model Structure on $\sSet$}.
    \begin{reference}
        \cite[Theorem 3.1.8 and Theorem 3.1.29]{cisinski_2019} and \cite[Chap. II, \S 3, Theorem 1]{Quillen1967}
    \end{reference}
\end{prop}
\begin{definition}[Simplicial Model Category]
    A \emph{simplicial model category} is a simplicial category $C$ that is 
    \begin{itemize}
        \item tensored and powered
        \item equipped with a model structure
    \end{itemize} 
    such that for a cofibration $i:A\rightarrowtail B$ and a fibration $p:X\twoheadrightarrow Y$ in the model structure the induced map of simplicial sets
    \begin{equation*}
        \Hom_C(B,X)\to\Hom_C(A,X)\times_{\Hom_C(A,Y)}\Hom_C(B,Y)
    \end{equation*}
    is a fibration that additionally is a weak equivalence whenever $i$ or $p$ is a weak equivalence.
    
    We say that a simplicial model category is a \emph{combinatorial simplicial model category} if its underlying model category is a combinatorial model category.
\end{definition}
\begin{construction}%[{\cite[Notation 2.4.3.1]{kerodon}}] TODO decide
    For $[n]\in\Delta$ we define a simplicial category $\C[n]$ as follows:
    \begin{itemize}
        \item the objects of $\C[n]$ are the elements of the set $\set{0,\ldots,n}$
        \item for $i,j\in\set{0,\ldots,n}$, the simplicial $\Hom$-objects are given by 
            \begin{equation*}
                \Hom_{\C[n]}(i,j)=\N(P_{i,j})
            \end{equation*}
            where $P_{i,j}$ is the partially ordered set of subsets of $\set{0,\ldots,n}$ which contain both $i$ and $j$ with the relation given by reverse inclusion
        \item for $i,j,k\in\set{0,\ldots,n}$ the composition is given by applying the nerve $\N(-)$ to the map
        \begin{align*}
                P_{i,j}\times P_{j,k}&\to P_{i,k}\\
                (I,J)&\mapsto I\cup J\;.
        \end{align*}
    \end{itemize}
    Given a morphism $[n]\to[m]\in\Delta$ we define
    \begin{equation*}
        \C[f]\colon\C[n]\to C[m]
    \end{equation*}
    which
    \begin{itemize}
        \item sends an object $i\in\C[n]$ to $f(i)\in\C[m]$
        \item for $i,j\in\C[n]$ induces a map between $\Hom$-objects by applying $\N(-)$ to the map
            \begin{align*}
                P_{i,j}&\to P_{f(i),f(j)}\\
                I&\mapsto f(I)
            \end{align*}
    \end{itemize}
    so we obtain a functor $\C[-]\colon\Delta\to\Cat_{\Delta}$.
\end{construction}
\begin{definition}[Homotopy Coherent Nerve]
    For a simplicial category $A$ we define the \emph{simplicial nerve (or homotopy coherent nerve) $\Nhc(A)$ of $A$} as $\Nhc(A)=\Hom_{\Cat_{\Delta}}(\C[-],A)$; 
    Thus we obtain a functor $\Nhc\colon\Cat_{\Delta}\to\sSet$.
\end{definition}
For completeness sake we will also describe the Joyal model structure on $\sSet$, as it plays a similar role for \inftycats/ as the Quillen model structure does for Kan-complexes.
\begin{prop}[Joyal Model Structure on $\sSet$]
    The following two classes of maps endow $\sSet$ with a unique model structure:
    \begin{itemize}
        \item Cofibrations are the monomorphisms.
        \item The weak equivalences are the maps $A\to B$ such that for every \inftycat/ $C$ the induced map $\ho(\Fun(B,C))\to\ho(\Fun(A,C))$ is a categorical equivalence.
    \end{itemize}
    The fibrant objects of this model structure are the \inftycats/.
    We refer to the weak equivalences of this model structure as \emph{weak categorical equivalences}.
    We refer to weak categorical equivalences whose domain and target are both \inftycats/ as \emph{equivalences of \inftycats/}.
    We call this model structure the \emph{Joyal Model Structure on $\sSet$}.
    \begin{reference}
        \cite[Definition 3.3.7 and Theorem 3.6.8]{cisinski_2019}
    \end{reference}
\end{prop}
\begin{prop}
    Let $F\colon C\to D$ be a map between \inftycats/.
    Then the following are equivalent:
    \begin{itemize}
        \item $F$ is a weak categorical equivalence (thus an equivalence of \inftycats/)
        \item $F$ is fully faithful and essentially surjective
        \item There exists a map $G\colon D\to C$ and invertible natural transformations $FG\simeq \id_D$ and $GF\simeq \id_C$.
    \end{itemize}
    \begin{reference}
        \cite[Corollary 3.6.6 and Theorem 3.9.7]{cisinski_2019}
    \end{reference}
\end{prop}
\begin{prop}
    Let $A$ be a simplicial category such that for every pair of objects $X,Y\in A$ the simplicial set $\Hom_A(X,Y)$ is a Kan-complex.
    Then $\Nhc(A)$ is an \inftycat/.
    \begin{reference}
        \cite[Theorem 2.4.5.1]{kerodon}
    \end{reference}
\end{prop}
\begin{corollary}
    Let $M$ be a simplicial model category.
    Then the simplicial set $\Nhc(M^°)$ where $M^°$ denotes the restriction to cofibrant-fibrant objects is an \inftycat/.
    \begin{proof}
        The axioms of a simplicial model category imply that for every pair of objects $X,Y\in M^°$ the simplicial set $\Hom_{M^°}(X,Y)$ is a Kan-complex.
    \end{proof}
\end{corollary}
\begin{definition}[Simplicial Localization]
    Let $C$ be a simplicial set and $W\subset C$ be a simplicial subset.

    For a given \inftycat/ $D$, let $\Fun_W(C,D)\subset\Fun(C,D)$ be the full subcategory of functors $f\colon C\to D$ such that for every map in $u\in W_1$, $f(u)$ is invertible.
    Then a \emph{localization of $C$ by $W\subset C$} is an \inftycat/ $C[W^{-1}]$ together with a functor $\gamma\in\Fun_W(C, C[W^{-1}])_0$ such that for every \inftycat/ $D$, $\gamma$ induces an equivalence of \inftycats/
    \begin{equation*}
        \Fun(C[W^{-1}],D)\xrightarrow{\sim}\Fun_W(C,D)\;.
    \end{equation*}
\end{definition}
\begin{lemma}\label{prop:simpLocEssSurj}
    Let $C$ be a simplicial set and $W\subset C$ be a simplicial subset.
    Then a simplicial localization exists, is essentially unique and the localization functor $\gamma\colon C\to C[W^{-1}]$ is essentially surjective.

    Furthermore, we can always assume $\gamma$ to be bijective on objects.
    \begin{reference}
        \cite[Proposition 7.1.3 and Remark 7.1.4]{cisinski_2019}
    \end{reference}
\end{lemma}
\begin{definition}[Join]
    For simplicial sets $X,Y\in\sSet$ let $X\star Y$ denote the \emph{join} which on simplices is given by $(X\star Y)_n=X_n\cup Y_n\cup\bigcup\limits_{i+j=n-1}X_i\times X_j$.
    The join is functorial in both arguments.
    
    For a simplicial set $X$, we further define $X^{\lhd}=\Delta^0\star X$ and $X^{\rhd}=X\star\Delta^0$.
\end{definition}
\begin{construction}[(Co)Slice Contraction]%TODO kerodon 4.3.5.12.
    Let $f\colon X\to Y$ be a morphism of simplicial sets.
    Then let $c\colon Y_{/f}\star X\to Y$ be defined as follows:
    \begin{itemize}
        \item the restriction to $Y_{/f}$ is the slice projection $Y_{/f}\to Y$
        \item the restriction to $X$ is $f$
        \item a simplex $\sigma\colon\Delta^n\to Y_{/f}\star X$ not covered by the above cases factors uniquely as $\sigma=\sigma^+\star\sigma^-$ where $\sigma^+\colon\colon\Delta^p\to Y_{/f}$ and $\sigma^-\colon\Delta^q\to X$.
              We can thus identify $\sigma^+\colon\Delta^p\to Y_{/f}$ with a morphism $\overline{f}\colon\Delta^p\star X\to Y$ such that $\overline{f}|_{X}=f$.
              We then set $c(\sigma)=\Delta^n\cong\Delta^p\star\Delta^q\xrightarrow{\id_{\Delta^p}\star\sigma^-}\Delta^p\star X\xrightarrow{\overline{f}} Y$.
    \end{itemize}
    We call $c$ the \emph{slice contraction morphism}.
    The analogous construction $c\colon c\colon X\star Y_{f/}\to Y$ is called the \emph{coslice contraction morphism}.
\end{construction}
\begin{definition} %kerodon Definition 7.3.2.1
    Let $F\colon C\to D$ be a functor between \inftycats/ and let $C_0\subset C$ be a full subcategory.
    Then $F$ is \emph{right Kan-extended from $C_0$} if for all objects $c\in C$ the diagram
    \begin{equation*}
        \left(C_{c/}\times_C C_0\right)^{\lhd}\to \Delta^0\star C_{c/}\xrightarrow{c} C\to D
    \end{equation*}
    where the map $c\colon\Delta^0\star C_{c/}\to C$ is the coslice contraction map, is a limit cone.

    Dually, $F$ is \emph{left Kan-extended from $C_0$} if for all objects $c\in C$ the diagram
    \begin{equation*}
        \left(C_{/c}\times_C C_0\right)^{\rhd}\to C_{/c}\star\Delta^0 \xrightarrow{c} C\to D
    \end{equation*}
    where the map $c\colon C_{/c}\star\Delta^0\to C$ is the slice contraction map, is a colimit cone.
\end{definition}
\begin{remark}
    There is also a notion of Kan-extensions along general functors. 
    We will not need this, but it can be found in \cite[\S 7.3.1]{kerodon}.
\end{remark}
\begin{prop}\label{prop:exKanExt}
    Let $C$ be an \inftycat/, $C_0\subset C$ a full subcategory and $F\colon C_0\to D$ a functor between \inftycats/.
    If $D$ is complete, then there exists a functor $F\colon C\to D$ that is right Kan-extended from $C_0$.

    Dually if $C$ is cocomplete, then there exists a functor $F\colon C\to D$ that is left Kan-extended from $C_0$.
    
    Furthermore, in both cases there is a fully faithful functor $\Fun(C_0,D)\to\Fun(C,D)$ that is right/left adjoint to the restriction functor and given by computing right/left Kan-extensions respectively.
    \begin{reference}
        The existence and adjointness for left Kan-extensions is a corollary of \cite[Corollary 7.3.6.4]{kerodon} (and the dual of this proves the statement for right Kan-extensions).
        The full faithfulness is implied by \cite[Corollary 7.3.1.16]{kerodon} since it is shown that counit/unit of the adjunction is an equivalence.
    \end{reference}
\end{prop}
