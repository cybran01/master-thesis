In the following we will describe three model structures that will be used throughout: 
The Quillen model structure on $\sSet$ (which was already introduced in the previous section), the Quillen model structure on $\Top$ and the \Strom/ model structure on $\Top$.
The first two will turn out to be Quillen equivalent and will be used to define the \inftycat/ of spaces $\spaces$.

The \Strom/ model structure will be important later on; its interaction with the Quillen model structure on $\Top$ will be an important part of our proof that $\spaces$ is an \inftytop/.
\begin{prop}\label{prop:sSetCombSimpModelStructure}
    The Quillen model structure on $\sSet$ together with its natural enrichment over itself gives it the structure of a combinatorial simplicial model category.
    \begin{proof}
        First we note that the category $\sSet$ together with the Quillen model structure is a combinatorial model category (see e.g. \cite[Remark 7.11.15]{cisinski_2019}).
        
        The category $\sSet$ is enriched over itself by $\Fun(A,B)_n=\hom_{\sSet}(A\times\Delta^n,B)$ so it is a simplicial category.
        It is further tensored (via the product) and powered (via $\Fun(-,-)$).
        By \cite[Corollary 1.4.5.6, Theorem 3.1.3.1 and Theorem 3.1.3.5]{kerodon} we have that for a cofibration $A\xtailrightarrow{} B$ and a fibration $X\xtwoheadrightarrow{} Y$ the induced map 
        \begin{equation*}
            \Fun(B,X)\to\Fun(A,X)\times_{\Fun(A,Y)}\Fun(B,Y)
        \end{equation*}
        is a fibration that additionally is a weak equivalence whenever $A\xtailrightarrow{} B$ or $X\xtwoheadrightarrow{} Y$ is a weak equivalence, so $\sSet$ is a simplicial model category.
    \end{proof}
\end{prop}
\begin{definition}[Serre Fibration]
    A map $f\colon X\to Y$ is called a \emph{Serre fibration} if has the right lifting property against the maps $\set{D^n\times\set{0}\xhookrightarrow{}D^n\times I\mid n\in\Nat}$.
\end{definition}
\begin{definition}[Hurewicz Fibration]
    A map $f\colon X\to Y$ is called a \emph{Hurewicz fibration} if has the right lifting property against the maps $\set{X\times\set{0}\xhookrightarrow{}X\times I\mid X\in\Top}$.
\end{definition}
\begin{definition}[Hurewicz Cofibration]
    Let $i\colon A\to B$ be a map such that for all spaces $X$ and maps $A\times I\to X$, $B\to X$ that make the diagram below commute, there exists a map $B\times I\to X$ such that
    \begin{center}
        \begin{tikzcd} [sep = 4em]
            A \arrow[r,"\id_A\times\set{0}"] \arrow[d, "i"'] & A\times I \arrow[d, "i\times\id_I"] \arrow[ddr, bend left] & \\
            B \arrow[r,"\id_B\times\set{0}"] \arrow[drr, bend right] & B\times I \arrow[dr, dashed, "\exists"] & \\
            &&X
        \end{tikzcd}
    \end{center}
    commutes.
    Then we say $i$ fulfills the \emph{homotopy extension property (HEP)} or $i$ is a \emph{Hurewicz cofibration}.
    If the image $i(A)$ is closed, we call $i$ a \emph{closed Hurewicz cofibration}.
\end{definition}
\begin{remark}\label{rmk:hurewiczCofibIsIncl}
    A Hurewicz cofibration is always an inclusion, see \cite[Proposition 4H.1]{hatcher2002algebraic}.
\end{remark}
%TODO w.e., htpy eq
When speaking of cofibrations, fibrations or weak equivalences between topological spaces, we are always referring to the following model structure.
\begin{thm}[Quillen Model Structure on $\Top$]
    The following classes of maps endow $\Top$ with a model structure:
    \begin{itemize}
        \item Weak equivalences are weak equivalences of topological spaces.
        \item Fibrations are the Serre fibrations.
    \end{itemize}
    We call this model structure the \emph{Quillen Model Structure on $\Top$}.
    
    The cofibrations are the smallest saturated class containing the maps $S^{n-1}\xhookrightarrow{}D^n$ for $n\in\Nat$ (where $S^{-1}=\emptyset$) and the trivial cofibrations are the smallest saturated class containing the maps $D^n\times\set{0}\xhookrightarrow{}D^n\times I$ for $n\in\Nat$.
    In particular, this model structure is cofibrantly generated with respect to these classes.
    \begin{reference}
        \cite[Chap. II, \S 3, Theorem 1]{Quillen1967} and \cite{hirschhorn_quillen}
    \end{reference}
\end{thm}
The other model structure on $\Top$ we will need is given by the following.
\begin{thm}[\Strom/ Model Structure on $\Top$]
    The following three classes of maps endow $\Top$ with a model structure:
    \begin{itemize}
        \item Cofibrations are the closed Hurewicz cofibrations, which we will refer to as \emph{h-cofibrations}.
        \item Weak equivalences are the homotopy equivalences.
        \item Fibrations are the Hurewicz fibrations, which we will refer to as \emph{h-fibrations}.
    \end{itemize}
    We call this model structure the \emph{\Strom/ Model Structure on $\Top$}.
    \begin{reference}
       \cite{Strom1972}%TODO maybe add statement about small object argument failure; (generators too big)
    \end{reference}
\end{thm}
\begin{remark}
    Since every h-fibration is also a fibration in the Quillen model structure and every homotopy equivalence is a weak equivalence, every cofibration of the Quillen model structure is an h-cofibration.
\end{remark}
\begin{remark}[Properness] 
    A model category whose object are all cofibrant is left proper (see \cite[Proposition A.2.4.2]{HTT}).
    Dually, a model category whose object are all fibrant is right proper.
    
    This implies that $\Top$ with the \Strom/ model structure is proper, $\sSet$ is left proper and $\Top$ with the Quillen model structure is right proper.
    In fact, all of the aforementioned model are proper:

    Left properness of $\Top$ with the Quillen model structure is proven in \cite[Theorem 13.1.10]{hirschhorn2003model} and right properness of $\sSet$ is proven in \cite[Theorem 13.1.13]{hirschhorn2003model}.
    Hence, whenever computing homotopy pushouts/pullbacks in these model categories, it will always suffice to only replace one map by a cofibration/fibration.
\end{remark}
\begin{prop}[Fiberwise Characterization of Homotopy Pullbacks]
    Let 
    \begin{center}
        \begin{tikzcd}[sep = 4em]
            A \ar[d, "q"'] \ar[r] & X \ar[d, "p"] \\
            B \ar[r, "f"] & Y
        \end{tikzcd}
    \end{center}
    be a commutative square in $\Top$. 
    Then the square is a homotopy pullback square if and only if for every $b\in B$, the induced map between the homotopy fiber $F_b$ of $q$ at $b$ and the homotopy fiber $G_f(b)$ of $p$ at $p(b)$ is a weak equivalence.

    More explicitly, by picking factorizations of $p$ and $q$ into trivial cofibrations followed by fibrations, we obtain a commutative diagram
    \begin{center}
        \begin{tikzcd}[sep = 4em]
            A \ar[>->,d,"\sim"'] \ar[r] & X \ar[>->,d,"\sim"] \\
            \overline{A} \ar[->>,d, "\overline{q}"'] \ar[r,dashed] & \overline{X} \ar[->>,d,"\overline{p}"] \\
            B \ar[r, "f"] & Y
        \end{tikzcd}
    \end{center}
    where the dashed arrow exists by the lifting properties.
    Then the starting square (which is the outer square) is a homotopy pullback if and only if for every $b\in B$ the induced map $F_b\simeq\set{b}\times_B\overline{A}\to\set{f(b)}\times_Y\overline{X}\simeq G_{f(b)}$ is a weak equivalence.
    (The choice of factorizations and lift does not matter.)
    \begin{reference}
        \cite[Proposition 3.3.18]{cubical_htpy_theory}
    \end{reference}
\end{prop}
\begin{thm}[Quillen Equivalence of Quillen Model Structures]\label{prop:quillenEqSSetTop}
    The functors
    \begin{equation*}
        \lvert-\rvert\colon\sSet\rightleftarrows\Top\mkern+3mu:\mkern-3mu\Sing(-)
    \end{equation*}
    form a Quillen equivalence, where $\lvert-\rvert$ denotes geometric realization and $\Sing(-)$ is the singular simplicial set functor.
    \begin{proof}
        By \cite[Corollary 1.1.8.5]{kerodon} the functors are adjoint.

        Thus $\lvert-\rvert\colon\sSet\to\Top$ preserves colimits and since $\lvert\partial\Delta^n\xhookrightarrow{}\Delta^n\rvert\;\cong\;S^{n-1}\xhookrightarrow{} D^n$, geometric realization preserves cofibrations.
        Since it also preserves weak equivalences by definition, it preserves trivial cofibrations and therefore the adjunction is a Quillen adjunction.

        To prove it is a Quillen equivalence, we need to prove that 
        \begin{itemize}
            \item for every cofibrant object $A\in\sSet$ and a fibrant replacement $\lvert A\rvert\xrightarrow{\sim}\widehat{\lvert A\rvert}$ the composition $A\to\Sing(\lvert A\rvert)\to\Sing(\widehat{\lvert A\rvert})$ where $A\to\Sing(\lvert A\rvert)$ is the adjuntion unit, is a weak equivalence.
            \item for every fibrant object $X\in\Top$ and a cofibrant replacement $\widehat{\Sing(X)}\xrightarrow{\sim}\Sing(X)$ the composition $\lvert\widehat{\Sing(X)}\rvert\to\lvert\Sing(X)\rvert\to X$ where $\lvert\Sing(X)\rvert\to X$ is the adjunction counit, is a weak equivalence.
        \end{itemize}
        Since all simplicial sets are cofibrant and all topological spaces are fibrant, we can pick the replacements to be the identity in both cases.
        Thus the statement is reduced to proving that unit and counit are objectwise weak equivalences. 
        This is proven in e.g. \cite[Theorem 3.5.4.1 and Corollary 3.5.4.2]{kerodon}.
    \end{proof}
\end{thm}
\begin{corollary}\label{prop:homotopyHypothesis}
    The Quillen equivalence 
    \begin{equation*}
        \lvert-\rvert\colon\sSet\rightleftarrows\Top\mkern+3mu:\mkern-3mu\Sing(-)
    \end{equation*}
    induces an equivalence of \inftycats/ 
    \begin{equation*}
        \N(\lvert-\rvert)\colon \N(\sSet)\lbrack W_{\sSet}^{-1}\rbrack\longleftrightarrow\N(\Top)\lbrack W_{\Top}^{-1}\rbrack\mkern+3mu:\mkern-3mu\N(\Sing(-))\;.
    \end{equation*}
    \begin{proof}
        Using \cref{prop:quillenEqSSetTop} this follows from the fact that Quillen equivalences induce equivalences of \inftycats/ after simplicial localization at the weak equivalences, see e.g. \cite[Corollary 1.3]{quillen_adj_15}.
    \end{proof}
\end{corollary}
\begin{corollary}\label{cor:htpyCoherentNerveIsLoc}
    There is an equivalence of \inftycats/ $\Nhc(\sSet^°)\to\N(\Top)[W_{\Top}^{-1}]$ induced by the universal property of simplicial localization.
    \begin{proof}
        Using that $\sSet$ is a simplicial model category, by \cite[Theorem 1.3.4.20]{higher_algebra} we have an equivalence of \inftycats/ $\Nhc(\sSet^°)\to\N(\sSet)[W_{\sSet}^{-1}]$.
        By \cref{prop:homotopyHypothesis} we also have an equivalence of \inftycats/ $\N(\sSet)[W_{\sSet}^{-1}]\to\N(\Top)[W_{\Top}^{-1}]$.
        Composition now gives the desired map.
    \end{proof}
\end{corollary}
\begin{definition}
    We set $\spaces=\N(\Top)[W_{\Top}^{-1}]$ to be the \emph{\inftycat/ of spaces}.
    For convenience of notation we pick a model of $\N(\Top)[W_{\Top}^{-1}]$ that has the same objects as $\Top$, which is possible by \cref{prop:simpLocEssSurj}.
\end{definition}