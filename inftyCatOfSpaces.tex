\begin{prop}[Quillen Model Structure on $\Top$]
    The following three classes of maps endow $\Top$ with a model structure:
    \begin{itemize}
        \item Cofibrations are the maps which are retracts of relative CW-complexes. %TODO relative CW-complex def
        \item Weak equivalences are the maps inducing isomorphisms on $\pi_n$ for $n\in\Nat$.
        \item Fibrations are the maps with right lifting property against the maps $\set{D^n\xhookrightarrow{\left(\id_{D^n},0\right)}D^n\times I\mid n\in\Nat}$
    \end{itemize}
    We call this model structure the \emph{Quillen Model Structure on $\Top$}.
\end{prop}
\begin{prop}
    The cofibrations of the Quillen Model Structure on $\Top$ are the smallest saturated class of maps containing the set $\set{\partial D^n\xhookrightarrow{}D^n}$.
\end{prop}
\begin{prop}[Quillen Model Structure on $\sSet$]
    The following three classes of maps endow $\sSet$ with a model structure:
    \begin{itemize}
        \item Cofibrations are the monomorphisms.
        \item Weak equivalences are the maps that are weak equivalences under geometric realization. %TODO define geometric realization
        \item Fibrations are the maps with right lifting property against the maps $\set{\Lambda_k^n\xhookrightarrow{}\Delta^n\mid 0\leq k\leq n}$.
    \end{itemize}
    We call this model structure the \emph{Quillen Model Structure on $\sSet$}.
\end{prop}
\begin{prop}
    The cofibrations of the Quillen Model Structure on $\sSet$ are the smallest saturated class of maps containing the set $\set{\partial\Delta^n\xhookrightarrow{}\Delta^n\mid n\in\Nat}$.
\end{prop}
\begin{prop}[Quillen Equivalence of the Model Structure on $\Top$ and $\sSet$]
    The functors
    \begin{equation*}
        \lvert-\rvert\colon\sSet\rightleftarrows\Top\colon\Sing(-)
    \end{equation*}
    form a Quillen equivalence. 
    \begin{proof}
        %TODO adjointness
        Since $\lvert-\rvert\colon\sSet\to\Top$ preserves as a left adjoint preserves colimits and $\lvert\partial\Delta^n\rvert\to\lvert\Delta^n\rvert\;\cong\;\partial D^n\xhookrightarrow{} D^n$, $\lvert-\rvert$ preserves cofibrations.
        Since it also preserves weak equivalences by definition, it preserves trivial cofibrations and thus the adjunction is a Quillen adjunction.

        To prove it is a Quillen equivalence, we need to prove that 
        \begin{itemize}
            \item for every cofibrant object $A\in\sSet$ and a fibrant replacement $\lvert A\rvert\xrightarrow{\sim}\widehat{\lvert A\rvert}$ the composition $A\to\Sing(\lvert A\rvert)\to\Sing(\widehat{\lvert A\rvert})$ where $A\to\Sing(\lvert A\rvert)$ is the adjuntion unit, is a weak equivalence.
            \item for every fibrant object $X\in\Top$ and a cofibrant replacement $\widehat{\Sing(X)}\xrightarrow{\sim}\Sing(X)$ the composition $\lvert\widehat{\Sing(X)}\rvert\to\lvert\Sing(X)\rvert\to X$ where $\lvert\Sing(X)\rvert\to X$ is the adjunction counit, is a weak equivalence.
        \end{itemize}
        Since all simplicial sets are cofibrant and all topological spaces are fibrant, we can pick the replacements to be the identity in both cases.
        Thus the statement is reduced to proving that unit and counit are objectwise weak equivalences. %TODO cite Kerodon 3.5.4.1 and 3.5.4.2 
    \end{proof}
\end{prop}
%TODO define W_{\sSet},W_{\Top}
\begin{prop}
    The Quillen equivalence 
    \begin{equation*}
        \lvert-\rvert\colon\sSet\rightleftarrows\Top\colon\Sing(-)
    \end{equation*}
    induces an equivalence of \inftycats/ 
    \begin{equation*}
        F\coloneqq\N(\lvert-\rvert)\colon \N(\sSet)\lbrack W_{\sSet}^{-1}\rbrack\longleftrightarrow\N(\Top)\lbrack W_{\Top}^{-1}\rbrack\colon \N(\Sing(-))\eqqcolon G\;.
    \end{equation*}
    \begin{proof}
        Since we have $\N(\lvert W_{\sSet}\rvert)\subset \N(W_{\Top})$ by definition, by the universal property of localization we obtain a localization $F\colon \N(\sSet)\lbrack W_{\sSet}^{-1}\rbrack\to\N(\Top)\lbrack W_{\Top}^{-1}\rbrack$.

        To prove $\N(\Sing(W_{\Top}))\subset\N(W_{\sSet})$ we have to check that for a weak equivalence $X\to Y$, the induced map $\lvert\Sing(X)\rvert\to \lvert\Sing(Y)\rvert$ is also a weak equivalence.
        This follows from that fact that a map is a weak equivalence if and only if it is an isomorphism in the homotopy category.
        Thus we also obtain a functor $G\colon\N(\Top)\lbrack W_{\Top}^{-1}\rbrack\to\N(\sSet)\lbrack W_{\sSet}^{-1}\rbrack$.

        Since unit and counit are objectwise weak equivalences, their induced maps $\epsilon\colon\id\to GF$ and $\eta\colon FG\to \id$ are objectwise invertible and thus invertible natural transformations, which proves the proposition. %TODO cite proof 
    \end{proof}
\end{prop}
\begin{prop} %TODO https://people.math.rochester.edu/faculty/doug/otherpapers/dwyer-kan-3.pdf prop 4.8 proves htpy coh nerve of fibcofib is localization at w.e.
    Let $\gamma\colon\N(\sSet)\to\Nhc(\sSet^°)$ be a map given by applying $\N$ to a fibrant replacement functor $R\colon\sSet\to\sSet^°$ and composing with the inclusion $\N(\sSet^°)\to\Nhc(\sSet^°)$.
    
    Then the \inftycat/ $\Nhc(\sSet^°)$ together with $\gamma$ is a localization of $\N(\sSet)$ at $W_{\sSet}$.
\end{prop}
\begin{corollary}
    There is an equivalence of \inftycats/ $\Nhc(\sSet^°)\to\N(\Top)[W_{\Top}^{-1}]$ induced by the universal property of simplicial localization. %TODO Theorem 1.3.4.20. HA
\end{corollary}
%TODO pullbacks in homotopy theory Proof of Theorem 25 for Second cube theorem (universality)
%TODO relation of h(co)lim of model cat and simpl cat is in Higher Algebra 1.3.4.24, also see HTT Corollary 4.2.4.8
%TODO prove htpy hypothesis