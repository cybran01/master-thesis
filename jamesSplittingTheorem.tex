In this section we will give a proof of the classical James Splitting Theorem \cite{james_splitting_original} for general \inftytops/.
The classical James Splitting Theorem in $\spaces$ states that for connected pointed objects $X$, there is a canonical equivalence $\bigvee\limits_{i=1}^{\infty}\Sigma X^{\wedge i}\to\Sigma\Omega\Sigma X$ of pointed objects.
Our main source is \cite[\S 2.2 and \S 4]{splittings_21}.

We begin by introducing some well known constructions that we will need throughout the rest of this section.
\begin{definition}
    Let $C$ be an \inftytop/.
    Then 
    \begin{itemize}
        \item for an object $X$ in $C$, we call a pushout
        \begin{center}
            \begin{tikzcd} [sep = 4em]
                X \arrow[r] \arrow[d] & * \arrow[d, "i_0"] \\
                * \arrow[r, "i_1"] & \Sigma X \\
            \end{tikzcd}
        \end{center}
        a \emph{suspension of $X$}. 
        Note that since both canonical maps $X\to *\to \Sigma X$ are equivalent, when $X$ is pointed then so is $\Sigma X$ via the map $*\to X\to\Sigma X$.
        \item for pointed objects $X$ and $Y$ in $C$, we call a pushout
        \begin{center}
            \begin{tikzcd} [sep = 4em]
                * \arrow[r] \arrow[d] & Y \arrow[d] \\
                X \arrow[r] & X\vee Y \\
            \end{tikzcd}
        \end{center}
        a \emph{wedge sum of $X$ and $Y$}.
        \item for pointed objects $X$ and $Y$ in $C$, we call a pushout
        \begin{center}
            \begin{tikzcd} [sep = 4em]
                X\vee Y \arrow[r] \arrow[d] & X\times Y \arrow[d] \\
                * \arrow[r] & X\wedge Y \\
            \end{tikzcd}
        \end{center}
        a \emph{smash product of $X$ and $Y$} (where the map $X\vee Y\to X\times Y$ is a canonical map from the universal property induced by the maps $X\vee Y\to X$ and $X\vee Y\to Y$).
    \end{itemize}
\end{definition}
We now want to establish some fundamental properties of the wedge sum and smash product.
It is clear from their definition that both are commutative.
We will prove that both are associative and that the smash product distributes over the wedge sum.
\begin{lemma}\label{lem:assocWedge}
    Let $C$ be an \inftytop/.
    Then the wedge sum is associative in the sense that for pointed objects $A,B,C$ of $C$ there is a canonical equivalence $(A\vee B)\vee C\simeq A\vee (B\vee C)$ of pointed objects.
    In fact, $(A\vee B)\vee C\simeq(A\vee B)\cup_{B}(B\vee C)\simeq A\vee (B\vee C)$ as pointed objects.
    \begin{proof}
        From the horizontal composite square of the diagram 
        \begin{center}
            \begin{tikzcd} [sep = 4em]
                & * \arrow[r] \arrow[d] & C \arrow[d] \\
                * \arrow[r] \arrow[d] & B \arrow[d] \arrow[r] & B\vee C \arrow[d] \\
                A \arrow[r] & A\vee B \arrow[r] & (A\vee B)\cup_B(B\vee C) \\
            \end{tikzcd}
        \end{center}
        it follows that $(A\vee B)\cup_B(B\vee C)\simeq A\vee (B\vee C)$ by the pasting law. 
        The pasting law applied to the vertical composite square shows that $(A\vee B)\cup_B(B\vee C)\simeq (A\vee B)\vee C$.
    \end{proof}
\end{lemma}
\begin{lemma}\label{lem:productPreservesPushout}
    Let $C$ be an \inftytop/, $X$ an object in $C$ and
    \begin{center}
        \begin{tikzcd} [sep = 4em]
            A \arrow[r] \arrow[d] & B \arrow[d] \\
            C \arrow[r] & D \\
        \end{tikzcd}
    \end{center} 
    and pushout in $C$.

    Then the induced square
    \begin{center}
        \begin{tikzcd} [sep = 4em]
            A\times X \arrow[r] \arrow[d] & B\times X \arrow[d] \\
            C\times X \arrow[r] & D\times X \\
        \end{tikzcd}
    \end{center} 
    is also a pushout.
    \begin{proof}
        The vertical faces of the cube in the diagram
        \begin{center}
            \begin{tikzcd} [sep = .5 cm]
                A\times X \arrow [dr] \arrow [rr] \arrow [dd] & & B\times X \arrow [dr] \arrow [dd] \\
                & C\times X \arrow [rr, crossing over] \arrow [dd] & & D\times X \arrow [dd] \arrow[r] &[2em] X \arrow [dd]\\
                A \arrow [dr] \arrow [rr] & & B \arrow [dr] \\
                & C \arrow [from=uu, crossing over] \arrow [rr] & & D \arrow[r] & *
            \end{tikzcd}
        \end{center}
        are all pullbacks by the pasting law.
        Hence by universality the top square of the cube is a pushout.
    \end{proof}
\end{lemma}
\begin{lemma}\label{lem:smashDist}
    Let $C$ be an \inftytop/.
    Then the smash product distributes over the wedge sum in the sense that for pointed objects $X,A,B$ of $C$ there is a canonical equivalence $(A\vee B)\wedge X\simeq (A\wedge X)\vee (B\wedge X)$ of pointed objects.
    \begin{proof}
        Because colimits commute, the colimit of a diagram indexed by $\Lambda_0^2\times\Lambda_0^2$ can be computed by either taking object-wise pushouts of the first argument and then the pushout of the resulting $\Lambda_0^2$-shaped diagram, or taking object-wise pushouts of the second argument and then taking pushout of the resulting $\Lambda_0^2$-shaped diagram.
        Hence given the diagram
        \begin{center}
            \begin{tikzcd} [sep = 4em]
                * & * \arrow[l] \arrow[r] & * \\
                A\vee X \arrow[d] \arrow[u] & X \arrow[l]  \arrow[r] \arrow[d, equal] \arrow[u] & B\vee X \arrow[d] \arrow[u] \\
                A\times X  & X \arrow[l] \arrow[r] & B\times X \\
            \end{tikzcd}
        \end{center}
        we can compute the colimit row-wise or column-wise:

        Row-wise evaluation gives the pushout 
        \begin{center}
            \begin{tikzcd}[sep=4em]
                (A\vee B)\vee X \ar[r] \ar[d] & (A\vee B)\times X \ar[d]\\
                * \ar[r] & (A\vee B)\wedge X
            \end{tikzcd}
        \end{center}
        where we use \cref{lem:productPreservesPushout} and the fact that the wedge sum is associative (\cref{lem:assocWedge}) and commutative.

        Column-wise evaluation gives the pushout 
        \begin{center}
            \begin{tikzcd}[sep=4em]
                * \ar[r] \ar[d] & B\wedge X \ar[d]\\
                A\wedge X \ar[r] & (A\wedge X)\vee (B\wedge X)
            \end{tikzcd}
        \end{center}
        hence $(A\vee B)\wedge X\simeq(A\wedge X)\vee (B\wedge X)$ follows.
    \end{proof}
\end{lemma}
\begin{lemma}\label{lem:pushoutProdSquareIsPushout}
    Let $C$ be an \inftytop/, $f\colon X\to Y\in C_1$ a morphism and 
    \begin{center}
        \begin{tikzcd} [sep = 4em]
            A \arrow[r] \arrow[d] & B \arrow[d] \\
            C \arrow[r] & D \\
        \end{tikzcd}
    \end{center} 
    a pushout.

    Then the induced square 
    \begin{center}
        \begin{tikzcd} [sep = 4em]
            A\times Y\cup_{A\times X} B\times X \arrow[r] \arrow[d] & B\times Y \arrow[d] \\
            C\times Y\cup_{C\times X} D\times X \arrow[r] & D\times Y
        \end{tikzcd}
    \end{center} 
    is also a pushout.
    \begin{proof}
        All vertical faces in the diagram
        \begin{center}
            \begin{tikzcd} [sep = .5 cm]
                A\times X \arrow [dr] \arrow [rr] \arrow [dd] & & B\times X \arrow [dr] \arrow [dd] \\
                & C\times X \arrow [rr, crossing over]  & & D\times X \arrow [dd] \arrow[r] &[2em] X \arrow [dd, "f"]\\
                A\times Y \arrow [dr] \arrow [rr] \arrow [dd] & & B\times Y \arrow [dr] \arrow [dd] \\
                & C\times Y \arrow [rr, crossing over] \arrow [from=uu, crossing over] \arrow [dd] & & D\times Y \arrow [dd] \arrow[r] &[2em] Y \arrow [dd]\\
                A \arrow [dr] \arrow [rr] & & B \arrow [dr] \\
                & C \arrow [from=uu, crossing over] \arrow [rr] & & D \arrow[r] & *
            \end{tikzcd}
        \end{center}
        are pullbacks by the pasting law.
        Hence by universality the middle and top square of the stacked cube are also pushouts.

        Now taking pushouts of the back and front square of the upper cube induces the square 
        \begin{center}
            \begin{tikzcd} [sep = 4em]
                A\times Y\cup_{A\times X} B\times X \arrow[r] \arrow[d] & B\times Y \arrow[d] \\
                C\times Y\cup_{C\times X} D\times X \arrow[r] & D\times Y
            \end{tikzcd}\;.
        \end{center} 

        The pushout square
        \begin{center}
            \begin{tikzcd} [sep = 4em]
                A\times X \arrow[r] \arrow[d] \ar[dr,phantom,"\Circled{0}"] & B\times X \arrow[d] \\
                C\times Y \arrow[r] & C\times Y\cup_{A\times X}B\times X
            \end{tikzcd}
        \end{center} 
        can be written as a composite square in two ways:
        \begin{center}
            \begin{tikzcd} [sep = 4em]
                A\times X \arrow[r] \arrow[d] \ar[dr,phantom,"\Circled{1}"] & B\times X \arrow[d] \\
                C\times X \arrow[r] \arrow[d] \ar[dr,phantom,"\Circled{2}"]& D\times X \arrow[d] \\
                C\times Y \arrow[r] & C\times Y\cup_{A\times X}B\times X
            \end{tikzcd}
            \begin{tikzcd} [sep = 4em]
                A\times X \arrow[r] \arrow[d] \ar[dr,phantom,"\Circled{3}"] & B\times X \arrow[d] \\
                A\times Y \arrow[r] \arrow[d] \ar[dr,phantom,"\Circled{4}"] & B\times Y \arrow[d] \\
                C\times Y \arrow[r] & C\times Y\cup_{A\times X}B\times X
            \end{tikzcd}
        \end{center} 
        The square $\Circled{1}$ is a pushout, thus by the pasting law $C\times Y\cup_{A\times X}B\times X\simeq C\times Y\cup_{C\times X} D\times X$.

        Using this identification and taking the pushout of $\Circled{3}$ we obtain 
        \begin{center}
            \begin{tikzcd} [sep = 4em]
                A\times X \arrow[r] \arrow[d] \ar[dr,phantom,"\Circled{3'}"] & B\times X \arrow[d] \\
                A\times Y \arrow[r] \arrow[d] \ar[dr,phantom,"\Circled{4'}"] & A\times Y\cup_{A\times X} B\times X \arrow[d] \\
                C\times Y \arrow[r] & C\times Y\cup_{C\times X} D\times X
            \end{tikzcd}
        \end{center}
        where the outer square is still equivalent to $\Circled{0}$.
        Thus by the pasting law $\Circled{4'}$ is a pushout.

        Since the composite square
        \begin{center}
            \begin{tikzcd} [sep = 4em]
                A\times Y \arrow[r] \arrow[d] \ar[dr,phantom,"\Circled{4'}"] & A\times Y\cup_{A\times X} B\times X \arrow[d] \arrow[r] & \arrow[d] B\times Y\\
                C\times Y \arrow[r] & C\times Y\cup_{C\times X} D\times X \arrow[r] & D\times Y
            \end{tikzcd}
        \end{center}
        is also a pushout, by the pasting law the right square is a pushout which proves the lemma.
    \end{proof}
\end{lemma}
\begin{lemma}\label{lem:assocSmash}
    Let $C$ be an \inftytop/.
    Then the smash product is associative in the sense that for pointed objects $A,B,C$ of $C$ there is a canonical equivalence $(A\wedge B)\wedge C\simeq A\wedge (B\wedge C)$ of pointed objects.
    \begin{proof}
        We apply \cref{lem:pushoutProdSquareIsPushout} to the pushout
        \begin{center}
            \begin{tikzcd} [sep = 4em]
                A\vee B \arrow[r] \arrow[d] & A\times B \arrow[d] \\
                * \arrow[r] & A\wedge B \\
            \end{tikzcd}
        \end{center} 
        and the canonical map $*\to C$ to obtain a pushout square
        \begin{center}
            \begin{tikzcd} [sep = 4em]
                \left((A\vee B)\times C\right)\cup_{(A\vee B)\times *} \left((A\times B)\times *\right) \arrow[r] \arrow[d] & (A\times B)\times C \arrow[d] \\
                *\times C\cup_{*\times *} \left((A\wedge B)\times *\right) \arrow[r] & (A\wedge B)\times C
            \end{tikzcd}
        \end{center}
        which we identify with
        \begin{center}
            \begin{tikzcd} [sep = 4em]
                \left((A\vee B)\times C\right)\cup_{A\vee B} A\times B \arrow[r] \arrow[d] & (A\times B)\times C \arrow[d] \\
                (A\wedge B)\vee C \arrow[r] & (A\wedge B)\times C
            \end{tikzcd}\;.
        \end{center}
        Then the composite square
        \begin{center}
            \begin{tikzcd} [sep = 4em]
                \left((A\vee B)\times C\right)\cup_{A\vee B} A\times B \arrow[r] \arrow[d] & (A\times B)\times C \arrow[d] \\
                (A\wedge B)\vee C \arrow[r] \arrow[d] & (A\wedge B)\times C \arrow[d] \\
                * \arrow[r] & (A\wedge B)\wedge C 
            \end{tikzcd}
        \end{center}
        is a pushout by the pasting law.

        The analogous construction starting with 
        \begin{center}
            \begin{tikzcd} [sep = 4em]
                B\vee C \arrow[r] \arrow[d] & B\times C \arrow[d] \\
                * \arrow[r] & B\wedge C \\
            \end{tikzcd}
        \end{center} 
        and the canonical map $*\to A$ gives a pushout square
        \begin{center}
            \begin{tikzcd} [sep = 4em]
                \left(A\times(B\vee C)\right)\cup_{B\vee C} B\times C \arrow[r] \arrow[d] & A\times(B\times C) \arrow[d] \\
                * \arrow[r] & A\wedge (B\wedge C)
            \end{tikzcd}\;.
        \end{center}
        There is a canonical equivalence $\left(A\times(B\vee C)\right)\cup_{B\vee C} B\times C\simeq \left((A\vee B)\times C\right)\cup_{A\vee B} A\times B$:

        Using \cref{lem:productPreservesPushout}, the row-wise colimit of 
        \begin{center}
            \begin{tikzcd} [sep = 4em]
                A\times B & A \arrow[l] \arrow[r] & A\times C \\
                B \arrow[d, equals] \arrow[u] & * \arrow[l] \arrow[r] \arrow[d] \arrow[u] & C \arrow[d] \arrow[u] \\
                B  & B \arrow[l, equals] \arrow[r] & B\times C \\
            \end{tikzcd}
        \end{center}
        evaluates to $\left(A\times(B\vee C)\right)\cup_{B\vee C} B\times C$ and the column-wise colimit evaluates to $\left((A\vee B)\times C\right)\cup_{A\vee B}A\times B$ which proves the canonical equivalence.

        Since the maps $\left(A\times(B\vee C)\right)\cup_{B\vee C} B\times C\to A\times (B\times C)$ and $\left((A\vee B)\times C\right)\cup_{A\vee B} A\times B\to (A\times B)\times C$ are universal maps, the top square of the cube
        \begin{center}
            \begin{tikzcd}[column sep=-2.5em]
                \left((A\vee B)\times C\right)\cup_{A\vee B} A\times B \arrow [dr,"\simeq"] \arrow [rr] \arrow [dd] & & (A\times B)\times C \arrow [dr,"\simeq"] \arrow [dd] \\
                & \left(A\times(B\vee C)\right)\cup_{B\vee C} B\times C \arrow [rr, crossing over] \arrow [dd] & &[5.5em] A\times (B\times C) \arrow [dd]\\
                * \arrow [dr,"\simeq"] \arrow [rr] & & (A\wedge B)\wedge C \arrow [dr] \\
                & * \arrow [from=uu, crossing over] \arrow [rr] & & A\wedge (B\wedge C)
            \end{tikzcd}
        \end{center}
        commutes and thus the induced map $(A\wedge B)\wedge C\to A\wedge(B\wedge C)$ of the pushout squares is an equivalence.
    \end{proof}
\end{lemma}
\addcontentsline{toc}{subsection}{The Fundamental James Splitting}
\subsection*{The Fundamental James Splitting}
As a first step to proving the James Splitting Theorem, we will show that for pointed objects $X$ there is a canonical equivalence $\Sigma\Omega\Sigma X\simeq\Sigma X\vee\left(X\wedge\Sigma\Omega\Sigma X\right)$ of pointed objects.
This is called the \emph{Fundamental James Splitting}.

The next lemma is a key ingredient of the proof of the Fundamental James Splitting.
There is some subtlety involved in it, see \cref{rmk:mapNotProj}.
\begin{lemma}\label{lem:existenceOfPoSq}
    Let $C$ be an \inftytop/ and let $X\in \left(C_*\right)_0$ be a pointed object.
    Then there exists a pushout square  
    \begin{center}
        \begin{tikzcd} [sep = 4em]
            X\times\Omega\Sigma X \arrow[r, "\pi_2"] \arrow[d, "\alpha"] & \Omega\Sigma X \arrow[d] \\
            \Omega\Sigma X \arrow[r] & * \\
        \end{tikzcd}\;.
    \end{center}
    \begin{proof}
        The definition of $\Sigma X$
        \begin{center}
            \begin{tikzcd} [sep = 4em]
                X \arrow[r] \arrow[d] & * \arrow[d, "i_0"] \\
                * \arrow[r, "i_1"] & \Sigma X \\
            \end{tikzcd}
        \end{center}
        shows that $i_0$ and $i_1$ are equivalent maps in $C$, since $X\to *\xrightarrow{i_0}\Sigma X$ and $X\to *\xrightarrow{i_1}\Sigma X$ are equivalent by construction and the canonical point $x\colon*\to X$ is a section of $X\to *$.
        Thus we get that the squares
        \begin{center}
            \begin{tikzcd} [sep = 4em]
                \Omega\Sigma X \arrow[r] \arrow[d] & * \arrow[d, "i_0"] \\
                * \arrow[r, "i_0"] & \Sigma X \\
            \end{tikzcd}
            \begin{tikzcd} [sep = 4em]
                \Omega\Sigma X \arrow[r] \arrow[d] & * \arrow[d, "i_0"] \\
                * \arrow[r, "i_1"] & \Sigma X \\
            \end{tikzcd}
        \end{center}
        are pullback squares.
        We pick a product together with projections to obtain a pullback square 
        \begin{center}
            \begin{tikzcd} [sep = 4em]
                X\times\Omega\Sigma X \arrow[r, "\pi_2"] \arrow[d, "\pi_1"] & \Omega\Sigma X \arrow[d] \\
                X \arrow[r] & * \\
            \end{tikzcd}
        \end{center}
        and assemble these pullback squares into a diagram
        \begin{center}
            \begin{tikzcd} [sep = .5 cm]
                X\times\Omega\Sigma X \arrow [rr, "\pi_2"] \arrow [dd, "\pi_1"] & & \Omega\Sigma X \arrow [dr] \arrow[dd] \\
                & \Omega\Sigma X \arrow [rr, crossing over] & & * \arrow [dd, "i_0"] & \\
                X \arrow [dr] \arrow [rr] & & * \arrow [dr, "i_0"] \\
                & * \arrow [from=uu,crossing over] \arrow [rr, "i_1"] & & \Sigma X &
            \end{tikzcd}
        \end{center}
        where the three vertical squares are pullbacks, and the bottom square is a pushout.

        Now let $\Fun(\Delta^1\times\Delta^1,C)_{pb}\subset\Fun(\Delta^1\times\Delta^1,C)$ denote the full subcategory of pullback squares.
        Then the functor $\lim\colon\Fun(\Lambda_0^2,C)\to\Fun(\Delta^1\times\Delta^1,C)_{pb}$ is an equivalence of \inftycats/ since it is essentially surjective by definition and fully faithful since it is a right Kan-extension along the inclusion $\Lambda_0^2\subset\Delta^1\times\Delta^1$.

        The previous diagram corresponds to a natural transformation $\lambda\in\Fun(\Lambda_2^2,C)_1$ where $\lambda_1$ and $\lambda_1 $ are the restrictions of the back square and the front square to $\Lambda_2^2$ respectively.
        The induced natural transformation $\overline{\lambda}=\lim(\lambda)$ is a map between pullback squares, and since back and front square in the above diagram were already pullbacks, they are equivalent to $\overline{\lambda}_0$ and $\overline{\lambda}_1$ respectively.
        The map $\overline{\lambda}$ together with the equivalences of pullback squares thus induces a cube
        \begin{center}
            \begin{tikzcd} [sep = .5 cm]
                X\times\Omega\Sigma X \arrow [dr, "\alpha"] \arrow [rr, "\pi_2"] \arrow [dd, "\pi_1"] & & \Omega\Sigma X \arrow [dr] \arrow[dd] \\
                & \Omega\Sigma X \arrow [rr, crossing over] & &[1em] * \arrow [dd, "i_0"] & \\
                X \arrow [dr] \arrow [rr] & & * \arrow [dr, "i_0"] \\
                & * \arrow [from=uu, crossing over] \arrow [rr, "i_1"] & & \Sigma X &
            \end{tikzcd}
        \end{center}
        and we claim that the left face is also a pullback:

        The outer square of the diagram
        \begin{center}
            \begin{tikzcd} [sep = 4em]
                X\times\Omega\Sigma X \arrow[d, "\pi_1"] \arrow[r, "\pi_2"] & \Omega\Sigma X \arrow[r] \arrow[d] & * \arrow[d, "i_0"] \\
                X \arrow[r] & * \arrow[r, "i_0"] & \Sigma X \\
            \end{tikzcd}
        \end{center}
        is a pullback since both squares are pullbacks by assumption.
        Furthermore, the outer square of this diagram is equivalent to the outer square of the diagram
        \begin{center}
            \begin{tikzcd} [sep = 4em]
                X\times\Omega\Sigma X \arrow[d, "\pi_1"] \arrow[r, "\alpha"] & \Omega\Sigma X \arrow[r] \arrow[d] & * \arrow[d, "i_0"] \\
                X \arrow[r] & * \arrow[r, "i_1"] & \Sigma X \\
            \end{tikzcd}
        \end{center}
        which shows that it is also a pullback.
        Since the right square in this diagram is a pullback, the pasting law for pullbacks implies that the left square is also a pullback.

        Thus all vertical squares of the cube
        \begin{center}
            \begin{tikzcd} [sep = .5 cm]
                X\times\Omega\Sigma X \arrow [dr, "\alpha"] \arrow [rr, "\pi_2"] \arrow [dd, "\pi_1"'] & & \Omega\Sigma X \arrow [dr] \arrow[dd] \\
                & \Omega\Sigma X \arrow [rr, crossing over] & &[1em] * \arrow [dd, "i_0"] & \\
                X \arrow [dr] \arrow [rr] & & * \arrow [dr, "i_0"] \\
                & * \arrow [from=uu, crossing over] \arrow [rr, "i_1"] & & \Sigma X &
            \end{tikzcd}
        \end{center}
        are pullbacks and since the bottom square is a pushout by assumption, by universality the top square is pushout which finishes the proof.
    \end{proof}
\end{lemma}
\begin{remark}\label{rmk:mapNotProj}
    The map $\alpha\colon X\times\Omega\Sigma X\to\Omega\Sigma X$ can in general not be identified with the projection $\pi_2$.
    We are not able to choose the map $\alpha$ freely since the higher simplices must be compatible along the ``diagonal'' of our cube (which our map $\alpha$ fulfills by construction, since it is a universal map).

    If we assume $\alpha=\pi_2$ then by universality we have that 
    \begin{center}
        \begin{tikzcd} [sep = 4em]
            X\times \Omega\Sigma X \arrow[r,"\pi_2"] \arrow[d, "\pi_2"] & \Omega\Sigma X \arrow[d] \\
            \Omega\Sigma X \arrow[r] & \Sigma X \times \Omega\Sigma X\\
        \end{tikzcd}
    \end{center}
    is a pushout.
    But $\Sigma X \times \Omega\Sigma X$ is not generally contractible. 
\end{remark}
\begin{corollary}\label{lem:poOfProductIsSuspension}
    Let $C$ be an \inftytop/ and let $X\in \left(C_*\right)_0$ be a pointed object. 
    Then there exists a square
    \begin{center}
        \begin{tikzcd} [sep = 4em]
            X\times\Omega\Sigma X \arrow[r, "\pi_2"] \arrow[d] & \Omega\Sigma X \arrow[d] \\
            * \arrow[r] & \Sigma\Omega\Sigma X \\
        \end{tikzcd}
    \end{center}
    that is a pushout.
    \begin{proof}
        This follows from \cref{lem:existenceOfPoSq} and the pasting lemma applied to 
        \begin{center}
            \begin{tikzcd} [sep = 4em]
                X\times\Omega\Sigma X \arrow[r, "\pi_2"] \arrow[d, "\alpha"] & \Omega\Sigma X \arrow[d] \\
                \Omega\Sigma X \arrow[r] \arrow[d] & * \arrow[d] \\
                * \arrow[r] & \Sigma\Omega\Sigma X \\
            \end{tikzcd}\;.
        \end{center}
    \end{proof}
\end{corollary}
We will also need to prove some basic facts about wedges and smash products.
\begin{lemma}\label{lem:poOfCollapseMapsIsTrivial}
    Let $C$ be an \inftytop/ and let $X,Y$ be pointed objects in $C$. 
    Then 
    \begin{center}
        \begin{tikzcd} [sep = 4em]
            X\vee Y \arrow[r] \arrow[d] & Y \arrow[d] \\
            X \arrow[r] & * \\
        \end{tikzcd}
    \end{center}
    is a pushout.
    \begin{proof}
        We want to show that the square $\Circled{4}$ in the diagram
        \begin{center}
            \begin{tikzcd} [sep = 4em]
                * \arrow[r] \arrow[d] \ar[dr,phantom,"\Circled{1}"] & Y \arrow[r] \arrow[d] \ar[dr,phantom,"\Circled{2}"] & * \arrow[d] \\
                X \arrow[r] \arrow[d] \ar[dr,phantom,"\Circled{3}"] & X\vee Y \arrow[r] \arrow[d] \ar[dr,phantom,"\Circled{4}"] & X \arrow[d]\\
                * \arrow[r] & Y \arrow[r] & * \\
            \end{tikzcd}
        \end{center}
        is a pushout.
        Since $\Circled{1}$ and the composite square $\Circled{1}+\Circled{2}$ are pushouts, $\Circled{2}$ is a pushout by the pasting law.
        But since $\Circled{2}$ and the composite square $\Circled{2}+\Circled{4}$ are pushouts, by the pasting law we have that $\Circled{4}$ is a pushout.
    \end{proof}
\end{lemma}
\begin{lemma}\label{lem:poSqWithFactoringTerminal}
    Let $C$ be an \inftytop/ and let $X,Y$ be pointed objects in $C$. 
    Then there is a pushout square
    \begin{center}
        \begin{tikzcd} [sep = 4em]
            X\times Y \arrow[r, "\pi_2"] \arrow[d, "\pi_1"'] & Y \arrow[d] \\
            X \arrow[r] & \Sigma\left(X\wedge Y\right) \\
        \end{tikzcd}
    \end{center}
    such that the unlabeled morphisms factor through the terminal object.
    \begin{proof}
        We compute the colimit of the diagram
        \begin{center}
            \begin{tikzcd} [sep = 4em]
                * & * \arrow[l] \arrow[r] & * \\
                X \arrow[d, equal] \arrow[u] & X\vee Y \arrow[l]  \arrow[r] \arrow[d] \arrow[u] & Y \arrow[d, equal] \arrow[u] \\
                X  & X\times Y \arrow[l] \arrow[r] & Y \\
            \end{tikzcd}
        \end{center}
        first row-wise and and then column-wise.
        
        Taking pushouts of the rows results in the diagram $*\xleftarrow{}*\rightarrow X\cup_{X\times Y} Y$ where we use \cref{lem:poOfCollapseMapsIsTrivial} and taking pushouts of the columns results in the diagram $*\xleftarrow{}X\wedge Y\rightarrow *$. 
        Since the pushout of the latter diagram is $\Sigma\left(X\wedge Y\right)$ and the pushout of the first diagram must be equivalent to this, we have $X\cup_{X\times Y} Y\simeq\Sigma\left(X\wedge Y\right)$.
        
        It remains to show that the maps $X\to\Sigma\left(X\wedge Y\right)$ and $Y\to\Sigma\left(X\wedge Y\right)$ factor through $*$.
        But this follows directly from the starting diagram since the colimit cone $\left(\Lambda_0^2\times\Lambda_0^2\right)^{\rhd}\to C$ has $2$-simplices of the form
        \begin{center}
            \begin{tikzcd}[sep = 3em]
                & \Sigma\left(X\wedge Y\right) & \\
                X \ar[rr] \ar[ur]& & * \ar[ul]
            \end{tikzcd}
            \begin{tikzcd}[sep = 3em]
                & \Sigma\left(X\wedge Y\right) & \\
                Y \ar[rr] \ar[ur]& & * \ar[ul]
            \end{tikzcd}\;.
        \end{center}
    \end{proof}
\end{lemma}
\begin{lemma}\label{lem:allSqArePo}
    Let $C$ be an \inftytop/ and let $X,Y$ be pointed objects in $C$. 
    Then all squares of the diagram         
    \begin{center}
        \begin{tikzcd} [sep = 4em]
            X\times Y \arrow[r, "\pi_2"] \arrow[d, "\pi_1"'] & Y \arrow[r] \arrow[d] & * \arrow[d] \\
            X \arrow[r] \arrow[d] & \Sigma\left(X\wedge Y\right) \arrow[r] \arrow[d] & \Sigma Y\vee\Sigma\left(X\wedge Y\right) \arrow[d]\\
            * \arrow[r] &  \Sigma X\vee\Sigma\left(X\wedge Y\right)\arrow[r] & \Sigma X\vee\Sigma Y\vee\Sigma\left(X\wedge Y\right) \\
        \end{tikzcd}
    \end{center}
    are pushouts.
    \begin{proof}
        By \cref{lem:poSqWithFactoringTerminal} the top left square is a pushout square.
        The top right square is a pushout by the pasting law applied to
        \begin{center}
            \begin{tikzcd} [sep = 4em]
                Y \arrow[r] \arrow[d] & * \arrow[d] \\
                * \arrow[r] \arrow[d] & \Sigma Y \arrow[d] \\
                \Sigma\left(X\wedge Y\right) \arrow[r] & \Sigma Y\vee\Sigma\left(X\wedge Y\right) \\
            \end{tikzcd}\
        \end{center}
        because $Y\to\Sigma\left(X\wedge Y\right)$ factors through $*$ (again by \cref{lem:poSqWithFactoringTerminal}).
        The analogous argument also shows that the bottom left square is a pushout.

        The bottom right square is a pushout by \cref{lem:assocWedge} and the fact that the wedge sum is commutative.
    \end{proof}
\end{lemma}
The following corollary is not needed for our proof, however it is useful and immediate from what we have already shown.
\begin{corollary}
    Let $C$ be an \inftytop/ and let $X,Y$ be pointed objects in $C$. 
    Then we have $\Sigma\left(X\times Y\right)\cong\Sigma X\vee\Sigma Y\vee\Sigma\left(X\wedge Y\right)$.
    \begin{proof}
        The outer square of \cref{lem:allSqArePo} proves this immediately. 
    \end{proof}
\end{corollary}
\begin{lemma}\label{lem:suspensionCommutesWithSmash}
    Let $C$ be an \inftytop/ and let $X,Y$ be pointed objects in $C$. 
    Then there is a canonical equivalence $\Sigma\left(X\wedge Y\right)\cong X\wedge\Sigma Y$ of pointed objects.
    \begin{proof}
        Let $i_0,i_1\colon *\to\Sigma Y$ be the canonical maps from the definition $\Sigma Y$ as a pushout.
        These induce maps $j_0,j_1\colon X \to X\vee\Sigma Y$ and $k_0,k_1\colon X\to X\times\Sigma Y$.
        The squares
        \begin{center}
            \begin{tikzcd} [sep = 4em]
                X\vee Y \arrow[r] \arrow[d] &  X \arrow[d, "j_0"] \\
                X \arrow[r, "j_1"] & X\vee\Sigma Y\\
            \end{tikzcd}
            \begin{tikzcd} [sep = 4em]
                X\times Y \arrow[r] \arrow[d] &  X \arrow[d, "k_0"] \\
                X \arrow[r, "k_1"] & X\times\Sigma Y\\
            \end{tikzcd}\
        \end{center}
        are both pushouts; the first one as colimits commute and the second one by \cref{lem:productPreservesPushout}.
        Computing the pushout of the diagram 
        \begin{center}
            \begin{tikzcd} [sep = 4em]
                * & * \arrow[l] \arrow[r] & * \\
                X \arrow[d, equal] \arrow[u] & X\vee Y \arrow[l]  \arrow[r] \arrow[d] \arrow[u] & X \arrow[d, equal] \arrow[u] \\
                X  & X\times Y \arrow[l] \arrow[r] & X \\
            \end{tikzcd}
        \end{center}
        as in \cref{lem:poSqWithFactoringTerminal} we get $X\wedge\Sigma Y$ by computing the pushout of the pushouts of the rows and $\Sigma\left(X\wedge Y\right)$ by computing the pushout of the pushouts of the columns, which proves the proposition.
    \end{proof}
\end{lemma}
\begin{corollary}[Fundamental James Splitting]\label{cor:fundamentalJamesSplitting}
    Let $C$ be an \inftytop/ and let $X$ be a pointed object in $C$. 
    Then there exists a canonical equivalence $\Sigma\Omega\Sigma X\simeq\Sigma X\vee\left(X\wedge\Sigma\Omega\Sigma X\right)$ of pointed objects.
    \begin{proof}
        Applying \cref{lem:allSqArePo} to $X$ and $Y=\Omega\Sigma X$, the big left square (top and bottom left squares combined) is equivalent to the square 
        \begin{center}
            \begin{tikzcd} [sep = 4em]
                X\times\Omega\Sigma X \arrow[r, "\pi_2"] \arrow[d] & \Omega\Sigma X \arrow[d] \\
                * \arrow[r] & \Sigma\Omega\Sigma X \\
            \end{tikzcd}
        \end{center}
        of \cref{lem:poOfProductIsSuspension}.
        Since both are pushout squares, we get an equivalence of pointed objects $\Sigma\Omega\Sigma X\simeq\Sigma X\vee\Sigma\left(X\wedge\Omega\Sigma X\right)$ and the proposition follows from \cref{lem:suspensionCommutesWithSmash}.
    \end{proof}
\end{corollary}
\addcontentsline{toc}{subsection}{The James Splitting Theorem in $\text{Fun}(A,\spaces)$}
\subsection*{The James Splitting Theorem in $\Fun(A,\spaces)$}
We will now prove the James Splitting Theorem for general \inftytops/.
By \cref{lem:genTopoiComparisonMapEq} we will see that it suffices to prove this for \inftytops/ of the form $\Fun(A,\spaces)$ where $A$ is a small \inftycat/ by a localization argument.

We will however work in the setting of general \inftytops/ whenever possible; 
in fact, only the very last step will require us to restrict to $C=\Fun(A,\spaces)$ since we need our \inftytop/ to be hypercomplete, which is not true generally.
\begin{corollary}[Fundamental James Splitting, iterated] 
    Let $C$ be an \inftytop/, let $X$ be a pointed object in $C$ and for $i\geq 1$ let $\Sigma X^{\wedge i}$ denote $\Sigma\left(X\wedge X\wedge\ldots\wedge X\right)$ with $X$ appearing $i$ times.
    Then for all $n\geq 1$ there is a canonical equivalence
    \begin{equation*}
        \bigvee\limits_{i=1}^n\Sigma X^{\wedge i}\vee\left(X^{\wedge n}\wedge\Sigma\Omega\Sigma X\right)\simeq\Sigma\Omega\Sigma X 
    \end{equation*}
    of pointed objects induced by \cref{cor:fundamentalJamesSplitting}, and these are compatible in the sense that the diagram
    \begin{center}
        \begin{tikzcd}
            \bigvee\limits_{i=1}^n\Sigma X^{\wedge i} \ar[r] \ar[dd, hook] & \bigvee\limits_{i=1}^n\Sigma X^{\wedge i}\vee\left(X^{\wedge n}\wedge\Sigma\Omega\Sigma X\right) \ar[dr,"\simeq"]\\
            && \Sigma\Omega\Sigma X \\
            \bigvee\limits_{i=1}^{n+1}\Sigma X^{\wedge i} \ar[r] & \bigvee\limits_{i=1}^{n+1}\Sigma X^{\wedge i}\vee\left(X^{\wedge (n+1)}\wedge\Sigma\Omega\Sigma X\right) \ar[ur,"\simeq"]
        \end{tikzcd}
    \end{center}
    commutes.
    \begin{proof}
        This follows by an inductive argument:
        Given 
        \begin{equation*}
            \Sigma\Omega\Sigma X\simeq\bigvee\limits_{i=1}^n\Sigma X^{\wedge i}\vee\left(X^{\wedge n}\wedge\Sigma\Omega\Sigma X\right)
        \end{equation*}
        we obtain 
        \begin{equation*}
            \Sigma\Omega\Sigma X\simeq\bigvee\limits_{i=1}^n\Sigma X^{\wedge i}\vee\left(X^{\wedge n}\wedge\left(\Sigma X\vee\left(X\wedge\Sigma\Omega\Sigma X\right)\right)\right) 
        \end{equation*}
        by the Fundamental James Splitting \cref{cor:fundamentalJamesSplitting}.
        Using the distributivity and associativity of the smash product (\cref{lem:smashDist} and \cref{lem:assocSmash}) we get
        \begin{equation*}
            \bigvee\limits_{i=1}^n\Sigma X^{\wedge i}\vee\left(X^{\wedge n}\wedge\left(\Sigma X\vee\left(X\wedge\Sigma\Omega\Sigma X\right)\right)\right)\simeq\bigvee\limits_{i=1}^n\Sigma X^{\wedge i}\vee\left(X^{\wedge n}\wedge\Sigma X\right)\vee\left(X^{\wedge (n+1)}\wedge\Sigma\Omega\Sigma X\right)\;.
        \end{equation*}
        Lastly using \cref{lem:suspensionCommutesWithSmash} we have $\left(X^{\wedge n}\wedge\Sigma X\right)\simeq \Sigma X^{\wedge (n+1)}$ and conclude
        \begin{equation*}
            \Sigma\Omega\Sigma X\simeq\bigvee\limits_{i=1}^n\Sigma X^{\wedge i}\vee\Sigma X^{\wedge (n+1)} \vee\left(X^{\wedge (n+1)}\wedge\Sigma\Omega\Sigma X\right)\simeq\bigvee\limits_{i=1}^{n+1}\Sigma X^{\wedge i}\vee\left(X^{\wedge (n+1)}\wedge\Sigma\Omega\Sigma X\right)\;.
        \end{equation*}
        The commutativity of the diagram now follows by construction.
    \end{proof}
\end{corollary}
\begin{definition}\label{def:jamesSplittingMap}
    Let $C$ be an \inftytop/ and let $X$ be a pointed object in $C$. 
    Then the canonical compatible maps 
    \begin{equation*}
        c_X^n\colon\bigvee\limits_{i=1}^n\Sigma X^{\wedge i}\to\bigvee\limits_{i=1}^n\Sigma X^{\wedge i}\vee\left(X^{\wedge n}\wedge\Sigma\Omega\Sigma X\right)\xrightarrow{\simeq}\Sigma\Omega\Sigma X
    \end{equation*}
    from the previous corollary induce a map
    \begin{equation*}
        c_X\colon\bigvee\limits_{i=1}^{\infty}\Sigma X^{\wedge i}\to\Sigma\Omega\Sigma X
    \end{equation*}
    of pointed objects which we refer to as the \emph{comparison map}.
\end{definition}
\begin{definition}\label{def:connected}
    Let $C$ be an \inftytop/.
    With the convention that every map is $(-2)$-connected, we recursively define that a map $f\colon X\to Y$ in $C$ is \emph{$k$-connected} for $k\geq -1$ if
    \begin{itemize}
        \item it is an effective epimorphism
        \item the diagonal map $X\to X\times_YX$ is $(k-1)$-connected.
    \end{itemize} 
    If $f$ is connected for all $k\in\Nat$, then we say that $f$ is \emph{$\infty$-connected}.
    We consider an object $X$ to be \emph{$k$-connected} if the map $X\to *$ is $k$-connected.
\end{definition}
\begin{remark}
    There is ``another'' definition of connectedness in $\Top$:

    A space $X$ in $\Top$ is 
    \begin{itemize}
        \item \emph{topologically $(-1)$-connected} if $X\neq\emptyset$
        \item \emph{topologically $n$-connected} for $n\in\Nat$ if $X\neq\emptyset$ and for all $x\in X$ and $k\leq n$, we have $\pi_k(X,x)\cong 0$
    \end{itemize}
    with the convention that all spaces are \emph{topologically $(-2)$-connected}.
    A map $f\colon X\to Y$ in $\Top$ is \emph{topologically $n$-conneced} for $n\in\Nat\cup\set{-2,-1}$ if all its homotopy fibers are $n$-connected. 
    %(For authors this definition means a topologically $(n+1)$-conneced map.) %TODO maybe mention that some authors use different definition; for them, this map is (n+1)-connected

    This definition agrees with the definition of connectedness in $\spaces$:
    \begin{itemize}
        \item For $n=-2$ there is nothing to prove.
        \item For $n\geq -1$ all homotopy fibers of $f$ are nonempty, hence $f$ is an effective epimorphism (see \cref{prop:effInSpaceIffSurj}).
            Furthermore, given the pullback square
            \begin{center}
                \begin{tikzcd} [sep = 4em]
                    X\times_{Y} X \arrow[r, "g"] \arrow[d, "g"] & X \arrow[d, "f"] \\
                    X \arrow[r, "f"] & Y \\
                \end{tikzcd}
            \end{center}
            in $\spaces$, we claim the map $g$ is topologically $n$-connected if and only if $f$ is topologically $n$-connected:
            
            From the fiberwise characterization of homotopy pullbacks \cref{prop:fiberwiseCharOfHtpyPb}, it follows directly that if $f$ is topologically $n$-connected, $g$ is as well.

            Since homotopy fibers are essentially constant on path-connected components and $f$ is an effective epimorphism, the fiberwise characterization of homotopy pullbacks also shows that $f$ is topologically $n$-connected if $g$ is topologically $n$-connected.

            Furthermore, the universal map $s\colon X\to X\times_{Y} X$ is a section of $g$ in $\spaces$.
            By inspection of the long exact sequences of homotopy groups, $g$ is topologically $n$-connected if and only if the section $s$ is topologically $(n-1)$-connected.

            This allows us to recursively reduce to the case of $f\colon X\to Y$ being topologically $(-2)$-connected, in which we have already seen that both definitions coincide.
    \end{itemize}
\end{remark}
\begin{definition}
    We say that an \inftytop/ $C$ is \emph{hypercomplete} if every $\infty$-connected map is an equivalence.
\end{definition}
\begin{remark}\label{rmk:spacesHypercomplete}
    Every equivalence is $\infty$-connected, however the converse is not true for general \inftytops/.
    For every small \inftycat/ $A$, $\Fun(A,\spaces)$ is an \inftytop/ that is hypercomplete.
    This can be seen using \cite[Remark 6.5.4.7]{HTT}.
\end{remark}
Our goal now is to prove that the comparison map is an equivalence whenever $X$ is a $0$-connected pointed object.
We will prove this only in $\Fun(A,\spaces)$, however this case is already enough to derive the result for general topoi (see \cref{lem:genTopoiComparisonMapEq}).

The main idea is to show that $X^{\wedge n}\wedge\Sigma\Omega\Sigma X$ becomes increasingly connected and vanishes as $n\to\infty$.
For this proof we need some stability properties of connectedness.
\begin{prop}\label{prop:conn}
    Let $C$ be an \inftytop/ and $k\geq -2$.
    Then
    \begin{enumerate}[label={(\roman*)}]
        \item $k$-connected morphisms are stable under pushout. \label{prop:connStableUnderPo}
        \item $k$-connected morphisms are stable under pullback.\label{prop:connStableUnderPb}
        \item for morphisms $f\colon X\to Y$ and $g\colon Y\to Z$ with $f$ $k$-connected, the morphism $g$ is $k$-connected if and only if $gf$ is $k$-connected. \label{prop:connRightCancel}
        \item for a morphism $f\colon X\to Y$ with a section $s\colon Y\to X$, $f$ is $(k+1)$-connected if and only if $s$ is $k$-connected. \label{prop:connSection}
        \item the class of $k$-connected objects is stable under filtered colimits. \label{prop:connStableFilteredColim}
    \end{enumerate}
    \begin{reference}
        \cite[Proposition 4.10]{splittings_21}
    \end{reference}
\end{prop}
\begin{thm}[Blakers-Massey Theorem]\label{prop:blakersMassey}
    Let $C$ be an \inftytop/ and let 
    \begin{center}
        \begin{tikzcd} [sep = 4em]
            A \arrow[r, "g"] \arrow[d, "f"] &  C \arrow[d] \\
            B \arrow[r] & D \\
        \end{tikzcd}
    \end{center}
    be a pushout square in $C$.
    Then if $f$ is $n$-connected and $g$ is $m$-connected, the canonical map $A\to B\times_{D}C$ is $(n+m)$-connected.
    \begin{reference}
        A very general version for \inftytops/ can be found in \cite[Corollary 4.3.1]{gen_blakers_massey}, but the version used here can also be proven by a reduction to topoi of the form $\Fun(A,\spaces)$ similar to \cref{lem:genTopoiComparisonMapEq} and thus derived from the ordinary Blakers-Massey Theorem in $\Top$.
        The argument for the latter can be found in \cite[Proposition 8.16]{toposes_and_htpy_toposes}.
    \end{reference}
\end{thm}
With these we can deduce the following statements:
\begin{corollary}\label{cor:connEst}
    Let $C$ be an \inftytop/, $X,Y$ pointed objects in $C$ and $k,l\geq 0$. 
    If $X$ is $k$-connected and $Y$ is $l$-connected, then
    \begin{enumerate}[label={(\roman*)}]
        \item the suspension $\Sigma X$ is $(k+1)$-connected. \label{cor:connEstSuspension}
        \item the canonical map $X\to X\vee Y$ is $(l-1)$-connected. \label{cor:connEstVeeIncl}
        \item the smash product $X\wedge Y$ is $(k + l + 1)$-connected. \label{cor:connEstSmash}
        \item for every $n\geq 0$, the smash product $X^{\wedge n}$ is $(n(k+1)-1)$-connected. \label{cor:connEstIteratedSmash}
    \end{enumerate}
    \begin{proof}
        \ref{cor:connEstSuspension}: Since $X$ is $k$-connected by \cref{prop:conn} \ref{prop:connStableUnderPo} the pushout of the map $X\to *$ which is a map $*\to \Sigma X$ is $k$-connected.
        Thus by \cref{prop:conn} \ref{prop:connSection} the map $\Sigma X\to *$ is $(k+1)$-connected.

        \ref{cor:connEstVeeIncl}: Since the map $*\to Y$ is $(l-1)$-connected by \cref{prop:conn} \ref{prop:connSection}, by \cref{prop:conn} \ref{prop:connStableUnderPo} the map $X\to X\vee Y$ is also $(l-1)$-connected.
        
        \ref{cor:connEstSmash}: By \cref{prop:blakersMassey} and \cref{lem:poOfCollapseMapsIsTrivial} we have that $X\vee Y\to X\times Y$ is  $(k+l)$-connected. 
        By \cref{prop:conn} \ref{prop:connStableUnderPo} the map $*\to X\wedge Y$ is thus also $(k+l)$-connected and so by \cref{prop:conn} \ref{prop:connSection} the space $X\wedge Y$ is $(k+l+1)$-connected.

        \ref{cor:connEstIteratedSmash}: This follows by induction on \ref{cor:connEstSmash}.
    \end{proof}
\end{corollary}
\begin{prop}[James Splitting for Presheaf Categories]\label{prop:jamesSplittingPresheaf} %ref Proposition 4.18 
    Let $A$ be a small \inftycat/ and let $X$ be a pointed and $0$-connected object in the \inftytop/ $\Fun(A,\spaces)$.
    Then the comparison map 
    \begin{equation*}
        c_X\colon\bigvee\limits_{i=1}^{\infty}\Sigma X^{\wedge i}\to\Sigma\Omega\Sigma X
    \end{equation*}
    is an equivalence of pointed objects. 
    \begin{proof}
        Since $\Fun(A,\spaces)$ is hypercomplete it suffices to prove that $c_X$ is $\infty$-connected.

        First for each $n\in\Nat_{\geq 1}$ we have $2$-simplices in $\Fun(A,\spaces)$
        \begin{center}
            \begin{tikzcd} [sep = 4em]
                \bigvee\limits_{i=1}^{n}\Sigma X^{\wedge i} \arrow[r] \arrow[dr] & \bigvee\limits_{i=1}^{\infty}\Sigma X^{\wedge i} \arrow[d, "c_X"] \\
                & \Sigma\Omega\Sigma X \\
            \end{tikzcd}
        \end{center}
        induced by the universal property of $\bigvee\limits_{i=1}^{\infty}\Sigma X^{\wedge i}$.

        By \cref{cor:connEst} \ref{cor:connEstIteratedSmash} $X^{\wedge n}$ is $(n-1)$-connected.
        By \cref{prop:conn} \ref{prop:connStableUnderPo} and \cref{prop:conn} \ref{prop:connStableUnderPb} we have that $\Omega\Sigma X$ is $0$-connected, so by \cref{cor:connEst} \ref{cor:connEstSuspension} $\Sigma\Omega\Sigma X$ is $1$-connected and combining this via \cref{cor:connEst} \ref{cor:connEstSmash} shows that $X^{\wedge n}\wedge\Sigma\Omega\Sigma X$ is $(n+1)$-connected.

        So the map 
        \begin{equation*}
            c_X^n\colon\bigvee\limits_{i=1}^n\Sigma X^{\wedge i}\to\bigvee\limits_{i=1}^n\Sigma X^{\wedge i}\vee\left(X^{\wedge n}\wedge\Sigma\Omega\Sigma X\right)\xrightarrow{\cong}\Sigma\Omega\Sigma X
        \end{equation*}
        corresponding to the diagonal map of the above triangle is $n$-connected by \cref{cor:connEst} \ref{cor:connEstVeeIncl}.

        Since by \cref{cor:connEst} \ref{cor:connEstIteratedSmash} and \cref{cor:connEst} \ref{cor:connEstSmash} every object of the filtered colimit $\bigvee\limits_{i=n+1}^{\infty}\Sigma X^{\wedge i}$ is at least $(n+1)$-connected, by \cref{prop:conn} \ref{prop:connStableFilteredColim} it is itself $(n+1)$-connected.
        So by \cref{cor:connEst} \ref{cor:connEstVeeIncl} we have that the horizontal map of the triangle $\bigvee\limits_{i=1}^n\Sigma X^{\wedge i}\to\bigvee\limits_{i=1}^n\Sigma X^{\wedge i}\vee\bigvee\limits_{i=n+1}^{\infty}\Sigma X^{\wedge i}\cong\bigvee\limits_{i=1}^{\infty}\Sigma X^{\wedge i}$ is $n$-connected.

        As the diagonal as well as the horizontal map of the triangle above are $n$-connected, from \cref{prop:conn} \ref{prop:connRightCancel} it follows that $c_X$ is $n$-connected. 
        As $n$ was arbitrary, $c_X$ is therefore an equivalence.
    \end{proof}
\end{prop}
\begin{corollary}[Classical James Splitting]\label{cor:classicalJamesSplitting}
    Let $X$ be a pointed and $0$-connected object in $\spaces$.
    Then     
    \begin{equation*}
        c_X\colon\bigvee\limits_{i=1}^{\infty}\Sigma X^{\wedge i}\to\Sigma\Omega\Sigma X
    \end{equation*}
    is an equivalence of pointed objects. 
    \begin{proof}
        This is \cref{prop:jamesSplittingPresheaf} with $A=\Delta^0$.
    \end{proof}
\end{corollary}
\begin{remark}
    The connectedness assumption on $X$ cannot be dropped.
    Setting $X=S^0$ in $\spaces$, the equivalence $\Sigma X\vee (X\wedge \Sigma\Omega\Sigma X)\simeq\Sigma\Omega\Sigma X$ from \cref{cor:fundamentalJamesSplitting} can be identified with
    \begin{equation*}
        S^1\vee \bigvee\limits_{\Z\setminus\set{1}}S^1\xrightarrow{\simeq}\bigvee\limits_{\Z}S^1
    \end{equation*}
    as $\Sigma\Omega\Sigma S^0\simeq\bigvee\limits_{\Z}S^1$.
    This can be used to show that $c_{S^0}$ is equivalent to the map
    \begin{equation*}
        \bigvee\limits_{i=1}^{\infty} S^1\to\bigvee\limits_{\Z} S^1
    \end{equation*}
    corresponding to the inclusion $\Nat_{\geq1}\subset\Z$ on components (see \cite[Warning 2.13]{splittings_21}).
    This map is not an equivalence:

    By \cite[Proposition 1A.2]{hatcher2002algebraic} we have that $\pi_1\left(\bigvee\limits_{i=1}^{\infty} S^1\right)$ and $\pi_1\left(\bigvee\limits_{\Z} S^1\right)$ are both free groups with generators the canonical loops corresponding to their components.
    Since the loops of generators corresponding to non-positive components of $\pi_1\left(\bigvee\limits_{\Z} S^1\right)$ are not in the image of the induced map $\pi_1(c_{S^0})\colon\pi_1\left(\bigvee\limits_{i=1}^{\infty} S^1\right)\to\pi_1\left(\bigvee\limits_{\Z} S^1\right)$, this means the map $\pi_1(c_{S^0})$ is not surjective.
\end{remark}
\begin{thm}\label{thm:charTopoiByAdj}
    Let $C$ be an \inftytop/.
    Then there exists a small \inftycat/ $A$ and an adjunction
    \begin{equation*}
        L\colon\Fun(A,\spaces)\rightleftarrows C\mkern+3mu:\mkern-3mu i
    \end{equation*}
    such that the left adjoint $L$ preserves finite limits and the right adjoint $i$ is fully faithful.

    Furthermore, given $-2\leq m\leq\infty$ and an $m$-connected morphism $f\colon x\to y$ in $C$, there exists an $m$-connected morphism $g\colon i(x)\to y'$ in $\Fun(A,\spaces)$ such that $L(g)$ is equivalent to $f$ in $\Fun(\Delta^1,C)$.
    \begin{reference}
        \cite[Theorem 6.1.0.6 (1) and (2)]{HTT} for the adjunction and \cite[Remark 6.5.1.15]{HTT} for connectedness statement.
    \end{reference}
\end{thm}
\begin{corollary}[James Splitting for \inftyTops/]\label{lem:genTopoiComparisonMapEq}
    Let $C$ be an \inftytop/ and $X$ be a pointed and $0$-connected object in $C$.
    Then the canonical map 
    \begin{equation*}
        c_X\colon\bigvee\limits_{i=1}^{\infty}\Sigma X^{\wedge i}\to\Sigma\Omega\Sigma X
    \end{equation*}
    from \cref{def:jamesSplittingMap} is an equivalence of pointed objects.
    \begin{proof}
        Let $L$ and $i$ be the induced adjoint functors from \cref{thm:charTopoiByAdj} and let $X$ be a pointed $0$-connected object in $C$ (in particular, by \cref{prop:conn} \ref{prop:connSection} $*\to C$ is $(-1)$-connected).
        As both functors $L$ and $i$ preserve finite limits and thus preserve terminal objects, they descend to the categories of pointed objects.
        
        Again using \cref{thm:charTopoiByAdj} we can pick a map $*\to Y$ in $\Fun(A,\spaces)$ that is $(-1)$-connected such that $L(Y)$ is equivalent to $X$ as pointed spaces.
        The space $Y$ is $0$-connected (again by \cref{prop:conn} \ref{prop:connSection}), thus the map $c_Y$ in $\Fun(A,\spaces)$ is an equivalence by \cref{prop:jamesSplittingPresheaf}.
        
        Note that since $L$ preserves colimits and finite limits, it preserve all colimits/limits appearing in $c_Y$, thus $L(c_Y)\simeq c_{L(Y)}$ in $C_*$.
        But since $L(Y)\simeq X$, this means that $c_{L(Y)}\simeq c_X$ in $C_*$ which proves that $c_X$ is an equivalence in $C_*$.
    \end{proof}
\end{corollary}
Lastly, we give an application of the James Splitting Theorem.
\begin{lemma}\label{lem:reducedHomLoopOfSpheres}
    Let $H_*$ be a reduced homology theory and let $S^n$ with $n\geq1$ be a model of the pointed $n$-sphere.
    Then
    \begin{equation*}
        H_k(\Omega S^{n+1})\cong\bigoplus\limits_{i=1}^{\infty}H_k(S^{ni})\;.
    \end{equation*}
    \begin{proof}
        Note that for $n\geq1$ the sphere is $0$-connected.
        Thus by \cref{cor:classicalJamesSplitting} we have an isomorphism 
        \begin{equation*}
            H_{k+1}\left(\bigvee\limits_{i=1}^{\infty}\Sigma \left(S^n\right)^{\wedge i}\right)\cong H_{k+1}\left(\Sigma\Omega\Sigma S^n\right)\;.
        \end{equation*}
        This gives an isomorphism
        \begin{equation*}
            \bigoplus\limits_{i=1}^{\infty}H_k\left(\left(S^n\right)^{\wedge i}\right)\cong H_k\left(\Omega S^{n+1}\right)
        \end{equation*}
        and the fact that $S^{n_1}\wedge S^{n_2}\cong S^{n_1+n_2}$ proves the claim.
    \end{proof}
\end{lemma}
\begin{corollary}
    For $H_*$ reduced ordinary homology and $S^n$ with $n\geq1$ a model of the pointed $n$-sphere, we have
    \begin{equation*}
        H_k(\Omega S^{n+1})\cong
        \begin{cases}
            \mathbb{Z} & k=ni ~for~ i\in\Nat\\
            0 & otherwise
        \end{cases}\;.
    \end{equation*}
    \begin{proof}
        This follows from \cref{lem:reducedHomLoopOfSpheres} and the reduced ordinary homology of the spheres.
    \end{proof}
\end{corollary}
