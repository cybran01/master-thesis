In this section we will give a proof of the James Splitting Theorem relying only on universality, therefore in particular proving it for general \inftytops/. 
The main source is \cite[\S 2.2 and \S 4]{splittings_21}.

We begin by introducing the objects that we will need to work with.
\begin{definition}
    Let $C$ be an \inftytop/ and let $X,Y\in \left(C_*\right)_0$ be pointed objects. 
    Then we let 
    \begin{itemize}
        \item the pushout
        \begin{center}
            \begin{tikzcd} [sep = 4em]
                X \arrow[r] \arrow[d] & * \arrow[d, "i_0"] \\
                * \arrow[r, "i_1"] & \Sigma X \\
            \end{tikzcd}
        \end{center}
        be the \emph{suspension of $X$}
        \item the pullback
        \begin{center}
            \begin{tikzcd} [sep = 4em]
                \Omega Y \arrow[r] \arrow[d] & * \arrow[d] \\
                * \arrow[r] & Y \\
            \end{tikzcd}
        \end{center}
        be the \emph{loop space of $Y$}
        \item the pushout
        \begin{center}
            \begin{tikzcd} [sep = 4em]
                * \arrow[r] \arrow[d] & Y \arrow[d] \\
                X \arrow[r] & X\vee Y \\
            \end{tikzcd}
        \end{center}
        be the \emph{wedge sum of $X$ and $Y$} 
        \item and the pushout 
        \begin{center}
            \begin{tikzcd} [sep = 4em]
                X\vee Y \arrow[r] \arrow[d] & X\times Y \arrow[d] \\
                * \arrow[r] & X\wedge Y \\
            \end{tikzcd}
        \end{center}
        be the \emph{smash product of $X$ and $Y$} (where the map $X\vee Y\to X\times Y$ is the canonical map from the universal property induced by the maps $X\vee Y\to X$ and $X\vee Y\to Y$).
    \end{itemize}
\end{definition}
The next lemma is a key lemma of the proof of the James Splitting Theorem.
There is some subtlety involved in it, see \cref{rmk:mapNotProj}.
\begin{lemma}\label{lem:existenceOfPoSq}
    Let $C$ be an \inftytop/ and let $X\in \left(C_*\right)_0$ be a pointed object.
    Then there exists a pushout square  
    \begin{center}
        \begin{tikzcd} [sep = 4em]
            X\times\Omega\Sigma X \arrow[r, "\pi_2"] \arrow[d, "\alpha"] & \Omega\Sigma X \arrow[d] \\
            \Omega\Sigma X \arrow[r] & * \\
        \end{tikzcd}\;.
    \end{center}
    \begin{proof}
        The definition of $\Sigma X$
        \begin{center}
            \begin{tikzcd} [sep = 4em]
                X \arrow[r] \arrow[d] & * \arrow[d, "i_0"] \\
                * \arrow[r, "i_1"] & \Sigma X \\
            \end{tikzcd}
        \end{center}
        shows that $i_1$ and $i_2$ are equivalent maps in $C$ (meaning identical in $\ho(C)$), since $X\to *\xrightarrow{i_0}\Sigma X$ and $X\to *\xrightarrow{i_1}\Sigma X$ are identical in $\ho(C)$ by construction and the canonical point $x\colon*\to X$ is a right inverse to $X\to *$.
        Thus we get that both
        \begin{center}
            \begin{tikzcd} [sep = 4em]
                \Omega\Sigma X \arrow[r] \arrow[d] & * \arrow[d, "i_0"] \\
                * \arrow[r, "i_0"] & \Sigma X \\
            \end{tikzcd}
            \begin{tikzcd} [sep = 4em]
                \Omega\Sigma X \arrow[r] \arrow[d] & * \arrow[d, "i_0"] \\
                * \arrow[r, "i_1"] & \Sigma X \\
            \end{tikzcd}
        \end{center}
        are pullback squares.
        Now we pick a product together with projections to obtain a pullback square 
        \begin{center}
            \begin{tikzcd} [sep = 4em]
                X\times\Omega\Sigma X \arrow[r, "\pi_2"] \arrow[d, "\pi_1"] & \Omega\Sigma X \arrow[d] \\
                X \arrow[r] & * \\
            \end{tikzcd}
        \end{center}
        and assemble these pullback squares into a diagram
        \begin{center}
            \begin{tikzcd} [sep = .5 cm]
                X\times\Omega\Sigma X \arrow [rr, "\pi_2"] \arrow [dd, "\pi_1"] & & \Omega\Sigma X \arrow [dr] \arrow[dd] \\
                & \Omega\Sigma X \arrow [rr, crossing over] & & * \arrow [dd, "i_0"] & \\
                X \arrow [dr] \arrow [rr] & & * \arrow [dr, "i_0"] \\
                & * \arrow [from=uu,crossing over] \arrow [rr, "i_1"] & & \Sigma X &
            \end{tikzcd}
        \end{center}
        where three vertical squares are pullback and the bottom square is a pushout.
        We now want to construct a map $\alpha\colon X\times\Omega\Sigma X\to\Omega\Sigma X$ such that we have a commutative cube with all vertical sides pullbacks.
        By e.g. \cite[Lemma 1.4.7.5]{kerodon} the map $\Lambda_1^2\times\Delta^1\cup_{\Lambda_1^2\times\set{1}}\Delta^2\times\set{1
        }\to\Delta^2\times\Delta^1$ is inner anodyne, so we can lift
        \begin{center}
            \begin{tikzcd} [sep = .5 cm]
                X\times\Omega\Sigma X \arrow [rr, "\pi_2"] \arrow [dd, "\pi_1"] & & \Omega\Sigma X \arrow [dr] \arrow[dd] \\
                & & & * \arrow [dd, "i_0"] & \\
                X \arrow [drrr] \arrow [rr] & & * \arrow [dr, "i_0"] \\
                & & & \Sigma X &
            \end{tikzcd}
        \end{center}
        and obtain 
        \begin{center}
            \begin{tikzcd} [sep = .5 cm]
                X\times\Omega\Sigma X \arrow [rr, "\pi_2"] \arrow [dd, "\pi_1"] & & \Omega\Sigma X \arrow [dr] \arrow[dd] \\
                & \Omega\Sigma X \arrow [rr, crossing over] & & * \arrow [from=ulll, crossing over, shorten >= 1.1em]  \arrow [dd, "i_0"] & \\
                X \arrow [dr] \arrow [rr] \arrow [drrr, shorten >= 3pt] & & * \arrow [dr, "i_0"] \\
                & * \arrow [from=uu, crossing over] \arrow [rr, swap, "i_1"] & & \Sigma X &
            \end{tikzcd}
        \end{center}
        where the back prism commutes.
        In the subdiagram
        \begin{center}
            \begin{tikzcd} [sep = .5 cm]
                X\times\Omega\Sigma X \arrow [dd] \arrow [dddrrr, "j"] & & \\
                & & & \\
                X \arrow [dr] \arrow [drrr, shorten >= 3pt] & & \\
                & * \arrow[from=uuul, crossing over] \arrow [rr, swap, "i_1"] & & \Sigma X\\
            \end{tikzcd}
        \end{center}
        all faces except the one opposite to $X$ commute and since this corresponds to a map $\Lambda^3_1\to C$, we can lift it to a map $\Delta^3\to C$.
        Thus we can construct the commutative square
        \begin{center}
            \begin{tikzcd} [sep = 4em]
                X\times\Omega\Sigma X \arrow[r] \arrow[d] \arrow[dr, "j"] & * \arrow[d, "i_0"] \\
                * \arrow[r, "i_1"] & \Sigma X \\
            \end{tikzcd}
        \end{center}
        and this induces a map $\Delta^1\star\left(\Delta^1\times\Delta^1\right)\to C$ corresponding to a diagram
        \begin{center}
            \begin{tikzcd} [sep = .5 cm]
                X\times\Omega\Sigma X \arrow[dr, near end, "\alpha", shorten >=-0.5em] \arrow [drrr] \arrow [dash, dddrrr, shorten >= 2.6cm] \arrow [dddrrr, near end, "j", shorten <= 2.3cm] & & \\
                & \Omega\Sigma X \arrow [dd,  crossing over] \arrow [rr] & & * \arrow[dd, "i_0"]\\
                & & \\
                & * \arrow[from=uuul] \arrow [rr, swap, "i_1"] & & \Sigma X\\
            \end{tikzcd}
        \end{center}
        since
        \begin{center}
            \begin{tikzcd} [sep = 4em]
                \Omega\Sigma X \arrow[r] \arrow[d] & * \arrow[d, "i_0"] \\
                * \arrow[r, "i_1"] & \Sigma X \\
            \end{tikzcd}
        \end{center}
        is a pullback square.
        This fills the remaining simplices of the desired cube.

        The outer square of 
        \begin{center}
            \begin{tikzcd} [sep = 4em]
                X\times\Omega\Sigma X \arrow[d, "\pi_1"] \arrow[r, "\alpha"] & \Omega\Sigma X \arrow[r] \arrow[d] & * \arrow[d, "i_0"] \\
                X \arrow[r] & * \arrow[r, "i_1"] & \Sigma X \\
            \end{tikzcd}
        \end{center}
        is a pullback square since 
        \begin{center}
            \begin{tikzcd} [sep = 4em]
                X\times\Omega\Sigma X \arrow[d, "\pi_1"] \arrow[r, "\pi_2"] & \Omega\Sigma X \arrow[r] \arrow[d] & * \arrow[d, "i_0"] \\
                X \arrow[r] & * \arrow[r, "i_0"] & \Sigma X \\
            \end{tikzcd}
        \end{center}
        is one and top and their top and bottom maps are equivalent.
        Thus by the pasting law all vertical squares of the cube
        \begin{center}
            \begin{tikzcd} [sep = .5 cm]
                X\times\Omega\Sigma X \arrow [dr, "\alpha"] \arrow [rr, "\pi_2"] \arrow [dd, "\pi_1"] & & \Omega\Sigma X \arrow [dr] \arrow[dd] \\
                & \Omega\Sigma X \arrow [rr, crossing over] & & * \arrow [dd, "i_0"] & \\
                X \arrow [dr] \arrow [rr] & & * \arrow [dr, "i_0"] \\
                & * \arrow [from=uu, crossing over] \arrow [rr, "i_1"] & & \Sigma X &
            \end{tikzcd}
        \end{center}
        are pullbacks and since the bottom square is a pushout, by universality the top square is pushout which finishes the proof.
    \end{proof}
\end{lemma}
\begin{remark}\label{rmk:mapNotProj}
    The map $\alpha\colon X\times\Omega\Sigma X\to\Omega\Sigma X$ can in general not be identified with the projection $\pi_2$.
    This is because we are not able to choose the map $\alpha$ freely since the higher simplices must be compatible along the ``diagonal'' of our cube (which our map $\alpha$ fulfills by construction).

    If we assume $\alpha=\pi_2$ then by universality we have that 
    \begin{center}
        \begin{tikzcd} [sep = 4em]
            X\times \Omega\Sigma X \arrow[r,"\pi_2"] \arrow[d, "\pi_2"] & \Omega\Sigma X \arrow[d] \\
            \Omega\Sigma X \arrow[r] & \Sigma X \times \Omega\Sigma X\\
        \end{tikzcd}
    \end{center}
    is a pushout.
    But $\Omega\Sigma X\times \Sigma X$ is not generally contractible. 
\end{remark}
\begin{corollary}\label{lem:poOfProductIsSuspension}
    Let $C$ be an \inftytop/ and let $X\in \left(C_*\right)_0$ be a pointed object. 
    Then there exists a square
    \begin{center}
        \begin{tikzcd} [sep = 4em]
            X\times\Omega\Sigma X \arrow[r, "\pi_2"] \arrow[d] & \Omega\Sigma X \arrow[d] \\
            * \arrow[r] & \Sigma\Omega\Sigma X \\
        \end{tikzcd}
    \end{center}
    that is a pushout.
    \begin{proof}
        This follows from \cref{lem:existenceOfPoSq} and the pasting lemma applied to 
        \begin{center}
            \begin{tikzcd} [sep = 4em]
                X\times\Omega\Sigma X \arrow[r, "\pi_2"] \arrow[d, "\alpha"] & \Omega\Sigma X \arrow[d] \\
                \Omega\Sigma X \arrow[r] \arrow[d] & * \arrow[d] \\
                * \arrow[r] & \Sigma\Omega\Sigma X \\
            \end{tikzcd}\;.
        \end{center}
    \end{proof}
\end{corollary}
We will also need to prove some basic facts about wedges and smash products.
\begin{lemma}\label{lem:poOfCollapseMapsIsTrivial}
    Let $C$ be an \inftytop/ and let $X,Y$ be pointed objects in $C$. 
    Then 
    \begin{center}
        \begin{tikzcd} [sep = 4em]
            X\vee Y \arrow[r] \arrow[d] & Y \arrow[d] \\
            X \arrow[r] & * \\
        \end{tikzcd}
    \end{center}
    is a pushout.
    \begin{proof}
        This follows from repeatedly applying the pasting law to 
        \begin{center}
            \begin{tikzcd} [sep = 4em]
                * \arrow[r] \arrow[d] & Y \arrow[r] \arrow[d] & * \arrow[d] \\
                X \arrow[r] \arrow[d] & X\vee Y \arrow[r] \arrow[d] & X \arrow[d]\\
                * \arrow[r] & Y \arrow[r] & * \\
            \end{tikzcd}\;.
        \end{center}
    \end{proof}
\end{lemma}
\begin{lemma}\label{lem:poSqWithFactoringTerminal}
    Let $C$ be an \inftytop/ and let $X,Y$ be pointed objects in $C$. 
    Then there is a pushout square
    \begin{center}
        \begin{tikzcd} [sep = 4em]
            X\times Y \arrow[r, "\pi_2"] \arrow[d, "\pi_1"] & Y \arrow[d] \\
            X \arrow[r] & \Sigma\left(X\wedge Y\right) \\
        \end{tikzcd}
    \end{center}
    such that the unlabeled morphisms factor through the terminal object.
    \begin{proof}
        The colimit of the diagram 
        \begin{center}
            \begin{tikzcd} [sep = 4em]
                * & * \arrow[l] \arrow[r] & * \\
                X \arrow[d, equal] \arrow[u] & X\vee Y \arrow[l]  \arrow[r] \arrow[d] \arrow[u] & Y \arrow[d, equal] \arrow[u] \\
                X  & X\times Y \arrow[l] \arrow[r] & Y \\
            \end{tikzcd}
        \end{center}
        can be computed by either taking pushouts of the rows and then the pushout of the resulting diagram, or taking pushouts of the columns and then taking pushout of the resulting diagram.
        
        Taking pushouts of the rows results in the diagram $*\xleftarrow{}*\rightarrow X\cup_{X\times Y} Y$ where we use \cref{lem:poOfCollapseMapsIsTrivial} and taking pushouts of the columns results in the diagram $*\xleftarrow{}X\wedge Y\rightarrow *$. 
        Since the pushout of the latter diagram is $\Sigma\left(X\wedge Y\right)$ and the pushout of the first diagram must be equivalent to this, we have $X\cup_{X\times Y} Y\cong\Sigma\left(X\wedge Y\right)$.
        The maps $X\to\Sigma\left(X\wedge Y\right)$ and $Y\to\Sigma\left(X\wedge Y\right)$ factor through $*$ by construction which proves the proposition.
    \end{proof}
\end{lemma}
\begin{lemma}\label{lem:allSqArePo}
    Let $C$ be an \inftytop/ and let $X,Y$ be pointed objects in $C$. 
    Then all squares of the diagram         
    \begin{center}
        \begin{tikzcd} [sep = 4em]
            X\times Y \arrow[r, "\pi_2"] \arrow[d, "\pi_1"] & Y \arrow[r] \arrow[d] & * \arrow[d] \\
            X \arrow[r] \arrow[d] & \Sigma\left(X\wedge Y\right) \arrow[r] \arrow[d] & \Sigma Y\vee\Sigma\left(X\wedge Y\right) \arrow[d]\\
            * \arrow[r] &  \Sigma X\vee\Sigma\left(X\wedge Y\right)\arrow[r] & \Sigma X\vee\Sigma Y\vee\Sigma\left(X\wedge Y\right) \\
        \end{tikzcd}
    \end{center}
    are pushouts.
    \begin{proof}
        By \cref{lem:poSqWithFactoringTerminal} the top left square is a pushout square.
        The top right square is a pushout by the pasting law applied to
        \begin{center}
            \begin{tikzcd} [sep = 4em]
                Y \arrow[r] \arrow[d] & * \arrow[d] \\
                * \arrow[r] \arrow[d] & \Sigma Y \arrow[d] \\
                \Sigma\left(X\wedge Y\right) \arrow[r] & \Sigma Y\vee\Sigma\left(X\wedge Y\right) \\
            \end{tikzcd}\
        \end{center}
        because $Y\to\Sigma\left(X\wedge Y\right)$ factors through $*$ (again by \cref{lem:poSqWithFactoringTerminal}).
        The analogous argument also shows that the bottom left square is a pushout.

        The bottom right square is a pushout since for pointed objects $A,B,C\in \left(C_*\right)_0$ the square
        \begin{center}
            \begin{tikzcd} [sep = 4em]
                B \arrow[r] \arrow[d] &  B\vee C \arrow[d] \\
                A\vee B \arrow[r] & A\vee B\vee C\\
            \end{tikzcd}
        \end{center}
        is always a pushout and the wedge sum is commutative.
    \end{proof}
\end{lemma}
The following corollary is not needed for the James Splitting Theorem, however it is useful and immediate from what we have already shown.
\begin{corollary}
    Let $C$ be an \inftytop/ and let $X,Y$ be pointed objects in $C$. 
    Then we have $\Sigma\left(X\times Y\right)\cong\Sigma X\vee\Sigma Y\vee\Sigma\left(X\wedge Y\right)$.
    \begin{proof}
        The outer square of \cref{lem:allSqArePo} proves this immediately. 
    \end{proof}
\end{corollary}
\begin{lemma}\label{lem:suspensionCommutesWithSmash}
    Let $C$ be an \inftytop/ and let $X,Y$ be pointed objects in $C$. 
    Then there is a canonical equivalence $\Sigma\left(X\wedge Y\right)\cong X\wedge\Sigma Y$.
    \begin{proof}
        The squares
        \begin{center}
            \begin{tikzcd} [sep = 4em]
                X\vee Y \arrow[r] \arrow[d] &  X\vee * \arrow[d] \\
                X\vee * \arrow[r] & X\vee\Sigma Y\\
            \end{tikzcd}
            \begin{tikzcd} [sep = 4em]
                X\times Y \arrow[r] \arrow[d] &  X\times * \arrow[d] \\
                X\times * \arrow[r] & X\times\Sigma Y\\
            \end{tikzcd}\
        \end{center}
        are both pushouts; the first one as colimits commute and the second one due to universality. %TODO maybe explain at some earlier point how universality is equivalent to base change
        Computing the pushout of the diagram as in \cref{lem:poSqWithFactoringTerminal}
        \begin{center}
            \begin{tikzcd} [sep = 4em]
                * & * \arrow[l] \arrow[r] & * \\
                X\vee * \arrow[d, equal] \arrow[u] & X\vee Y \arrow[l]  \arrow[r] \arrow[d] \arrow[u] & X\vee * \arrow[d, equal] \arrow[u] \\
                X\times *  & X\times Y \arrow[l] \arrow[r] & X\times * \\
            \end{tikzcd}
        \end{center}
        we get $X\wedge\Sigma Y$ by computing the pushout of the pushouts of the rows and $\Sigma\left(X\wedge Y\right)$ by computing the pushout of the pushouts of the columns, which proves the proposition.
    \end{proof}
\end{lemma}
\begin{corollary}[Fundamental James Splitting]\label{cor:fundamentalJamesSplitting}
    Let $C$ be an \inftytop/ and let $X$ be a pointed object in $C$. 
    Then there exists a canonical equivalence $\Sigma\Omega\Sigma X\cong\Sigma X\vee\left(X\wedge\Sigma\Omega\Sigma X\right)$.
    \begin{proof}
        Applying \cref{lem:allSqArePo} to $X$ and $Y=\Omega\Sigma X$, the left vertical and top horizontal maps of the big left square (top and bottom left squares combined) are equivalent to the maps of the square in \cref{lem:poOfProductIsSuspension}.
        Since both are pushout squares, we get an equivalence $\Sigma\Omega\Sigma X\cong\Sigma X\vee\Sigma\left(X\wedge\Omega\Sigma X\right)$ and the proposition follows from \cref{lem:suspensionCommutesWithSmash}.
    \end{proof}
\end{corollary}
\begin{corollary}[Fundamental James Splitting, iterated]
    Let $C$ be an \inftytop/ and let $X$ be a pointed object in $C$. 
    Then for all $n\geq 1$ there is a canonical equivalence
    \begin{equation*}
        \Sigma\Omega\Sigma X \cong\bigvee\limits_{i=1}^n\Sigma X^{\wedge i}\vee\left(X^{\wedge n}\wedge\Sigma\Omega\Sigma X\right)
    \end{equation*}
    induced by \cref{cor:fundamentalJamesSplitting}.
    Here $\Sigma X^{\wedge i}$ denotes $\Sigma\left(X\wedge X\wedge\ldots\wedge X\right)$ with $X$ appearing $i$ times.
    \begin{proof}
        This follows inductively from \cref{cor:fundamentalJamesSplitting} and \cref{lem:suspensionCommutesWithSmash}.
    \end{proof}
\end{corollary}
\begin{definition}
    Let $C$ be an \inftytop/ and let $X$ be a pointed object in $C$. 
    Then the canonical maps 
    \begin{equation*}
        c_X^n\colon\bigvee\limits_{i=1}^n\Sigma X^{\wedge i}\to\bigvee\limits_{i=1}^n\Sigma X^{\wedge i}\vee\left(X^{\wedge n}\wedge\Sigma\Omega\Sigma X\right)\xrightarrow{\cong}\Sigma\Omega\Sigma X
    \end{equation*}
    induce a map
    \begin{equation*}
        c_X\colon\bigvee\limits_{i=1}^{\infty}\Sigma X^{\wedge i}\to\Sigma\Omega\Sigma X
    \end{equation*}
    which we refer to as the \emph{comparison map}.
\end{definition}
%TODO remark similarity to truncatedness
\begin{definition}\label{def:connected}
    Let $C$ be an \inftytop/.
    With the convention that every map is $(-2)$-connected, we recursively define that a map $f\colon X\to Y$ in $C$ is \emph{$k$-connected} for $k\geq -1$ if
    \begin{itemize}
        \item it is an effective epimorphism
        \item the diagonal map $X\to X\times_YX$ is $(k-1)$-connected.
    \end{itemize} 
    If $f$ is connected for all $k\in\Nat$, then we say that $f$ is \emph{$\infty$-connected}.
    We consider an object $X$ to be \emph{$k$-connected} if the map $X\to *$ is $k$-connected.
\end{definition}
\begin{remark}
    Note that in $\spaces$ this definition agrees with the following:
    \begin{definition*}
        A map of topological spaces $f\colon X\to Y$ is $k$-connected if and only if for all its homotopy fibers $F$, all $x\in F$ and $n\leq k$ we have $\pi_n(F,x)\cong 0$.
        We say that every space is $(-2)$-connected and that a space is $(-1)$-connected if it is inhabited.
    \end{definition*}
    This is because for a pullback square in $\spaces$ 
    \begin{center}
        \begin{tikzcd} [sep = 4em]
            X\times_{Y} X \arrow[r, "g"] \arrow[d, "g"] & X \arrow[d, "f"] \\
            X \arrow[r, "f"] & Y \\
        \end{tikzcd}
    \end{center}
    $g$ is $k$-connected if and only if $f$ is $k$-connected since
    \begin{itemize}
        \item for $k=-2$ there is nothing to prove
        \item for $k\geq -1$ the homotopy fiber is inhabited for all chosen basepoints and thus $f$ is an effective epimorphism (see \cref{prop:effInSpaceIffSurj}).
            Since homotopy fibers are essentially constant on connected components, this allows us to prove that $f$ is $k$-connected if $g$ is $k$-connected from the fiberwise characterization of homotopy pullbacks. %TODO maybe say how to do this
            The other implication does not need $f$ to be an effective epimorphism and follows directly from the fiberwise characterization of homotopy pullbacks. 
    \end{itemize}
    
    Furthermore, a map $f\colon X\to Y$ equipped with a section $s\colon Y\to X$ is $n$-connected if and only if the section $s$ is $(n-1)$-connected (as can be proven by inspection of the long exact sequences of homotopy groups).
    
    Since $X\to X\times_YX$ is the canonical section of $X\times_{Y} X\to Y$, this allows us to recursively reduce to the case of $f\colon X\to Y$ being $(-2)$-connected, which in both definitions means that it is an arbitrary map.
\end{remark}
\begin{definition}
    We say that an \inftytop/ $C$ is \emph{hypercomplete} if every $\infty$-connected map is an equivalence.
\end{definition}
\begin{remark}
    Every equivalence is $\infty$-connected, however the converse is not true for general \inftytops/.
    For every small \inftycat/ $A$ the \inftytop/ $\Fun(A,\spaces)$ is hypercomplete. %TODO reference (HTT ??) 
\end{remark}
Our goal now is to prove that the comparison map is an equivalence whenever $X$ is a $0$-connected pointed object.
We will prove this only in $\spaces$, however this case is already enough to derive the result for general topoi (see \cref{rem:genTopoiComparisonMapEq}).

The main idea is to show that $X^{\wedge n}\wedge\Sigma\Omega\Sigma X$ becomes increasingly connected and vanishes as $n\to\infty$.
For this proof we need some stability properties of connectedness.
\begin{prop}\label{prop:conn}
    Let $C$ be an \inftytop/ and $k\geq -2$.
    Then
    \begin{enumerate}[label={(\roman*)}]
        \item $k$-connected morphisms are stable under pushout. \label{prop:connStableUnderPo}
        \item $k$-connected morphisms are stable under pullback.\label{prop:connStableUnderPb}
        \item for morphisms $f\colon X\to Y$ and $g\colon Y\to Z$ with $f$ $k$-connected, the morphism $g$ is $k$-connected if and only if $gf$ is $k$-connected. \label{prop:connRightCancel}
        \item for a morphism $f\colon X\to Y$ with a section $s\colon Y\to X$, $f$ is $(k+1)$-connected if and only if $s$ is $k$-connected. \label{prop:connSection}
        \item the class of $k$-connected objects is stable under filtered colimits. \label{prop:connStableFilteredColim}
    \end{enumerate}
    \begin{reference}
        \cite[Proposition 4.10]{splittings_21}
    \end{reference}
\end{prop}
\begin{prop}[Blakers-Massay Theorem]\label{prop:blakersMassay}
    Let $C$ be an \inftytop/ and let 
    \begin{center}
        \begin{tikzcd} [sep = 4em]
            A \arrow[r, "g"] \arrow[d, "f"] &  C \arrow[d] \\
            B \arrow[r] & D \\
        \end{tikzcd}
    \end{center}
    be a pushout square in $C$.
    Then if $f$ is $n$-connected and $g$ is $m$-connected, the universal map $A\to B\times_{D}C$ is $(n+m)$-connected.
    \begin{reference}
        A very general version for \inftytops/ can be found in \cite[Corollary 4.3.1]{gen_blakers_massey}, but the version used here can also be proven by a reduction to topoi of the form $\Fun(A,\spaces)$ similar to \cref{rem:genTopoiComparisonMapEq} and thus derived from the ordinary Blakers-Massay Theorem in $\Top$.
    \end{reference}
\end{prop}
With these we can deduce the following statements:
\begin{corollary}\label{cor:connEst}
    Let $C$ be an \inftytop/, $X,Y$ pointed objects in $C$ and $k,l\geq 0$. 
    If $X$ is $k$-connected and $Y$ is $l$-connected, then
    \begin{enumerate}[label={(\roman*)}]
        \item the suspension $\Sigma X$ is $(k+1)$-connected. \label{cor:connEstSuspension}
        \item the canonical map $X\to X\vee Y$ is $(l-1)$-connected. \label{cor:connEstVeeIncl}
        \item the smash product $X\wedge Y$ is $(k + l + 1)$-connected. \label{cor:connEstSmash}
        \item for every $n\geq 0$, the smash product $X^{\wedge n}$ is $(n(k+1)-1)$-connected. \label{cor:connEstIteratedSmash}
    \end{enumerate}
    \begin{proof}
        \ref{cor:connEstSuspension}: Since $X$ is $k$-connected by \cref{prop:conn} \ref{prop:connStableUnderPo} the pushout of the map $X\to *$ which is a map $*\to \Sigma X$ is $k$-connected.
        Thus by \cref{prop:conn} \ref{prop:connSection} the map $\Sigma X\to *$ is $(k+1)$-connected.

        \ref{cor:connEstVeeIncl}: Since the map $*\to Y$ is $(l-1)$-connected by \cref{prop:conn} \ref{prop:connSection}, by \cref{prop:conn} \ref{prop:connStableUnderPo} the map $X\to X\vee Y$ is also $(l-1)$-connected.
        
        \ref{cor:connEstSmash}: By \cref{prop:blakersMassay} and \cref{lem:poOfCollapseMapsIsTrivial} we have that $X\vee Y\to X\times Y$ is  $(k+l)$-connected. 
        By \cref{prop:conn} \ref{prop:connStableUnderPo} the map $*\to X\wedge Y$ is thus also $(k+l)$-connected and so by \cref{prop:conn} \ref{prop:connSection} the space $X\wedge Y$ is $(k+l+1)$-connected.

        \ref{cor:connEstIteratedSmash}: This follows by induction on \ref{cor:connEstSmash}.
    \end{proof}
\end{corollary}
\begin{prop}[James Splitting for Presheaf Categories]\label{prop:jamesSplittingPresheaf} %ref Proposition 4.18 
    Let $A$ be an \inftycat/ and let $X$ be a pointed and $0$-connected object in the \inftytop/ $\Fun(A,\spaces)$. %TODO why is this an \inftytop/?
    Then the comparison morphism 
    \begin{equation*}
        c_X\colon\bigvee\limits_{i=1}^{\infty}\Sigma X^{\wedge i}\to\Sigma\Omega\Sigma X
    \end{equation*}
    is an equivalence. 
    \begin{proof}
        Since $\Fun(A,\spaces)$ is hypercomplete %TODO reference
        it suffices to prove that $c_X$ is $\infty$-connected.

        First for each $n\in\Nat_{\geq 1}$ we have $2$-simplices in $\Fun(A,\spaces)$
        \begin{center}
            \begin{tikzcd} [sep = 4em]
                \bigvee\limits_{i=1}^{n}\Sigma X^{\wedge i} \arrow[r] \arrow[dr] & \bigvee\limits_{i=1}^{\infty}\Sigma X^{\wedge i} \arrow[d, "c_X"] \\
                & \Sigma\Omega\Sigma X \\
            \end{tikzcd}
        \end{center}
        induced by the universal property of $\bigvee\limits_{i=1}^{\infty}\Sigma X^{\wedge i}$.

        By \cref{cor:connEst} \ref{cor:connEstIteratedSmash} $X^{\wedge n}$ is $(n-1)$-connected.
        By \cref{prop:conn} \ref{prop:connStableUnderPo} and \cref{prop:conn} \ref{prop:connStableUnderPb} we have that $\Omega\Sigma X$ is $0$-connected, so by \cref{cor:connEst} \ref{cor:connEstSuspension} $\Sigma\Omega\Sigma X$ is $1$-connected and combining this via \cref{cor:connEst} \ref{cor:connEstSmash} shows that $X^{\wedge n}\wedge\Sigma\Omega\Sigma X$ is $(n+1)$-connected.

        So the map 
        \begin{equation*}
            c_X^n\colon\bigvee\limits_{i=1}^n\Sigma X^{\wedge i}\to\bigvee\limits_{i=1}^n\Sigma X^{\wedge i}\vee\left(X^{\wedge n}\wedge\Sigma\Omega\Sigma X\right)\xrightarrow{\cong}\Sigma\Omega\Sigma X
        \end{equation*}
        corresponding to the diagonal map of the above triangle is $n$-connected by \cref{cor:connEst} \ref{cor:connEstVeeIncl}.

        Since by \cref{cor:connEst} \ref{cor:connEstIteratedSmash} and \cref{cor:connEst} \ref{cor:connEstSmash} every object of the filtered colimit $\bigvee\limits_{i=n+1}^{\infty}\Sigma X^{\wedge i}$ is at least $(n+1)$-connected, by \cref{prop:conn} \ref{prop:connStableFilteredColim} it is itself $(n+1)$-connected.
        So by \cref{cor:connEst} \ref{cor:connEstVeeIncl} we have that the horizontal map of the triangle $\bigvee\limits_{i=1}^n\Sigma X^{\wedge i}\to\bigvee\limits_{i=1}^n\Sigma X^{\wedge i}\vee\bigvee\limits_{i=n+1}^{\infty}\Sigma X^{\wedge i}\cong\bigvee\limits_{i=1}^{\infty}\Sigma X^{\wedge i}$ is $n$-connected.

        As the diagonal as well as the horizontal map of the triangle above are $n$-connected, from \cref{prop:conn} \ref{prop:connRightCancel} it follows that $c_X$ is $n$-connected. 
        As $n$ was arbitrary, $c_X$ is therefore an equivalence.
    \end{proof}
\end{prop}
\begin{corollary}[Classical James Splitting]\label{cor:classicalJamesSplitting}
    Let $X$ be a pointed and $0$-connected object in $\spaces$.
    Then     
    \begin{equation*}
        c_X\colon\bigvee\limits_{i=1}^{\infty}\Sigma X^{\wedge i}\to\Sigma\Omega\Sigma X
    \end{equation*}
    is an equivalence. 
    \begin{proof}
        This is \cref{prop:jamesSplittingPresheaf} with $A=\Delta^0$.
    \end{proof}
\end{corollary}
\begin{remark}\label{rem:genTopoiComparisonMapEq}
    Due to the equivalent characterizations of \inftytops/ in \cite[Definition 6.1.0.6]{HTT} for every \inftytop/ $C$ there exists a small \inftycat/ $A$ and map $f\colon\Fun(A,\spaces)\to C$ that preserves finite limits and has a fully faithful right adjoint $i$.
    
    Then if for a pointed object $X$ in $C$ the comparison morphism $c_{i(X)}$ is an equivalence in $Fun(A,\spaces)$, so is $c_X$ in $C$ (see \cite[Lemma 4.17]{splittings_21}).
    If $X$ is $0$-connected then so is $i(X)$, so it suffices to prove that it $c_{i(X)}$ is $\infty$-connected since $\Fun(A,\spaces)$ is hypercomplete. %TODO prove that X 0-conn => i(X) 0-conn
    But we have proven this in \cref{prop:jamesSplittingPresheaf}, so it is true for all \inftytops/.
\end{remark}
Lastly, we give an application of the James Splitting Theorem.
\begin{lemma}\label{lem:reducedHomLoopOfSpheres}
    Let $H_*$ be a reduced homology theory and let $S^n$ with $n>0$ be a model of the pointed sphere.
    Then
    \begin{equation*}
        H_k(\Omega S^{n+1})\cong\bigoplus\limits_{i=1}^{\infty}H_k(S^{ni})\;.
    \end{equation*}
    \begin{proof}
        Note that for $n>0$ the sphere is $0$-connected.
        Thus by \cref{cor:classicalJamesSplitting} we have an isomorphism 
        \begin{equation*}
            H_{k+1}\left(\bigvee\limits_{i=1}^{\infty}\Sigma \left(S^n\right)^{\wedge i}\right)\cong H_{k+1}\left(\Sigma\Omega\Sigma S^n\right)\;.
        \end{equation*}
        This gives an isomorphism
        \begin{equation*}
            \bigoplus\limits_{i=1}^{\infty}H_k\left(\left(S^n\right)^{\wedge i}\right)\cong H_k\left(\Omega S^{n+1}\right)
        \end{equation*}
        and the fact that $S^{n_1}\wedge S^{n_2}\cong S^{n_1+n_2}$ proves the claim.
    \end{proof}
\end{lemma}
\begin{corollary}
    For $H_*$ reduced ordinary homology and $S^n$ with $n>0$ a model of the pointed sphere, we have
    \begin{equation*}
        H_k(\Omega S^{n+1})\cong
        \begin{cases}
            \mathbb{Z} & k=ni ~for~ i\in\Nat\\
            0 & else
        \end{cases}\;.
    \end{equation*}
    \begin{proof}
        This follows from \cref{lem:reducedHomLoopOfSpheres} and the reduced ordinary homology of the spheres.
    \end{proof}
\end{corollary}
