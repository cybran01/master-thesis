%TODO order the statements correctly
%TODO do the proofs in the Strom model structure and argue at the end that this gives the statements in the Quillen model structure
In this section we will prove that our \inftycat/ of spaces $\spaces$ is an \inftytop/.
We will keep our proofs topological; for a different approach using simplicial sets see e.g. %TODO
In order to transfer reasoning back and forth from the model category $\Top$ and the \inftycat/ $\spaces$ we need the following lemma.
\begin{lemma}\label{lem:replaceWithStrictDiagram} %TODO is the statement even true in this generality?
    Let $J$ be an ordinary category and let $M$ be a model category.
    Then a diagram $\N(J)\to\N(M)[W^{-1}]$ can be replaced with a diagram $J\to M$ that is equivalent in the sense that under the composition
    \begin{equation*}
        \Fun(\N(J),\N(M))\to\Fun(\N(J),\N(M)[W^{-1}]) %TODO is this correct?
    \end{equation*} 
    they are equivalent as objects of the \inftycat/ $\Fun(\N(J),\N(M)[W^{-1}])$.

    Furthermore the map $\N(\Fun(J,M))\to\Fun(\N(J),\N(M)[W^{-1}])$ is natural in $J$.
    \begin{proof}
        %TODO Proposition 1.3.4.25. HA
    \end{proof}
\end{lemma}
Since both coproducts and pushouts in \inftycats/ are indexed by nerves of ordinary categories and we are able to work in the model category by the following corollary:
\begin{corollary}\label{cor:sufficientToProveInModCat}
    Let $M$ be a model category with weak equivalences $W$. 
    Then $\N(M)[W^{-1}]$ has universal colimits if and only if $M$ has universal homotopy pushouts and universal homotopy coproducts as a model category.
    
    If further $\N(M)[W^{-1}]$ has universal colimits, then $M$ has descent if and only if it has descent for homotopy pushouts and homotopy coproducts as a model category.
    \begin{proof}
        A diagram $\N(J)^{\rhd}\to\N(M)[W^{-1}]$ is a colimit cone if and only if some equivalent replacement $J^{\rhd}\to M$ as in \cref{lem:replaceWithStrictDiagram} is a homotopy colimit cone in $M$.
        Given a diagram $J^{\rhd}\to M$, is is a homotopy colimit cone if and only if $\N(J)^{\rhd}\to\N(M)[W^{-1}]$ is a colimit cone.
        The analogous statement is also true for limits and homotopy limits.
        Thus the proposition follows from \cref{lem:univColimIffUnivPoAndCoprod} and \cref{lem:descentIffDescentPoAndCoprod} respectively.
    \end{proof}
\end{corollary}
We will use the interaction of the \Strom/ model structure and the Quillen model structure to prove universality and descent in $\Top$.
Note that whenever referring to $\Top$ as a model category, we implicitly mean the Quillen model structure. 
Cofibrations and fibrations of the \Strom/ model structure are tagged by the letteer $h$.
Our most important (and very non trivial) part of this interaction is the following proposition.
% \begin{lemma}[Universality for Pushouts]\label{lem:univForPushouts}
%     Let $C$ be an \inftycat/ that admits pushouts and pullbacks.
%     Then for all natural transformations $\alpha\colon p\to q\in\Fun\left(\left(\Lambda^2_0\right)^{\rhd},C\right)_1$ such that $\alpha$ is cartesian and $q$ is a colimit cone we have that $p$ is also a colimit cone if and only if for all cubes
%     \begin{center}
%         \begin{tikzcd} [sep = .5 cm]
%             A^{\prime} \arrow [dr] \arrow [rr] \arrow [dd] & & B^{\prime} \arrow [dr] \arrow [dd] \\
%             & C^{\prime} \arrow [rr, crossing over] \arrow [dd] & & D^{\prime} \arrow [dd] & \\
%             A \arrow [dr] \arrow [rr] & & B \arrow [dr] \\
%             & C \arrow [from=uu, crossing over] \arrow [rr] & & D &
%         \end{tikzcd}
%     \end{center}
%     in $C$ where the vertial faces are pullbacks and the bottom square is a pushout, the top square is also a pushout.
%     \begin{proof}
%         %TODO
%     \end{proof}
% \end{lemma}
% \begin{lemma}[Descent for Pushouts]\label{lem:descForPushouts}
%     Let $C$ be an \inftycat/ that admits pushouts and pullbacks.
%     Then all natural transformations $\alpha\colon p\to q\in\Fun\left(\left(\Lambda^2_0\right)^{\rhd},C\right)_1$ such that $\alpha|_{\Lambda^2_0}$ is cartesian and $p,q$ are colimit cones are cartesian and if and only if for all cubes
%     \begin{center}
%         \begin{tikzcd} [sep = .5 cm]
%             A^{\prime} \arrow [dr] \arrow [rr] \arrow [dd] & & B^{\prime} \arrow [dr] \arrow [dd] \\
%             & C^{\prime} \arrow [rr, crossing over] \arrow [dd] & & D^{\prime} \arrow [dd] & \\
%             A \arrow [dr] \arrow [rr] & & B \arrow [dr] \\
%             & C \arrow [from=uu, crossing over] \arrow [rr] & & D &
%         \end{tikzcd}
%     \end{center}
%     in $C$ where the left and back face are pullbacks and the bottom and top square are a pushouts, the front and right squares are pullbacks.
%     \begin{proof}
%         %TODO
%     \end{proof}
% \end{lemma}
\begin{prop}\label{prop:poAlongHCofibIsHtpyPo}
    A pushout along an h-cofibration is a homotopy pushout.
    \begin{proof}
        Let
        \begin{center}
            \begin{tikzcd} [sep = 4em]
                A \arrow[>->, r, "h"] \arrow[d] & B \arrow[d]\\
                C \arrow[>->, r, "h"] & D \\
            \end{tikzcd}
        \end{center}
        be a pushout with the inducated maps being h-cofibrations.
        We factor the map $A\to C'\to C$ into a cofibration followed by a weak equivalence and obtain the diagram
        \begin{center}
            \begin{tikzcd} [sep = 4em]
                A \arrow[>->, r, "h"] \arrow[>->,d] & B \arrow[d]\\
                C' \arrow[>->, r, "h"] \arrow[d, "\sim"] & D' \arrow[d, "\sim"] \\
                C \arrow[>->, r, "h"] & D
            \end{tikzcd}
        \end{center}
        by taking pushouts.
        By %TODO hocolim_bar
        we have that pushouts of weak equivalences along h-cofibrations are again weak equivalences.

        The top square is a homotopy pushout as it is a pushout along a cofibration. 
        The lower square is a homotopy pushout by %TODO HTT ??
        .
        Thus the outer square is a homotopy pushout which proves the proposition.
    \end{proof}
\end{prop}
\begin{prop}\label{lem:topUniversalPo}
    The model category $\Top$ has universality for homotopy pushouts.
    \begin{proof}
        By factoring the map $D'\xrightarrow{\sim}\overline{D}\xtwoheadrightarrow{h} D$ into a weak equivalence followed by an h-fibration (which is also a Serre fibration) and pulling back successively, we obtain the diagram %TODO explain why we we use weak equivalence here: since the faces are pb in the SERRE model str, the 3 remaining vertical maps are only w.e.
        \begin{center}
            \begin{tikzcd} [sep = .5 cm]
                A^{\prime} \arrow [dr] \arrow [rr] \arrow [dd] & & B^{\prime} \arrow [dr] \arrow [dd] \\
                & C^{\prime} \arrow [rr, crossing over] & & D^{\prime} \arrow [dd] & \\
                \overline{A} \arrow [dr] \arrow [rr] \arrow [->>, dd] & & \overline{B} \arrow [dr] \arrow [->>, dd] \\
                & \overline{C} \arrow [rr, crossing over] \arrow [from=uu, crossing over] \arrow [dd] & & \overline{D} \arrow [->>, dd] & \\
                A \arrow [dr] \arrow [rr] & & B \arrow [dr] \\
                & C \arrow [->>,from=uu, crossing over] \arrow [rr] & & D & \\
            \end{tikzcd}
        \end{center}
        where $\overline{A}=A'\times_{D'}\overline{D}$, $\overline{B}=B'\times_{D'}\overline{D}$ and $\overline{C}=C'\times_{D'}\overline{D}$, and the indicated maps are fibrations.
        Since by the pasting lemma for homotopy pullbacks the upper vertical squares are also homotopy pullbacks and $D'\to\overline{D}$ is a weak equivalence, all vertical maps in the upper cube are weak equivalences.
        Thus by %HTT ???
        it is sufficent to show that the square
        \begin{center}
            \begin{tikzcd} [sep = 4em]
                \overline{A} \arrow[r] \arrow[d] & \overline{B} \arrow[d] \\
                \overline{C} \arrow[r] & \overline{D} \\
            \end{tikzcd}
        \end{center}
        is a homotopy pushout.

        We factor the map $A\xtailrightarrow{h}\widehat{B}\xrightarrow{\sim} B$ into an h-cofibration followed by a weak equivalence.
        We obtain a diagram 
        \begin{center}
            \begin{tikzcd} [sep = 4em]
                A \arrow[>->, r, "h"] \arrow[d] & \widehat{B} \arrow[r, "\sim"] \arrow[d] & B \arrow[d] \\
                C \arrow[>->, r, "h"] & C\cup_{A}\widehat{B} \arrow[r] & D \\
            \end{tikzcd}
        \end{center}
        where the left square is a homotopy pushout as its a pushout along an h-cofibration by \cref{prop:poAlongHCofibIsHtpyPo} and the outer square is homotopy pushout by definition, thus by the pasting law the right square is also homotopy pushout. 
        This implies by %HTT
        that $C\cup_{A}\widehat{B}\to D$ is a weak equivalence as well.
        Pulling back this diagram along the h-fibration $\overline{D}\twoheadrightarrow D$ gives a diagram 
        \begin{center}
            \begin{tikzcd} [sep = 4em]
                \overline{A} \arrow[>->, r, "h"] \arrow[d] & \widehat{B}\times_{D}\overline{D} \arrow[r] \arrow[d] & \overline{B} \arrow[d] \\
                \overline{C} \arrow[>->, r, "h" ] & \left(C\cup_{A}\widehat{B}\right)\times_{D}\overline{D} \arrow[r] & \overline{D} \\
            \end{tikzcd}
        \end{center}
        where the horizontal maps in the left diagram are h-cofibrations due to % Jeffrey Strom - Modern Classical Homotopy Theory Theorem 14.1 for pb along fib of cofib is cofib, TODO is this true in the serre model str?
        and its a pushout since $\Top$ has universality (as an ordinary category, so pulling back a pushout square gives a pushout square again). %TODO maybe cite somewhere?
        The right square is also homotopy pushout since $\Top$ is right proper so pullbacks of weak equivalences along fibrations are again weak equivalences. 
        Thus the outer square is a homotopy pushout which proves the proposition. 
    \end{proof}
\end{prop}
\begin{lemma}\label{lem:topUniversalCoproduct}
    The model category $\Top$ has universal homotopy coproducts.
    \begin{proof}
        	Let $\left(X_i\right)_{i\in I}$ be a collection of topological spaces with $I$ small. 
            Then the homotopy coproduct is given by the coproduct over cofibrant replacements $\bigcup\limits_{i\in I}\widehat{X_i}$ for each of the $X_i$.

            However since in $\Top$ weak equivalences are closed under coproducts, the ordinary coproduct is already a homotopy coproduct.
            Also a map $X_i\to\bigcup\limits_{i\in I} X_i$ is a Serre fibration since the domain of the generating fibrations is connected and coproducts in $\Top$ are disjoint.
            Thus the ordinary pullback
            \begin{center}
                \begin{tikzcd} [sep = 1 cm]
                    Y\times_{\bigcup\limits_{i\in I} X_i}X_i \arrow [r] \arrow [->>,d] & X_i \arrow [->>,d]\\
                    Y \arrow [r] & \bigcup\limits_{i\in I} X_i
                \end{tikzcd}
            \end{center}
            is a homotopy pullback and so $Y_i\to Y\times_{\bigcup\limits_{i\in I} X_i}X_i$ is a weak equivalence.
            Since in $\Top$ coproducts commute with pullbacks, $\bigcup\limits_{i\in I}\left(Y\times_{\bigcup\limits_{i\in I} X_i}X_i\right)\cong Y$.
            Thus $\bigcup\limits_{i\in I}Y\to Y$ is a weak equivalence, which proves that $Y$ is a homotopy coproduct.
    \end{proof}
\end{lemma}
Thus by \cref{cor:sufficientToProveInModCat} we $\spaces$ has universal colimits.
It now remains to show that it also has descent.

We start proving the case for pushouts by the following reduction step. 
\begin{prop}\label{prop:reductionStepDescent}
    If for all cubes in the model category $\Top$
    \begin{center}
        \begin{tikzcd} [sep = .5 cm]
            \overline{A} \arrow [dr] \arrow [rr] \arrow [->>,dd, "h"] & & \overline{B} \arrow [dr] \arrow[->>]{dd}[near start]{h} \\
            & \overline{C} \arrow [rr, crossing over] & & \overline{D} \arrow [dd] & \\
            A \arrow [dr] \arrow [rr] & & B \arrow [dr] \\
            & C \arrow [->>,from=uu,crossing over]{}[near start]{h} \arrow [rr] & & D &
        \end{tikzcd}
    \end{center}
    where 
    \begin{itemize}
        \item left and back face are homotopy pullbacks
        \item bottom and top square are pushouts and homotopy pushouts %TODO is ordinary po necessary?
        \item the maps $\overline{A}\to A$, $\overline{B}\to B$ and $\overline{C}\to C$ are h-fibrations
    \end{itemize}
    the front and right squares are homotopy pullbacks, then $\Top$ has descent for homotopy pushouts.
    \begin{proof}
        Starting from a cube 
        \begin{center}
            \begin{tikzcd} [sep = .5 cm]
                A^{\prime} \arrow [dr] \arrow [rr] \arrow [dd] & & B^{\prime} \arrow [dr] \arrow [dd] \\
                & C^{\prime} \arrow [rr, crossing over] \arrow [dd] & & D^{\prime} \arrow [dd] & \\
                A \arrow [dr] \arrow [rr] & & B \arrow [dr] \\
                & C \arrow [from=uu, crossing over] \arrow [rr] & & D &
            \end{tikzcd}
        \end{center}
        where the back and left square are homotopy pullbacks and top and bottom are homotopy pushouts, we show that we can construct a cube of the required form such that all faces are equivalent to the corresponding faces of the starting cube.
        
        We first factor the maps $A\xtailrightarrow{h}\hat{C}\xtwoheadrightarrow[\sim]{h} C$ and $B'\xrightarrow{\sim}\overline{B}\xtwoheadrightarrow{h} B$ to obtain the diagram
        \begin{center}
            \begin{tikzcd} [sep = 4em]
                C^{\prime} \arrow[dd] & \hat{C}\times_CC' \arrow[->>]{l}[swap]{\sim} \arrow[dd] & A^{\prime} \arrow[l] \arrow[r] \arrow[d, "\sim"] & B^{\prime} \arrow[d, "\sim"] \\
                && \overline{A} \arrow [->>,d, "h"] \arrow[r] & \overline{B} \arrow[->>,d, "h"] \\
                C & \hat{C} \arrow[->>]{l}{h}[swap]{\sim} & A \arrow[>->,l, "h"] \arrow[r] & B 
            \end{tikzcd}
        \end{center}
        where 
        \begin{itemize}
            \item $\overline{A}=A\times_{B}\overline{B}$
            \item $A'\to \overline{A}$ is a weak equivalence since $B'\to\overline{B}$ is one by the pasting law for homotopy pullbacks as the outer and lower square are homotopy pullback.
        \end{itemize}
        Next we factor the map $A'\xtailrightarrow{h}X\xrightarrow{\sim}\hat{C}\times_CC'$. 
        Note that the diagram 
        \begin{center}
            \begin{tikzcd} [sep = 4em]
                X \arrow[r, "\sim"] \arrow[d] & C' \arrow[d] \\
                \hat{C} \arrow[->>]{r}{\sim}[swap]{h} & C \\
            \end{tikzcd}
        \end{center}
        is a homotopy pullback square.
        We form the diagram
        \begin{center}
            \begin{tikzcd} [sep = 4em]
                C^{\prime} \arrow[dd] & X \arrow{l}[swap]{\sim} \arrow[d, "\sim"] & A^{\prime} \arrow[>->,l, "h"] \arrow[r] \arrow[d, "\sim"] & B^{\prime} \arrow[d, "\sim"] \\
                & X\cup_{A'}\overline{A} \arrow[d] & \overline{A} \arrow[>->,l, "h"] \arrow [->>,d, "h"] \arrow[r] & \overline{B} \arrow[->>,d, "h"] \\
                C & \hat{C} \arrow[->>]{l}{h}[swap]{\sim} & A \arrow[>->,l, "h"] \arrow[r] & B 
            \end{tikzcd}
        \end{center}
        where $X\to X\cup_{A'}\overline{A}$ is a weak equivalence as its a pushout of a weak equivalence along an h-cofibration. %TODO reference
        By the pasting law we have that
        \begin{center}
            \begin{tikzcd} [sep = 4em]
                A' \arrow[>->,r, "h"] \arrow[d] & X \arrow[d] \\
                A \arrow[>->,r, "h"] & \hat{C} \\
            \end{tikzcd}
        \end{center}
        is homotopy pullback and since this square is equivalent %TODO maybe define what that means
        to the square
        \begin{center}
            \begin{tikzcd} [sep = 4em]
                \overline{A} \arrow[>->,r, "h"] \arrow[->>,d, "h"] & X\cup_{A'}\overline{A} \arrow[d] \\
                A \arrow[>->,r, "h"] & \hat{C} \\
            \end{tikzcd}
        \end{center}
        we know that both are homotopy pullbacks.
        Finally, we factor the map $X\cup_{A'}\overline{A}\xtailrightarrow[\sim]{h}\overline{C}\xtwoheadrightarrow{h}\hat{C}$.
        Then 
        \begin{center}
            \begin{tikzcd} [sep = 4em]
                \overline{A} \arrow[>->,r, "h"] \arrow[->>,d, "h"] & \overline{C} \arrow[->>,d, "h"] \\
                A \arrow[>->,r, "h"] & \hat{C} \\
            \end{tikzcd}
        \end{center}
        is again a homotopy pullback square. %TODO maybe say that this reduction is possible in any model cat

        Thus we have obtained a diagram
        \begin{center}
            \begin{tikzcd} [sep = 4em]
                \overline{C} \arrow[->>,d, "h"] & \overline{A} \arrow[>->,l, "h"] \arrow[r] \arrow[->>,d, "h"] & \overline{B} \arrow[->>,d, "h"] \\
                \hat{C} & A \arrow[>->,l, "h"] \arrow[r] & B \\
            \end{tikzcd}
        \end{center}
        and we can form the cube 
        \begin{center}
            \begin{tikzcd} [sep = .5 cm]
                \overline{A} \arrow[>->,dr, "h"] \arrow [rr] \arrow [->>,dd,"h"] & & \overline{B} \arrow [dr] \arrow[->>]{dd}[near start]{h} \\
                & \overline{C} \arrow [rr, crossing over] & & \overline{C}\cup_{\overline{A}}\overline{B} \arrow [dd] & \\
                A \arrow[>->,dr, "h"] \arrow [rr] & & B \arrow [dr] \\
                & \hat{C} \arrow[->>, from=uu, crossing over]{}[near start]{h} \arrow [rr] & & \hat{C}\cup_A B &
            \end{tikzcd}
        \end{center}
        by taking pushouts.
        Since these are pushouts along h-cofibrations, they are already homotopy pushouts. 
        Following the construction of this cube, we see that all faces are equivalent to their corresponding faces of the starting cube which proves the proposition.
    \end{proof}
\end{prop}
For the last part of the proof we need the following definitions.
\begin{definition}
    Let $M$ be a model category and $f\colon X\to Y$ be a map.
    Then we call $f$ a \emph{quasifibration} if for every map $*\to Y$ (where $*$ is the terminal object) the ordinary pullback square
    \begin{center}
        \begin{tikzcd} [sep = 4em]
            F \arrow[r] \arrow[d] & X \arrow[d, "f"] \\
            * \arrow[r] & Y\\
        \end{tikzcd}
    \end{center}
    is a homotopy pullback.
\end{definition}
\begin{definition}
    Let $f\colon A\to B$ be a map of topological spaces and let $I=[0,1]$.
    Then we let
    \begin{equation*}
        \M(f)\coloneqq\faktor{\left(A\times I\right)\cup B}{(a,0)\sim f(a)}
    \end{equation*}
    denote the \emph{mapping cylinder of $f$}.

    It will be convenient to also allow other intervalls instead of $I=[0,1]$; we will allow $I=[0,x]$ and $I=[0,x)$ for $x>0$.
    The analogous construction with $I=[0,x]$ will still referred to as a \emph{closed mapping cylinder of $f$};
    the construction with half open intervalls $I=[0,x)$ will be referred to as an \emph{open mapping cylinder of $f$}. %TODO perhaps describe factorization
\end{definition}
\begin{lemma}\label{lem:mapOfCylIsQuasiFib} %TODO Cubical homotopy Theory Lemma 5.10.6, TODO prove for open and closed mapping cylinder, TODO prove assertions on pullbacks
    Let 
    \begin{center}
        \begin{tikzcd} [sep = 4em]
            X \arrow[d,->>, "h"] \arrow[r, "f"] & Y \arrow[d,->>, "h"] \\
            A \arrow[r, "g"] & B \\
        \end{tikzcd}
    \end{center}
    be a homotopy pullback (but not necessarily a pullback).
    Then the induced map $\M(f)\to\M(g)$ between mapping cylinders is a quasifibration.
    The same holds true for closed and open mapping cylinders.
    \begin{proof}
        For convenience we will only prove the case $I=[0,1]$ since the other cases follow by the analogous argument.

        Let $\hat{b}\in\M(g)$. 
        Then either $\hat{b}=(\hat{a},t)\in A\times(0,1]\subset\M(g)$ or $\hat{b}\in B\subset\M(g)$.
        In the first case, we have a diagram
        \begin{center}
            \begin{tikzcd} [sep = 4em]
                X \arrow[r, "{x\mapsto (x,t)}"] \arrow[->>,d, "h"] & \M(f) \arrow[r, "\sim"] \arrow[d] & Y \arrow[->>,d, "h"] \\
                A \arrow[r, "{a\mapsto (a,t)}"] & \M(g) \arrow[r, "\sim"] & B \\
            \end{tikzcd}
        \end{center}
        where the outer square is the starting square and in particular homotopy pullback, so by the pasting law the left square is homotopy pullback as well.
        Since the left square is also an ordinary pullback, in the diagram
        \begin{center}
            \begin{tikzcd} [sep = 4em]
                F \arrow[r] \arrow[d] & X \arrow[r, "{x\mapsto (x,t)}"] \arrow[->>,d, "h"] & \M(f) \arrow[d] \\
                * \arrow[r, "\hat{b}"] & A \arrow[r, "{a\mapsto (a,t)}"] & \M(g) \\
            \end{tikzcd}
        \end{center}
        the left square is homotopy pullback since its the pullback along a fibration.
        Thus the outer square is a homotopy pullback which proves that the fiber $F$ is a homotopy fiber.

        In the second case we have the diagram
        \begin{center}
            \begin{tikzcd} [sep = 4em]
                F \arrow[r] \arrow[d] & Y \arrow[r, hook, "\sim"] \arrow[->>,d, "h"] & \M(f) \arrow[d] \\
                * \arrow[r, "\hat{b}"] & B \arrow[r, hook, "\sim"] & \M(g) \\
            \end{tikzcd}
        \end{center}
        where the indicated maps are weak equivalences since they are right inverse to the maps $\M(f)\to Y$ and $\M(g)\to B$ respectively, which are weak equivalences themselves.
        Since the right square is pullback and homotopy pullback and the left side is as well, this means that the fiber $F$ is already a homotopy fiber which proves the proposition.
    \end{proof}
\end{lemma}
\begin{prop}\label{lem:topDescentPo}
    The model category $\Top$ has descent for homotopy pushouts. 
    \begin{proof}
        By \cref{prop:reductionStepDescent} it is sufficient to prove that for all cubes 
        \begin{center}
            \begin{tikzcd} [sep = .5 cm]
                \overline{A} \arrow [dr] \arrow [rr] \arrow [->>,dd, "h"] & & \overline{B} \arrow [dr] \arrow[->>]{dd}[near start]{h} \\
                & \overline{C} \arrow [rr, crossing over] & & \overline{D} \arrow [dd] & \\
                A \arrow [dr] \arrow [rr] & & B \arrow [dr] \\
                & C \arrow [->>,from=uu,crossing over]{}[near start]{h} \arrow [rr] & & D &
            \end{tikzcd}
        \end{center}
        where 
        \begin{itemize}
            \item left and back face are homotopy pullbacks
            \item bottom and top square are pushouts and homotopy pushouts
            \item the maps $\overline{A}\to A$, $\overline{B}\to B$ and $\overline{C}\to C$ are h-fibrations
        \end{itemize}
        the front and right squares are homotopy pullbacks.

        By \cref{lem:mapOfCylIsQuasiFib} we know that the induced maps $\M(f_{\overline{A}\overline{C}})\to\M(f_{AC})$ and $\M(f_{\overline{A}\overline{B}})\to\M(f_{AB})$ are quasifibrations. %TODO why are inclusions cofibs
        So we can form the cube 
        \begin{center}
            \begin{tikzcd} [sep = .75 cm]
                \overline{A} \arrow [>->, dr, "h"] \arrow [>->, rr, "h"] \arrow [->>,dd, "h"] & & \M(f_{\overline{A}\overline{B}}) \arrow [>->, dr, "h"] \arrow[dd] \\
                & \M(f_{\overline{A}\overline{C}}) \arrow [>->, rr, crossing over, near start,  "h"] & & \M(f_{\overline{A}\overline{C}})\cup_{\overline{A}}\M(f_{\overline{A}\overline{B}}) \arrow [dd] & \\
                A \arrow [>->, dr, "h"] \arrow [>->, rr, near start, "h"] & & \M(f_{AB}) \arrow [>->, dr, "h"] \\
                & \M(f_{AC}) \arrow [from=uu,crossing over] \arrow [>->, rr, "h"] & & \M(f_{AC})\cup_A\M(f_{AB}) &
            \end{tikzcd}
        \end{center}
        where all faces are again equivalent to the corresponding faces of the original cube and the indicated maps are h-cofibrations by %TODO reference %TODO confirm this statement
        .
        
        Next we prove that the map $p\colon\M(f_{\overline{A}\overline{C}})\cup_{\overline{A}}\M(f_{\overline{A}\overline{B}})\to\M(f_{AC})\cup_A\M(f_{AB})$ is a quasifibration.
        By %TODO Hatcher Lemma 4K.3
        we only need to find open sets $U_1,U_2\subset\M(f_{AC})\cup_A\M(f_{AB})$ covering $\M(f_{AC})\cup_A\M(f_{AB})$ such that the induced maps $p^{-1}(U_1)\to U_1$, $p^{-1}(U_2)\to U_2$ and $p^{-1}(U_1\cap U_2)\to U_1\cap U_2$ are quasifibrations.

        We take $U_1=\M(f_{AC})\cup_A\M(f_{AB})\setminus{B}$ and $U_2=\M(f_{AC})\cup_A\M(f_{AB})\setminus{C}$. 

        We identify $U_1\cap U_2=A\times [0,1)\cup_A A\times[0,1)\cong A\times (0,2)$ via the ''obvious'' homeomorphism. %TODO explain
        Then $p^{-1}(U_1)=\M(f_{\overline{A}\overline{B}})\cup_{\overline{A}}\M(f_{\overline{A}\overline{B}})\setminus{\overline{B}}$, $p^{-1}(U_2)=\M(f_{\overline{A}\overline{B}})\cup_{\overline{A}}\M(f_{\overline{A}\overline{B}})\setminus{\overline{C}}$ and $p^{-1}(U_1\cap U_2)=\overline{A}\times (0,2)$ (identified with $\overline{A}\times [0,1)\cup_{\overline{A}} \overline{A}\times[0,1)$ as before).
        It follows $p^{-1}(U_1)\to U_1$ and $p^{-1}(U_2)\to U_2$ are quasifibrations by \cref{lem:mapOfCylIsQuasiFib} since they are both maps of open mapping cylinders induced by the squares 
        \begin{center}
            \begin{tikzcd} [sep = 4em]
                \overline{A} \arrow[d,->>, "h"] \arrow[r, "f_{\overline{A}\overline{C}}"] & \overline{C} \arrow[d,->>, "h"] \\
                A \arrow[r, "f_{AC}"] & C \\
            \end{tikzcd} 
            \begin{tikzcd} [sep = 4em]
                \overline{A} \arrow[d,->>, "h"] \arrow[r, "f_{\overline{A}\overline{B}}"] & \overline{B} \arrow[d,->>, "h"] \\
                A \arrow[r, "f_{AB}"] & B \\
            \end{tikzcd}
        \end{center}
        respectively.
        Since the map $p^{-1}(U_1\cap U_2)\to U_1\cap U_2$ is a product of fibrations $f_{\overline{A}A}\times\id_{(0,2)}$, it is again a fibration and thus a quasifibration.

        Finally, since the squares 
        \begin{center}
            \begin{tikzcd} [sep = 4em]
                \M(f_{\overline{A}\overline{C}}) \arrow[d] \arrow[r] & \M(f_{\overline{A}\overline{C}})\cup_{\overline{A}}\M(f_{\overline{A}\overline{B}}) \arrow[d] \\
                \M(f_{AC}) \arrow[r] & \M(f_{AC})\cup_A\M(f_{AB}) \\
            \end{tikzcd}
            \begin{tikzcd} [sep = 4em]
                \M(f_{\overline{A}\overline{B}}) \arrow[d] \arrow[r] & \M(f_{\overline{A}\overline{C}})\cup_{\overline{A}}\M(f_{\overline{A}\overline{B}}) \arrow[d] \\
                \M(f_{AB}) \arrow[r] & \M(f_{AC})\cup_A\M(f_{AB}) \\
            \end{tikzcd}
        \end{center}
        are both ordinary pullbacks %TODO maybe explain why
        with both vertical maps quasifibrations, they are already homotopy pullbacks. %TODO maybe put this as an extra statement somewhere after introducing quasifibs
        Taking the left square as an example, this follows from the fact that the fiber
        \begin{center}
            \begin{tikzcd} [sep = 4em]
                F \arrow[r] \arrow[d] & \M(f_{\overline{A}\overline{C}}) \arrow[d] \arrow[r] & \M(f_{\overline{A}\overline{C}})\cup_{\overline{A}}\M(f_{\overline{A}\overline{B}}) \arrow[d] \\
                * \arrow[r] & \M(f_{AC}) \arrow[r] & \M(f_{AC})\cup_A\M(f_{AB}) \\
            \end{tikzcd}
        \end{center}
        is already a homotopy fiber and so the map between homotopy fibers of the vertical maps is the identity and thus a weak equivalence.
        But this is an equivalent characterization of a homotopy pullback. %TODO reference
    \end{proof}
\end{prop}
Lastly, we need to prove descent for homotopy pullbacks. 
For this we again need some preliminaries.
\begin{lemma}
    Homotopy coproducts in the model category $\Top$ are disjoint.
    \begin{proof}
        Since homotopy coproducts are given by ordinary coproducts in $\Top$, we have to check whether a pushout
        \begin{center}
            \begin{tikzcd} [sep = 1 cm]
                \emptyset \arrow [r] \arrow [d] & X \arrow [d]\\
                Y \arrow [r] & X\cup Y
            \end{tikzcd}
        \end{center}
        is a homotopy pullback.
        As in the proof for universality of coproducts in $\Top$, we see that the map $X\to X\cup Y$ is a Serre fibration.
        Since the square is an ordinary pullback, this already implies that it is a homotopy pullback.
    \end{proof}
\end{lemma}
\begin{corollary}\label{cor:genCoproductComponentPb} %TODO fix indexing
    Let $\left(X_i\right)_{i\in I}$ be a collection of topological spaces and for ease of notation, set $Y_i=\bigcup\limits_{i\in I\setminus{\set{i}}}X_i$.
    Then in the pullback square (which is also a homotopy pullback)
    \begin{center}
        \begin{tikzcd} [sep = 1 cm]
            P_{ij} \arrow [r] \arrow [d] & X_i \arrow [d]\\
            X_j \arrow [r] & \bigcup\limits_{i\in I} X_i
        \end{tikzcd}
    \end{center}
    we have $P_{ij}=\emptyset$ for $i\neq j$ and $P_{ii}\cong X_i$.
    \begin{proof}
        First let $i\neq j$.
        Then the lower square of the tower of pullback squares 
        \begin{center}
            \begin{tikzcd} [sep = 1 cm]
                P_{ij} \arrow [r] \arrow [d] & X_i \arrow [d]\\
                \overline{P_{ij}} \arrow [d] \arrow [r] & Y_j \arrow [d]\\
                X_j \arrow [r] & Y_j\cup X_j
            \end{tikzcd}
        \end{center}
        shows that $\overline{P_{ij}}=\emptyset$ by disjointness of (binary) coproducts and thus $P_{ij}=\emptyset$ as well.

        Next we prove the case $i=j$. 
        We need to prove that for the pullback square (that is also a homotopy pullback)
        \begin{center}
            \begin{tikzcd} [sep = 1 cm]
                P_{ii} \arrow [r] \arrow [d] & X_i \arrow [d]\\
                X_i \arrow [r] & X_i\cup Y_i
            \end{tikzcd}
        \end{center}
        we have $P_{ii}\cong X_i$. 

        By universality of coproducts we have $\left(X_i\times_{X_i\cup Y}X_i\right)\cup\left(Y_i\times_{X_i\cup Y_i}X_i\right)\cong X_i$.
        But by disjointness of coproducts $Y_i\times_{X_i\cup Y_i}X_i=\emptyset$, and since $\left(X_i\times_{X_i\cup Y_i}X_i\right)\cup\emptyset\cong\left(X_i\times_{X_i\cup Y_i}X_i\right)$.
        So $X_i\cong X_i\times_{X_i\cup Y_i}X_i$ which completes the proof.
    \end{proof}
\end{corollary}
\begin{lemma}\label{lem:topDescentCoproduct}
    The model category $\Top$ has descent for coproducts.
    \begin{proof}
        	Let $\left(X_i\right)_{i\in I}, \left(Y_i\right)_{i\in I}$ be families of topological spaces and let $\left(\alpha_i\colon X_i\to Y_i\right)_{i\in I}$ be a collection of maps.
            Then define $P_{ij}$ as the pullback
            \begin{center}
                \begin{tikzcd} [sep = 1 cm]
                    P_{ij} \arrow [r] \arrow [dd] & X_i \arrow [d]\\
                    & \bigcup\limits_{i\in I}X_i \arrow [d]\\
                    Y_j \arrow [r] & \bigcup\limits_{i\in I}Y_i
                \end{tikzcd}\;.
            \end{center}
            Since $Y_i\to\bigcup\limits_{i\in I}Y_i$ is a Serre fibration, this means it is already a homotopy pullback.
            
            By universality of coproducts we have that the canonical map $\bigcup\limits_{i\in I} P_{ij}=\bigcup\limits_{i\in I} \left(X_i\times_{\bigcup\limits_{i\in I}Y_i}Y_j\right)\to \left(\bigcup\limits_{i\in I}X_i\right)\times_{\bigcup\limits_{i\in I}Y_i}Y_j$ is a weak equivalence.
            So it remains to show that $P_{ij}\cong\emptyset$ for $i\neq j$ and $P_{ii}\to X_i$ is a weak equivalence.

            Since $X_i\to\bigcup\limits_{i\in I}X_i\to\bigcup\limits_{i\in I}Y_i$ is equal to $X_i\to Y_i\to\bigcup\limits_{i\in I}Y_i$, we can also compute $P_{ij}$ as the successive pullback
            \begin{center}
                \begin{tikzcd} [sep = 1 cm]
                    P_{ij} \arrow [r] \arrow [d] & X_i \arrow [d]\\
                    Y_j\times_{\bigcup\limits_{i\in I}Y_i} Y_i \arrow [d] \arrow [r] & Y_i \arrow [d]\\
                    Y_j \arrow [r] & \bigcup\limits_{i\in I}Y_i
                \end{tikzcd}
            \end{center}
            by the pasting law for pullbacks.
            (Note that all pullbacks here are already homotopy pullbacks.)

            So it suffices to prove that $Y_j\times_{\bigcup\limits_{i\in I}Y_i} Y_i\cong\emptyset$ for $i\neq j$ and $Y_i\times_{\bigcup\limits_{i\in I}Y_i} Y_i\cong Y_i$. 
            But we have already shown this in \cref{cor:genCoproductComponentPb}.
    \end{proof}
\end{lemma}
\begin{corollary}
    The \inftycat/ of spaces $\spaces$ is an \inftytop/.
    \begin{proof}
        We know $\spaces$ is locally presentable from \cref{cor:spacesIsLocPres}.
        \Cref{cor:sufficientToProveInModCat} together with \cref{lem:topUniversalPo} and \cref{lem:topUniversalCoproduct} proves that $\spaces$ has universal colimits, and together with \cref{lem:topDescentPo} with \cref{lem:topDescentCoproduct} proves that it has descent.
    \end{proof}
\end{corollary}