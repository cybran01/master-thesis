In this section we will prove that our \inftycat/ of spaces $\spaces$ is an \inftytop/.
We will keep our proofs topological; for a different approach using simplicial sets see e.g. \cite[\S 6]{HTT}.
In order to transfer reasoning back and forth from the model category $\Top$ and the \inftycat/ $\spaces$ we need the following lemma.
\begin{lemma}\label{lem:replaceWithStrictDiagram} %TODO is the statement even true in this generality?
    Let $J$ be an ordinary category and let $M$ be a model category.
    Then a diagram $\N(J)\to\N(M)[W^{-1}]$ can be replaced with a diagram $J\to M$ that is equivalent in the sense that under the composition
    \begin{equation*}
        \Fun(\N(J),\N(M))\to\Fun(\N(J),\N(M)[W^{-1}]) %TODO is this correct?
    \end{equation*} 
    they are equivalent as objects of the \inftycat/ $\Fun(\N(J),\N(M)[W^{-1}])$.

    Furthermore the map $\Fun(\N(J),\N(M))\to\Fun(\N(J),\N(M)[W^{-1}])$ is natural in $J$.
    \begin{reference}
        \cite[Proposition 1.3.4.25]{higher_algebra}
    \end{reference}
\end{lemma}
Since both coproducts and pushouts in \inftycats/ are indexed by nerves of ordinary categories, we are able to work in the underlying model category $\Top$ by the following corollary:
\begin{corollary}\label{cor:sufficientToProveInModCat}
    Let $M$ be a model category with weak equivalences $W$. 
    Then $\N(M)[W^{-1}]$ has universal colimits if and only if $M$ has universal homotopy pushouts and universal homotopy coproducts as a model category.
    
    If further $\N(M)[W^{-1}]$ has universal colimits, then $M$ has descent if and only if it has descent for homotopy pushouts and homotopy coproducts as a model category.
    \begin{proof}
        A diagram $\N(J)^{\rhd}\to\N(M)[W^{-1}]$ is a colimit cone if and only if some equivalent replacement $J^{\rhd}\to M$ as in \cref{lem:replaceWithStrictDiagram} is a homotopy colimit cone in $M$.
        Given a diagram $J^{\rhd}\to M$, is is a homotopy colimit cone if and only if $\N(J)^{\rhd}\to\N(M)[W^{-1}]$ is a colimit cone.
        The analogous statement is also true for limits and homotopy limits.
        Thus the proposition follows from \cref{lem:univColimIffUnivPoAndCoprod} and \cref{lem:descentIffDescentPoAndCoprod} respectively.
    \end{proof}
\end{corollary}
We will use the interaction of the \Strom/ model structure and the Quillen model structure to prove universality and descent in $\Top$.
Note that whenever referring to $\Top$ as a model category, we implicitly mean the Quillen model structure. 
Cofibrations and fibrations of the \Strom/ model structure are tagged by the letter $h$.
Our most important (and quite non trivial) part of this interaction is the following proposition.
\begin{prop}\label{prop:poAlongHCofibIsHtpyPo}
    A pushout along an h-cofibration is a homotopy pushout.
    \begin{proof}
        Let
        \begin{center}
            \begin{tikzcd} [sep = 4em]
                A \arrow[>->, r, "h"] \arrow[d] & B \arrow[d]\\
                C \arrow[>->, r, "h"] & D \\
            \end{tikzcd}
        \end{center}
        be a pushout with the indicated maps being h-cofibrations.
        We factor the map $A\xtailrightarrow{} C'\xrightarrow{\sim} C$ into a cofibration followed by a weak equivalence and obtain the diagram
        \begin{center}
            \begin{tikzcd} [sep = 4em]
                C' \arrow[d,"\sim"] & A \arrow[>->, r, "h"] \arrow[>->,l] \arrow[d, equal] & B \arrow[d, equal]\\
                C & A \arrow[>->, r, "h"] \arrow[l] & B \\
            \end{tikzcd}\;.
        \end{center}
        By \cite[Proposition 1.1]{hcolim_bar} we have that the map induced by the pushouts $C'\cup_AB\to C\cup_AB$ is a again a weak equivalence.
        But since $C'\cup_AB$ is a homotopy pushout (since its a pushout along a cofibration), $C\cup_AB$ is thus also a homotopy pushout which proves the proposition.
    \end{proof}
\end{prop}
\begin{prop}\label{lem:topUniversalPo}
    The model category $\Top$ has universality for homotopy pushouts.
    \begin{proof}
        By factoring the map $D'\xrightarrow{\sim}\overline{D}\xtwoheadrightarrow{h} D$ into a weak equivalence followed by an h-fibration (which is also a fibration) and pulling back successively, we obtain the diagram %TODO explain why we we use weak equivalence here: since the faces are pb in the SERRE model str, the 3 remaining vertical maps are only w.e.
        \begin{center}
            \begin{tikzcd} [sep = .5 cm]
                A^{\prime} \arrow [dr] \arrow [rr] \arrow [dd] & & B^{\prime} \arrow [dr] \arrow [dd] \\
                & C^{\prime} \arrow [rr, crossing over] & & D^{\prime} \arrow [dd, "\sim"] & \\
                \overline{A} \arrow [dr] \arrow [rr] \arrow [->>, dd, "h"] & & \overline{B} \arrow [dr] \arrow [->>, dd, near start, "h"] \\
                & \overline{C} \arrow [rr, crossing over] \arrow [from=uu, crossing over] & & \overline{D} \arrow [->>, dd, "h"] & \\
                A \arrow [dr] \arrow [rr] & & B \arrow [dr] \\
                & C \arrow [->>,from=uu, crossing over, near start, "h"] \arrow [rr] & & D & \\
            \end{tikzcd}
        \end{center}
        where $\overline{A}=A\times_{D}\overline{D}$, $\overline{B}=B\times_{D}\overline{D}$ and $\overline{C}=C\times_{D}\overline{D}$.
        Since by the pasting lemma for homotopy pullbacks the upper vertical squares are also homotopy pullbacks and $D'\xrightarrow{\sim}\overline{D}$ is a weak equivalence, all vertical maps in the upper cube are weak equivalences.
        Thus it is sufficent to show that the square
        \begin{center}
            \begin{tikzcd} [sep = 4em]
                \overline{A} \arrow[r] \arrow[d] & \overline{B} \arrow[d] \\
                \overline{C} \arrow[r] & \overline{D} \\
            \end{tikzcd}
        \end{center}
        is a homotopy pushout.

        We factor the maps $A\xtailrightarrow{h}\widehat{B}\xrightarrow{\sim} B$ and $A\xtailrightarrow{h}\widehat{C}\xrightarrow{\sim} C$ into an h-cofibration followed by a weak equivalence.
        We obtain a diagram
        \begin{center}
            \begin{tikzcd} [sep = 4em]
                A \arrow[>->, r, "h"] \arrow[>->, d, "h"] & \widehat{B} \arrow[r, "\sim"] \arrow[>->, d, "h"] & B \arrow[dd] \\
                \widehat{C} \arrow[>->, r, "h"] \arrow[d, "\sim"] & \widehat{B}\cup_A\widehat{C} \arrow[dr] & \\
                C \arrow[rr] & & D
            \end{tikzcd}
        \end{center}
        where the small square is a homotopy pushout as its a pushout along an h-cofibration by \cref{prop:poAlongHCofibIsHtpyPo} so $\widehat{B}\cup_A\widehat{C}\to D$ is a weak equivalence.
        Pulling back the small square along the h-fibration $\overline{D}\xtwoheadrightarrow{h} D$ gives the square
        \begin{center}
            \begin{tikzcd} [sep = 4em]
                \overline{A} \arrow[>->, r, "h"] \arrow[>->, d, "h"] & \widehat{B}\times_{D}\overline{D} \arrow[>->, d, "h"] \\
                \widehat{C}\times_{D}\overline{D} \arrow[>->, r, "h" ] & \left(\widehat{B}\cup_A\widehat{C}\right)\times_{D}\overline{D}\\
            \end{tikzcd}
        \end{center}
        where the horizontal maps are h-cofibrations since pullbacks of h-cofibrations along h-fibrations are again h-cofibrations by \cite[Theorem 12]{note_on_cofibs_2}.
        
        Let $q\colon\left(\widehat{B}\cup_A\widehat{C}\right)\times_{D}\overline{D}\xtwoheadrightarrow{h}\widehat{B}\cup_A\widehat{C}$ denote the pullback of $\overline{D}\xtwoheadrightarrow{h} D$ along  $\widehat{B}\cup_A\widehat{C}\to D$.
        Since h-cofibrations are in particular closed embeddings we have $\widehat{B}\times_{D}\overline{D}\cong q^{-1}(\widehat{B})$, $\widehat{C}\times_{D}\overline{D}\cong q^{-1}(\widehat{C})$ and $\overline{A}\cong q^{-1}(A)\cong q^{-1}(\widehat{B})\cap q^{-1}(\widehat{C})$. %TODO why h-cofib closed embeddings?
        As $q^{-1}(\widehat{B})$ and $q^{-1}(\widehat{C})$ are closed, this implies by point set topology that the square is an ordinary pushout.

        Thus it is a pushout along an h-cofibration, hence by \cref{prop:poAlongHCofibIsHtpyPo} a homotopy pushout.
        Because by right properness of $\Top$ the square is equivalent to 
        \begin{center}
            \begin{tikzcd} [sep = 4em]
                \overline{A} \arrow[r] \arrow[d] & \overline{B} \arrow[d] \\
                \overline{C} \arrow[r] & \overline{D} \\
            \end{tikzcd}
        \end{center}
        this proves the proposition.
    \end{proof}
\end{prop}
\begin{lemma}\label{lem:topUniversalCoproduct}
    The model category $\Top$ has universal homotopy coproducts.
    \begin{proof}
        	Let $\left(X_i\right)_{i\in I}$ and $\left(Y_i\right)_{i\in I}$ be collections of topological spaces with $I$ small. 
            In a model category the homotopy coproduct of such a family is given by the ordinary coproduct of cofibrant replacements $\bigcup\limits_{i\in I}\widehat{X_i}$ for each of the $X_i$.
            However since in $\Top$ weak equivalences are closed under coproducts, the ordinary coproduct $\bigcup\limits_{i\in I}X_i$ is already a homotopy coproduct.

            Let $f\colon Y\to\bigcup\limits_{i\in I}X_i$ be a map and let $\left(Y_i\to X_i\right)_{i\in I}$ and $\left(Y_i\to Y\right)_{i\in I}$ be families of maps such that for all $k\in I$ we have a commutative square
            \begin{center}
                \begin{tikzcd} [sep = 4em]
                    Y_k \arrow [r] \arrow [d] & Y \arrow [d]\\
                    X_k \arrow [r] & \bigcup\limits_{i\in I} X_i
                \end{tikzcd}
            \end{center}
            that is a homotopy pullback.

            Note that every inclusion map $X_k\to\bigcup\limits_{i\in I} X_i$ is already a fibration as can be checked by its lifting properties.
            Therefore the ordinary pullback
            \begin{center}
                \begin{tikzcd} [sep = 1 cm]
                    Y\times_{\bigcup\limits_{i\in I} X_i}X_k \arrow [->>,r] \arrow [d] & Y \arrow [d]\\
                    X_k \arrow [->>,r] & \bigcup\limits_{i\in I} X_i
                \end{tikzcd}
            \end{center}
            is a homotopy pullback and so $Y_k\to Y\times_{\bigcup\limits_{i\in I} X_i}X_k$ is a weak equivalence.
            Since the canonical map $\bigcup\limits_{k\in I}\left(Y\times_{\bigcup\limits_{i\in I} X_i}X_k\right)\cong Y$ is a homeomorphism in $\Top$, $\bigcup\limits_{i\in I}Y_i\to Y$ is a weak equivalence.
            This proves that $Y$ is a homotopy coproduct.
    \end{proof}
\end{lemma}
Therefore by \cref{cor:sufficientToProveInModCat} we know $\spaces$ has universal colimits.
It remains to show that it also has descent.
%Insertion of proof of new stuff starts here -------------------------------------
\begin{definition}[Pairs and Triples of Spaces]
    Let $\Top^{(2)}$ denote the category of \emph{pairs of topological spaces} $(X,A)$ with $A\subset X$ a subspace and maps $f\colon(X_1,A_1)\to (X_2,A_2)$ being continuous maps $f\colon X_1\to X_2$ such that $\im f|_{A_1}\subset A_2$.
    We call these maps \emph{maps of pairs}.
    
    A \emph{homotopy of maps of pairs} $f,g\colon(X_1,A_1)\to (X_2,A_2)$ is a map $h\colon(X_1\times I,A_1\times I)\to (X_2,A_2)$ such that $h_0=f$ and $h_1=g$.
    We define $(X,A)\times I$ to mean $(X\times I,A\times I)$.

    In a similar fashion we define maps of triples:
    Let $\Top^{(3)}$ denote the category of \emph{triples of topological spaces} $(X,A,B)$ with $B\subset A\subset X$ subspaces and maps $f\colon(X_1,A_1,B_1)\to (X_2,A_2,B_2)$ being continuous maps $f\colon X_1\to X_2$ such that $\im f|_{A_1}\subset A_2$ and $\im f|_{B_1}\subset B_2$.
    We call these maps \emph{maps of triples}.
    
    A \emph{homotopy of maps of triples} $f,g\colon(X_1,A_1,B_1)\to (X_2,A_2,B_2)$ is a map $h\colon(X_1\times I,A_1\times I,B_1\times I)\to (X_2,A_2,B_2)$ such that $h_0=f$ and $h_1=g$.
    We define $(X,A,B)\times I$ to mean $(X\times I,A\times I, B\times I)$.
\end{definition}
\begin{definition}[Triad]
    A triple of spaces $(X;X_1,X_2)$ is called a \emph{triad} if $X_1\subset X$ and $X_2\subset X$ are subspaces of $X$.
    A triad $(X;X_1,X_2)$ is called \emph{excisive} if $X_1^°\cup X_2^°=X$.

    A map of triads $f\colon (X;X_1,X_2)\to (Y;Y_1,Y_2)$ is a continuous map $f\colon X\to Y$ such that $f(X_1)\subset Y_1$ and $f(X_2)\subset Y_2$.
\end{definition}
\begin{lemma}[Relative HEP]\label{lem:rHEP}
    Let $(X;X_1,X_2)$ be a triad such that $X_1\cap X_2\to X_2$ and $X_1\cup X_2\to X$ are h-cofibrations. 
    Then given a homotopy $h\colon(X_1,X_1\cap X_2)\times I\to(Y,A)$ and a map $f\colon(X,X_2)\to(Y,A)$ such that $h_0=f|_{X_1}$, there exists a homotopy $H\colon(X,X_2)\times I\to(Y,A)$ such that $H|_{X_1\times I}=h$ and $H_0=f$.
    \begin{proof}
        We apply the homotopy extension property twice: First we extend $h|_{\left(X_1\cap X_2\right)\times I}$ to a homotopy $h'\colon X_2\times I\to A$ with $h'_0=f|_{X_2}$.
        Then we can extend $h\cup_{X_1\cap X_2} h'\colon \left(X_1\cup X_2\right)\times I\to Y$ to the desired homotopy $H$.
    \end{proof}
\end{lemma}
%TODO add all used triads as examples HERE
%TODO define concat and inverses on htpies 
\begin{definition}[Weak Equivalence of Pairs]\label{def:weOfPairs}
    Let $f\colon(X,A)\to (Y,B)$ be a map such that 
    \begin{itemize}
        \item for all $n\geq 1$ and $a\in A$ the induced map $\pi_n(X,A,a)\to\pi_n(Y,B,f(a))$ is a bijection
        \item $\pi_0(f)^{-1}\im\left(\pi_0(B)\to\pi_0(Y)\right)=\im\left(\pi_0(A)\to\pi_0(X)\right)$.
    \end{itemize}
    Then we call $f$ a \emph{weak equivalence of pairs}.
\end{definition}
\begin{remark}\label{rmk:emptyFiber}
    For a map $f\colon(X,A)\to (Y,B)$ the inclusion $\im\left(\pi_0(A)\to\pi_0(X)\right)\subset\pi_0(f)^{-1}\im\left(\pi_0(B)\to\pi_0(Y)\right)$ always holds.
    Having an equality means that for all $x\in X$ such that there is a path from $f(x)$ to some element in $B$, there also exists a path from $x$ to some element of $A$.
    This is equivalent to saying that if the homotopy fiber of $A\hookrightarrow X$ at $x\in X$ is empty, then so is the homotopy fiber of $B\hookrightarrow Y$ at $f(x)$.
    This is because the homotopy fiber of a map at a point is empty if and only if the point is in a connected component that does not intersect the image of the map.
\end{remark}
\begin{lemma}\label{lem:weOfaPairsIsHtpyPb}
    Let $f\colon(X,A)\to (Y,B)$ be a map.
    Then the following are equivalent:
    \begin{enumerate}[label={(\roman*)}]
        \item The map $f$ is a weak equivalence of pairs. \label{itm:weOfPairs}
        \item The induced commutative square \begin{center} 
            \begin{tikzcd} [sep = 4em]
                A \arrow [r, "f|_A"] \arrow [d, hook] & B \arrow [d, hook]\\
                X \arrow [r, "f"] & Y
            \end{tikzcd}
        \end{center}
        is a homotopy pullback square. \label{itm:htpyPb}
    \end{enumerate}
    \begin{proof}
        We model the homotopy fibers for given points $x_0\in X$ and $y_0\in Y$ as the pullbacks of the squares
        \begin{center} 
            \begin{tikzcd} [sep = 4em]
                F_{x_0} \arrow [r, hook] \arrow [->>, d, "h"', "\ev_1"] & P_{x_0}X \arrow [->>, d, "h"', "\ev_1"]\\
                A \arrow [r, hook] & X
            \end{tikzcd}
            \begin{tikzcd} [sep = 4em]
                G_{y_0} \arrow [r, hook] \arrow [->>, d, "h"', "\ev_1"] & P_{y_0}Y \arrow [->>, d, "h"', "\ev_1"]\\
                B \arrow [r, hook] & Y
            \end{tikzcd}
        \end{center}
        where $P_{x_0}X$ and $P_{y_0}Y$ are the path mapping spaces of the maps $x_0\colon *\to X$ and  $y_0\colon *\to Y$ respectively.
        We can identify $F_{x_0}$ with subspace of $P_{x_0}X$ of paths that start at $x_0$ and end at a point in $A$ (and analogously for $G$).
        Let $\alpha\colon F_{x_0}\to G_{f(x_0)}, \gamma\mapsto f\gamma$ denote the induced map between fibers.
        Proving \ref{itm:htpyPb} is equivalent to showing that this map is a weak equivalence.

        If $F_{x_0}\neq\emptyset$ we have that for every point $z\in F_{x_0}$ and $n\geq 1$ the induced maps $\pi_n(X,A,z(1))\to \pi_{n-1}(F_{z(1)},\const_{z(1)})$ and $\pi_n(Y,B,f(p(z)))\to \pi_{n-1}(G_{\alpha(z)(1)},\const_{\alpha(z)(1)})$ are isomorphisms by \cite[Theorem 5.1.8]{lectures_htpy_thy}.
        The map $\Lambda\colon F_{x_0}\to F_{z(1)}, \gamma\mapsto \overline{z}*\gamma$ is a homotopy equivalence with homotopy inverse $\Lambda^{-1}\colon F_{z(1)}\to F_{x_0}, \gamma\mapsto z*\gamma$, hence it induces an isomorphism $\pi_{n-1}(F_{x_0},z)\to\pi_{n-1}(F_{z(1)},\overline{z}*z)$.
        Since there is a homotopy $H\colon I\times I\to X$ contracting the loop $\overline{z}*z$ to the point $z(1)$ such that $H_0=\overline{z}*z$, $H_1=\const_{z(1)}$ and $\partial I\times I=\const_{z(1)}$ and thus corresponds to a path $I\to F$ from $\overline{z}*z$ to $\const_{z(1)}$, we get an isomorphism $\pi_{n-1}(F_{z(1)},\overline{z}*z)\to\pi_{n-1}(F_{z(1)},\const_{z(1)})$.
        
        These isomorphisms induce a commutative square
        \begin{center} 
            \begin{tikzcd} [sep = 4em]
                \pi_n(X,A,z(1)) \arrow [r, "\cong"] \arrow [d, "\pi_n(f)"] & \pi_{n-1}(F_{x_0},z) \arrow [d, "\pi_{n-1}(\alpha)"]\\
                \pi_n(Y,B,f(z(1))) \arrow [r, "\cong"] & \pi_{n-1}(G_{f(x_0)},\alpha(z))
            \end{tikzcd}
        \end{center}
        which together with \cref{rmk:emptyFiber} for the case $F_{x_0}=\emptyset$ proves that \ref{itm:weOfPairs} and \ref{itm:htpyPb} are equivalent.
    \end{proof}
\end{lemma}
\begin{definition}
    Let $J^n=\set{0}\times I^n\cup I\times\partial I^n\subset\partial I^{n+1}\subset I^{n+1}$.

    For a pair of spaces $(X,A)$, $a\in A$  and $n\geq 1$ we set $\pi_n(X,A,a)$ to be the set of equivalence classes of maps $(I^n,\partial I^n,J^{n-1})\to(X,A,a)$ modulo homotopies of triples $(I^n,\partial I^n,J^{n-1})\times I\to(X,A,a)$ between such maps.
\end{definition}
\begin{lemma}\label{lem:compressionVariation}
    Let $n\geq 1$, $e\colon(I^{n+1},\set{1}\times I^n)\to (X,A)$, $*=(1,0,\ldots,0)\in\set{1}\times\partial I^n$ and $a=e(*)$. 
    Then $e|_{J^n}\colon(J^n,\set{1}\times\partial I^n,*)\to(X,A,a)$ represents the equivalence class of the constant map in $\pi_n(X,A,a)$. %TODO different def of \pi_n used here! J^n instead of I^n
    \begin{proof}
        Fix a homeomorphism $I^n\cong J^n$ which restricts to a homeomorphism $\partial I^n\cong \partial J^n$.
        Then the square  
        \begin{center}
            \begin{tikzcd} [sep = 4em]
                J^n\times\partial I\cup\partial J^n\times I \arrow [r] \arrow [>->, d, "h"] & I^{n+1} \arrow [->>, d, "h","\simeq"']\\
                J^n\times I \arrow[r,"\simeq"] & *
            \end{tikzcd}
        \end{center}
        where the top horizontal map on
        \begin{itemize}
            \item $J^n\times\set{0}$ is the inclusion $J^n\hookrightarrow I^{n+1}$
            \item $J^n\times\set{1}$ is the inclusion $J^n\cong\set{1}\times I^n\hookrightarrow I^{n+1}$
            \item on $\partial J^n \times I$ is the map $\partial J^n \times I\to\partial J^n\cong \set{1}\times\partial I^n \hookrightarrow I^{n+1}$
        \end{itemize}
        admits a lift $H$. 
        This lift is a homotopy $H\colon(J^n,\set{1}\times \partial I^n,*)\times I\to (X,A,a)$ from $e|_{J^n}$ to some map $H_1=f$ with $\im f\subset A$ and is constant on $\partial J^n\times I$.
        Since $\im f\subset A$ and $J^n$ is contractible with respect to $*$, $f$ represents $\const_a$ which proves the proposition. 
    \end{proof}
\end{lemma}
\begin{prop}\label{prop:eqCharInjSurj} %TODO cite w.e. and quasifibs lemma 3.3
    Given a map $f\colon(X,A)\to (Y,B)$ and $n\geq0$, the following are equivalent:
    \begin{enumerate}[label={(\roman*)}]
        \item For every $a\in A$ the map $\pi_n(X,A,a)\to \pi_n(Y,B,f(a))$ is injective and $\pi_{n+1}(X,A,a)\to \pi_{n+1}(Y,B,f(a))$ is surjective.
            If $n=0$, replace the injectivity statement by $\pi_0(f)^{-1}\im\left(\pi_0(B)\to\pi_0(Y)\right)=\im\left(\pi_0(A)\to\pi_0(X)\right)$. \label{itm:injAndSurjOfHtpyGrps}
        \item Given maps $e\colon(I^{n+1},\set{1}\times I^n)\to (Y,B)$, $g(J^n,\set{1}\times\partial I^n)\to(X,A)$ and a homotopy $h\colon(J^n,\set{1}\times\partial I^n)\times I\to(Y,B)$ from $e|_{J^n}$ to $fg$, there exists a map $G\colon(I^{n+1},\set{1}\times I^n)\to(X,A)$ and homotopy $H\colon(I^{n+1},\set{1}\times I^n)\times I\to(Y,B)$ from $e$ to $fG$ such that $H|_{J^n\times I}=h$ and $G|_{J^n}=g$. \label{itm:liftingMapsAndHtpy}
        \item The statement of \ref{itm:liftingMapsAndHtpy} holds with the additional assumption that $e|_{J^n}=fg$ and $h$ is the constant homotopy. \label{itm:liftingMapsAndHtpyConst}
    \end{enumerate}
    \begin{proof}
        \ref{itm:liftingMapsAndHtpyConst} implies \ref{itm:injAndSurjOfHtpyGrps}:

        We first prove the injectivity statement for $n=0$. 
        By \cref{rmk:emptyFiber} we need to show that when given $x\in X$ and path $f(x)$ to some element of $B$, there exists a path from $x$ to some element of $A$.

        So let $e$ be a path from $f(x)$ to $b\in B$ and $g=x$.
        Then by \ref{itm:liftingMapsAndHtpyConst} there is a map $G\colon (I,\set{1})\to(X,A)$ such that $G(0)=g=x$.
        This is the desired path.

        We next prove that for $n\geq 1$ \ref{itm:liftingMapsAndHtpyConst} implies injectivity of $\pi_n(X,A,a)\to \pi_n(Y,B,f(b))$.
        Given $[\lambda],[\mu]\in\pi_n(X,A,a)$ such that $\pi_n(f)([\lambda])=\pi_n(f)([\mu])$, we have a homotopy $\widetilde{h}\colon(I^n,\partial I^n,J^{n-1})\times I\to (Y,B,b)$ from $f\lambda$ to $f\mu$.
        Set $e=\widetilde{h}$ and using $J^n=J^{n-1}\times I\cup I^n\times\partial I$ define $g$ to be $\lambda$ on $I^n\times\set{0}$, $\mu$ on $I^n\times\set{1}$ and $\const_a$ on $J^{n-1}\times I$.
        Then $e|_{J^n}=fg$ and so by \ref{itm:liftingMapsAndHtpyConst} we get a map $G\colon(I^{n+1},\set{1}\times I)\to(X,A)$ with $G|_{J^n}=g$. 
        Thus $G$ is a homotopy $(I^n,\partial I^n,J^{n-1})\times I\to(X,A,a)$ from $\lambda$ to $\mu$, hence $[\lambda]=[\mu]$ in $\pi_n(X,A,a)$.

        Lastly we show that for $n\geq 0$ \ref{itm:liftingMapsAndHtpyConst} implies surjectivity of $\pi_{n+1}(X,A,a)\to \pi_{n+1}(Y,B,f(b))$.
        Let $[\lambda]\in\pi_n(Y,B,f(a))$ and set $e=\lambda$ and $g=\const_a$. 
        Then we get a map $G\colon(I^{n+1},\set{1}\times I^n)\to(X,A)$ and homotopy $H\colon(I^{n+1},\set{1}\times I^n)\times I\to(Y,B)$ from $e$ to $fG$ such that $H|_{J^n\times I}$ is $\const_{f(a)}$ and $G|_{J^n}=\const_a$.
        Thus $G$ is a map $(I^{n+1},\partial I^{n+1},J^n)\to (X,A,a)$ and $H$ is a homotopy $(I^{n+1},\partial I^{n+1},J^n)\times I\to(Y,B,f(a))$ from $\lambda$ to $fG$ which shows $[\lambda]=[fG]$ in $\pi_n(Y,B,f(a))$.

        \ref{itm:liftingMapsAndHtpyConst} implies \ref{itm:liftingMapsAndHtpy}:

        Let $e\colon(I^{n+1},\set{1}\times I^n)\to (Y,B)$, $g(J^n,\set{1}\times\partial I^n)\to(X,A)$ and $h\colon(J^n,\set{1}\times\partial I^n)\times I\to(Y,B)$ be a homotopy from $e|_{J^n}$ to $fg$.
        By the relative homotopy extension property (\cref{lem:rHEP}) applied to the triad $(I^{n+1};J^n,\set{1}\times I^n)$, there exists a homotopy $j\colon(I^{n+1},\set{1}\times I^n)\times I\to(Y,B)$ extending $h$ such that $j_0=e$.
        Since $j_1|_{J^n}=fg$ we can apply \ref{itm:liftingMapsAndHtpyConst} to get a map $G\colon(I^{n+1},\set{1}\times I^n)\to(X,A)$ such that $G|_{J^n}=g$ and a homotopy $k\colon(I^{n+1},\set{1}\times I^n)\times I\to(Y,B)$ such that $k|_{J^n\times I}$ is the constant homotopy of $j_1|_{J^n}=fg$. 
        Let $\const_{fg}\colon(J^n\times I,\set{1}\times\partial I^n\times I)\to (Y,B)$ denote the constant homotopy of $j_1|_{J^n}=fg$.
        Then we can pick a homotopy $L\colon(J^n\times I,\set{1}\times\partial I^n\times I)\times I\to (Y,B)$ from $h*\const_{fg}$ to $h$ such that $L|_{J^n\times\set{0}\times I}=e|_{J^n}\times\id_I$ and $L|_{J^n\times\set{1}\times I}=fg\times\id_I$. %TODO why?
        We now apply \cref{lem:rHEP} again to the triad $(I^{n+1}\times I;J^{n+1},\set{1}\times I^n\times I)$, noting that $J^{n+1}=J^n\times I\cup I^{n+1}\times\partial I$, to extend the homotopy given by the union of $L$, $e$ on $I^{n+1}\times\set{0}\times I$ and $fG$ on $I^{n+1}\times\set{1}\times I$ to a homotopy $\widetilde{L}\colon(I^{n+1}\times I,\set{1}\times I^n\times I)\times I\to(Y,B)$ such that $\widetilde{L}_0=j*k$.
        Setting $H=\widetilde{L}_1$ gives the desired homotopy.

        \ref{itm:injAndSurjOfHtpyGrps} implies \ref{itm:liftingMapsAndHtpyConst}:

        For $n=0$ we view $e$ as a path from $f(g)$ to some element of $b\in B$, so by \cref{rmk:emptyFiber} we can choose a path $G'\colon I\to X$ which goes from $g$ to some element $a\in A$.
        Since $\pi_1(f)\colon\pi_1(X,A,a)\to\pi_1(Y,B,f(a))$ is surjective we can pick a map $\lambda\colon(I^1,\partial I^1,J^0)\to(X,A,a)$ and a homotopy $L\colon(I^1,\partial I^1,J^0)\times I\to(Y,B,f(a))$ from $\overline{fG'}*e$ to $f\lambda$.
        Adding the constant homotopy, we obtain a homotopy $M\colon(I^1,\partial I^1,J^0)\times I\to (Y,B,f(g))$ from $fG'*(\overline{fG'}*e)$ to $fG'*f\lambda=f(G'*\lambda)$.
        We can now pick a homotopy $N\colon(I^1,\partial I^1,J^0)\times I\to(Y,B,f(g))$ from $fG'*(\overline{fG'}*e)$ to $e$. 
        Setting $G=G'*\lambda$ and $H=\overline{N}*M$ then gives the desired maps.

        For $n\geq 1$, we fix a homotopy equivalence of triples $(I^n,\partial I^n, J^{n-1})\simeq (J^n,\set{1}\times\partial I^n,*)$ where $*=(1,0,\ldots,0)\in I^{n+1}$.
        Given maps $e\colon(I^{n+1},\set{1}\times I^n)\to (Y,B)$ and $g(J^n,\set{1}\times\partial I^n)\to(X,A)$ with $e|_{J^n}=fg$, setting $a=g(*)$ we can view $g$ as an element of $\pi_n(X,A,a)$ via the aforementioned homotopy equivalence of triples.
        Then $fg$ is homotopic to the constant map in $\pi_n(Y,B,f(a))$ by \cref{lem:compressionVariation}.
        By injectivity, we thus get a homotopy $j\colon(J^n,\set{1}\times\partial I^n)\times I\to(X,A,a)$ from $g$ to $\const_a$.
        Next we apply the relative homotopy extension property to the triad $(I^{n+1};J^n,\set{1}\times I^n)$ to extend the homotopy $fj$ to a homotopy $K\colon(I^{n+1},\set{1}\times I^n)\times I\to(Y,B)$ such that $K_0=e$.
        Since $K_1|_{J^n}=\const_{f(a)}$, it is a map $K_1\colon(I^{n+1},\partial I^{n+1},J^n)\to(Y,B,f(a))$ and represents an element in $\pi_{n+1}(Y,B,f(a))$.
        By surjectivity of $\pi_{n+1}(f)$ we then get a map $J_1\colon(I^{n+1},\partial I^{n+1},J^n)\to(X,A,a)$ and a homotopy of triples $L$ from $K_1$ to $fJ_1$.
        By the  relative homotopy extension property for $(I^{n+1};J^n,\set{1}\times I^n)$ we extend the reverse homotopy of $j$ to a homotopy of $J\colon(I^{n+1},\set{1}\times I^n)\to(X,A)$ which ends at $J_1$.
        Setting $G=J_0$ we have $G|_{J^n}=j_0=g$.
        Furthermore we have the homotopy $(K*L)*\overline{fJ}\colon(J^n,\set{1}\times\partial I^n)\times I\to(Y,B)$ from $fg$ to $fg$.
        We pick a homotopy $M\colon(J^n\times I,\set{1}\times\partial I^n\times I)\times I\to(Y,B)$ from $(fj*\const_{f(a)})*\overline{fj}$ to the constant homotopy at $fg$ such that $M|_{J^n\times\set{0}\times I}$ and $M|_{J^n\times\set{1}\times I}$ are both the constant homotopies at $fg$.
        Using the relative homotopy extension property on $(I^{n+2};J^n\times I\cup I^{n+1}\times\set{0}\cup I^{n+1}\times\set{1},\set{1}\times I^{n+1})$, we extend the homotopy given by the constant homotopy at $e$ on $I^{n+1}\times\set{0}\times I$, the constant homotopy at $fG$ on $I^{n+1}\times\set{1}\times I$ and $M$ on $J^n\times I\times I$ to a homotopy $\widetilde{M}\colon(I^{n+2},\set{1}\times I^{n+1})\times I\to(Y,B)$ with $\widetilde{M}_0=(K*L)*\overline{fJ}$.
        Now setting $H=\widetilde{M_1}$ gives the desired homotopy from $e$ to $fG$ that is the constant homotopy at $fg$ on $J^n\times I$.
    \end{proof}
\end{prop}
%TODO put mapping cylinder here
\begin{lemma}\label{lem:replaceByEmbedding}
    Let $(X;X_1,X_2)$ and $(Y;Y_1,Y_2)$ be excisive triads and let $f\colon (X;X_1,X_2)\to (Y;Y_1,Y_2)$ be a map of triads.
    Then there exists an excisive triad $(\widetilde{Y};\widetilde{Y_1},\widetilde{Y_2})$ and maps of triads $\widetilde{f}\colon(X;X_1,X_2)\to(\widetilde{Y};\widetilde{Y_1},\widetilde{Y_2})$, $j\colon(\widetilde{Y};\widetilde{Y_1},\widetilde{Y_2})\to(Y;Y_1,Y_2)$ factoring $j\widetilde{f}=f$ such that
    \begin{itemize}
        \item $\widetilde{f}$ is an inclusion and $j$ is a weak equivalence 
        \item $f(X_i^°)=f(X)\cap \widetilde{Y_i}^°$ for $i\in\set{1,2}$
        \item $j$ induces a weak equivalence of pairs $(\widetilde{Y_i},\widetilde{Y_1}\cap \widetilde{Y_2})\to(Y_i,Y_1\cap Y_2)$ for $i\in\set{1,2}$
        \item and $j$ induces a weak equivalence of pairs $(\widetilde{Y},\widetilde{Y_i})\to(Y,Y_i)$ for $i\in\set{1,2}$.
    \end{itemize}
    \begin{proof}
        Let $\widetilde{Y}=\M(f)$ be the mapping cylinder of $f$.
        We set $\widetilde{Y_i}=\M(f|_{X_i})\cup f^{-1}(Y_i)\times\left(0,\frac{1}{2}\right)$ where we consider $f|_{X_i}$ to be a map $X_i\to Y_i$.
        Then the triad $(\widetilde{Y};\widetilde{Y_1},\widetilde{Y_2})$ together with the choices of $\widetilde{f}\colon X\to \widetilde{Y}$ as the inclusion into the mapping cylinder and as $g\colon\widetilde{Y}\to Y$ the contraction map give a factorization of $f$.
        
        We first show that $(\widetilde{Y};\widetilde{Y_1},\widetilde{Y_2})$ is excisive:
        First assume $x\in X\times(0,1]\subset\M(f)$, then $x\in X_i^°\times(0,1]$ for some $i\in\set{1,2}$.
        Since $X_i^°\times(0,1]$ is open and contained in $\widetilde{Y_i}$, we have $x\in \widetilde{Y_i}^°$.
        
        If $x\in Y\subset\M(f)$, then $x\in Y_i^°\subset\M(f)$ for some $i\in\set{1,2}$.
        In this case $x$ is contained in the open set $f^{-1}(Y_i^°)\times (0,\frac{1}{2})\cup Y_i^°\subset\widetilde{Y_i}$, hence $x\in \widetilde{Y_i}^°$.

        We now show $f(X_i^°)=f(X)\cap \widetilde{Y_i}^°$: 
        Since $X_i^°\times(\frac{1}{2},1]=Y_i^°\cap X\times(\frac{1}{2},1]$, we have $f(X_i^°)\supset f(X)\cap \widetilde{Y_i}^°$.
        The other inclusion follows from the definition of $\widetilde{Y_i}$.

        The remaining assumptions on $j$ follow from the following observation: 
        The canonical deformation retraction of $M(f)$ to $Y$ restricts to the subspaces $\widetilde{Y_1},\widetilde{Y_2}$ and $\widetilde{Y_1}\cap\widetilde{Y_2}$.
        Hence for all $i\in\set{1,2}$ the indicated maps in the commutative squares
        \begin{center} 
            \begin{tikzcd} [sep = 4em]
                \widetilde{Y_i} \arrow [r, "j|_{\widetilde{Y_i}}", "\simeq"'] \arrow [d, hook] & Y_i \arrow [d, hook]\\
                \widetilde{Y} \arrow [r, "j", "\simeq"'] & Y
            \end{tikzcd}
            \begin{tikzcd} [sep = 4em]
                \widetilde{Y_1}\cap\widetilde{Y_2} \arrow [r, "j|_{\widetilde{Y_1}\cap\widetilde{Y_2}}", "\simeq"'] \arrow [d, hook] & Y_1\cap Y_2 \arrow [d, hook]\\
                \widetilde{Y_i} \arrow [r, "j|_{\widetilde{Y_i}}", "\simeq"'] & Y_i
            \end{tikzcd}
        \end{center} 
        are homotopy equivalences (hence weak equivalences).
        Therefore the proposition follows by \cref{lem:weOfaPairsIsHtpyPb}.
    \end{proof}
\end{lemma}
%TODO mention lebesgue number lemma and J^n\subset I^{n+1} DR-pair
\begin{prop}%TODO cite quasifibs Thm 1.2
    Let $(X;X_1,X_2)$ and $(Y;Y_1,Y_2)$ be excisive triads and let $f\colon (X;X_1,X_2)\to (Y;Y_1,Y_2)$ be a map of triads.
    Then if the maps $f\colon(X_i,X_1\cap X_2)\to(Y_i,Y_1\cap Y_2)$ are weak equivalences of pairs for $i\in\set{1,2}$, the maps $f\colon(X,X_i)\to(Y,Y_i)$ are weak equivalences of pairs for $i\in\set{1,2}$ as well.
    \begin{proof}
        We can assume $f$ to be an inclusion such that $f(X_i^°)=f(X)\cap Y_i^°$ for $i\in\set{1,2}$ by applying \cref{lem:replaceByEmbedding} and replacing it with $\widetilde{f}$.
        The assumptions on $g$ together with \cref{lem:weOfaPairsIsHtpyPb} and the pasting law for homtopy pullbacks show that proving the statement of this proposition for $\widetilde{f}$ also proves it for $f$.
        
        By the definition of a weak equivalence of pairs, it suffices to show that under the given assumptions the statement of \cref{prop:eqCharInjSurj} \ref{itm:liftingMapsAndHtpyConst} holds for all $n\geq 0$.
    
        So let $g\colon(J^n,\set{1}\times\partial I^n)\to(X,X_i)$ and $g\colon(I^{n+1},\set{1}\times I^n)\to(Y,Y_i)$ be given such that $e|_{J^n}=g$.
        As $J^n\subset I^{n+1}$ are both compact metric spaces, we can find a Lebesgue number $\delta$ for the cover $Y_1^°,Y_2^°\subset Y$ and the map $e$.
        We pick a cubical subdivision $\mathcal{C}$ of $I^{n+1}$ such that the diameter of every cube $c\in\mathcal{C}$ is smaller then $\delta$.
        This ensures that for every cube $c\in\mathcal{C}$ there is a $i\in\set{1,2}$ such that $e(c)\subset Y_i^°$.
        Whenever $e(c)\subset Y_i^°$ we also have $g(c\cap J^n)\subset X_i^°$ for the same $i\in\set{1,2}$, since $fg(c\cap J^n)\subset f(X)\cap Y_i^°=f(X_i^°)$ and $f$ is an inclusion.
        Note that this cubical subdivision gives a CW-structure on $I^{n+1}$ such that $J^n\subset I^{n+1}$ is a subcomplex.
    \end{proof}
\end{prop}
%TODO conclude intersection stable version from descent, hypercover prop 4.6
%and ends here -------------------------------------
We start proving the case for pushouts by the following reduction step. 
\begin{prop}\label{prop:reductionStepDescent}
    If for all cubes in the model category $\Top$
    \begin{center}
        \begin{tikzcd} [sep = .5 cm]
            \overline{A} \arrow [dr] \arrow [rr] \arrow [->>,dd] & & \overline{B} \arrow [dr] \arrow[->>,dd] \\
            & \overline{C} \arrow [rr, crossing over] & & \overline{D} \arrow [dd] & \\
            A \arrow [dr] \arrow [rr] & & B \arrow [dr] \\
            & C \arrow [->>,from=uu,crossing over] \arrow [rr] & & D &
        \end{tikzcd}
    \end{center}
    where 
    \begin{itemize}
        \item left and back face are homotopy pullbacks
        \item bottom and top square homotopy pushouts %TODO is ordinary po necessary?
        \item the maps $\overline{A}\to A$, $\overline{B}\to B$ and $\overline{C}\to C$ are fibrations
    \end{itemize}
    the front and right squares are homotopy pullbacks, then $\Top$ has descent for homotopy pushouts.
    \begin{proof}
        Starting from a cube 
        \begin{center}
            \begin{tikzcd} [sep = .5 cm]
                A^{\prime} \arrow [dr] \arrow [rr] \arrow [dd] & & B^{\prime} \arrow [dr] \arrow [dd] \\
                & C^{\prime} \arrow [rr, crossing over] \arrow [dd] & & D^{\prime} \arrow [dd] & \\
                A \arrow [dr] \arrow [rr] & & B \arrow [dr] \\
                & C \arrow [from=uu, crossing over] \arrow [rr] & & D &
            \end{tikzcd}
        \end{center}
        where the back and left square are homotopy pullbacks and top and bottom are homotopy pushouts, we show that we can construct a cube of the required form such that all faces are equivalent to the corresponding faces of the starting cube.
        
        We first factor the maps $A\xtailrightarrow{}\hat{C}\xtwoheadrightarrow[]{\sim} C$ and $B'\xrightarrow{\sim}\overline{B}\xtwoheadrightarrow{} B$ to obtain the diagram
        \begin{center}
            \begin{tikzcd} [sep = 4em]
                C^{\prime} \arrow[dd] & \hat{C}\times_CC' \arrow[->>]{l}[swap]{\sim} \arrow[dd] & A^{\prime} \arrow[l] \arrow[r] \arrow[d, "\sim"] & B^{\prime} \arrow[d, "\sim"] \\
                && \overline{A} \arrow [->>,d, ""] \arrow[r] & \overline{B} \arrow[->>,d, ""] \\
                C & \hat{C} \arrow[->>]{l}{}[swap]{\sim} & A \arrow[>->,l, ""] \arrow[r] & B 
            \end{tikzcd}
        \end{center}
        where 
        \begin{itemize}
            \item $\overline{A}=A\times_{B}\overline{B}$
            \item $A'\to \overline{A}$ is a weak equivalence since $B'\to\overline{B}$ is one by the pasting law for homotopy pullbacks as the outer and lower square are homotopy pullbacks.
        \end{itemize}
        Next we factor the map $A'\xtailrightarrow{}X\xrightarrow{\sim}\hat{C}\times_CC'$. 
        Note that the diagram 
        \begin{center}
            \begin{tikzcd} [sep = 4em]
                X \arrow[r, "\sim"] \arrow[d] & C' \arrow[d] \\
                \hat{C} \arrow[->>]{r}{\sim}[swap]{} & C \\
            \end{tikzcd}
        \end{center}
        is a homotopy pullback square.
        We form the diagram
        \begin{center}
            \begin{tikzcd} [sep = 4em]
                C^{\prime} \arrow[dd] & X \arrow{l}[swap]{\sim} \arrow[d, "\sim"] & A^{\prime} \arrow[>->,l, ""] \arrow[r] \arrow[d, "\sim"] & B^{\prime} \arrow[d, "\sim"] \\
                & X\cup_{A'}\overline{A} \arrow[d] & \overline{A} \arrow[>->,l, ""] \arrow [->>,d, ""] \arrow[r] & \overline{B} \arrow[->>,d, ""] \\
                C & \hat{C} \arrow[->>]{l}{}[swap]{\sim} & A \arrow[>->,l, ""] \arrow[r] & B 
            \end{tikzcd}
        \end{center}
        where $X\to X\cup_{A'}\overline{A}$ is a weak equivalence as its a pushout of a weak equivalence along a cofibration.
        By the pasting law we have that
        \begin{center}
            \begin{tikzcd} [sep = 4em]
                A' \arrow[>->,r, ""] \arrow[d] & X \arrow[d] \\
                A \arrow[>->,r, ""] & \hat{C} \\
            \end{tikzcd}
        \end{center}
        is a homotopy pullback and since this square is equivalent to the square
        \begin{center}
            \begin{tikzcd} [sep = 4em]
                \overline{A} \arrow[>->,r, ""] \arrow[->>,d, ""] & X\cup_{A'}\overline{A} \arrow[d] \\
                A \arrow[>->,r, ""] & \hat{C} \\
            \end{tikzcd}
        \end{center}
        we know that both are homotopy pullbacks.
        Finally, we factor the map $X\cup_{A'}\overline{A}\xtailrightarrow[]{\sim}\overline{C}\xtwoheadrightarrow{}\hat{C}$.
        Then 
        \begin{center}
            \begin{tikzcd} [sep = 4em]
                \overline{A} \arrow[>->,r, ""] \arrow[->>,d, ""] & \overline{C} \arrow[->>,d, ""] \\
                A \arrow[>->,r, ""] & \hat{C} \\
            \end{tikzcd}
        \end{center}
        is again a homotopy pullback square. %TODO maybe say that this reduction is possible in any model cat

        Thus we have obtained a diagram
        \begin{center}
            \begin{tikzcd} [sep = 4em]
                \overline{C} \arrow[->>,d, ""] & \overline{A} \arrow[>->,l, ""] \arrow[r] \arrow[->>,d, ""] & \overline{B} \arrow[->>,d, ""] \\
                \hat{C} & A \arrow[>->,l, ""] \arrow[r] & B \\
            \end{tikzcd}
        \end{center}
        and we can form the cube 
        \begin{center}
            \begin{tikzcd} [sep = .5 cm]
                \overline{A} \arrow[>->,dr, ""] \arrow [rr] \arrow [->>,dd,""] & & \overline{B} \arrow [>->,dr,""] \arrow[->>]{dd}[near start]{} \\
                & \overline{C} \arrow [rr, crossing over] & & \overline{C}\cup_{\overline{A}}\overline{B} \arrow [dd] & \\
                A \arrow[>->,dr, ""] \arrow [rr] & & B \arrow [>->,dr,""] \\
                & \hat{C} \arrow[->>, from=uu, crossing over]{}[near start]{} \arrow [rr] & & \hat{C}\cup_A B &
            \end{tikzcd}
        \end{center}
        by taking pushouts.
        Since these are pushouts along cofibrations, they are already homotopy pushouts. 
        Following the construction of this cube, we see that all faces are equivalent to their corresponding faces of the starting cube which proves the proposition.
    \end{proof}
\end{prop}
For the last part of the proof we need the following definitions.
\begin{definition}[Quasifibration]
    Let $M$ be a model category and $f\colon X\to Y$ be a map.
    Then we call $f$ a \emph{quasifibration} if for every map $*\to Y$ (where $*$ is a terminal object) the ordinary pullback square
    \begin{center}
        \begin{tikzcd} [sep = 4em]
            F \arrow[r] \arrow[d] & X \arrow[d, "f"] \\
            * \arrow[r] & Y\\
        \end{tikzcd}
    \end{center}
    is a homotopy pullback.
\end{definition}
\begin{definition}[Mapping Cylinder]
    Let $f\colon A\to B$ be a map of topological spaces and let $I=[0,1]$.
    Then we let
    \begin{equation*}
        \M(f)\coloneqq\faktor{\left(A\times I\right)\cup B}{(a,0)\sim f(a)}
    \end{equation*}
    denote the \emph{mapping cylinder of $f$}.

    It will be convenient to also allow other intervalls instead of $I=[0,1]$; we will allow $I=[0,x]$ and $I=[0,x)$ for $x>0$.
    The analogous construction with $I=[0,x]$ will be referred to as a \emph{closed mapping cylinder of $f$};
    the construction with half open intervalls $I=[0,x)$ will be referred to as an \emph{open mapping cylinder of $f$}. %TODO perhaps describe factorization
\end{definition}
%TODO reference for inclusion into mapping cylinder is closed Hurewicz cofib
\begin{lemma}\label{lem:mapOfCylIsQuasiFib} %TODO prove that the claimed ordinary pbs are actuall ordinary pbs
    Let 
    \begin{center}
        \begin{tikzcd} [sep = 4em]
            X \arrow[d,->>] \arrow[r, "f"] & Y \arrow[d,->>] \\
            A \arrow[r, "g"] & B \\
        \end{tikzcd}
    \end{center}
    be a homotopy pullback (but not necessarily a pullback).
    Then the induced map $\M(f)\to\M(g)$ between mapping cylinders is a quasifibration.
    The same holds true for closed and open mapping cylinders.
    \begin{proof}
        For convenience we will only prove the case $I=[0,1]$ since the other cases follow by the analogous argument.
        The proof follows \cite[Lemma 5.10.6]{cubical_htpy_theory}.

        Let $\hat{b}\in\M(g)$. 
        Then either $\hat{b}=(\hat{a},\hat{t})\in A\times(0,1]\subset\M(g)$ or $\hat{b}\in B\subset\M(g)$.
        In the first case, we have a diagram
        \begin{center}
            \begin{tikzcd} [sep = 4em]
                X \arrow[r, "{x\mapsto (x,\hat{t})}"] \arrow[->>,d] & \M(f) \arrow[r, "\sim"] \arrow[d] & Y \arrow[->>,d] \\
                A \arrow[r, "{a\mapsto (a,\hat{t})}"] & \M(g) \arrow[r, "\sim"] & B \\
            \end{tikzcd}
        \end{center}
        where the outer square is the starting square and in particular homotopy pullback, so by the pasting law the left square is homotopy pullback as well.
        Since the left square is also an ordinary pullback, in the diagram
        \begin{center}
            \begin{tikzcd} [sep = 4em]
                F \arrow[r] \arrow[d] & X \arrow[r, "{x\mapsto (x,\hat{t})}"] \arrow[->>,d] & \M(f) \arrow[d] \\
                * \arrow[r, "\hat{b}"] & A \arrow[r, "{a\mapsto (a,\hat{t})}"] & \M(g) \\
            \end{tikzcd}
        \end{center}
        the left square is homotopy pullback since its the pullback along a fibration.
        Thus the outer square is a homotopy pullback which proves that the fiber $F$ is a homotopy fiber.

        In the second case we have the diagram
        \begin{center}
            \begin{tikzcd} [sep = 4em]
                F \arrow[r] \arrow[d] & Y \arrow[r, "\sim"] \arrow[->>,d] & \M(f) \arrow[d] \\
                * \arrow[r, "\hat{b}"] & B \arrow[r, "\sim"] & \M(g) \\
            \end{tikzcd}
        \end{center}
        where the indicated maps are weak equivalences since they are right inverse to the maps $\M(f)\to Y$ and $\M(g)\to B$ respectively, which are weak equivalences themselves.
        Since the right square is pullback and homotopy pullback and the left side is as well, this means that the fiber $F$ is already a homotopy fiber which proves the proposition.
    \end{proof}
\end{lemma}
\begin{prop}\label{lem:topDescentPo}
    The model category $\Top$ has descent for homotopy pushouts. 
    \begin{proof}
        By \cref{prop:reductionStepDescent} it is sufficient to prove that for all cubes 
        \begin{center}
            \begin{tikzcd} [sep = .5 cm]
                \overline{A} \arrow [dr] \arrow [rr] \arrow [->>,dd] & & \overline{B} \arrow [dr] \arrow[->>]{dd}[near start]{} \\
                & \overline{C} \arrow [rr, crossing over] & & \overline{D} \arrow [dd] & \\
                A \arrow [dr] \arrow [rr] & & B \arrow [dr] \\
                & C \arrow [->>,from=uu,crossing over]{}[near start]{} \arrow [rr] & & D &
            \end{tikzcd}
        \end{center}
        where 
        \begin{itemize}
            \item left and back face are homotopy pullbacks
            \item bottom and top square are homotopy pushouts
            \item the maps $\overline{A}\to A$, $\overline{B}\to B$ and $\overline{C}\to C$ are fibrations
        \end{itemize}
        the front and right squares are homotopy pullbacks.

        By \cref{lem:mapOfCylIsQuasiFib} we know that the induced maps $\M(f_{\overline{A}\overline{C}})\to\M(f_{AC})$ and $\M(f_{\overline{A}\overline{B}})\to\M(f_{AB})$ are quasifibrations. %TODO why are inclusions cofibs
        So we can form the cube 
        \begin{center}
            \begin{tikzcd} [sep = .75 cm]
                \overline{A} \arrow [>->, dr, "h"] \arrow [>->, rr, "h"] \arrow [->>,dd] & & \M(f_{\overline{A}\overline{B}}) \arrow [>->, dr, "h"] \arrow[dd] \\
                & \M(f_{\overline{A}\overline{C}}) \arrow [>->, rr, crossing over, near start,  "h"] & & \M(f_{\overline{A}\overline{C}})\cup_{\overline{A}}\M(f_{\overline{A}\overline{B}}) \arrow [dd] & \\
                A \arrow [>->, dr, "h"] \arrow [>->, rr, near start, "h"] & & \M(f_{AB}) \arrow [>->, dr, "h"] \\
                & \M(f_{AC}) \arrow [from=uu,crossing over] \arrow [>->, rr, "h"] & & \M(f_{AC})\cup_A\M(f_{AB}) &
            \end{tikzcd}
        \end{center}
        where the indicated maps are h-cofibrations as they are either inclusions into the mapping cylinder (see e.g. \cite[Theorem 2]{note_on_cofibs_1}) or pushouts of h-cofibrations.
        All faces are again equivalent to the corresponding faces of the original cube since top and bottom squares are homotopy pushouts by \cref{prop:poAlongHCofibIsHtpyPo}.
        
        Next we prove that the map $p\colon\M(f_{\overline{A}\overline{C}})\cup_{\overline{A}}\M(f_{\overline{A}\overline{B}})\to\M(f_{AC})\cup_A\M(f_{AB})$ is a quasifibration.
        By \cite[Lemma 4K.3]{hatcher2002algebraic} we only need to find open sets $U_1,U_2\subset\M(f_{AC})\cup_A\M(f_{AB})$ covering $\M(f_{AC})\cup_A\M(f_{AB})$ such that the induced maps $p^{-1}(U_1)\to U_1$, $p^{-1}(U_2)\to U_2$ and $p^{-1}(U_1\cap U_2)\to U_1\cap U_2$ are quasifibrations.

        We take $U_1=\M(f_{AC})\cup_A\M(f_{AB})\setminus{B}$ and $U_2=\M(f_{AC})\cup_A\M(f_{AB})\setminus{C}$. 

        We identify $U_1\cap U_2=A\times [0,1)\cup_{A\times\set{0}} A\times[0,1)\cong A\times (0,2)$ via the ``obvious'' homeomorphism. %TODO explain
        Then $p^{-1}(U_1)=\M(f_{\overline{A}\overline{B}})\cup_{\overline{A}}\M(f_{\overline{A}\overline{B}})\setminus{\overline{B}}$, $p^{-1}(U_2)=\M(f_{\overline{A}\overline{B}})\cup_{\overline{A}}\M(f_{\overline{A}\overline{B}})\setminus{\overline{C}}$ and $p^{-1}(U_1\cap U_2)=\overline{A}\times (0,2)$ (identified with $\overline{A}\times [0,1)\cup_{\overline{A}\times\set{0}} \overline{A}\times[0,1)$ as before).
        It follows $p^{-1}(U_1)\to U_1$ and $p^{-1}(U_2)\to U_2$ are quasifibrations by \cref{lem:mapOfCylIsQuasiFib} since they are both maps of open mapping cylinders induced by the squares 
        \begin{center}
            \begin{tikzcd} [sep = 4em]
                \overline{A} \arrow[d,->>] \arrow[r, "f_{\overline{A}\overline{C}}"] & \overline{C} \arrow[d,->>] \\
                A \arrow[r, "f_{AC}"] & C \\
            \end{tikzcd} 
            \begin{tikzcd} [sep = 4em]
                \overline{A} \arrow[d,->>] \arrow[r, "f_{\overline{A}\overline{B}}"] & \overline{B} \arrow[d,->>] \\
                A \arrow[r, "f_{AB}"] & B \\
            \end{tikzcd}
        \end{center}
        respectively.
        Since the map $p^{-1}(U_1\cap U_2)\to U_1\cap U_2$ is a product of fibrations $f_{\overline{A}A}\times\id_{(0,2)}$, it is again a fibration and thus a quasifibration.

        Finally, since the squares 
        \begin{center}
            \begin{tikzcd} [sep = 4em]
                \M(f_{\overline{A}\overline{C}}) \arrow[d] \arrow[r] & \M(f_{\overline{A}\overline{C}})\cup_{\overline{A}}\M(f_{\overline{A}\overline{B}}) \arrow[d] \\
                \M(f_{AC}) \arrow[r] & \M(f_{AC})\cup_A\M(f_{AB}) \\
            \end{tikzcd}
            \begin{tikzcd} [sep = 4em]
                \M(f_{\overline{A}\overline{B}}) \arrow[d] \arrow[r] & \M(f_{\overline{A}\overline{C}})\cup_{\overline{A}}\M(f_{\overline{A}\overline{B}}) \arrow[d] \\
                \M(f_{AB}) \arrow[r] & \M(f_{AC})\cup_A\M(f_{AB}) \\
            \end{tikzcd}
        \end{center}
        are both ordinary pullbacks %TODO maybe explain why
        with both vertical maps quasifibrations, they are already homotopy pullbacks. %TODO maybe put this as an extra statement somewhere after introducing quasifibs
        Taking the left square as an example, this follows from the fact that the fiber
        \begin{center}
            \begin{tikzcd} [sep = 4em]
                F \arrow[r] \arrow[d] & \M(f_{\overline{A}\overline{C}}) \arrow[d] \arrow[r] & \M(f_{\overline{A}\overline{C}})\cup_{\overline{A}}\M(f_{\overline{A}\overline{B}}) \arrow[d] \\
                * \arrow[r] & \M(f_{AC}) \arrow[r] & \M(f_{AC})\cup_A\M(f_{AB}) \\
            \end{tikzcd}
        \end{center}
        is already a homotopy fiber and so the map between homotopy fibers of the vertical maps is the identity and thus a weak equivalence.
        But this is an equivalent characterization of a homotopy pullback. %TODO reference
    \end{proof}
\end{prop}
Lastly, we need to prove descent for homotopy coproducts.
For readability purposes, we will abuse notation a little:
We will always work with seemingly fixed diagrams throughout the proofs without replacing it before computing limits/colimits.
This is of course not possible in general, since one cannot e.g. extend a given span to a homotopy pushout without replacing it first.
We will do this implicitly; if one is uncomfortable doing that, one can view the proofs as being in the \inftycat/ presented by the given model category, where the issue disappears.

Our proof of descent for homotopy coproducts will again demonstrate that descent is somewhat linked to universality.
We will show that descent for homotopy coproducts is a consequence of having disjoint binary homotopy coproducts and universality for homotopy coproducts.
To avoid confusion between homotopy and ordinary limits/colimits, we will tag homotopy limits/colimits with ``h'' (in particular, the ``h'' here does not refer to the \Strom/ model structure).
\begin{definition}
    Let $M$ be a model category.
    We say that \emph{binary homotopy coproducts are disjoint} in $M$ if every homotopy pushout square
    \begin{center}
        \begin{tikzcd} [sep = 1 cm]
            \emptyset \arrow [r] \arrow [d] & X \arrow [d]\\
            Y \arrow [r] & X\cup^h Y
        \end{tikzcd}
    \end{center}
    (where $\emptyset$ is the initial object and $X\cup^h Y$ a homotopy coproduct) is a homotopy pullback.
\end{definition}
\begin{lemma}\label{lem:binCoprodDisjoint}
    Binary homotopy coproducts in the model category $\Top$ are disjoint.
    \begin{proof}
        Since homotopy coproducts are given by ordinary coproducts in $\Top$, we have to check that ordinary pushouts
        \begin{center}
            \begin{tikzcd} [sep = 1 cm]
                \emptyset \arrow [r] \arrow [d] & X \arrow [d]\\
                Y \arrow [r] & X\cup Y
            \end{tikzcd}
        \end{center}
        are already homotopy pullbacks.
        As in the proof for universality of coproducts in $\Top$ \cref{lem:topUniversalCoproduct}, we note that the map $X\to X\cup Y$ is a fibration.
        Since the square is an ordinary pullback, this already implies that it is a homotopy pullback.
    \end{proof}
\end{lemma}
\begin{corollary}\label{cor:genCoproductComponentPb}
    Let $M$ be a model category with universal homotopy coproducts and disjoint binary homotopy coproducts.
    Let $\left(X_i\right)_{i\in I}$ be a small family of objects and let $\bigcup\limits_{i\in I}^h X_i$ be its homotopy coproduct.
    Then
    \begin{center}
        \begin{tikzcd} [sep = 1 cm]
            X_k \arrow [r] \arrow [d] & X_k \arrow [d]\\
            X_k \arrow [r] & \bigcup\limits_{i\in I}^h X_i
        \end{tikzcd}
    \end{center}
    is a homotopy pullback square and for $k\neq j$
    \begin{center}
        \begin{tikzcd} [sep = 1 cm]
            \emptyset \arrow [r] \arrow [d] & X_j \arrow [d]\\
            X_k \arrow [r] & \bigcup\limits_{i\in I}^h X_i
        \end{tikzcd}
    \end{center}
    is a homotopy pullback square.
    \begin{proof}
        For ease of notation, set $Y_k=\bigcup\limits_{i\in I\setminus{\set{k}}}^hX_i$.

        We first consider the case $k\neq j$.
        We define $P_k$ and $P_{kj}$ to be the homotopy pullbacks 
        \begin{center}
            \begin{tikzcd} [sep = 1 cm]
                P_{kj} \arrow [r] \arrow [d] & X_j \arrow [d]\\
                P_{k} \arrow [d] \arrow [r] & Y_k \arrow [d]\\
                X_k \arrow [r] & Y_k\cup^h X_k\simeq\bigcup\limits_{i\in I}^h X_i
            \end{tikzcd}
        \end{center}
        and by disjointness of binary homotopy coproducts we have $P_k\simeq\emptyset$.
        Since $\emptyset$ is the homotopy coproduct over the empty indexing set, by universality for homotopy coproducts the existence of a map $A\to\emptyset$ already implies that $A$ is also a homotopy coproduct over the empty indexing set.
        This proves that $P_{kj}\simeq\emptyset$.

        Next we prove the case $k=j$.
        We need to show that
        \begin{center}
            \begin{tikzcd} [sep = 1 cm]
                X_k \arrow [r] \arrow [d] & X_k \arrow [d]\\
                X_k \arrow [r] & X_k\cup^h Y_k
            \end{tikzcd}
        \end{center}
        is a homotopy pullback square.
        By universality of homotopy coproducts we have that $\left(X_k\times_{X_k\cup^h Y_k}^h X_k\right)\cup^h\left(X_k\times_{X_k\cup^h Y_k}^hY_k\right)\simeq X_k$.
        But by disjointness of homotopy coproducts we also have $X_k\times_{X_k\cup^h Y_k}^hY_k\simeq\emptyset$.
        Since $\left(X_k\times_{X_k\cup^h Y_k}^h X_k\right)\cup^h\emptyset\simeq\left(X_k\times_{X_k\cup^h Y_k}^hX_k\right)$, we get $X_k\times_{X_k\cup^h Y_k}^hX_k\simeq X_k$ which completes the proof.
    \end{proof}
\end{corollary}
\begin{corollary}\label{cor:disjointImpliesDescent}
    Let $M$ be a model category with universal homotopy coproducts and disjoint binary homotopy coproducts.
    Then $M$ has descent for homotopy coproducts.
    \begin{proof}
        Let $\left(X_i\right)_{i\in I}$ and $\left(Y_i\right)_{i\in I}$ be families of objects and let $\left(f_i\colon X_i\to Y_i\right)_{i\in I}$ be a collection of maps.
        We have to prove that 
        \begin{center}
            \begin{tikzcd} [sep = 1 cm]
                X_k \arrow [r] \arrow [d] & \bigcup\limits_{i\in I}^hX_i \arrow [d, "\bigcup\limits_{i\in I}f_i"]\\
                Y_k \arrow [r] & \bigcup\limits_{i\in I}^hY_i
            \end{tikzcd}
        \end{center}
        is a homotopy pullback for all $k\in I$.
        
        Let $P_{kj}$ be the homotopy pullback
        \begin{center}
            \begin{tikzcd} [sep = 1 cm]
                P_{kj} \arrow [r] \arrow [dd] & X_j \arrow [d]\\
                & \bigcup\limits_{i\in I}^hX_i \arrow [d]\\
                Y_k \arrow [r] & \bigcup\limits_{i\in I}^hY_i
            \end{tikzcd}
        \end{center}
        and since $X_j\to\bigcup\limits_{i\in I}X_i\to\bigcup\limits_{i\in I}Y_i$ is equal to $X_j\to Y_j\to\bigcup\limits_{i\in I}Y_i$, we can also compute $P_{kj}$ as the successive homotopy pullback
        \begin{center}
            \begin{tikzcd} [sep = 1 cm]
                P_{kj} \arrow [r] \arrow [d] & X_j \arrow [d]\\
                Y_k\times_{\bigcup\limits_{i\in I}^hY_i}^h Y_j \arrow [d] \arrow [r] & Y_j \arrow [d]\\
                Y_k \arrow [r] & \bigcup\limits_{i\in I}^hY_i
            \end{tikzcd}
        \end{center}
        by the pasting law for homotopy pullbacks.

        By \cref{cor:genCoproductComponentPb} we have that $Y_k\times_{\bigcup\limits_{i\in I}Y_i}^h Y_j\simeq\emptyset$ for $k\neq j$ and $Y_k\times_{\bigcup\limits_{i\in I}Y_i}^h Y_k\simeq Y_k$ for $k=j$.
        This in particular shows that $P_{kj}\simeq\emptyset$ for $j\neq k$ (again by universality of homotopy coproducts for the empty indexing set) and $P_{kk}\simeq X_k$.
        By universality of homotopy coproducts we have that $\bigcup\limits_{j\in I}^hP_{kj}\simeq\bigcup\limits_{j\in I}^h\left(Y_k\times_{\bigcup\limits_{i\in I}^hY_i}^hX_j\right)\simeq Y_k\times_{\bigcup\limits_{i\in I}^hY_i}^h\left(\bigcup\limits_{j\in I}^hX_j\right)$ is a weak equivalence.
        But since $P_{kj}\simeq\emptyset$ for $j\neq k$ and $P_{kk}\simeq X_k$ we obtain $\bigcup\limits_{j\in I}P_{kj}\simeq X_k$ and this proves the proposition.
    \end{proof}
\end{corollary}
\begin{corollary}\label{cor:topDescentCoproduct}
    The model category $\Top$ has descent for homotopy coproducts.
    \begin{proof}
        By \cref{cor:disjointImpliesDescent} this follows from \cref{lem:binCoprodDisjoint} and \cref{lem:topUniversalCoproduct}.
    \end{proof}
\end{corollary}
\begin{remark}
    One can prove descent for coproducts much more directly in $\Top$:
    
    Since homotopy coproducts can be computed by the ordinary coproducts in $\Top$, in the setting of \cref{cor:disjointImpliesDescent} the square
    \begin{center}
        \begin{tikzcd} [sep = 1 cm]
            X_k \arrow [r] \arrow [d] & \bigcup\limits_{i\in I}X_i \arrow [d, "\bigcup\limits_{i\in I}f_i"]\\
            Y_k \arrow [r] & \bigcup\limits_{i\in I}Y_i
        \end{tikzcd}
    \end{center}
    is a pullback square as can be checked by point set topology and since the map $Y_k\to\bigcup\limits_{i\in I}Y_i$ is a fibration, it is already a homotopy pullback which proves the proposition.
\end{remark}
\begin{corollary}
    The \inftycat/ of spaces $\spaces$ is an \inftytop/.
    \begin{proof}
        We know $\spaces$ is locally presentable from \cref{cor:spacesIsLocPres}.
        \Cref{cor:sufficientToProveInModCat} together with \cref{lem:topUniversalPo} and \cref{lem:topUniversalCoproduct} prove that $\spaces$ has universal colimits, and \cref{lem:topDescentPo} together with \cref{cor:topDescentCoproduct} then prove that it has descent.
    \end{proof}
\end{corollary}