In this section we will prove that our \inftycat/ of spaces $\spaces$ is an \inftytop/.
We will keep our proofs topological; for a different approach using simplicial sets see e.g. \cite[\S 6]{HTT}.
In order to transfer reasoning back and forth from the model category $\Top$ and the \inftycat/ $\spaces$ we need the following lemma.
\begin{lemma}\label{lem:replaceWithStrictDiagram} %TODO is the statement even true in this generality?
    Let $J$ be an ordinary category and let $M$ be a model category.
    Then a diagram $\N(J)\to\N(M)[W^{-1}]$ can be replaced with a diagram $J\to M$ that is equivalent in the sense that under the composition
    \begin{equation*}
        \Fun(\N(J),\N(M))\to\Fun(\N(J),\N(M)[W^{-1}]) %TODO is this correct?
    \end{equation*} 
    they are equivalent as objects of the \inftycat/ $\Fun(\N(J),\N(M)[W^{-1}])$.

    Furthermore the map $\Fun(\N(J),\N(M))\to\Fun(\N(J),\N(M)[W^{-1}])$ is natural in $J$.
    \begin{reference}
        \cite[Proposition 1.3.4.25]{higher_algebra}
    \end{reference}
\end{lemma}
Since both coproducts and pushouts in \inftycats/ are indexed by nerves of ordinary categories and we are able to work in the underlying model category $\Top$ by the following corollary:
\begin{corollary}\label{cor:sufficientToProveInModCat}
    Let $M$ be a model category with weak equivalences $W$. 
    Then $\N(M)[W^{-1}]$ has universal colimits if and only if $M$ has universal homotopy pushouts and universal homotopy coproducts as a model category.
    
    If further $\N(M)[W^{-1}]$ has universal colimits, then $M$ has descent if and only if it has descent for homotopy pushouts and homotopy coproducts as a model category.
    \begin{proof}
        A diagram $\N(J)^{\rhd}\to\N(M)[W^{-1}]$ is a colimit cone if and only if some equivalent replacement $J^{\rhd}\to M$ as in \cref{lem:replaceWithStrictDiagram} is a homotopy colimit cone in $M$.
        Given a diagram $J^{\rhd}\to M$, is is a homotopy colimit cone if and only if $\N(J)^{\rhd}\to\N(M)[W^{-1}]$ is a colimit cone.
        The analogous statement is also true for limits and homotopy limits.
        Thus the proposition follows from \cref{lem:univColimIffUnivPoAndCoprod} and \cref{lem:descentIffDescentPoAndCoprod} respectively.
    \end{proof}
\end{corollary}
We will use the interaction of the \Strom/ model structure and the Quillen model structure to prove universality and descent in $\Top$.
Note that whenever referring to $\Top$ as a model category, we implicitly mean the Quillen model structure. 
Cofibrations and fibrations of the \Strom/ model structure are tagged by the letter $h$.
Our most important (and quite non trivial) part of this interaction is the following proposition.
\begin{prop}\label{prop:poAlongHCofibIsHtpyPo}
    A pushout along an h-cofibration is a homotopy pushout.
    \begin{proof}
        Let
        \begin{center}
            \begin{tikzcd} [sep = 4em]
                A \arrow[>->, r, "h"] \arrow[d] & B \arrow[d]\\
                C \arrow[>->, r, "h"] & D \\
            \end{tikzcd}
        \end{center}
        be a pushout with the inducated maps being h-cofibrations.
        We factor the map $A\to C'\to C$ into a cofibration followed by a weak equivalence and obtain the diagram
        \begin{center}
            \begin{tikzcd} [sep = 4em]
                A \arrow[>->, r, "h"] \arrow[>->,d] & B \arrow[d]\\
                C' \arrow[>->, r, "h"] \arrow[d, "\sim"] & D' \arrow[d, "\sim"] \\
                C \arrow[>->, r, "h"] & D
            \end{tikzcd}
        \end{center}
        by taking pushouts.
        By \cite[Proposition 1.1]{hcolim_bar} we have that pushouts of weak equivalences along h-cofibrations are again weak equivalences.

        The top square is a homotopy pushout as it is a pushout along a cofibration. 
        Since the lower square is a also a homotopy pushout, the outer square is a homotopy pushout which proves the proposition.
    \end{proof}
\end{prop}
\begin{prop}\label{lem:topUniversalPo}
    The model category $\Top$ has universality for homotopy pushouts.
    \begin{proof}
        By factoring the map $D'\xrightarrow{\sim}\overline{D}\xtwoheadrightarrow{h} D$ into a weak equivalence followed by an h-fibration (which is also a fibration) and pulling back successively, we obtain the diagram %TODO explain why we we use weak equivalence here: since the faces are pb in the SERRE model str, the 3 remaining vertical maps are only w.e.
        \begin{center}
            \begin{tikzcd} [sep = .5 cm]
                A^{\prime} \arrow [dr] \arrow [rr] \arrow [dd] & & B^{\prime} \arrow [dr] \arrow [dd] \\
                & C^{\prime} \arrow [rr, crossing over] & & D^{\prime} \arrow [dd, "\sim"] & \\
                \overline{A} \arrow [dr] \arrow [rr] \arrow [->>, dd, "h"] & & \overline{B} \arrow [dr] \arrow [->>, dd, near start, "h"] \\
                & \overline{C} \arrow [rr, crossing over] \arrow [from=uu, crossing over] & & \overline{D} \arrow [->>, dd, "h"] & \\
                A \arrow [dr] \arrow [rr] & & B \arrow [dr] \\
                & C \arrow [->>,from=uu, crossing over, near start, "h"] \arrow [rr] & & D & \\
            \end{tikzcd}
        \end{center}
        where $\overline{A}=A\times_{D}\overline{D}$, $\overline{B}=B\times_{D}\overline{D}$ and $\overline{C}=C\times_{D}\overline{D}$.
        Since by the pasting lemma for homotopy pullbacks the upper vertical squares are also homotopy pullbacks and $D'\xrightarrow{\sim}\overline{D}$ is a weak equivalence, all vertical maps in the upper cube are weak equivalences.
        Thus it is sufficent to show that the square
        \begin{center}
            \begin{tikzcd} [sep = 4em]
                \overline{A} \arrow[r] \arrow[d] & \overline{B} \arrow[d] \\
                \overline{C} \arrow[r] & \overline{D} \\
            \end{tikzcd}
        \end{center}
        is a homotopy pushout.

        We factor the maps $A\xtailrightarrow{h}\widehat{B}\xrightarrow{\sim} B$ and $A\xtailrightarrow{h}\widehat{C}\xrightarrow{\sim} C$ into an h-cofibration followed by a weak equivalence.
        We obtain a diagram
        \begin{center}
            \begin{tikzcd} [sep = 4em]
                A \arrow[>->, r, "h"] \arrow[>->, d, "h"] & \widehat{B} \arrow[r, "\sim"] \arrow[>->, d, "h"] & B \arrow[dd] \\
                \widehat{C} \arrow[>->, r, "h"] \arrow[d, "\sim"] & \widehat{B}\cup_A\widehat{C} \arrow[dr] & \\
                C \arrow[rr] & & D
            \end{tikzcd}
        \end{center}
        where the small square is a homotopy pushout as its a pushout along an h-cofibration by \cref{prop:poAlongHCofibIsHtpyPo} so $\widehat{B}\cup_A\widehat{C}\to D$ is a weak equivalence.
        Pulling back the small square along the h-fibration $\overline{D}\xtwoheadrightarrow{h} D$ gives the square
        \begin{center}
            \begin{tikzcd} [sep = 4em]
                \overline{A} \arrow[>->, r, "h"] \arrow[>->, d, "h"] & \widehat{B}\times_{D}\overline{D} \arrow[>->, d, "h"] \\
                \widehat{C}\times_{D}\overline{D} \arrow[>->, r, "h" ] & \left(\widehat{B}\cup_A\widehat{C}\right)\times_{D}\overline{D}\\
            \end{tikzcd}
        \end{center}
        where the horizontal maps are h-cofibrations since pullbacks of h-cofibrations along h-fibrations are again h-cofibrations by \cite[Theorem 12]{note_on_cofibs_2}.
        
        Let $q\colon\left(\widehat{B}\cup_A\widehat{C}\right)\times_{D}\overline{D}\xtwoheadrightarrow{h}\widehat{B}\cup_A\widehat{C}$ denote the pullback of $\overline{D}\xtwoheadrightarrow{h} D$ along  $\widehat{B}\cup_A\widehat{C}\to D$.
        Since h-cofibrations are in particular closed embeddings we have $\widehat{B}\times_{D}\overline{D}\cong q^{-1}(\widehat{B})$, $\widehat{C}\times_{D}\overline{D}\cong q^{-1}(\widehat{C})$ and $\overline{A}\cong q^{-1}(A)\cong q^{-1}(\widehat{B})\cap q^{-1}(\widehat{C})$. %TODO why h-cofib closed embeddings?
        As $q^{-1}(\widehat{B})$ and $q^{-1}(\widehat{C})$ are closed, this implies by point set topology that the square is an ordinary pushout.

        Thus it is a pushout along an h-cofibration, hence by \cref{prop:poAlongHCofibIsHtpyPo} a homotopy pushout.
        Because by right properness of $\Top$ the square is equivalent to 
        \begin{center}
            \begin{tikzcd} [sep = 4em]
                \overline{A} \arrow[r] \arrow[d] & \overline{B} \arrow[d] \\
                \overline{C} \arrow[r] & \overline{D} \\
            \end{tikzcd}
        \end{center}
        this proves the proposition.
    \end{proof}
\end{prop}
\begin{lemma}\label{lem:topUniversalCoproduct}
    The model category $\Top$ has universal homotopy coproducts.
    \begin{proof}
        	Let $\left(X_i\right)_{i\in I}$ and $\left(Y_i\right)_{i\in I}$ be collections of topological spaces with $I$ small. 
            In a model category the homotopy coproduct of such a family is given by the ordinary coproduct of cofibrant replacements $\bigcup\limits_{i\in I}\widehat{X_i}$ for each of the $X_i$.
            However since in $\Top$ weak equivalences are closed under coproducts, the ordinary coproduct $\bigcup\limits_{i\in I}X_i$ is already a homotopy coproduct.

            Let $f\colon Y\to\bigcup\limits_{i\in I}X_i$ be a map and let $\left(Y_i\to X_i\right)_{i\in I}$ and $\left(Y_i\to Y\right)_{i\in I}$ be families of maps such that for all $k\in I$ we have a commutative square
            \begin{center}
                \begin{tikzcd} [sep = 1 cm]
                    Y_k \arrow [r] \arrow [d] & Y \arrow [d]\\
                    X_k \arrow [r] & \bigcup\limits_{i\in I} X_i
                \end{tikzcd}
            \end{center}
            that is a homotopy pullback.

            Note that every inclusion map $X_k\to\bigcup\limits_{i\in I} X_i$ is already a fibration as can be checked by its lifting properties.
            Therefore the ordinary pullback
            \begin{center}
                \begin{tikzcd} [sep = 1 cm]
                    Y\times_{\bigcup\limits_{i\in I} X_i}X_k \arrow [->>,r] \arrow [d] & Y \arrow [d]\\
                    X_k \arrow [->>,r] & \bigcup\limits_{i\in I} X_i
                \end{tikzcd}
            \end{center}
            is a homotopy pullback and so $Y_k\to Y\times_{\bigcup\limits_{i\in I} X_i}X_k$ is a weak equivalence.
            Since the canonical map $\bigcup\limits_{k\in I}\left(Y\times_{\bigcup\limits_{i\in I} X_i}X_k\right)\cong Y$ is a homeomorphism in $\Top$, $\bigcup\limits_{i\in I}Y_i\to Y$ is a weak equivalence.
            This proves that $Y$ is a homotopy coproduct.
    \end{proof}
\end{lemma}
Therefore by \cref{cor:sufficientToProveInModCat} we know $\spaces$ has universal colimits.
It remains to show that it also has descent.

We start proving the case for pushouts by the following reduction step. 
\begin{prop}\label{prop:reductionStepDescent}
    If for all cubes in the model category $\Top$
    \begin{center}
        \begin{tikzcd} [sep = .5 cm]
            \overline{A} \arrow [dr] \arrow [rr] \arrow [->>,dd, "h"] & & \overline{B} \arrow [dr] \arrow[->>]{dd}[near start]{h} \\
            & \overline{C} \arrow [rr, crossing over] & & \overline{D} \arrow [dd] & \\
            A \arrow [dr] \arrow [rr] & & B \arrow [dr] \\
            & C \arrow [->>,from=uu,crossing over]{}[near start]{h} \arrow [rr] & & D &
        \end{tikzcd}
    \end{center}
    where 
    \begin{itemize}
        \item left and back face are homotopy pullbacks
        \item bottom and top square homotopy pushouts %TODO is ordinary po necessary?
        \item the maps $\overline{A}\to A$, $\overline{B}\to B$ and $\overline{C}\to C$ are h-fibrations
    \end{itemize}
    the front and right squares are homotopy pullbacks, then $\Top$ has descent for homotopy pushouts.
    \begin{proof}
        Starting from a cube 
        \begin{center}
            \begin{tikzcd} [sep = .5 cm]
                A^{\prime} \arrow [dr] \arrow [rr] \arrow [dd] & & B^{\prime} \arrow [dr] \arrow [dd] \\
                & C^{\prime} \arrow [rr, crossing over] \arrow [dd] & & D^{\prime} \arrow [dd] & \\
                A \arrow [dr] \arrow [rr] & & B \arrow [dr] \\
                & C \arrow [from=uu, crossing over] \arrow [rr] & & D &
            \end{tikzcd}
        \end{center}
        where the back and left square are homotopy pullbacks and top and bottom are homotopy pushouts, we show that we can construct a cube of the required form such that all faces are equivalent to the corresponding faces of the starting cube.
        
        We first factor the maps $A\xtailrightarrow{h}\hat{C}\xtwoheadrightarrow[\sim]{h} C$ and $B'\xrightarrow{\sim}\overline{B}\xtwoheadrightarrow{h} B$ to obtain the diagram
        \begin{center}
            \begin{tikzcd} [sep = 4em]
                C^{\prime} \arrow[dd] & \hat{C}\times_CC' \arrow[->>]{l}[swap]{\sim} \arrow[dd] & A^{\prime} \arrow[l] \arrow[r] \arrow[d, "\sim"] & B^{\prime} \arrow[d, "\sim"] \\
                && \overline{A} \arrow [->>,d, "h"] \arrow[r] & \overline{B} \arrow[->>,d, "h"] \\
                C & \hat{C} \arrow[->>]{l}{h}[swap]{\sim} & A \arrow[>->,l, "h"] \arrow[r] & B 
            \end{tikzcd}
        \end{center}
        where 
        \begin{itemize}
            \item $\overline{A}=A\times_{B}\overline{B}$
            \item $A'\to \overline{A}$ is a weak equivalence since $B'\to\overline{B}$ is one by the pasting law for homotopy pullbacks as the outer and lower square are homotopy pullback.
        \end{itemize}
        Next we factor the map $A'\xtailrightarrow{h}X\xrightarrow{\sim}\hat{C}\times_CC'$. 
        Note that the diagram 
        \begin{center}
            \begin{tikzcd} [sep = 4em]
                X \arrow[r, "\sim"] \arrow[d] & C' \arrow[d] \\
                \hat{C} \arrow[->>]{r}{\sim}[swap]{h} & C \\
            \end{tikzcd}
        \end{center}
        is a homotopy pullback square.
        We form the diagram
        \begin{center}
            \begin{tikzcd} [sep = 4em]
                C^{\prime} \arrow[dd] & X \arrow{l}[swap]{\sim} \arrow[d, "\sim"] & A^{\prime} \arrow[>->,l, "h"] \arrow[r] \arrow[d, "\sim"] & B^{\prime} \arrow[d, "\sim"] \\
                & X\cup_{A'}\overline{A} \arrow[d] & \overline{A} \arrow[>->,l, "h"] \arrow [->>,d, "h"] \arrow[r] & \overline{B} \arrow[->>,d, "h"] \\
                C & \hat{C} \arrow[->>]{l}{h}[swap]{\sim} & A \arrow[>->,l, "h"] \arrow[r] & B 
            \end{tikzcd}
        \end{center}
        where $X\to X\cup_{A'}\overline{A}$ is a weak equivalence as its a pushout of a weak equivalence along an h-cofibration. %TODO reference
        By the pasting law we have that
        \begin{center}
            \begin{tikzcd} [sep = 4em]
                A' \arrow[>->,r, "h"] \arrow[d] & X \arrow[d] \\
                A \arrow[>->,r, "h"] & \hat{C} \\
            \end{tikzcd}
        \end{center}
        is homotopy pullback and since this square is equivalent %TODO maybe define what that means
        to the square
        \begin{center}
            \begin{tikzcd} [sep = 4em]
                \overline{A} \arrow[>->,r, "h"] \arrow[->>,d, "h"] & X\cup_{A'}\overline{A} \arrow[d] \\
                A \arrow[>->,r, "h"] & \hat{C} \\
            \end{tikzcd}
        \end{center}
        we know that both are homotopy pullbacks.
        Finally, we factor the map $X\cup_{A'}\overline{A}\xtailrightarrow[\sim]{h}\overline{C}\xtwoheadrightarrow{h}\hat{C}$.
        Then 
        \begin{center}
            \begin{tikzcd} [sep = 4em]
                \overline{A} \arrow[>->,r, "h"] \arrow[->>,d, "h"] & \overline{C} \arrow[->>,d, "h"] \\
                A \arrow[>->,r, "h"] & \hat{C} \\
            \end{tikzcd}
        \end{center}
        is again a homotopy pullback square. %TODO maybe say that this reduction is possible in any model cat

        Thus we have obtained a diagram
        \begin{center}
            \begin{tikzcd} [sep = 4em]
                \overline{C} \arrow[->>,d, "h"] & \overline{A} \arrow[>->,l, "h"] \arrow[r] \arrow[->>,d, "h"] & \overline{B} \arrow[->>,d, "h"] \\
                \hat{C} & A \arrow[>->,l, "h"] \arrow[r] & B \\
            \end{tikzcd}
        \end{center}
        and we can form the cube 
        \begin{center}
            \begin{tikzcd} [sep = .5 cm]
                \overline{A} \arrow[>->,dr, "h"] \arrow [rr] \arrow [->>,dd,"h"] & & \overline{B} \arrow [dr] \arrow[->>]{dd}[near start]{h} \\
                & \overline{C} \arrow [rr, crossing over] & & \overline{C}\cup_{\overline{A}}\overline{B} \arrow [dd] & \\
                A \arrow[>->,dr, "h"] \arrow [rr] & & B \arrow [dr] \\
                & \hat{C} \arrow[->>, from=uu, crossing over]{}[near start]{h} \arrow [rr] & & \hat{C}\cup_A B &
            \end{tikzcd}
        \end{center}
        by taking pushouts.
        Since these are pushouts along h-cofibrations, they are already homotopy pushouts. 
        Following the construction of this cube, we see that all faces are equivalent to their corresponding faces of the starting cube which proves the proposition.
    \end{proof}
\end{prop}
For the last part of the proof we need the following definitions.
\begin{definition}[Quasifibration]
    Let $M$ be a model category and $f\colon X\to Y$ be a map.
    Then we call $f$ a \emph{quasifibration} if for every map $*\to Y$ (where $*$ is a terminal object) the ordinary pullback square
    \begin{center}
        \begin{tikzcd} [sep = 4em]
            F \arrow[r] \arrow[d] & X \arrow[d, "f"] \\
            * \arrow[r] & Y\\
        \end{tikzcd}
    \end{center}
    is a homotopy pullback.
\end{definition}
\begin{definition}[Mapping Cylinder]
    Let $f\colon A\to B$ be a map of topological spaces and let $I=[0,1]$.
    Then we let
    \begin{equation*}
        \M(f)\coloneqq\faktor{\left(A\times I\right)\cup B}{(a,0)\sim f(a)}
    \end{equation*}
    denote the \emph{mapping cylinder of $f$}.

    It will be convenient to also allow other intervalls instead of $I=[0,1]$; we will allow $I=[0,x]$ and $I=[0,x)$ for $x>0$.
    The analogous construction with $I=[0,x]$ will be referred to as a \emph{closed mapping cylinder of $f$};
    the construction with half open intervalls $I=[0,x)$ will be referred to as an \emph{open mapping cylinder of $f$}. %TODO perhaps describe factorization
\end{definition}
%TODO reference for inclusion into mapping cylinder is closed Hurewicz cofib
\begin{lemma}\label{lem:mapOfCylIsQuasiFib} %TODO prove assertions on pullbacks
    Let 
    \begin{center}
        \begin{tikzcd} [sep = 4em]
            X \arrow[d,->>, "h"] \arrow[r, "f"] & Y \arrow[d,->>, "h"] \\
            A \arrow[r, "g"] & B \\
        \end{tikzcd}
    \end{center}
    be a homotopy pullback (but not necessarily a pullback).
    Then the induced map $\M(f)\to\M(g)$ between mapping cylinders is a quasifibration.
    The same holds true for closed and open mapping cylinders.
    \begin{proof}
        For convenience we will only prove the case $I=[0,1]$ since the other cases follow by the analogous argument.
        The proof follows \cite[Lemma 5.10.6]{cubical_htpy_theory}.

        Let $\hat{b}\in\M(g)$. 
        Then either $\hat{b}=(\hat{a},t)\in A\times(0,1]\subset\M(g)$ or $\hat{b}\in B\subset\M(g)$.
        In the first case, we have a diagram
        \begin{center}
            \begin{tikzcd} [sep = 4em]
                X \arrow[r, "{x\mapsto (x,t)}"] \arrow[->>,d, "h"] & \M(f) \arrow[r, "\sim"] \arrow[d] & Y \arrow[->>,d, "h"] \\
                A \arrow[r, "{a\mapsto (a,t)}"] & \M(g) \arrow[r, "\sim"] & B \\
            \end{tikzcd}
        \end{center}
        where the outer square is the starting square and in particular homotopy pullback, so by the pasting law the left square is homotopy pullback as well.
        Since the left square is also an ordinary pullback, in the diagram
        \begin{center}
            \begin{tikzcd} [sep = 4em]
                F \arrow[r] \arrow[d] & X \arrow[r, "{x\mapsto (x,t)}"] \arrow[->>,d, "h"] & \M(f) \arrow[d] \\
                * \arrow[r, "\hat{b}"] & A \arrow[r, "{a\mapsto (a,t)}"] & \M(g) \\
            \end{tikzcd}
        \end{center}
        the left square is homotopy pullback since its the pullback along a fibration.
        Thus the outer square is a homotopy pullback which proves that the fiber $F$ is a homotopy fiber.

        In the second case we have the diagram
        \begin{center}
            \begin{tikzcd} [sep = 4em]
                F \arrow[r] \arrow[d] & Y \arrow[r, hook, "\sim"] \arrow[->>,d, "h"] & \M(f) \arrow[d] \\
                * \arrow[r, "\hat{b}"] & B \arrow[r, hook, "\sim"] & \M(g) \\
            \end{tikzcd}
        \end{center}
        where the indicated maps are weak equivalences since they are right inverse to the maps $\M(f)\to Y$ and $\M(g)\to B$ respectively, which are weak equivalences themselves.
        Since the right square is pullback and homotopy pullback and the left side is as well, this means that the fiber $F$ is already a homotopy fiber which proves the proposition.
    \end{proof}
\end{lemma}
\begin{prop}\label{lem:topDescentPo}
    The model category $\Top$ has descent for homotopy pushouts. 
    \begin{proof}
        By \cref{prop:reductionStepDescent} it is sufficient to prove that for all cubes 
        \begin{center}
            \begin{tikzcd} [sep = .5 cm]
                \overline{A} \arrow [dr] \arrow [rr] \arrow [->>,dd, "h"] & & \overline{B} \arrow [dr] \arrow[->>]{dd}[near start]{h} \\
                & \overline{C} \arrow [rr, crossing over] & & \overline{D} \arrow [dd] & \\
                A \arrow [dr] \arrow [rr] & & B \arrow [dr] \\
                & C \arrow [->>,from=uu,crossing over]{}[near start]{h} \arrow [rr] & & D &
            \end{tikzcd}
        \end{center}
        where 
        \begin{itemize}
            \item left and back face are homotopy pullbacks
            \item bottom and top square are homotopy pushouts
            \item the maps $\overline{A}\to A$, $\overline{B}\to B$ and $\overline{C}\to C$ are h-fibrations
        \end{itemize}
        the front and right squares are homotopy pullbacks.

        By \cref{lem:mapOfCylIsQuasiFib} we know that the induced maps $\M(f_{\overline{A}\overline{C}})\to\M(f_{AC})$ and $\M(f_{\overline{A}\overline{B}})\to\M(f_{AB})$ are quasifibrations. %TODO why are inclusions cofibs
        So we can form the cube 
        \begin{center}
            \begin{tikzcd} [sep = .75 cm]
                \overline{A} \arrow [>->, dr, "h"] \arrow [>->, rr, "h"] \arrow [->>,dd, "h"] & & \M(f_{\overline{A}\overline{B}}) \arrow [>->, dr, "h"] \arrow[dd] \\
                & \M(f_{\overline{A}\overline{C}}) \arrow [>->, rr, crossing over, near start,  "h"] & & \M(f_{\overline{A}\overline{C}})\cup_{\overline{A}}\M(f_{\overline{A}\overline{B}}) \arrow [dd] & \\
                A \arrow [>->, dr, "h"] \arrow [>->, rr, near start, "h"] & & \M(f_{AB}) \arrow [>->, dr, "h"] \\
                & \M(f_{AC}) \arrow [from=uu,crossing over] \arrow [>->, rr, "h"] & & \M(f_{AC})\cup_A\M(f_{AB}) &
            \end{tikzcd}
        \end{center}
        where all faces are again equivalent to the corresponding faces of the original cube and the indicated maps are h-cofibrations as they are either pushouts of h-cofibrations or inclusions into the mapping cylinder. %TODO why does this mean they are cofibs?
        
        Next we prove that the map $p\colon\M(f_{\overline{A}\overline{C}})\cup_{\overline{A}}\M(f_{\overline{A}\overline{B}})\to\M(f_{AC})\cup_A\M(f_{AB})$ is a quasifibration.
        By \cite[Lemma 4K.3]{hatcher2002algebraic} we only need to find open sets $U_1,U_2\subset\M(f_{AC})\cup_A\M(f_{AB})$ covering $\M(f_{AC})\cup_A\M(f_{AB})$ such that the induced maps $p^{-1}(U_1)\to U_1$, $p^{-1}(U_2)\to U_2$ and $p^{-1}(U_1\cap U_2)\to U_1\cap U_2$ are quasifibrations.

        We take $U_1=\M(f_{AC})\cup_A\M(f_{AB})\setminus{B}$ and $U_2=\M(f_{AC})\cup_A\M(f_{AB})\setminus{C}$. 

        We identify $U_1\cap U_2=A\times [0,1)\cup_A A\times[0,1)\cong A\times (0,2)$ via the ``obvious'' homeomorphism. %TODO explain
        Then $p^{-1}(U_1)=\M(f_{\overline{A}\overline{B}})\cup_{\overline{A}}\M(f_{\overline{A}\overline{B}})\setminus{\overline{B}}$, $p^{-1}(U_2)=\M(f_{\overline{A}\overline{B}})\cup_{\overline{A}}\M(f_{\overline{A}\overline{B}})\setminus{\overline{C}}$ and $p^{-1}(U_1\cap U_2)=\overline{A}\times (0,2)$ (identified with $\overline{A}\times [0,1)\cup_{\overline{A}} \overline{A}\times[0,1)$ as before).
        It follows $p^{-1}(U_1)\to U_1$ and $p^{-1}(U_2)\to U_2$ are quasifibrations by \cref{lem:mapOfCylIsQuasiFib} since they are both maps of open mapping cylinders induced by the squares 
        \begin{center}
            \begin{tikzcd} [sep = 4em]
                \overline{A} \arrow[d,->>, "h"] \arrow[r, "f_{\overline{A}\overline{C}}"] & \overline{C} \arrow[d,->>, "h"] \\
                A \arrow[r, "f_{AC}"] & C \\
            \end{tikzcd} 
            \begin{tikzcd} [sep = 4em]
                \overline{A} \arrow[d,->>, "h"] \arrow[r, "f_{\overline{A}\overline{B}}"] & \overline{B} \arrow[d,->>, "h"] \\
                A \arrow[r, "f_{AB}"] & B \\
            \end{tikzcd}
        \end{center}
        respectively.
        Since the map $p^{-1}(U_1\cap U_2)\to U_1\cap U_2$ is a product of fibrations $f_{\overline{A}A}\times\id_{(0,2)}$, it is again a fibration and thus a quasifibration.

        Finally, since the squares 
        \begin{center}
            \begin{tikzcd} [sep = 4em]
                \M(f_{\overline{A}\overline{C}}) \arrow[d] \arrow[r] & \M(f_{\overline{A}\overline{C}})\cup_{\overline{A}}\M(f_{\overline{A}\overline{B}}) \arrow[d] \\
                \M(f_{AC}) \arrow[r] & \M(f_{AC})\cup_A\M(f_{AB}) \\
            \end{tikzcd}
            \begin{tikzcd} [sep = 4em]
                \M(f_{\overline{A}\overline{B}}) \arrow[d] \arrow[r] & \M(f_{\overline{A}\overline{C}})\cup_{\overline{A}}\M(f_{\overline{A}\overline{B}}) \arrow[d] \\
                \M(f_{AB}) \arrow[r] & \M(f_{AC})\cup_A\M(f_{AB}) \\
            \end{tikzcd}
        \end{center}
        are both ordinary pullbacks %TODO maybe explain why
        with both vertical maps quasifibrations, they are already homotopy pullbacks. %TODO maybe put this as an extra statement somewhere after introducing quasifibs
        Taking the left square as an example, this follows from the fact that the fiber
        \begin{center}
            \begin{tikzcd} [sep = 4em]
                F \arrow[r] \arrow[d] & \M(f_{\overline{A}\overline{C}}) \arrow[d] \arrow[r] & \M(f_{\overline{A}\overline{C}})\cup_{\overline{A}}\M(f_{\overline{A}\overline{B}}) \arrow[d] \\
                * \arrow[r] & \M(f_{AC}) \arrow[r] & \M(f_{AC})\cup_A\M(f_{AB}) \\
            \end{tikzcd}
        \end{center}
        is already a homotopy fiber and so the map between homotopy fibers of the vertical maps is the identity and thus a weak equivalence.
        But this is an equivalent characterization of a homotopy pullback. %TODO reference
    \end{proof}
\end{prop}
Lastly, we need to prove descent for homotopy coproducts.
For readability purposes, we will abuse notation a little:
We will always work with seemingly fixed diagrams throughout the proofs without replacing it before computing limits/colimits.
This is of course not possible in general, since one cannot e.g. extend a given span to a homotopy pushout without replacing it first.
We will do this implicitly; if one is uncomfortable doing that, one can view the proofs as being in the \inftycat/ presented by the given model category, where the issue disappears.

Our proof of descent for homotopy coproducts will again demonstrate that descent is somewhat linked to universality.
We will show that descent for homotopy coproducts is a consequence of having disjoint binary homotopy coproducts and universality for homotopy coproducts.
To avoid confusion between homotopy and ordinary limits/colimits, we will tag homotopy limits/colimits with ``h''.
\begin{definition}
    Let $M$ be a model category.
    We say that \emph{binary homotopy coproducts are disjoint} in $M$ if every homotopy pushout square
    \begin{center}
        \begin{tikzcd} [sep = 1 cm]
            \emptyset \arrow [r] \arrow [d] & X \arrow [d]\\
            Y \arrow [r] & X\cup^h Y
        \end{tikzcd}
    \end{center}
    (where $\emptyset$ is the initial object and $X\cup^h Y$ a homotopy coproduct) is a homotopy pullback.
\end{definition}
\begin{lemma}\label{lem:binCoprodDisjoint}
    Binary homotopy coproducts in the model category $\Top$ are disjoint.
    \begin{proof}
        Since homotopy coproducts are given by ordinary coproducts in $\Top$, we have to check that ordinary pushouts
        \begin{center}
            \begin{tikzcd} [sep = 1 cm]
                \emptyset \arrow [r] \arrow [d] & X \arrow [d]\\
                Y \arrow [r] & X\cup Y
            \end{tikzcd}
        \end{center}
        are already homotopy pullbacks.
        As in the proof for universality of coproducts in $\Top$ \cref{lem:topUniversalCoproduct}, we note that the map $X\to X\cup Y$ is a fibration.
        Since the square is an ordinary pullback, this already implies that it is a homotopy pullback.
    \end{proof}
\end{lemma}
\begin{corollary}\label{cor:genCoproductComponentPb}
    Let $M$ be a model category with universal homotopy coproducts and disjoint binary homotopy coproducts.
    Let $\left(X_i\right)_{i\in I}$ be a small family of objects and let $\bigcup\limits_{i\in I}^h X_i$ be its homotopy coproduct.
    Then
    \begin{center}
        \begin{tikzcd} [sep = 1 cm]
            X_k \arrow [r] \arrow [d] & X_k \arrow [d]\\
            X_k \arrow [r] & \bigcup\limits_{i\in I}^h X_i
        \end{tikzcd}
    \end{center}
    is a homotopy pullback square and for $k\neq j$
    \begin{center}
        \begin{tikzcd} [sep = 1 cm]
            \emptyset \arrow [r] \arrow [d] & X_j \arrow [d]\\
            X_k \arrow [r] & \bigcup\limits_{i\in I}^h X_i
        \end{tikzcd}
    \end{center}
    is a homotopy pullback square.
    \begin{proof}
        For ease of notation, set $Y_k=\bigcup\limits_{i\in I\setminus{\set{k}}}^hX_i$.

        We first consider the case $k\neq j$.
        We get homotopy pullbacks 
        \begin{center}
            \begin{tikzcd} [sep = 1 cm]
                P_{kj} \arrow [r] \arrow [d] & X_j \arrow [d]\\
                P_{k} \arrow [d] \arrow [r] & Y_k \arrow [d]\\
                X_k \arrow [r] & Y_k\cup^h X_k\simeq\bigcup\limits_{i\in I}^h X_i
            \end{tikzcd}
        \end{center}
        and by disjointness of binary homotopy coproducts we have $P_k\simeq\emptyset$.
        Since $\emptyset$ is the homotopy coproduct over the empty indexing set, by universality for homotopy coproducts the existence of a map $A\to\emptyset$ already implies that $A$ is also a homotopy coproduct over the empty indexing set.
        This proves that $P_{kj}\simeq\emptyset$.

        Next we prove the case $k=j$.
        We need to prove that
        \begin{center}
            \begin{tikzcd} [sep = 1 cm]
                X_k \arrow [r] \arrow [d] & X_k \arrow [d]\\
                X_k \arrow [r] & X_k\cup^h Y_k
            \end{tikzcd}
        \end{center}
        is a homotopy pullback square.
        By universality of homotopy coproducts we have that $\left(X_k\times_{X_k\cup^h Y_k}^h X_k\right)\cup^h\left(X_k\times_{X_k\cup^h Y_k}^hY_k\right)\simeq X_k$.
        But by disjointness of homotopy coproducts we also have $X_k\times_{X_k\cup^h Y_k}^hY_k\simeq\emptyset$.
        Since $\left(X_k\times_{X_k\cup^h Y_k}^h X_k\right)\cup^h\emptyset\simeq\left(X_k\times_{X_k\cup^h Y_k}^hX_k\right)$, we get $X_k\times_{X_k\cup^h Y_k}^hX_k\simeq X_k$ which completes the proof.
    \end{proof}
\end{corollary}
\begin{corollary}\label{cor:disjointImpliesDescent}
    Let $M$ be a model category with universal homotopy coproducts and disjoint binary homotopy coproducts.
    Then $M$ has descent for homotopy coproducts.
    \begin{proof}
        Let $\left(X_i\right)_{i\in I}$ and $\left(Y_i\right)_{i\in I}$ be families of objects and let $\left(f_i\colon X_i\to Y_i\right)_{i\in I}$ be a collection of maps.
        We have to prove that 
        \begin{center}
            \begin{tikzcd} [sep = 1 cm]
                X_k \arrow [r] \arrow [d] & \bigcup\limits_{i\in I}^hX_i \arrow [d, "\bigcup\limits_{i\in I}f_i"]\\
                Y_k \arrow [r] & \bigcup\limits_{i\in I}^hY_i
            \end{tikzcd}
        \end{center}
        is a homotopy pullback for all $k\in I$.
        
        Let $P_{kj}$ be the homotopy pullback
        \begin{center}
            \begin{tikzcd} [sep = 1 cm]
                P_{kj} \arrow [r] \arrow [dd] & X_j \arrow [d]\\
                & \bigcup\limits_{i\in I}^hX_i \arrow [d]\\
                Y_k \arrow [r] & \bigcup\limits_{i\in I}^hY_i
            \end{tikzcd}
        \end{center}
        and since $X_j\to\bigcup\limits_{i\in I}X_i\to\bigcup\limits_{i\in I}Y_i$ is equal to $X_j\to Y_j\to\bigcup\limits_{i\in I}Y_i$, we can also compute $P_{kj}$ as the successive homotopy pullback
        \begin{center}
            \begin{tikzcd} [sep = 1 cm]
                P_{kj} \arrow [r] \arrow [d] & X_j \arrow [d]\\
                Y_k\times_{\bigcup\limits_{i\in I}^hY_i}^h Y_j \arrow [d] \arrow [r] & Y_j \arrow [d]\\
                Y_k \arrow [r] & \bigcup\limits_{i\in I}^hY_i
            \end{tikzcd}
        \end{center}
        by the pasting law for homotopy pullbacks.

        By \cref{cor:genCoproductComponentPb} we have that $Y_k\times_{\bigcup\limits_{i\in I}Y_i}^h Y_j\simeq\emptyset$ for $k\neq j$ and $Y_k\times_{\bigcup\limits_{i\in I}Y_i}^h Y_k\simeq Y_k$ for $k=j$.
        This in particular shows that $P_{kj}\simeq\emptyset$ for $j\neq k$ (again by universality of homotopy coproducts for the empty indexing set) and $P_{kk}\simeq X_k$.
        By universality of homotopy coproducts we have that $\bigcup\limits_{j\in I}^hP_{kj}\simeq\bigcup\limits_{j\in I}^h\left(Y_k\times_{\bigcup\limits_{i\in I}^hY_i}^hX_j\right)\simeq Y_k\times_{\bigcup\limits_{i\in I}^hY_i}^h\left(\bigcup\limits_{j\in I}^hX_j\right)$ is a weak equivalence.
        But since $P_{kj}\simeq\emptyset$ for $j\neq k$ and $P_{kk}\simeq X_k$ we obtain $\bigcup\limits_{j\in I}P_{kj}\simeq X_k$ and this proves the proposition.
    \end{proof}
\end{corollary}
\begin{corollary}\label{cor:topDescentCoproduct}
    The model category $\Top$ has descent for homotopy coproducts.
    \begin{proof}
        By \cref{cor:disjointImpliesDescent} this follows from \cref{lem:binCoprodDisjoint} and \cref{lem:topUniversalCoproduct}.
    \end{proof}
\end{corollary}
\begin{remark}
    One can prove descent for coproducts much more directly in $\Top$:
    
    In the setting of \cref{cor:disjointImpliesDescent} the square
    \begin{center}
        \begin{tikzcd} [sep = 1 cm]
            X_k \arrow [r] \arrow [d] & \bigcup\limits_{i\in I}X_i \arrow [d, "\bigcup\limits_{i\in I}f_i"]\\
            Y_k \arrow [r] & \bigcup\limits_{i\in I}Y_i
        \end{tikzcd}
    \end{center}
    is a pullback square as can be checked by point set topology.
    Since the map $Y_k\to\bigcup\limits_{i\in I}Y_i$ is a fibration, it is already a homotopy pullback which proves the proposition.
\end{remark}
\begin{corollary}
    The \inftycat/ of spaces $\spaces$ is an \inftytop/.
    \begin{proof}
        We know $\spaces$ is locally presentable from \cref{cor:spacesIsLocPres}.
        \Cref{cor:sufficientToProveInModCat} together with \cref{lem:topUniversalPo} and \cref{lem:topUniversalCoproduct} proves that $\spaces$ has universal colimits, and together with \cref{lem:topDescentPo} and \cref{cor:topDescentCoproduct} proves that it has descent.
    \end{proof}
\end{corollary}