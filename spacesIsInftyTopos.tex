In this section we will prove that the \inftycat/ of spaces $\spaces$ defined in \cref{def:inftyCatOfSpaces} is an \inftytop/.
Since we have already shown that $\spaces$ is locally presentable in \cref{cor:spacesIsLocPres}, it remains to show universality and descent.
We will keep our proofs topological; for a different approach using simplicial sets see e.g. \cite[\S 6]{HTT}.

In order to transfer reasoning back and forth between the model category $\Top$ and the \inftycat/ $\spaces$ we first prove some technical preliminaries showing that homotopy limits/colimits in model categories correspond to limits/colimits its simplicial localization at the weak equivalences.
The cases of pushouts, pullbacks and coproducts are especially important by \cref{lem:univColimIffUnivPoAndCoprod} and \cref{lem:descentIffDescentPoAndCoprod}.
\begin{thm}\label{thm:locCommutesWithHom} 
    Let $J$ be an ordinary small category and let $M$ be a model category with weak equivalences $W$.
    Let $\overline{W}\subset\Fun(\N(J),\N(M))$ denote the $1$-dimensional simplicial subset spanned by the natural transformations that are objectwise weak equivalences.
    
    Then there is a natural equivalence of \inftycats/
    \begin{equation*}
        \Fun(\N(J),\N(M))\left[\overline{W}^{-1}\right]\to\Fun(\N(J),\N(M)[W^{-1}])\;.
    \end{equation*} 
    \begin{reference}
        \cite[Remark 7.9.7 and Theorem 7.9.8]{cisinski_2019}
    \end{reference}
\end{thm}
\begin{corollary}\label{cor:replaceWithStrictDiagram}
    Let $J$ be a small category and let $M$ be a model category with weak equivalences $W$.
    Then a diagram $\N(J)\to\N(M)[W^{-1}]$ can be replaced with a diagram $J\to M$ that is equivalent in the sense that under the composition
    \begin{equation*}
        \Fun(\N(J),\N(M))\to\Fun(\N(J),\N(M)[W^{-1}])
    \end{equation*} 
    they are equivalent as objects of the \inftycat/ $\Fun(\N(J),\N(M)[W^{-1}])$.
    \begin{proof}
        By \cref{thm:locCommutesWithHom}, \cref{thm:eqCharEqOfInftycats} and \cref{prop:simpLocEssSurj}, the map $\Fun(\N(J),\N(M))\to\Fun(\N(J),\N(M)[W^{-1}])$ is essentially surjective.
        Since the nerve functor is fully faithful by \cref{lem:nerveFF}, this proves the corollary.
    \end{proof}
\end{corollary}
\begin{definition}[Direct/Inverse Category]
    Let $C$ be a small category.
    Then we call $C$ a \emph{direct category} if there exists an ordinal $\lambda$ and a functor $F\colon C\to\lambda$ (where we view $\lambda$ as a category by \cref{rmk:linOrderedAsCat}) such that $F$ sends non-identity maps in $C$ to non-identity maps in $\lambda$.
    
    Dually, we call $C$ an \emph{inverse category} if $C^{\op}$ is a direct category.
\end{definition}
\begin{definition}[Injective/Projective Model Structure]
    Let $M$ be a model category and $C$ be a small category.
    Then if there exists a model structure on $\Fun(C,M)$ whose
    \begin{itemize}
        \item weak equivalences are the objectwise weak equivalences 
        \item fibrations are the objectwise fibrations
    \end{itemize}
    we call this model structure the \emph{projective model structure on $\Fun(C,M)$} and denote the model category as $\Fun(C,M)_{proj}$.

    Dually, if there exists a model structure on $\Fun(C,M)$ whose
    \begin{itemize}
        \item weak equivalences are the objectwise weak equivalences 
        \item cofibrations are the objectwise cofibrations
    \end{itemize}
    we call this model structure the \emph{injective model structure on $\Fun(C,M)$} and denote the model category as $\Fun(C,M)_{inj}$.

    Note that the model structures are already uniquely determined by this data, provided they exist.
\end{definition}
\begin{thm}\label{thm:exProjInjModelStr}
    Let $C$ be a model category and $D$ be a small category.

    If $D$ is a direct category, then the projective model structure on $\Fun(D,C)$ exists. 
    Furthermore, the functor $\colim\colon\Fun(D,C)_{proj}\to C$ is left Quillen with respect to the constant functor.

    If $D$ is a inverse model category, then the injective model structure on $\Fun(D,C)$ exists.
    Furthermore, the functor $\lim\colon\Fun(D,C)_{inj}\to C$ is right Quillen with respect to the constant functor.
    \begin{reference}
        \cite[Theorem 5.1.3 and Corollary 5.1.6]{hovey2007model}
    \end{reference}
\end{thm}
\begin{lemma}\label{lem:htpyLimAndColimAgree}
    Let $M$ be a model category and let $J$ be a small category.
    
    If $J$ is a direct category, then a diagram $X\colon J^{\rhd}\to M$ is a homotopy colimit diagram if and only if the induced map $\N(J)^{\rhd}\cong\N(J^{\rhd})\to\N(M)[W^{-1}]$ is a colimit diagram.

    Dually, if $J$ is an inverse category, then a diagram $X\colon J^{\lhd}\to M$ is a homotopy limit diagram if and only if the induced map $\N(J)^{\lhd}\cong\N(J^{\lhd})\to\N(M)[W^{-1}]$ is a limit diagram.
    \begin{proof}
        We will only prove the case of (homotopy) colimits over direct categories $J$, since the case for limits follows by duality.

        We have an adjunction of \inftycats/ given by 
        \begin{equation*}
            \colim\colon\Fun(\N(J),\N(M)[W^{-1}])\rightleftarrows\N(M)[W^{-1}]\mkern+3mu:\mkern-3mu c
        \end{equation*}
        where $c$ is the constant functor.

        By \cite[Remark 6.1.5]{cisinski_2019}, this induces an adjunction of ordinary categories
        \begin{equation*}
            \ho(\Fun(\N(J),\N(M)[W^{-1}]))\rightleftarrows\ho(\N(M)[W^{-1}])\;.
        \end{equation*}
        Let $\Ho(C)$ denote the homotopy category for a model category $C$. 
        Since $\Ho(C)$ is a model of the (ordinary) localization of $C$ by the weak equivalences, there is a canonical equivalence $\Ho(C)\cong\ho(\N(C)[W^{-1}])$ since both fulfill the same universal property.
        
        Thus $\ho(\N(M)[W^{-1}])\cong\Ho(M)$ and using \cref{thm:locCommutesWithHom} we also have $\ho(\Fun(\N(J),\N(M)[W^{-1}]))\cong\Ho(\Fun(J,M)_{proj})$.
        This yields an adjunction
        \begin{equation*}
            \Ho(\Fun(J,M)_{proj})\rightleftarrows\Ho(M)\mkern+3mu:\mkern-3mu c
        \end{equation*}
        where the functor from right to left is the constant functor and the projective model structure exists by \cref{thm:exProjInjModelStr}.

        This means that the functor from left to right is (equivalent to) a left derived functor of the $\colim$-functor.

        Hence given a diagram $X\colon J^{\rhd}\to M$, the colimit of the induced map $\N(J)\to\N(M)[W^{-1}]$ in $\ho(\N(M)[W^{-1}])\cong\Ho(M)$ is isomorphic to the homotopy colimit of $X|_J$ computed via the derived colimit functor in $\Ho(M)\cong\ho(\N(M)[W^{-1}])$.
        Since objects are equivalent in $\N(M)[W^{-1}]$ if and only if they are isomorphic in $\ho(\N(M)[W^{-1}])$ (see \cref{rmk:boardmanVogt}), this proves the lemma. 
    \end{proof}
\end{lemma}
Since coproducts and pushouts/pullbacks in \inftycats/ are indexed by nerves of direct/inverse categories, we are able to work in the underlying model category by the following corollary.
\begin{corollary}\label{cor:sufficientToProveInModCat}
    Let $M$ be a model category with weak equivalences $W$. 
    Then $\N(M)[W^{-1}]$ has universal colimits if and only if $M$ has universal homotopy pushouts and universal homotopy coproducts as a model category.
    
    If further $\N(M)[W^{-1}]$ has universal colimits, then $M$ has descent if and only if it has descent for homotopy pushouts and homotopy coproducts as a model category.
    \begin{proof}
        First note that the indexing categories of pushouts $\bullet\xleftarrow{}\bullet\xrightarrow{}\bullet$ and small coproducts are a direct categories (the latter is inverse as well), and the indexing category for pullbacks $\bullet\xrightarrow{}\bullet\xleftarrow{}\bullet$ is an inverse category.
        By \cref{lem:univColimIffUnivPoAndCoprod} and \cref{lem:descentIffDescentPoAndCoprod} we only need to check universality and descent for pushouts and small coproducts.
        So let $I$ be either the indexing category for pushouts $\bullet\xleftarrow{}\bullet\xrightarrow{}\bullet$ or a small discrete category.

        Then given a natural transformation $\alpha\colon\N(I)^{\rhd}\times\Delta^1\to\N(M)[W^{-1}]$, we can replace $\alpha$ by an equivalent natural transformation represented by a functor $\overline{\alpha}\colon I^{\rhd}\times[1]\to M$ via \cref{cor:replaceWithStrictDiagram}.
        Then by \cref{lem:htpyLimAndColimAgree}
        \begin{itemize}
            \item for a map $f\colon i\to j\in I^{\rhd}$ (which we can identify with $f\colon\Delta^1\to\N(I)^{\rhd}$), the induced square $\Delta^1\times\Delta^1\xrightarrow{\alpha(f,\id_{\Delta^1})}\N(M)[W^{-1}]$ is a pullback square if and only if the square $[1]\times[1]\xrightarrow{\overline{\alpha}(f,\id_{[1]})}M$ is a homotopy pullback
            \item $\alpha_0$ is a colimit cone if and only if $\overline{\alpha}_0$ is a homotopy colimit cone
            \item $\alpha_1$ is a colimit cone if and only if $\overline{\alpha}_1$ is a homotopy colimit cone
        \end{itemize}
        which proves this corollary.
    \end{proof}
\end{corollary}
We will use the interaction of the \Strom/ model structure and the Quillen model structure to prove universality and descent in $\Top$.
Note that whenever referring to $\Top$ as a model category, we implicitly mean the Quillen model structure. 
Cofibrations and fibrations of the \Strom/ model structure in diagrams are tagged by the letter ``h''.
The most important (and quite non-trivial) part of the interaction between Quillen and \Strom/ model structure on $\Top$ is the following proposition.
\begin{prop}\label{prop:poOfWeIsWe}
    Let 
    \begin{center}
        \begin{tikzcd} [sep = 4em]
            C' \arrow[d,"\sim"] & A' \arrow[>->, r, "h"] \arrow[l] \arrow[d,"\sim"] & B' \arrow[d,"\sim"]\\
            C & A \arrow[>->, r, "h"] \arrow[l] & B \\
        \end{tikzcd}
    \end{center}
    be a commutative square in $\Top$ such that the indicated maps are weak equivalences and h-cofibrations respectively.

    Then the induced map of pushouts $C'\cup_{A'}B'\xrightarrow{\sim} C\cup_AB$ is a weak equivalence.
    \begin{reference}
        \cite[Proposition 1.1]{hcolim_bar}
    \end{reference}
\end{prop}
This preceding proposition has the following useful consequence.
\begin{corollary}\label{prop:poAlongHCofibIsHtpyPo}
    A pushout along an h-cofibration is a homotopy pushout.
    \begin{proof}
        Let
        \begin{center}
            \begin{tikzcd} [sep = 4em]
                A \arrow[>->, r, "h"] \arrow[d] & B \arrow[d]\\
                C \arrow[>->, r, "h"] & D \\
            \end{tikzcd}
        \end{center}
        be a pushout with the indicated maps being h-cofibrations.
        We factor the map $A\xtailrightarrow{} C'\xrightarrow{\sim} C$ into a cofibration followed by a weak equivalence and obtain the diagram
        \begin{center}
            \begin{tikzcd} [sep = 4em]
                C' \arrow[d,"\sim"] & A \arrow[>->, r, "h"] \arrow[>->,l] \arrow[d, equal] & B \arrow[d, equal]\\
                C & A \arrow[>->, r, "h"] \arrow[l] & B \\
            \end{tikzcd}\;.
        \end{center}
        By \cref{prop:poOfWeIsWe} we have that the map induced by the pushouts $C'\cup_AB\to C\cup_AB$ is again a weak equivalence.
        But since $C'\cup_AB$ is a homotopy pushout (it is a pushout along a cofibration and $\Top$ is left proper), $C\cup_AB$ is thus also a homotopy pushout which proves the proposition.
    \end{proof}
\end{corollary}
\addcontentsline{toc}{subsection}{Universality}
\subsection*{Universality}
Using \cref{prop:poAlongHCofibIsHtpyPo} and the following theorem, we first prove universality for pushouts.
\begin{thm}\label{thm:pbOfHcofibIsHcofib}
    A pullback of an h-cofibration along an h-fibration is again an h-cofibration.
    \begin{reference}
        \cite[Theorem 12]{note_on_cofibs_2}
    \end{reference}
\end{thm}
\begin{prop}\label{lem:topUniversalPo}
    The model category $\Top$ has universality for homotopy pushouts.
    \begin{proof}
        Let 
        \begin{center}
            \begin{tikzcd} [sep = .5 cm]
                A' \arrow [dr] \arrow [rr] \arrow [dd] & & B' \arrow [dr] \arrow [dd] \\
                & C' \arrow [rr, crossing over] \arrow [dd] & & D' \arrow [dd] & \\
                A \arrow [dr] \arrow [rr] & & C \arrow [dr] \\
                & C \arrow [from=uu, crossing over] \arrow [rr] & & D
            \end{tikzcd}
        \end{center}
        be a commutative cube where the bottom face is a homotopy pushout and the vertical faces are homotopy pullbacks.
        
        By factoring the map $D'\to D$ into a weak equivalence followed by an h-fibration $D'\xrightarrow{\sim}\overline{D}\xtwoheadrightarrow{h} D$ (which is always possible since we can factor any map into a trivial h-cofibration followed by an h-fibration by \cref{thm:stromModelStructure}) and pulling back successively, we obtain the diagram
        \begin{center}
            \begin{tikzcd} [sep = .5 cm]
                A^{\prime} \arrow [dr] \arrow [rr] \arrow [dd] & & B^{\prime} \arrow [dr] \arrow [dd] \\
                & C^{\prime} \arrow [rr, crossing over] & & D^{\prime} \arrow [dd, "\sim"] & \\
                \overline{A} \arrow [dr] \arrow [rr] \arrow [->>, dd, "h"] & & \overline{B} \arrow [dr] \arrow [->>, dd, near start, "h"] \\
                & \overline{C} \arrow [rr, crossing over] \arrow [from=uu, crossing over] & & \overline{D} \arrow [->>, dd, "h"] & \\
                A \arrow [dr] \arrow [rr] & & B \arrow [dr] \\
                & C \arrow [->>,from=uu, crossing over, near start, "h"] \arrow [rr] & & D & \\
            \end{tikzcd}
        \end{center}
        where $\overline{A}=A\times_{D}\overline{D}$, $\overline{B}=B\times_{D}\overline{D}$ and $\overline{C}=C\times_{D}\overline{D}$.
        Since by the pasting lemma for homotopy pullbacks the upper vertical squares are also homotopy pullbacks and $D'\xrightarrow{\sim}\overline{D}$ is a weak equivalence, all vertical maps in the upper cube of the above diagram are weak equivalences.
        Thus it is sufficient to show that the square
        \begin{center}
            \begin{tikzcd} [sep = 4em]
                \overline{A} \arrow[r] \arrow[d] & \overline{B} \arrow[d] \\
                \overline{C} \arrow[r] & \overline{D} \\
            \end{tikzcd}
        \end{center}
        is a homotopy pushout.

        We factor the maps $A\to B$ and $A\to C$ into into h-cofibrations followed by a weak equivalences $A\xtailrightarrow{h}\widehat{B}\xrightarrow{\sim} B$ and $A\xtailrightarrow{h}\widehat{C}\xrightarrow{\sim} C$ respectively (which is again possible by \cref{thm:stromModelStructure}).
        We obtain a diagram
        \begin{center}
            \begin{tikzcd} [sep = 4em]
                A \arrow[>->, r, "h"] \arrow[>->, d, "h"] & \widehat{B} \arrow[r, "\sim"] \arrow[>->, d, "h"] & B \arrow[dd] \\
                \widehat{C} \arrow[>->, r, "h"] \arrow[d, "\sim"] & \widehat{B}\cup_A\widehat{C} \arrow[dr] & \\
                C \arrow[rr] & & D
            \end{tikzcd}
        \end{center}
        where the small square is a homotopy pushout as it is a pushout along an h-cofibration by \cref{prop:poAlongHCofibIsHtpyPo}, so $\widehat{B}\cup_A\widehat{C}\to D$ is a weak equivalence.
        Since h-cofibrations are inclusions by \cref{rmk:hurewiczCofibIsIncl}, we have $\widehat{B}\cap\widehat{C}=A$ as subspaces of $\widehat{B}\cup_A\widehat{C}$ by the explicit construction of pushouts.

        Let $q\colon\left(\widehat{B}\cup_A\widehat{C}\right)\times_{D}\overline{D}\xtwoheadrightarrow{h}\widehat{B}\cup_A\widehat{C}$ denote the pullback of $\overline{D}\xtwoheadrightarrow{h} D$ along $\widehat{B}\cup_A\widehat{C}\xrightarrow{\sim} D$.
        Pulling back the square 
        \begin{center}
            \begin{tikzcd} [sep = 4em]
                \widehat{B}\cap\widehat{C}=A \arrow[>->, r, "h"] \arrow[>->, d, "h"] & \widehat{B}\arrow[>->, d, "h"] \\
                \widehat{C} \arrow[>->, r, "h"] & \widehat{B}\cup_A\widehat{C} \\
            \end{tikzcd}
        \end{center}
        along $q$ by the pasting law for pullbacks gives the cube
        \begin{center}
            \begin{tikzcd} [sep = .5 cm]
                \overline{A} \arrow [>->,dr,"h"] \arrow [>->,rr,"h"] \arrow [->>,dd,"h"] & & \widehat{B}\times_{D}\overline{D} \arrow [>->,dr,"h"] \arrow [->>,dd,near start,"h"] \\
                & \widehat{C}\times_{D}\overline{D} \arrow [>->,rr, crossing over, near start, "h"] & &[-1em] \left(\widehat{B}\cup_A\widehat{C}\right)\times_{D}\overline{D} \arrow [->>,dd,"h"',"q"] & \\
                A=\widehat{B}\cap\widehat{C} \arrow [>->,dr,"h"] \arrow [>->,rr,near end,"h"] & & \widehat{B} \arrow [>->,dr,"h"] \\
                & \widehat{C} \arrow [->>,from=uu, crossing over,near start,"h"] \arrow [>->,rr,"h"] & & \widehat{B}\cup_A\widehat{C}
            \end{tikzcd}
        \end{center}
        where the top horizontal maps are h-cofibrations since pullbacks of h-cofibrations along h-fibrations are again h-cofibrations by \cref{thm:pbOfHcofibIsHcofib}.
        We now claim that the top square of this cube is an ordinary pushout.

        First note that the pullback of a given map $f\colon X\to B$ along an inclusion $A\xhookrightarrow{} B$ can be identified with the map $f^{-1}(A)\xrightarrow{f|_{f^{-1}(A)}}A$ where $f^{-1}(A)$ is equipped with the subspace topology of $X$. 
        
        Hence we have $\widehat{B}\times_{D}\overline{D}\cong q^{-1}(\widehat{B})$, $\widehat{C}\times_{D}\overline{D}\cong q^{-1}(\widehat{C})$ and $\overline{A}\cong q^{-1}(A)\cong q^{-1}(\widehat{B})\cap q^{-1}(\widehat{C})$ as subspaces of $\left(\widehat{B}\cup_A\widehat{C}\right)\times_{D}\overline{D}$.
        Since h-cofibrations are in particular closed inclusions by \cref{rmk:hurewiczCofibIsIncl}, $q^{-1}(\widehat{B})$ and $q^{-1}(\widehat{C})$ form a closed cover of $\left(\widehat{B}\cup_A\widehat{C}\right)\times_{D}\overline{D}$.
        
        By point-set topology, given a space $X$ with closed cover $X_1,X_2\subset X$ one recovers $X_1\cup_{X_1\cap X_2}X_2\cong X$ as the pushout.
        Hence $q^{-1}(\widehat{B})\cup_{q^{-1}(A)}q^{-1}(\widehat{C})\cong\left(\widehat{B}\cup_A\widehat{C}\right)\times_{D}\overline{D}$ and thus the top square of the cube is an ordinary pushout, proving our claim.

        As the top square of the cube is a pushout along an h-cofibration, it is a homotopy pushout by \cref{prop:poAlongHCofibIsHtpyPo}.
        Since $\Top$ is right proper, the indicated maps in the diagram 
        \begin{center}
            \begin{tikzcd} [sep = 3em]
                \overline{A} \arrow[>->, r, "h"] \arrow[>->, d, "h"] & \widehat{B}\times_{D}\overline{D} \arrow[r, "\sim"] \arrow[>->, d, "h"] & \overline{B} \arrow[dd] \\
                \widehat{C}\times_{D}\overline{D} \arrow[>->, r, "h"] \arrow[d, "\sim"] & \left(\widehat{B}\cup_A\widehat{C}\right)\times_{D}\overline{D} \arrow[dr, "\sim"] & \\
                \overline{C} \arrow[rr] & & \overline{D}
            \end{tikzcd}
        \end{center}
        are weak equivalences as they are pullbacks of weak equivalences along fibrations.
        
        Thus
        \begin{center}
            \begin{tikzcd} [sep = 4em]
                \overline{A} \arrow[r] \arrow[d] & \overline{B} \arrow[d] \\
                \overline{C} \arrow[r] & \overline{D} \\
            \end{tikzcd}
        \end{center}
        is also a homotopy pushout, which proves the proposition.
    \end{proof}
\end{prop}
\begin{lemma}\label{lem:topUniversalCoproduct}
    The model category $\Top$ has universal homotopy coproducts.
    \begin{proof}
        	Let $\left(X_i\right)_{i\in I}$ and $\left(Y_i\right)_{i\in I}$ be collections of topological spaces with $I$ a small set. 
            In a model category the homotopy coproduct of such a family is given by the ordinary coproduct of cofibrant replacements $\coprod\limits_{i\in I}\widehat{X_i}$ for each of the $X_i$.
            However, since in $\Top$ weak equivalences are closed under coproducts, the ordinary coproduct $\coprod\limits_{i\in I}X_i$ is already a homotopy coproduct.

            Let $f\colon Y\to\coprod\limits_{i\in I}X_i$ be a map and let $\left(Y_i\to X_i\right)_{i\in I}$ and $\left(Y_i\to Y\right)_{i\in I}$ be families of maps such that for all $k\in I$ we have a commutative square
            \begin{center}
                \begin{tikzcd} [sep = 4em]
                    Y_k \arrow [r] \arrow [d] & Y \arrow [d, "f"]\\
                    X_k \arrow [r,hook] & \coprod\limits_{i\in I} X_i
                \end{tikzcd}
            \end{center}
            that is a homotopy pullback.

            Note that every inclusion map $X_k\xhookrightarrow{}\coprod\limits_{i\in I} X_i$ is already a fibration as can be checked by its lifting properties.
            Therefore, the ordinary pullback
            \begin{center}
                \begin{tikzcd} [sep = 1 cm]
                    f^{-1}(X_k) \arrow [->>,hook,r] \arrow [d, "f|_{f^{-1}(X_k)}"'] & Y \arrow [d, "f"]\\
                    X_k \arrow [->>,hook,r] & \coprod\limits_{i\in I} X_i
                \end{tikzcd}
            \end{center}
            is a homotopy pullback since $\Top$ is right proper, and so $Y_k\to f^{-1}(X_k)$ is a weak equivalence.
            We have $Y=f^{-1}\left(\dot{\bigcup\limits_{k\in I}} X_k\right)=\dot{\bigcup\limits_{k\in I}}f^{-1}(X_k)$ since preimages commute with unions and preserve disjointness.
            Since $X_k\subset\coprod\limits_{i\in I} X_i$ is open for all $k\in I$, $f^{-1}(X_k)\subset Y$ is open for all $k\in I$ as well. 
            By point-set topology it now follows that $Y\cong\coprod\limits_{k\in I}f^{-1}(X_k)$.
            
            Hence $\coprod\limits_{i\in I}Y_i\to Y$ is a weak equivalence, again because weak equivalences are closed under coproducts in $\Top$.
            This proves that $Y$ is a homotopy coproduct.
    \end{proof}
\end{lemma}
Therefore, by \cref{cor:sufficientToProveInModCat} we know $\spaces$ has universal colimits.
It remains to show that it also has descent.
\addcontentsline{toc}{subsection}{Descent for Homotopy Pushouts}
\subsection*{Descent for Homotopy Pushouts}
Proving descent will be more involved then the proof of universality.
The main theorem we use to achieve this will be \cref{prop:descentForPairs}, where we use a subdivision argument from \cite{may1990weak} to show a version of descent for pairs of spaces.
There are two routes that lead from this theorem to descent for pushouts:
One can either give a direct proof via decomposition of mapping cylinders (see \cref{prop:topDescentPo}) or it can be used to prove a weak version of the locality of quasifibrations (see \cref{cor:locOfQuasifib}).

We will show both; 
the first since it is very direct and the second since descent is closely related to locality of quasifibrations as \cref{rmk:locOfQuasifibDescent} demonstrates.

As a first step we will fix some notation related to pairs and triples of spaces. 
\begin{definition}[Pairs and Triples of Spaces]
    Let $\Top^{(2)}$ denote the category of \emph{pairs of topological spaces} $(X,A)$ with $A\subset X$ a subspace and maps $f\colon(X_1,A_1)\to (X_2,A_2)$ being continuous maps $f\colon X_1\to X_2$ such that $f(A_1)\subset A_2$.
    We refer to these maps as \emph{maps of pairs}.
    
    We define $(X,A)\times I$ to mean $(X\times I,A\times I)$.
    A \emph{homotopy of maps of pairs} $f,g\colon(X_1,A_1)\to (X_2,A_2)$ is a map $h\colon(X_1,A_1)\times I\to (X_2,A_2)$ such that $h_0=f$ and $h_1=g$.

    Similarly, we define maps of triples:
    Let $\Top^{(3)}$ denote the category of \emph{triples of topological spaces} $(X,A,B)$ with $B\subset A\subset X$ subspaces and maps $f\colon(X_1,A_1,B_1)\to (X_2,A_2,B_2)$ being continuous maps $f\colon X_1\to X_2$ such that $f(A_1)\subset A_2$ and $f(B_1)\subset B_2$.
    We refer to these maps as \emph{maps of triples}.
    
    We define $(X,A,B)\times I$ to mean $(X\times I,A\times I, B\times I)$.
    A \emph{homotopy of maps of triples} $f,g\colon(X_1,A_1,B_1)\to (X_2,A_2,B_2)$ is a map $h\colon(X_1,A_1,B_1)\times I\to (X_2,A_2,B_2)$ such that $h_0=f$ and $h_1=g$.
\end{definition}
\begin{definition}[Triad]
    Let $X$ be a topological space and $X_1,X_2\subset X$ be subspaces.
    Then we call $(X;X_1,X_2)$ a \emph{triad of topological spaces}.
    A triad $(X;X_1,X_2)$ is called \emph{excisive} if $X_1^°\cup X_2^°=X$.

    A \emph{map of triads} $f\colon (X;X_1,X_2)\to (Y;Y_1,Y_2)$ is a continuous map $f\colon X\to Y$ such that $f(X_1)\subset Y_1$ and $f(X_2)\subset Y_2$.
    (Note that this definition is distinct from (maps of) triples of topological spaces.)
\end{definition}
\begin{lemma}[Relative HEP]\label{lem:rHEP}
    Let $(X;X_1,X_2)$ be a triad such that $X_1\cap X_2\to X_2$ and $X_1\cup X_2\to X$ are h-cofibrations. 
    Then given a homotopy $h\colon(X_1,X_1\cap X_2)\times I\to(Y,A)$ and a map $f\colon(X,X_2)\to(Y,A)$ such that $h_0=f|_{X_1}$, there exists a homotopy $H\colon(X,X_2)\times I\to(Y,A)$ such that $H|_{X_1\times I}=h$ and $H_0=f$.
    \begin{proof}
        We apply the homotopy extension property twice: First we extend $h|_{\left(X_1\cap X_2\right)\times I}$ to a homotopy $h'\colon X_2\times I\to A$ with $h'_0=f|_{X_2}$.
        Then we can extend $h\cup_{X_1\cap X_2} h'\colon \left(X_1\cup X_2\right)\times I\to Y$ to the desired homotopy $H$.
    \end{proof}
\end{lemma}
\begin{remark}\label{rmk:cwInclIsHCofib}
    Let $J^n=\set{0}\times I^n\cup I\times\partial I^n\subset\partial I^{n+1}\subset I^{n+1}$.
    The triads $(I^{n+1};J^n,\set{1}\times I^n)$ fulfill the requirements of \cref{lem:rHEP} for all $n\geq 0$.
    This is because all the required maps can be identified with maps $S^{n-1}\hookrightarrow D^n$ (where $S^{-1}=\emptyset$) which are cofibrations and thus also h-cofibrations.
    
    The inclusion $J^n\subset I^{n+1}$ is an h-cofibration as well as a homotopy equivalence (since both spaces are contractible), hence by \cite[Corollary 0.20]{hatcher2002algebraic} $J^n$ is a (strong) deformation retract of $I^{n+1}$.
\end{remark}
\begin{definition}
    Let $H_1,H_2\colon X\times I\to Y$ be homotopies such that $(H_1)_1=(H_2)_0$.
    Then we define 
    \begin{align*}
        H_1*H_2\colon X\times I&\to Y\\
        (x,t)&\mapsto\begin{cases}
            H_1(x,2t) & t\in[0,\frac{1}{2}]\\
            H_2(x,2t-1) & t\in[\frac{1}{2},1]
        \end{cases}
    \end{align*}
    to be the \emph{concatenation of homotopies $H_1, H_2$}.

    Furthermore, for a homotopy $H\colon X\times I\to Y$ we define $\overline{H}(x,t)=H(x,1-t)$ to be the \emph{reverse homotopy of $H$} and for a map $f\colon X\to Y$ we define $\const_f\colon X\times I\to Y$ to be the constant homotopy $X\times I\to X\xrightarrow{f}Y$.
    
    Note that these definitions also preserve homotopies/maps of pairs and triples.
\end{definition}
\begin{remark}\label{rmk:choiceOfConstantHtpy}
    Let $H,H_1,H_2,H_3\colon X\times I\to Y$ be homotopies with $(H_1)_1=(H_2)_0$ and $(H_2)_1=(H_3)_0$ and let $f\colon X\to Y$ be a map.
    Then one can give explicit homotopies between homotopies $\const_f*H\simeq H$, $H*\const_f\simeq H$, $H*\overline{H}\simeq \const_{H_0}$, $\overline{H}*H\simeq\const_{H_1}$ and $(H_1*H_2)*H_3\simeq H_1*(H_2*H_3)$ that are constant at $X\times\set{0}\times I$ and $X\times\set{1}\times I$.

    Note that this is also true for homotopies of pairs and triples.
\end{remark}
We next fix a model for the homotopy groups/sets of pairs.
For us the model as maps of triples from $(I^n,\partial I^n,J^{n-1})$ is most convenient.
\begin{definition}[Homotopy Groups/Sets of Pairs]
    For a pair of spaces $(X,A)$, $a\in A$ and $n\geq 1$ we set $\pi_n(X,A,a)$ to be the set of equivalence classes of maps $(I^n,\partial I^n,J^{n-1})\to(X,A,a)$ modulo homotopies of triples $(I^n,\partial I^n,J^{n-1})\times I\to(X,A,a)$ between such maps.
\end{definition}
We will however also use $(J^n,\set{1}\times\partial I^n,*)$ with $*=(1,0,\ldots,0)$ and $(I^n,\partial I^n,(0,\ldots,0))$ as equivalent models whenever convenient.
There is a canonical group structure on these sets for $n\geq 2$ (that is even abelian for $n\geq 3$).
\begin{definition}[Weak Equivalence of Pairs]\label{def:weOfPairs}
    Let $f\colon(X,A)\to (Y,B)$ be a map such that 
    \begin{itemize}
        \item for all $n\geq 1$ and $a\in A$ the induced map $\pi_n(X,A,a)\to\pi_n(Y,B,f(a))$ is a bijection
        \item $\pi_0(f)^{-1}\im\left(\pi_0(B)\to\pi_0(Y)\right)=\im\left(\pi_0(A)\to\pi_0(X)\right)$.
    \end{itemize}
    Then we call $f$ a \emph{weak equivalence of pairs}.
\end{definition}
\begin{remark}\label{rmk:emptyFiber}
    For a map $f\colon(X,A)\to (Y,B)$ the inclusion $\im\left(\pi_0(A)\to\pi_0(X)\right)\subset\pi_0(f)^{-1}\im\left(\pi_0(B)\to\pi_0(Y)\right)$ always holds.
    Having an equality means that for all $x\in X$ such that there is a path from $f(x)$ to some element in $B$, there also exists a path from $x$ to some element of $A$.
    This is equivalent to saying that if the homotopy fiber of $A\hookrightarrow X$ at $x\in X$ is empty, then so is the homotopy fiber of $B\hookrightarrow Y$ at $f(x)$.
    This is because the homotopy fiber of a map at a point is empty if and only if the point is in a path-connected component that does not intersect the image of the map.
\end{remark}
We will need the following lemma to translate our statements about homotopy pullbacks to the language of pairs of maps and back.
\begin{lemma}\label{lem:weOfaPairsIsHtpyPb}
    Let $f\colon(X,A)\to (Y,B)$ be a map.
    Then the following are equivalent:
    \begin{enumerate}[label={(\roman*)}]
        \item The map $f$ is a weak equivalence of pairs. \label{itm:weOfPairs}
        \item The induced commutative square \begin{center} 
            \begin{tikzcd} [sep = 4em]
                A \arrow [r, "f|_A"] \arrow [d, hook] & B \arrow [d, hook]\\
                X \arrow [r, "f"] & Y
            \end{tikzcd}
        \end{center}
        is a homotopy pullback square. \label{itm:htpyPb}
    \end{enumerate}
    \begin{proof}
        We model the homotopy fibers for given points $x_0\in X$ and $y_0\in Y$ as the pullback squares
        \begin{center} 
            \begin{tikzcd} [sep = 4em]
                F_{x_0} \arrow [r, hook] \arrow [->>, d, "h"', "\ev_1"] & P_{x_0}X \arrow [->>, d, "h"', "\ev_1"]\\
                A \arrow [r, hook] & X
            \end{tikzcd}
            \begin{tikzcd} [sep = 4em]
                G_{y_0} \arrow [r, hook] \arrow [->>, d, "h"', "\ev_1"] & P_{y_0}Y \arrow [->>, d, "h"', "\ev_1"]\\
                B \arrow [r, hook] & Y
            \end{tikzcd}
        \end{center}
        where $P_{x_0}X$ and $P_{y_0}Y$ are the path mapping spaces of the maps $x_0\colon *\to X$ and  $y_0\colon *\to Y$ respectively.
        We can identify $F_{x_0}$ with the subspace of $P_{x_0}X$ of paths that start at $x_0$ and end at a point in $A$ (and analogously for $G_{y_0}$).
        Let $\alpha\colon F_{x_0}\to G_{f(x_0)}, \gamma\mapsto f\gamma$ denote the induced map between fibers.
        Proving \ref{itm:htpyPb} is equivalent to showing that this map is a weak equivalence for all $x_0\in X$.

        If $F_{x_0}\neq\emptyset$ we have that for every point $z\in F_{x_0}$ and $n\geq 1$ the induced maps $\pi_n(X,A,z(1))\to \pi_{n-1}(F_{z(1)},\const_{z(1)})$ and $\pi_n(Y,B,f(z(1)))\to \pi_{n-1}(G_{\alpha(z)(1)},\const_{\alpha(z)(1)})$ are isomorphisms by \cite[Theorem 5.1.8]{lectures_htpy_thy}.
        The map $\Lambda\colon F_{x_0}\to F_{z(1)}, \gamma\mapsto \overline{z}*\gamma$ is a homotopy equivalence with homotopy inverse $\Lambda^{-1}\colon F_{z(1)}\to F_{x_0}, \gamma\mapsto z*\gamma$, hence it induces an isomorphism $\pi_{n-1}(F_{x_0},z)\to\pi_{n-1}(F_{z(1)},\overline{z}*z)$.
        By \cref{rmk:choiceOfConstantHtpy} there is a homotopy $H\colon I\times I\to X$ contracting the loop $\overline{z}*z$ to the point $z(1)$ such that $H_0=\overline{z}*z$, $H_1=\const_{z(1)}$ and $\partial I\times I=\const_{z(1)}$.
        This homotopy thus corresponds to a path $\lambda\colon I\to F_{z(1)}$ from $\overline{z}*z$ to $\const_{z(1)}$ which induces isomorphisms $\pi_{n-1}(F_{z(1)},\overline{z}*z)\to\pi_{n-1}(F_{z(1)},\const_{z(1)})$ for all $n\geq 1$.

        This also gives a homotopy equivalence $\Gamma\colon G_{f(x_0)}\to F_{f(z(1))}, \gamma\mapsto f\overline{z}*\gamma$ and a path $f\lambda$ from $\overline{fz}*fz$ to $\const_{\alpha(z)(1)}$ which induce the analogous isomorphisms.
        
        Together these isomorphisms fit into a commutative square
        \begin{center} 
            \begin{tikzcd} [sep = 4em]
                \pi_n(X,A,z(1)) \arrow [r, "\cong"] \arrow [d, "\pi_n(f)"] & \pi_{n-1}(F_{x_0},z) \arrow [d, "\pi_{n-1}(\alpha)"]\\
                \pi_n(Y,B,f(z(1))) \arrow [r, "\cong"] & \pi_{n-1}(G_{f(x_0)},\alpha(z))
            \end{tikzcd}
        \end{center}
        which with \cref{rmk:emptyFiber} for the case $F_{x_0}=\emptyset$ proves that \ref{itm:weOfPairs} and \ref{itm:htpyPb} are equivalent.
    \end{proof}
\end{lemma}
The next lemma is a variant of what is commonly referred to as the ``compression criterion'' (see e.g. \cite[343]{hatcher2002algebraic}).
\begin{lemma}\label{lem:compressionVariation}
    Let $n\geq 1$, $e\colon(I^{n+1},\set{1}\times I^n)\to (X,A)$, $*=(1,0,\ldots,0)\in\set{1}\times\partial I^n$ and $a=e(*)$. 
    Then $e|_{J^n}\colon(J^n,\set{1}\times\partial I^n,*)\to(X,A,a)$ represents the equivalence class of the constant map in $\pi_n(X,A,a)$.
    \begin{proof}
        First pick a homeomorphism $\alpha\colon I^n\cong J^n$ such that $\alpha|_{\partial I^n}(x)=(1,x)\in J^n$.
        Then we can pick a map $F\colon I^{n+1}\cong I\times I^n\to I^{n+1}$ such that 
        \begin{itemize}
            \item $F|_{\set{0}\times I^n}(0,x)=\alpha$
            \item $F|_{\set{1}\times I^n}=\id_{\set{1}\times I^n}$
            \item $F|_{I\times \partial I^n}(t,x)=(1,x)\in I^{n+1}$
        \end{itemize}
        because this data determines a map $\partial I^{n+1}\to I^{n+1}$ that can be lifted along the cofibration $\partial I^{n+1}\xtailrightarrow{}I^{n+1}$ since $I^{n+1}$ is contractible.

        An explicit construction is also possible:
        As indicated below, one can pick $F$ such that for all $t_0\in I$, the map $F|_{\set{t_0}\times I^n}$ is a homeomorphism onto $J_{t_0}^n=\set{t_0}\times I^n\cup[t_0,1]\times\partial I^n$ and such that $F|_{\set{t_0}\times\partial I^n}(t_0,x)=(1,x)$.
        \begin{center}
            \begin{tikzpicture}[scale = 2, opacity = .7]
                \newcommand\diagx{.71}
                \newcommand\diagy{.71}
                \newcommand\lenMult{.71}
                \newcommand\relHeight{.7}
                \newcommand\xshift{0}

                \draw (0+\xshift,0) -- (0+\xshift,1);
                \draw (0+\xshift,0) -- (1+\xshift,0);
                \draw (1+\xshift,0) -- (1+\xshift,1);
                \draw (0+\xshift,1) -- (1+\xshift,1);

                \draw (0+\xshift,0) -- (-\diagx*\lenMult+\xshift,\diagy*\lenMult);
                \draw (0+\xshift,1) -- (-\diagx*\lenMult+\xshift,1+\diagy*\lenMult);
                \draw[dashed] (1+\xshift,0) -- (1-\diagx*\lenMult+\xshift,\diagy*\lenMult);
                \draw (1+\xshift,1) -- (1-\diagx*\lenMult+\xshift,1+\diagy*\lenMult);

                \draw (-\diagx*\lenMult+\xshift,\diagy*\lenMult) -- (-\diagx*\lenMult+\xshift,1+\diagy*\lenMult);
                \draw[dashed] (-\diagx*\lenMult+\xshift,\diagy*\lenMult) -- (1-\diagx*\lenMult+\xshift,\diagy*\lenMult);
                \draw[dashed] (1-\diagx*\lenMult+\xshift,\diagy*\lenMult) -- (1-\diagx*\lenMult+\xshift,1+\diagy*\lenMult);
                \draw (-\diagx*\lenMult+\xshift,1+\diagy*\lenMult) -- (1-\diagx*\lenMult+\xshift,1+\diagy*\lenMult);

                \draw[line width = 1pt, fill opacity = .25, fill = blue, draw = red] (0+\xshift,1-\relHeight) -- (1+\xshift,1-\relHeight) -- (1-\diagx*\lenMult+\xshift,1+\diagy*\lenMult-\relHeight) --(-\diagx*\lenMult+\xshift,1+\diagy*\lenMult-\relHeight) -- cycle;

                \renewcommand\xshift{3}

                \draw (0+\xshift,0) -- (0+\xshift,1);
                \draw (0+\xshift,0) -- (1+\xshift,0);
                \draw (1+\xshift,0) -- (1+\xshift,1);
                \draw (0+\xshift,1) -- (1+\xshift,1);

                \draw (0+\xshift,0) -- (-\diagx*\lenMult+\xshift,\diagy*\lenMult);
                \draw (0+\xshift,1) -- (-\diagx*\lenMult+\xshift,1+\diagy*\lenMult);
                \draw[dashed] (1+\xshift,0) -- (1-\diagx*\lenMult+\xshift,\diagy*\lenMult);
                \draw (1+\xshift,1) -- (1-\diagx*\lenMult+\xshift,1+\diagy*\lenMult);

                \draw (-\diagx*\lenMult+\xshift,\diagy*\lenMult) -- (-\diagx*\lenMult+\xshift,1+\diagy*\lenMult);
                \draw[dashed] (-\diagx*\lenMult+\xshift,\diagy*\lenMult) -- (1-\diagx*\lenMult+\xshift,\diagy*\lenMult);
                \draw[dashed] (1-\diagx*\lenMult+\xshift,\diagy*\lenMult) -- (1-\diagx*\lenMult+\xshift,1+\diagy*\lenMult);
                \draw (-\diagx*\lenMult+\xshift,1+\diagy*\lenMult) -- (1-\diagx*\lenMult+\xshift,1+\diagy*\lenMult);

                \draw[opacity = .2, fill = blue, draw = none] (0+\xshift,1) -- (1+\xshift,1) -- (1+\xshift,1-\relHeight) -- (0+\xshift,1-\relHeight) -- cycle;
                \draw[opacity = .2, fill = blue, draw = none] (-\diagx*\lenMult+\xshift,1+\diagy*\lenMult) -- (0+\xshift,1) -- (0+\xshift,1-\relHeight) -- (-\diagx*\lenMult+\xshift,1+\diagy*\lenMult-\relHeight) -- cycle;
                \draw[opacity = .2, fill = blue, draw = none] (-\diagx*\lenMult+\xshift,1+\diagy*\lenMult) -- (1-\diagx*\lenMult+\xshift,1+\diagy*\lenMult) -- (1-\diagx*\lenMult+\xshift,1+\diagy*\lenMult-\relHeight) -- (-\diagx*\lenMult+\xshift,1+\diagy*\lenMult-\relHeight) -- cycle;
                \draw[opacity = .2, fill = blue, draw = none] (1-\diagx*\lenMult+\xshift,1+\diagy*\lenMult) -- (1+\xshift,1) -- (1+\xshift,1-\relHeight) -- (1-\diagx*\lenMult+\xshift,1+\diagy*\lenMult-\relHeight) -- cycle;
                \draw[line width = .1, opacity = .25, fill = blue, draw = blue] (0+\xshift,1-\relHeight) -- (1+\xshift,1-\relHeight) -- (1-\diagx*\lenMult+\xshift,1+\diagy*\lenMult-\relHeight) --(-\diagx*\lenMult+\xshift,1+\diagy*\lenMult-\relHeight) -- cycle;
                \draw[line width = 1pt, draw = red] (0+\xshift,1) -- (1+\xshift,1) -- (1-\diagx*\lenMult+\xshift,1+\diagy*\lenMult) --(-\diagx*\lenMult+\xshift,1+\diagy*\lenMult) -- cycle;

                \draw[->, opacity = 1] (1.2,0.75) -- (1.3+1,0.75) node [above,midway]{$F|_{\set{t_0}\times I^n}$};
                \node[opacity = 1, align = center] at (1.4,1-\relHeight+0.1) {\small$\set{t_0}\times I^n$};
                \node[opacity = 1, align = center] at (1.2+\xshift,1-\relHeight+0.3) {\small$J_{t_0}^n$};
            \end{tikzpicture}
        \end{center}
        Then $e\circ F\colon I\times I^n\to X$ is a homotopy $\rel \partial I^n$ from $e|_{J^n}\circ F_0$ to $e|_{\set{1}\times I^n}\circ F_1$. 
        Since the image of the latter is contained in $A$, by the compression criterion $e|_{J^n}\circ F_0$ and hence $e|_{J^n}$ represent the equivalence class of the constant map in $\pi_n(X,A,a)$.
        % Fix a homeomorphism $\alpha\colon J^n\cong I^n$ such that $\alpha|_{\set{1}\times I^n}=\id_{I^n}$.
        % Then the square  
        % \begin{center}
        %     \begin{tikzcd} [sep = 4em]
        %         J^n\times\partial I\cup\partial J^n\times I \arrow [r, "f"] \arrow [>->, d, "h"] & I^{n+1} \arrow [->>, d, "h","\simeq"']\\
        %         J^n\times I \arrow[ur, dashed, "\exists H"] \arrow[r,"\simeq"] & *
        %     \end{tikzcd}
        % \end{center}
        % where the top horizontal map $f$ is given by
        % \begin{itemize}
        %     \item the inclusion $J^n\hookrightarrow I^{n+1}$ on $J^n\times\set{0}$ 
        %     \item the inclusion $J^n\cong\set{1}\times I^n\hookrightarrow I^{n+1}$ on $J^n\times\set{1}$ 
        %     \item the map $\partial J^n \times I=\left(\set{1}\times\partial I^n\right)\times I\to\set{1}\times\partial I^n \hookrightarrow I^{n+1}$ on $\partial J^n \times I$ 
        % \end{itemize}
        % admits a lift $H$. 
        % This gives a homotopy $\widetilde{H}=e\circ H\colon(J^n,\set{1}\times \partial I^n,*)\times I\to (X,A,a)$ from $e|_{J^n}$ to some map $\widetilde{H}_1=f$ with $\im f\subset A$, and $\widetilde{H}$ is constant on $\partial J^n\times I$ .
        % Since $\im f\subset A$ and $J^n$ is contractible with respect to $*$, $f$ represents $\const_a$ which proves the proposition. 
    \end{proof}
\end{lemma}
The next proposition is the most important part of the proof of \cref{prop:descentForPairs} from which we will derive descent for pushouts.
\begin{prop}\label{prop:eqCharInjSurj}
    Given a map $f\colon(X,A)\to (Y,B)$ and $n\geq0$, the following are equivalent:
    \begin{enumerate}[label={(\roman*)}]
        \item For every $a\in A$ the map $\pi_n(X,A,a)\to \pi_n(Y,B,f(a))$ is injective and $\pi_{n+1}(X,A,a)\to \pi_{n+1}(Y,B,f(a))$ is surjective.
            If $n=0$, replace the injectivity statement by $\pi_0(f)^{-1}\im\left(\pi_0(B)\to\pi_0(Y)\right)=\im\left(\pi_0(A)\to\pi_0(X)\right)$. \label{itm:injAndSurjOfHtpyGrps}
        \item Given maps $e\colon(I^{n+1},\set{1}\times I^n)\to (Y,B)$, $g\colon(J^n,\set{1}\times\partial I^n)\to(X,A)$ and a homotopy $h\colon(J^n,\set{1}\times\partial I^n)\times I\to(Y,B)$ from $e|_{J^n}$ to $fg$, there exists a map $G\colon(I^{n+1},\set{1}\times I^n)\to(X,A)$ and homotopy $H\colon(I^{n+1},\set{1}\times I^n)\times I\to(Y,B)$ from $e$ to $fG$ such that $H|_{J^n\times I}=h$ and $G|_{J^n}=g$. \label{itm:liftingMapsAndHtpy} 
        \begin{center}
            \begin{adjustbox}{width=\linewidth}
                \tikzcdset{every label/.append style = {font = \fontsize{11pt}{11pt}\selectfont}}
                \begin{tikzcd} [column sep = -1em]
                    &|[xshift=-1.5em, yshift=1.5em]| (X,A) \ar[rrr, "f"] &&&|[xshift=2.5em, yshift=1.5em]| (Y,B)\\
                    (J^n,\set{1}\times\partial I^n) \ar[dr, hook] \ar[rr, near end, "i_1"] \ar[ru, "g"] && (J^n,\set{1}\times\partial I^n)\times I \ar[dr, hook] \ar[urr, "h"] && (J^n,\set{1}\times\partial I^n) \ar[ll,"i_0"', shorten <=0.5em] \ar[dr, hook] \ar[u, "e|_{J^n}"']\\
                    & (I^{n+1},\set{1}\times I^n) \ar[rr, "i_1"] \ar[uu, crossing over, near end, dashed, "\exists G"'] && (I^{n+1},\set{1}\times I^n)\times I \ar[uur, shift left=0.5em, near start, dashed, shorten >=-0.1em, shorten <=-1em,"\exists H"'] && (I^{n+1},\set{1}\times I^n) \ar[ll,"i_0"'] \ar[luu, "e"']\\
                \end{tikzcd}
            \end{adjustbox}        
        \end{center}
        \item The statement of \ref{itm:liftingMapsAndHtpy} holds with the additional assumption that $e|_{J^n}=fg$ and $h$ is the constant homotopy. \label{itm:liftingMapsAndHtpyConst}
    \end{enumerate}
    \begin{proof}
        The proof follows \cite[Lemma 3.3]{may1990weak}.

        \ref{itm:liftingMapsAndHtpyConst} implies \ref{itm:injAndSurjOfHtpyGrps}:

        We first prove the injectivity statement for $n=0$. 
        By \cref{rmk:emptyFiber} we need to show that when given $x\in X$ and path $f(x)$ to some element of $B$, there exists a path from $x$ to some element of $A$.

        So let $g=x$ and $e$ be a path from $f(x)$ to $b\in B$.
        Then by \ref{itm:liftingMapsAndHtpyConst} there is a map $G\colon (I,\set{1})\to(X,A)$ such that $G(0)=x=g$.
        This is the desired path.

        We next prove that for $n\geq 1$, \ref{itm:liftingMapsAndHtpyConst} implies injectivity of $\pi_n(X,A,a)\to \pi_n(Y,B,f(b))$.
        Given $[\lambda],[\mu]\in\pi_n(X,A,a)$ such that $\pi_n(f)([\lambda])=\pi_n(f)([\mu])$, we have a homotopy $\widetilde{h}\colon(I^n,\partial I^n,J^{n-1})\times I\to (Y,B,b)$ from $f\lambda$ to $f\mu$.
        Set $e=\widetilde{h}$ and using $J^n=J^{n-1}\times I\cup I^n\times\partial I$ define $g$ to be $\lambda$ on $I^n\times\set{0}$, $\mu$ on $I^n\times\set{1}$ and $\const_a$ on $J^{n-1}\times I$.
        Then $e|_{J^n}=fg$ and so by \ref{itm:liftingMapsAndHtpyConst} we get a map $G\colon(I^{n+1},\set{1}\times I^n)\to(X,A)$ with $G|_{J^n}=g$. 
        Thus $G$ is a homotopy $(I^n,\partial I^n,J^{n-1})\times I\to(X,A,a)$ from $\lambda$ to $\mu$, hence $[\lambda]=[\mu]$ in $\pi_n(X,A,a)$.

        Lastly we show that for $n\geq 0$, \ref{itm:liftingMapsAndHtpyConst} implies surjectivity of $\pi_{n+1}(X,A,a)\to \pi_{n+1}(Y,B,f(b))$.
        Let $[\lambda]\in\pi_n(Y,B,f(a))$ and set $e=\lambda$ and $g=\const_a$. 
        Then we get a map $G\colon(I^{n+1},\set{1}\times I^n)\to(X,A)$ and homotopy $H\colon(I^{n+1},\set{1}\times I^n)\times I\to(Y,B)$ from $e$ to $fG$ such that $H|_{J^n\times I}$ is $\const_{f(a)}$ and $G|_{J^n}=\const_a$.
        Thus $G$ is a map $(I^{n+1},\partial I^{n+1},J^n)\to (X,A,a)$ and $H$ is a homotopy $(I^{n+1},\partial I^{n+1},J^n)\times I\to(Y,B,f(a))$ from $\lambda$ to $fG$ which shows $[\lambda]=[fG]$ in $\pi_{n+1}(Y,B,f(a))$.

        \ref{itm:injAndSurjOfHtpyGrps} implies \ref{itm:liftingMapsAndHtpyConst}:

        For $n=0$ we view $e$ as a path from $f(g)$ to some element of $b\in B$, so by \cref{rmk:emptyFiber} we can choose a path $G'\colon I\to X$ which goes from $g$ to some element $a\in A$.
        Since $\pi_1(f)\colon\pi_1(X,A,a)\to\pi_1(Y,B,f(a))$ is surjective we can pick a map $\lambda\colon(I^1,\partial I^1,J^0)\to(X,A,a)$ and a homotopy $L\colon(I^1,\partial I^1,J^0)\times I\to(Y,B,f(a))$ from $\overline{fG'}*e$ to $f\lambda$.
        
        Concatenating the constant homotopy at $fG'$, we obtain a homotopy $M\colon(I^1,\partial I^1,J^0)\times I\to (Y,B,f(g))$ from $fG'*(\overline{fG'}*e)$ to $fG'*f\lambda=f(G'*\lambda)$.
        By \cref{rmk:choiceOfConstantHtpy} can now pick a homotopy $N\colon(I^1,\partial I^1,J^0)\times I\to(Y,B,f(g))$ from $fG'*(\overline{fG'}*e)$ to $e$. 
        Setting $G=G'*\lambda$ and $H=\overline{N}*M$ then gives the desired maps.

        For $n\geq 1$, we fix a homotopy equivalence of triples $(I^n,\partial I^n, J^{n-1})\simeq (J^n,\set{1}\times\partial I^n,*)$ where $*=(1,0,\ldots,0)$.
        Given maps $e\colon(I^{n+1},\set{1}\times I^n)\to (Y,B)$ and $g(J^n,\set{1}\times\partial I^n)\to(X,A)$ with $e|_{J^n}=fg$, setting $a=g(*)$ we can view $g$ as an element of $\pi_n(X,A,a)$ via the aforementioned homotopy equivalence of triples.
        Then $fg$ is homotopic to the constant map in $\pi_n(Y,B,f(a))$ by \cref{lem:compressionVariation}.

        By the injectivity statement in \ref{itm:injAndSurjOfHtpyGrps}, we thus get a homotopy $j\colon(J^n,\set{1}\times\partial I^n,*)\times I\to(X,A,a)$ from $g$ to $\const_a$.
        Next we apply relative homotopy extension to the triad $(I^{n+1};J^n,\set{1}\times I^n)$ to extend the homotopy $fj$ to a homotopy $K\colon(I^{n+1},\set{1}\times I^n)\times I\to(Y,B)$ such that $K_0=e$.
        Since $K_1|_{J^n}=\const_{f(a)}$, it is a map $K_1\colon(I^{n+1},\partial I^{n+1},J^n)\to(Y,B,f(a))$ and represents an element in $\pi_{n+1}(Y,B,f(a))$.
        
        By surjectivity of $\pi_{n+1}(f)$ we then get a map $J_1\colon(I^{n+1},\partial I^{n+1},J^n)\to(X,A,a)$ and a homotopy of triples $L\colon(I^{n+1},\partial I^{n+1},J^n)\times I\to(Y,B,f(a))$ from $K_1$ to $fJ_1$.
        By the relative homotopy extension property for $(I^{n+1};J^n,\set{1}\times I^n)$ and by reversing the unit intervall, we extend the $j$ to a homotopy $J\colon(I^{n+1},\set{1}\times I^n)\to(X,A)$ which ends at $J_1$.
        Setting $G=J_0$ we have $G|_{J^n}=j_0=g$.

        Furthermore, we have the homotopy $(K*L)*\overline{fJ}\colon(J^n,\set{1}\times\partial I^n)\times I\to(Y,B)$ from $fg$ to $fg$.
        By \cref{rmk:choiceOfConstantHtpy} we can pick a homotopy $M\colon(J^n\times I,\set{1}\times\partial I^n\times I)\times I\to(Y,B)$ from $(fj*\const_{f(a)})*\overline{fj}$ to the constant homotopy at $fg$ such that $M|_{J^n\times\set{0}\times I}$ and $M|_{J^n\times\set{1}\times I}$ are both also the constant homotopies at $fg$.
        
        Using the relative homotopy extension property on $(I^{n+2};J^n\times I\cup I^{n+1}\times\set{0}\cup I^{n+1}\times\set{1},\set{1}\times I^{n+1})$, we extend the homotopy given by the constant homotopy at $e$ on $I^{n+1}\times\set{0}\times I$, the constant homotopy at $fG$ on $I^{n+1}\times\set{1}\times I$ and $M$ on $J^n\times I\times I$ to a homotopy $\widetilde{M}\colon(I^{n+2},\set{1}\times I^{n+1})\times I\to(Y,B)$ with $\widetilde{M}_0=(K*L)*\overline{fJ}$.
        Now setting $H=\widetilde{M_1}$ gives the desired homotopy from $e$ to $fG$ that is the constant homotopy at $fg$ on $J^n\times I$.
        
        \ref{itm:liftingMapsAndHtpyConst} implies \ref{itm:liftingMapsAndHtpy}:

        Let $e\colon(I^{n+1},\set{1}\times I^n)\to (Y,B)$, $g(J^n,\set{1}\times\partial I^n)\to(X,A)$ and $h\colon(J^n,\set{1}\times\partial I^n)\times I\to(Y,B)$ be a homotopy from $e|_{J^n}$ to $fg$.
        By relative homotopy extension applied to the triad $(I^{n+1};J^n,\set{1}\times I^n)$, there exists a homotopy $j\colon(I^{n+1},\set{1}\times I^n)\times I\to(Y,B)$ extending $h$ such that $j_0=e$.
        
        Since $j_1|_{J^n}=fg$ we can apply \ref{itm:liftingMapsAndHtpyConst} to get a map $G\colon(I^{n+1},\set{1}\times I^n)\to(X,A)$ such that $G|_{J^n}=g$ and a homotopy $k\colon(I^{n+1},\set{1}\times I^n)\times I\to(Y,B)$ such that $k|_{J^n\times I}$ is the constant homotopy of $j_1|_{J^n}=fg$. 
        Then by \cref{rmk:choiceOfConstantHtpy} we can pick a homotopy $L\colon(J^n\times I,\set{1}\times\partial I^n\times I)\times I\to (Y,B)$ from $h*\const_{fg}$ to $h$ such that $L|_{J^n\times\set{0}\times I}=\const_{e|_{J^n}}$ and $L|_{J^n\times\set{1}\times I}=\const_{fg}$.
        
        We now apply relative homotopy extension to the triad $(I^{n+1}\times I;J^{n+1},\set{1}\times I^n\times I)$, noting that $J^{n+1}=J^n\times I\cup I^{n+1}\times\partial I$, to extend the homotopy given by the union of $L$ on $J^n\times I\times I$, $e$ on $I^{n+1}\times\set{0}\times I$ and $fG$ on $I^{n+1}\times\set{1}\times I$ to a homotopy $\widetilde{L}\colon(I^{n+1}\times I,\set{1}\times I^n\times I)\times I\to(Y,B)$ such that $\widetilde{L}_0=j*k$.
        Setting $H=\widetilde{L}_1$ gives the desired homotopy.


        Since \ref{itm:liftingMapsAndHtpy} implies \ref{itm:liftingMapsAndHtpyConst} trivially we have proven the proposition.
    \end{proof}
\end{prop}
For convenience we fix notation for some variations of the mapping cylinder construction.
\begin{definition}[Mapping Cylinder]
    Let $f\colon A\to B$ be a map of topological spaces and let $I=[0,1]$.
    Then we let
    \begin{equation*}
        \M(f)=\newfaktor{\left(A\times I\right)\amalg B}{(a,0)\sim f(a)}
    \end{equation*}
    denote the \emph{mapping cylinder of $f$}.

    It will be convenient to also allow other intervalls instead of $I=[0,1]$; we will allow $I=[0,x]$ and $I=[0,x)$ for $x>0$.
    The analogous construction with $I=[0,x]$ will be referred to as a \emph{closed mapping cylinder of $f$} and denoted $\M^{[0,x]}(f)$;
    the construction with half open intervalls $I=[0,x)$ will be referred to as an \emph{open mapping cylinder of $f$} and denoted $\M^{[0,x)}(f)$.
\end{definition}
\begin{remark}\label{rmk:inclIntoMappingCylIsHCofib}
    Note that a mapping cylinder $\M^I(f)$ (with $f\colon A\to B$ and $I=[0,x]$ or $I=[0,x)$ for $x>0$) is a model of the pushout 
    \begin{center}
        \begin{tikzcd}
            A\times\set{0} \ar[r,"f"] \ar[d,hook] & B \ar[d,hook]\\
            A\times I \ar[r] & \M^I(f)
        \end{tikzcd}\;.
    \end{center}
    If $I=[0,1]$, both inclusions are h-cofibrations and inclusion $A\times\set{1}\xhookrightarrow{}\M(f)$ is also an h-cofibration by \cite[Theorem 2]{note_on_cofibs_1}.
\end{remark}
\begin{definition}[Double Mapping Cylinder]\label{def:doubleMappingCylinder}
    Let $f\colon A\to B$ and $b\colon A\to C$ be maps of topological spaces and let $I=[0,2]$.
    Then we let
    \begin{equation*} %TODO perhaps make sure that \D(f,g) is more in the middle
        \D(f,g)=\newfaktor{\left(A\times I\right)\amalg B\amalg C}{\stackbox{$(a,0)\sim f(a)$\\$(a,2)\sim g(a)$}}[-1]
    \end{equation*}
    denote the \emph{double mapping cylinder of $f,g$}.
\end{definition}
\begin{remark}\label{rmk:doubleMappingCylinderAlt}
    Of course one could also define the double mapping cylinder for intervalls other then $I=[0,2]$, but we will not need this.

    Similarly to the mapping cylinder, the double mapping cylinder of maps $f\colon A\to B$ and $b\colon A\to C$ is a model of the pushout
    \begin{center}
        \begin{tikzcd}
            A\times\set{0,2} \ar[r,"f\amalg g"] \ar[d, hook] & B\amalg C \ar[d, hook]\\
            A\times [0,2] \ar[r] & \D(f,g)
        \end{tikzcd}\;.
    \end{center}
    It can also be constructed by glueing closed or open mapping cylinders of $f$ and $g$ as follows:

    For closed mapping cylinders, the commutative square
    \begin{center}
        \begin{tikzcd}
            A\times\set{1} \ar[>->,r, "h"] \ar[>->,d,"h"] & \M(f) \ar[>->,d,"h"]\\
            \M(g) \ar[>->,r, "h"] & \D(f,g)
        \end{tikzcd}
    \end{center}
    where 
    \begin{itemize}
        \item the map $\M(f)\to\D(f,g)$ is the universal map induced by $B\xhookrightarrow{}B\amalg C$ and $A\times[0,1]\xhookrightarrow{}A\times[0,2]$
        \item the map $\M(g)\to\D(f,g)$ is the universal map induced by $C\xhookrightarrow{}B\amalg C$ and 
        \begin{align*}
            A\times[0,1]&\to A\times[0,2]\\
            (a,t)&\mapsto(a,2-t)
        \end{align*}
    \end{itemize}
    is a pushout. 
    This follows from point-set topology because $\M(f),\M(g)\subset\D(f,g)$ form a closed cover and $\M(f)\cap\M(g)=A\times\set{1}\subset\D(f,g)$.
    Thus it is also a homotopy pushout by \cref{prop:poAlongHCofibIsHtpyPo}. 
    The maps $\M(f)\xtailrightarrow{h}\D(f,g)$ and $\M(g)\xtailrightarrow{h}\D(f,g)$ are h-cofibrations since the top and left map of the square are h-cofibrations by \cref{rmk:inclIntoMappingCylIsHCofib}.

    Similarly for open mapping cylinders, the commutative square
    \begin{center}
        \begin{tikzcd}
            A\times(0,2) \ar[r,hook] \ar[d,hook] & \M^{[0,2)}(f) \ar[d,hook]\\
            \M^{[0,2)}(g) \ar[r,hook] & \D(f,g)
        \end{tikzcd}
    \end{center}
    where 
    \begin{itemize}
        \item the map $A\times(0,2)\to\M^{[0,2)}(f)$ is the canonical inclusion
        \item the map $A\times(0,2)\to\M^{[0,2)}(g)$ is the map
        \begin{align*}
            A\times(0,2)&\to\M^{[0,2)}(g)\\
            (a,t)&\mapsto(a,2-t)
        \end{align*}
        \item the map $\M^{[0,2)}(f)\to\D(f,g)$ is the universal map induced by $B\xhookrightarrow{}B\amalg C$ and $A\times[0,2)\xhookrightarrow{}A\times[0,2]$
        \item the map $\M^{[0,2)}(g)\to\D(f,g)$ is the universal map induced by $C\xhookrightarrow{}B\amalg C$ and 
        \begin{align*}
            A\times[0,2)&\to A\times[0,2]\\
            (a,t)&\mapsto(a,2-t)
        \end{align*}
    \end{itemize}
    is a pushout.
    This follows from point-set topology because $\M^{[0,2)}(f),\M^{[0,2)}(g)\subset\D(f,g)$ form an open cover and $\M^{[0,2)}(f)\cap\M^{[0,2)}(g)=A\times(0,2)\subset\D(f,g)$.
\end{remark}
\begin{lemma}\label{lem:replaceByEmbedding}
    Let $(X;X_1,X_2)$ and $(Y;Y_1,Y_2)$ be excisive triads and let $f\colon (X;X_1,X_2)\to (Y;Y_1,Y_2)$ be a map of triads.
    Then there exists an excisive triad $(\widetilde{Y};\widetilde{Y_1},\widetilde{Y_2})$ and maps of triads $\widetilde{f}\colon(X;X_1,X_2)\to(\widetilde{Y};\widetilde{Y_1},\widetilde{Y_2})$, $j\colon(\widetilde{Y};\widetilde{Y_1},\widetilde{Y_2})\to(Y;Y_1,Y_2)$ factoring $j\widetilde{f}=f$ such that
    \begin{itemize}
        \item $\widetilde{f}$ is an inclusion and $j$ is a weak equivalence in $\Top$
        \item $\widetilde{f}(X_i^°)=\widetilde{f}(X)\cap \widetilde{Y_i}^°$ and $\widetilde{f}(X_i)=\widetilde{f}(X)\cap \widetilde{Y_i}$ for $i\in\set{1,2}$
        \item $j$ induces a weak equivalence of pairs $(\widetilde{Y_i},\widetilde{Y_1}\cap \widetilde{Y_2})\to(Y_i,Y_1\cap Y_2)$ for $i\in\set{1,2}$
        \item and $j$ induces a weak equivalence of pairs $(\widetilde{Y},\widetilde{Y_i})\to(Y,Y_i)$ for $i\in\set{1,2}$.
    \end{itemize}
    \begin{proof}
        Let $\widetilde{Y}=\M(f)$ be the mapping cylinder of $f$.
        We set $\widetilde{Y_i}=\M(f|_{X_i})\cup f^{-1}(Y_i)\times\left(0,\frac{1}{2}\right)$ where we consider $f|_{X_i}$ to be a map $X_i\to Y_i$.
        Then the triad $(\widetilde{Y};\widetilde{Y_1},\widetilde{Y_2})$ together with the choices of $\widetilde{f}\colon X\to \widetilde{Y}$ as the inclusion into the mapping cylinder and as $g\colon\widetilde{Y}\to Y$ the contraction map give a factorization of $f$.
        This also immediately gives $\widetilde{f}(X_i)=\widetilde{f}(X)\cap \widetilde{Y_i}$.
        
        We first show that $(\widetilde{Y};\widetilde{Y_1},\widetilde{Y_2})$ is excisive:
        First assume $x\in X\times(0,1]\subset\M(f)$, then $x\in X_i^°\times(0,1]$ for some $i\in\set{1,2}$.
        Since $X_i^°\times(0,1]$ is open and contained in $\widetilde{Y_i}$, we have $x\in \widetilde{Y_i}^°$.
        
        If $x\in Y\subset\M(f)$, then $x\in Y_i^°\subset\M(f)$ for some $i\in\set{1,2}$.
        In this case $x$ is contained in the open set $\widetilde{f}^{-1}(Y_i^°)\times (0,\frac{1}{2})\cup Y_i^°\subset\widetilde{Y_i}$, hence $x\in \widetilde{Y_i}^°$.

        We now show $\widetilde{f}(X_i^°)=\widetilde{f}(X)\cap \widetilde{Y_i}^°$: 
        Since $X_i^°\times(\frac{1}{2},1]=Y_i^°\cap X\times(\frac{1}{2},1]$, we have $\widetilde{f}(X_i^°)\supset \widetilde{f}(X)\cap \widetilde{Y_i}^°$.
        The other inclusion follows from the definition of $\widetilde{Y_i}$.

        The remaining assumptions on $j$ follow from the following observation: 
        The canonical deformation retraction of $\M(f)$ to $Y$ restricts to a deformation retraction from $\widetilde{Y_1}$ to $Y_1$, $\widetilde{Y_2}$ to $Y_2$ and $\widetilde{Y_1}\cap\widetilde{Y_2}$ to $Y_1\cap Y_2$.
        Hence, for all $i\in\set{1,2}$ the indicated maps in the commutative squares
        \begin{center} 
            \begin{tikzcd} [sep = 4em]
                \widetilde{Y_i} \arrow [r, "j|_{\widetilde{Y_i}}", "\sim"'] \arrow [d, hook] & Y_i \arrow [d, hook]\\
                \widetilde{Y} \arrow [r, "j", "\sim"'] & Y
            \end{tikzcd}
            \begin{tikzcd} [sep = 4em]
                \widetilde{Y_1}\cap\widetilde{Y_2} \arrow [r, "j|_{\widetilde{Y_1}\cap\widetilde{Y_2}}", "\sim"'] \arrow [d, hook] & Y_1\cap Y_2 \arrow [d, hook]\\
                \widetilde{Y_i} \arrow [r, "j|_{\widetilde{Y_i}}", "\sim"'] & Y_i
            \end{tikzcd}
        \end{center} 
        are weak equivalences (in fact even homotopy equivalences).
        Therefore, the proposition follows by \cref{lem:weOfaPairsIsHtpyPb}.
    \end{proof}
\end{lemma}
\begin{prop}\label{prop:descentForPairs}
    Let $(X;X_1,X_2)$ and $(Y;Y_1,Y_2)$ be excisive triads and let $f\colon (X;X_1,X_2)\to (Y;Y_1,Y_2)$ be a map of triads.
    Then, if the maps $f|_{X_i}\colon(X_i,X_1\cap X_2)\to(Y_i,Y_1\cap Y_2)$ are weak equivalences of pairs for $i\in\set{1,2}$, the maps $f_i\colon(X,X_i)\to(Y,Y_i)$ are weak equivalences of pairs for $i\in\set{1,2}$ as well.
    \begin{proof}
        The proof will follow \cite[Theorem 1.2]{may1990weak}.

        We can assume $f$ to be an inclusion such that $f(X_i^°)=f(X)\cap Y_i^°$ and $f(X_i)=f(X)\cap Y_i$ for $i\in\set{1,2}$ by applying \cref{lem:replaceByEmbedding} and replacing it with $\widetilde{f}$.
        The assumptions on the map $g$ from \cref{lem:replaceByEmbedding} together with \cref{lem:weOfaPairsIsHtpyPb} and the pasting law for homotopy pullbacks show that proving the statement of this proposition for $\widetilde{f}$ also proves it for $f$.
        
        By the definition of a weak equivalence of pairs, it suffices to show that under the given assumptions the statement of \cref{prop:eqCharInjSurj} \ref{itm:liftingMapsAndHtpyConst} holds for all $n\geq 0$.
    
        So fix an $i\in\set{1,2}$ and let $g\colon(J^n,\set{1}\times\partial I^n)\to(X,X_i)$ and $e\colon(I^{n+1},\set{1}\times I^n)\to(Y,Y_i)$ be given such that $e|_{J^n}=f_ig$.
        As $J^n\subset I^{n+1}$ are both compact metric spaces, we can find a Lebesgue number $\delta$ for the cover $Y_1^°,Y_2^°\subset Y$ and the map $e$.
        We pick a cubical subdivision $\mathcal{C}$ of $I^{n+1}$ such that the diameter of every cube is smaller than $\delta$.
        This ensures that for every cubical cell $c\in\mathcal{C}$ there is a $j\in\set{1,2}$ such that $e(c)\subset Y_j^°$.
        Note that this cubical subdivision gives a CW-structure on $I^{n+1}$ such that $J^n\subset I^{n+1}$ is a subcomplex.

        We identify $I^{n+1}=I\times I^n$.
        The cubical subdivision $\mathcal{C}$ gives a cubical subdivision $\mathcal{D}$ of $I^n$ and a partition of $I$ into subintervalls of equal length $I_r=[v_r,v_{r+1}]\subset I$ with $0=v_0<v_1<\ldots<v_s=1$.
        For $m\in\set{0,\ldots,n}$, let $\Gamma^m$ denote the collection of cells of dimension $m$ of the CW-structure on $I^n$ induced by $\mathcal{D}$ which are not contained in $\partial I^n$, and define $\Gamma^{-1}=\emptyset$.
        
        Then for $r\in\set{0,\ldots,s-1}$ and $m\in\set{0,\ldots,n+1}$ let
        \begin{equation*}
            L_r^m=\left([0,v_r]\times I^n\right)\cup J^n\cup\bigcup\limits_{K\in\Gamma^{m-1}}\left(I_r\times K\right)\subset I^{n+1}\;.
        \end{equation*}
        We also note that $L_r^{n+1}=\left([0,v_{r+1}]\times I^n\right)\cup J^n=L_{r+1}^0$ and $L_{s-1}^{n+1}=I^{n+1}$.

        We will use a nested induction argument to construct the desired map $G$ and homotopy $H$ of \cref{prop:eqCharInjSurj} \ref{itm:liftingMapsAndHtpyConst} as extensions of $g$ and the constant homotopy $h$ respectively, where 
        \begin{itemize}
            \item given a fixed $r\in\set{0,\ldots,s-1}$ and $m\in\set{0,\ldots,n}$, the inner induction step extends them from $L_r^m$ to $L_r^{m+1}$, hence it extends them over all cells of the form $I_r\times K$ with $K\in\Gamma^m$
            \item and given a fixed $r\in\set{0,\ldots,s-2}$, the outer induction step extends them from $L_r^{n+1}$ to $L_{r+1}^{n+1}$.
        \end{itemize}
        As an example, provided $n=1$ and $s=3$, the inner and outer induction correspond to horizontal and vertical arrows respectively in the following diagram:
        \begin{center}
            \begin{tikzpicture}[scale = .7, opacity = .7]
                \newcommand\xshift{5}
                \newcommand\yshift{0}
                \newcommand\step{\size/3*0}
                \newcommand\size{3}

                \draw (0,\size+\yshift) -- (\size,\size+\yshift);
                \draw (0,0+\yshift) -- (0,\size+\yshift);
                \draw (0,0+\yshift) -- (\size,0+\yshift);
                \node[opacity = 1, align = center] at (1.5,-.5+\yshift) {$L_{\pgfmathparse{\step}\pgfmathprintnumber{\pgfmathresult}}^0=J^1$};

                \draw[->, decorate, decoration={zigzag, segment length=4, amplitude=.9,post=lineto,post length=2pt}] (\size+0.5,\size/2+\yshift) -- (\size+1.5,\size/2+\yshift);

                \draw[color = blue] (0+\xshift*1+\step,\size/3+\yshift) -- (1+\xshift*1+\step,\size/3+\yshift);
                \draw[color = blue] (0+\xshift*1+\step,\size/3*2+\yshift) -- (1+\xshift*1+\step,2*\size/3+\yshift);
                \draw (0+\xshift*1,\size+\yshift) -- (\size+\xshift*1,\size+\yshift);
                \draw (0+\xshift*1,0+\yshift) -- (0+\xshift*1,\size+\yshift);
                \draw (0+\xshift*1,0+\yshift) -- (\size+\xshift*1,0+\yshift);
                \node[opacity = 1, align = center] at (1.5+\xshift*1,-.5+\yshift) {$L_{\pgfmathparse{\step}\pgfmathprintnumber{\pgfmathresult}}^1$};

                \draw[->, decorate, decoration={zigzag, segment length=4, amplitude=.9,post=lineto,post length=2pt}] (\size+0.5+\xshift*1,\size/2+\yshift) -- (\size+1.5+\xshift*1,\size/2+\yshift);

                \draw[fill = red, color = red, opacity=0.4, draw = none] (0+\xshift*2+\step,\size/3*0+\yshift) rectangle (1+\xshift*2+\step,\size/3*1+\yshift);
                \draw[fill = red, color = red, opacity=0.4, draw = none] (0+\xshift*2+\step,\size/3*1+\yshift) rectangle (1+\xshift*2+\step,\size/3*2+\yshift);
                \draw[fill = red, color = red, opacity=0.4, draw = none] (0+\xshift*2+\step,\size/3*2+\yshift) rectangle (1+\xshift*2+\step,\size/3*3+\yshift);
                \draw[color = blue] (0+\xshift*2+\step,\size/3*1+\yshift) -- (1+\xshift*2+\step,\size/3*1+\yshift);
                \draw[color = blue] (0+\xshift*2+\step,\size/3*2+\yshift) -- (1+\xshift*2+\step,\size/3*2+\yshift);
                \draw (0+\xshift*2,\size+\yshift) -- (\size+\xshift*2,\size+\yshift);
                \draw (0+\xshift*2,0+\yshift) -- (0+\xshift*2,\size+\yshift);
                \draw (0+\xshift*2,0+\yshift) -- (\size+\xshift*2,0+\yshift);
                \node[opacity = 1, align = center] at (1.5+\xshift*2,-.5+\yshift) {$L_{\pgfmathparse{\step}\pgfmathprintnumber{\pgfmathresult}}^2=L_1^0$};

                \draw[->, decorate, decoration={zigzag, segment length=4, amplitude=.9,post=lineto,post length=2pt}] (3*\xshift-\size/2,\size/2+\yshift) to[bend left] (3*\xshift-\size/2,\size/2-4.25+0.2);
                \renewcommand\yshift{-4.25}
                \renewcommand\step{\size/3*1}

                \draw[fill = black, color = black, opacity=0.4, draw = none] (0+\xshift*0,0+\yshift) rectangle (\step+\xshift*0,\size+\yshift);
                \draw (0,\size+\yshift) -- (\size,\size+\yshift);
                \draw (0,0+\yshift) -- (0,\size+\yshift);
                \draw (0,0+\yshift) -- (\size,0+\yshift);
                \node[opacity = 1, align = center] at (1.5,-.5+\yshift) {$L_{\pgfmathparse{\step}\pgfmathprintnumber{\pgfmathresult}}^0$};

                \draw[->, decorate, decoration={zigzag, segment length=4, amplitude=.9,post=lineto,post length=2pt}] (\size+0.5,\size/2+\yshift) -- (\size+1.5,\size/2+\yshift);

                \draw[fill = black, color = black, opacity=0.4, draw = none] (0+\xshift*1,0+\yshift) rectangle (\step+\xshift*1,\size+\yshift);
                \draw[color = blue] (0+\xshift*1+\step,\size/3+\yshift) -- (1+\xshift*1+\step,\size/3+\yshift);
                \draw[color = blue] (0+\xshift*1+\step,\size/3*2+\yshift) -- (1+\xshift*1+\step,2*\size/3+\yshift);
                \draw (0+\xshift*1,\size+\yshift) -- (\size+\xshift*1,\size+\yshift);
                \draw (0+\xshift*1,0+\yshift) -- (0+\xshift*1,\size+\yshift);
                \draw (0+\xshift*1,0+\yshift) -- (\size+\xshift*1,0+\yshift);
                \node[opacity = 1, align = center] at (1.5+\xshift*1,-.5+\yshift) {$L_{\pgfmathparse{\step}\pgfmathprintnumber{\pgfmathresult}}^1$};

                \draw[->, decorate, decoration={zigzag, segment length=4, amplitude=.9,post=lineto,post length=2pt}] (\size+0.5+\xshift*1,\size/2+\yshift) -- (\size+1.5+\xshift*1,\size/2+\yshift);

                \draw[fill = black, color = black, opacity=0.4, draw = none] (0+\xshift*2,0+\yshift) rectangle (\step+\xshift*2,\size+\yshift);
                \draw[fill = red, color = red, opacity=0.4, draw = none] (0+\xshift*2+\step,\size/3*0+\yshift) rectangle (1+\xshift*2+\step,\size/3*1+\yshift);
                \draw[fill = red, color = red, opacity=0.4, draw = none] (0+\xshift*2+\step,\size/3*1+\yshift) rectangle (1+\xshift*2+\step,\size/3*2+\yshift);
                \draw[fill = red, color = red, opacity=0.4, draw = none] (0+\xshift*2+\step,\size/3*2+\yshift) rectangle (1+\xshift*2+\step,\size/3*3+\yshift);
                \draw[color = blue] (0+\xshift*2+\step,\size/3*1+\yshift) -- (1+\xshift*2+\step,\size/3*1+\yshift);
                \draw[color = blue] (0+\xshift*2+\step,\size/3*2+\yshift) -- (1+\xshift*2+\step,\size/3*2+\yshift);
                \draw (0+\xshift*2,\size+\yshift) -- (\size+\xshift*2,\size+\yshift);
                \draw (0+\xshift*2,0+\yshift) -- (0+\xshift*2,\size+\yshift);
                \draw (0+\xshift*2,0+\yshift) -- (\size+\xshift*2,0+\yshift);
                \node[opacity = 1, align = center] at (1.5+\xshift*2,-.5+\yshift) {$L_{\pgfmathparse{\step}\pgfmathprintnumber{\pgfmathresult}}^2=L_2^0$};

                \draw[->, decorate, decoration={zigzag, segment length=4, amplitude=.9,post=lineto,post length=2pt}] (3*\xshift-\size/2,\size/2+\yshift) to[bend left] (3*\xshift-\size/2,\size/2-8.5-0.2);
                \renewcommand\yshift{-4.25*2}
                \renewcommand\step{\size/3*2}

                \draw[fill = black, color = black, opacity=0.4, draw = none] (0+\xshift*0,0+\yshift) rectangle (\step+\xshift*0,\size+\yshift);
                \draw (0,\size+\yshift) -- (\size,\size+\yshift);
                \draw (0,0+\yshift) -- (0,\size+\yshift);
                \draw (0,0+\yshift) -- (\size,0+\yshift);
                \node[opacity = 1, align = center] at (1.5,-.5+\yshift) {$L_{\pgfmathparse{\step}\pgfmathprintnumber{\pgfmathresult}}^0$};

                \draw[->, decorate, decoration={zigzag, segment length=4, amplitude=.9,post=lineto,post length=2pt}] (\size+0.5,\size/2+\yshift) -- (\size+1.5,\size/2+\yshift);

                \draw[fill = black, color = black, opacity=0.4, draw = none] (0+\xshift*1,0+\yshift) rectangle (\step+\xshift*1,\size+\yshift);
                \draw[color = blue] (0+\xshift*1+\step,\size/3+\yshift) -- (1+\xshift*1+\step,\size/3+\yshift);
                \draw[color = blue] (0+\xshift*1+\step,\size/3*2+\yshift) -- (1+\xshift*1+\step,2*\size/3+\yshift);
                \draw (0+\xshift*1,\size+\yshift) -- (\size+\xshift*1,\size+\yshift);
                \draw (0+\xshift*1,0+\yshift) -- (0+\xshift*1,\size+\yshift);
                \draw (0+\xshift*1,0+\yshift) -- (\size+\xshift*1,0+\yshift);
                \node[opacity = 1, align = center] at (1.5+\xshift*1,-.5+\yshift) {$L_{\pgfmathparse{\step}\pgfmathprintnumber{\pgfmathresult}}^1$};

                \draw[->, decorate, decoration={zigzag, segment length=4, amplitude=.9,post=lineto,post length=2pt}] (\size+0.5+\xshift*1,\size/2+\yshift) -- (\size+1.5+\xshift*1,\size/2+\yshift);

                \draw[fill = black, color = black, opacity=0.4, draw = none] (0+\xshift*2,0+\yshift) rectangle (\step+\xshift*2,\size+\yshift);
                \draw[fill = red, color = red, opacity=0.4, draw = none] (0+\xshift*2+\step,\size/3*0+\yshift) rectangle (1+\xshift*2+\step,\size/3*1+\yshift);
                \draw[fill = red, color = red, opacity=0.4, draw = none] (0+\xshift*2+\step,\size/3*1+\yshift) rectangle (1+\xshift*2+\step,\size/3*2+\yshift);
                \draw[fill = red, color = red, opacity=0.4, draw = none] (0+\xshift*2+\step,\size/3*2+\yshift) rectangle (1+\xshift*2+\step,\size/3*3+\yshift);
                \draw[color = blue] (0+\xshift*2+\step,\size/3*1+\yshift) -- (1+\xshift*2+\step,\size/3*1+\yshift);
                \draw[color = blue] (0+\xshift*2+\step,\size/3*2+\yshift) -- (1+\xshift*2+\step,\size/3*2+\yshift);
                \draw (0+\xshift*2,\size+\yshift) -- (\size+\xshift*2,\size+\yshift);
                \draw (0+\xshift*2,0+\yshift) -- (0+\xshift*2,\size+\yshift);
                \draw (0+\xshift*2,0+\yshift) -- (\size+\xshift*2,0+\yshift);
                \node[opacity = 1, align = center] at (1.5+\xshift*2,-.5+\yshift) {$L_{\pgfmathparse{\step}\pgfmathprintnumber{\pgfmathresult}}^2=I^2$};
            \end{tikzpicture}
        \end{center}
        The desired maps $G$ and $H$ will be constructed such that for all cells $K\in\Gamma^m$ and $r\in\set{0,\ldots,s-1}$
        \begin{itemize}
            \item if $e(I_r\times K)\subset Y_j^°$ then $G(I_r\times K)\subset X_j$ and $H(I_r\times K\times I)\subset Y_j$ for the same $j\in\set{1,2}$
            \item if $e(\set{v_{r+1}}\times K)\subset Y_1\cap Y_2$ then $G(\set{v_{r+1}}\times K)\subset X_1\cap X_2$ and $H(\set{v_{r+1}}\times K\times I)\subset Y_1\cap Y_2$.
        \end{itemize}
        We will refer to these conditions as the \emph{patching conditions}.
        Note that for a cubical cell $c\in\mathcal{C}$ that is contained in $J^n$
        \begin{itemize}
            \item if $e(c)\subset Y_j^°$ then $e(c)=fg(c)\subset f(X)\cap Y_j^°=f(X_j^°)$ and thus $g(c)\subset X_j^°\subset X_j$ since $f$ is an inclusion
            \item if $e(c)\subset Y_1\cap Y_2$ then $e(c)=fg(c)\subset f(X)\cap Y_1\cap Y_2=f(X_1\cap X_2)$ and $g(c)\subset X_1\cap X_2$ since $f$ is an inclusion
        \end{itemize}
        and thus the patching conditions are in particular fulfilled for cells of the shape $I_r\times d$ where $d$ is a cell in $\partial I^n$.
        
        So for the inner induction step with given $r\in\set{0,\ldots,s-1}$ and $m\in\set{0,\ldots,n}$, by the induction hypothesis we have maps $G\colon L_r^m\to X$ and $H\colon L_r^m\times I\to Y$ fulfilling the patching conditions and extending $g$ and the constant homotopy at $f_ig=e|_{J^n}$ respectively, such that $H$ is a homotopy from $e|_{L_r^m}$ to $f_iG$.

        We now pick a $K\in\Gamma^m$ and proceed via a case distinction:
        \begin{itemize}
            \item First, if $e(\set{v_{r+1}}\times K)\subset Y_1\cap Y_2$ and $e(I_r\times K)\subset Y_j^°$ is contained in exactly one of the $Y_j^°$, we have that $G$ and $H$ restrict to maps $(I_r\times\partial K\cup\set{v_r}\times K, \set{v_{r+1}}\times\partial K)\to(X_j,X_1\cap X_2)$ and $(I_r\times\partial K\cup\set{v_r}\times K, \set{v_{r+1}}\times\partial K)\times I\to(X_j,X_1\cap X_2)$ respectively by the patching conditions, which can be identified with maps $g'\colon(J^m, \set{1}\times\partial{I^m})\to(X_j,X_1\cap X_2)$ and $h'\colon (J^m, \set{1}\times\partial{I^m})\times I\to(X_j,X_1\cap X_2)$.
        
                    Together with $e|_{I_r\times K}\cong e'\colon(I^{m+1},\set{1}\times I^m)\to(Y_j,Y_1\cap Y_2)$ and the weak equivalence of pairs $f|_{X_j}\colon(X_j,X_1\cap X_2)\to(Y_j,Y_1\cap Y_2)$ these fulfill the conditions of \cref{prop:eqCharInjSurj} \ref{itm:liftingMapsAndHtpy}. 
                    So by \cref{prop:eqCharInjSurj} there exist extensions which we can then identify with maps $G'_K\colon(I_r\times K,\set{v_{r+1}}\times K)\to(X_j,X_1\cap X_2)$ and $H'_K\colon(I_r\times K,\set{v_{r+1}}\times K)\times I\to(Y_j,Y_1\cap Y_2)$.
                    Note that $G'_K$ and $H'_K$ fulfill the patching conditions.
            \item In the remaining cases either $e(\set{v_{r+1}}\times K)\not\subset Y_1\cap Y_2$ or $e(I_r\times K)\subset Y_1^°\cap Y_2^°$. %TODO maybe be more explicit with the cases here
                    By \cref{rmk:cwInclIsHCofib} we can pick a deformation retract $F\colon (I_r\times K)\times I\to I_r\times K$ of $J_K^m= I_r\times\partial K\cup\set{v_r}\times K$ (hence $F_1\colon I_r\times K\to J_K^m$) and we set $G'_K=G|_{J_K^m}\circ F_1$ and $H'_K=(e|_{I_r\times K}\bullet F)*(H|_{J_K^m}\bullet F_1)$.
                    Note that $G'_K$ and $H'_K$ fulfill the patching conditions since $G$ and $H$ fulfill them by induction hypothesis.
        \end{itemize}
        Now note that since we work ourselves upwards from the dimensions, the constructed extensions $G'_K$ and $H'_K$ are always compatible along boundaries of $I_r\times K$, and thus we can glue them together to obtain extensions to $L_r^m\cup\bigcup\limits_{K\in\Gamma^m}I^r\times K=L_r^{m+1}$.
        Hence we get extensions of $G\colon L_r^m\to X$ and $H\colon L_r^m\times I\to Y$ to $L_r^{m+1}$ and $L_r^{m+1}\times I$ respectively (the latter also uses that $L_r^{m+1}\times I\cong\left(L_r^m\times I\right)\cup\bigcup\limits_{K\in\Gamma^m}\left(I_r\times K\times I\right)$).
        These extensions fulfill the patching conditions since $G'_K$ and $H'_K$ fulfill them by construction.

        It only remains to check that the final extensions $G$ and $H$ are maps of pairs $(I^{n+1},\set{1}\times I^n)\to(X,X_i)$ and $(I^{n+1},\set{1}\times I^n)\times I\to(Y,Y_i)$ respectively.
        Let $K\in\Gamma^n$ and note that $\bigcup\limits_{K\in\Gamma^n}\set{1}\times K=\set{1}\times I^n$.
        Since $e(\set{1}\times I^n)\subset Y_i$, we always have $e(I_{s-1}\times K)\subset Y_i^°$ or $e(\set{1}\times K)\subset Y_1\cap Y_2$ and both cases imply $G(\set{1}\times K)\subset X_i$ and $H(\set{1}\times K\times I)\subset Y_i$ by the patching conditions.
        Thus we have proven the proposition.
    \end{proof}
\end{prop}
This concludes the technical preliminaries. 
We are now ready to give a proof of descent for pushouts.
\begin{prop}\label{prop:topDescentPo}
    The model category $\Top$ has descent for homotopy pushouts. 
    \begin{proof}
        We start with a cube
        \begin{center}
            \begin{tikzcd} [sep = .5 cm]
                \overline{A} \arrow [dr] \arrow [rr] \arrow [dd] & & \overline{B} \arrow [dr] \arrow[dd] \\
                & \overline{C} \arrow [rr, crossing over] & & \overline{D} \arrow [dd] & \\
                A \arrow [dr] \arrow [rr] & & B \arrow [dr] \\
                & C \arrow [from=uu,crossing over] \arrow [rr] & & D
            \end{tikzcd}
        \end{center}
        where bottom and top square homotopy pushouts and left and back face are homotopy pullbacks.
        The pushouts of open/closed mapping cylinders of \cref{rmk:doubleMappingCylinderAlt} induce a cube
        \begin{center}
            \begin{tikzcd} [sep = .5 cm]
                \overline{A}\times\left(0,2\right) \arrow [dr, hook] \arrow [rr, hook] \arrow [dd] & & \M^{[0,2)}(f_{\overline{A}\overline{B}}) \arrow [dr, hook] \arrow[dd] \\
                & \M^{[0,2)}(f_{\overline{A}\overline{C}}) \arrow [rr, crossing over, hook] & & \D(f_{\overline{A}\overline{B}},f_{\overline{A}\overline{C}}) \arrow [dd, "f"] & \\
                A\times(0,2) \arrow [dr, hook] \arrow [rr, hook] & & \M^{[0,2)}(f_{AB}) \arrow [dr, hook] \\
                & \M^{[0,2)}(f_{AC}) \arrow [from=uu,crossing over] \arrow [rr, hook] & & \D(f_{AB},f_{AC})
            \end{tikzcd}
        \end{center}
        where $f\colon\left(\D(f_{\overline{A}\overline{B}},f_{\overline{A}\overline{C}});\M^{[0,2)}(f_{\overline{A}\overline{B}}),\M^{[0,2)}(f_{\overline{A}\overline{C}})\right)\to\left(\D(f_{AB},f_{AC});\M^{[0,2)}(f_{AB}),\M^{[0,2)}(f_{AC})\right)$ is a map of excisive triads.
        We claim all faces of this cube are equivalent to the starting cube:
        
        The equivalence of the top square follows from the fact that the map $\D(f_{\overline{A}\overline{B}},f_{\overline{A}\overline{C}})\to\overline{D}$ in the commutative diagram (see again \cref{rmk:doubleMappingCylinderAlt} for square in the middle)
        \begin{center}
            \begin{tikzcd} [sep = .5 cm]
                \overline{A} \arrow [dr] \arrow [rr] \arrow [from=dd, "\sim"] & & \overline{B} \arrow [dr] \arrow [from=dd, near end, "\sim"] \\
                & \overline{C} \arrow [rr, crossing over] & & \overline{D} \arrow [from=dd] & \\
                \overline{A}\times\set{1} \arrow [>->, dr, "h"] \arrow [>->, rr, near start, "h"] \arrow [dd, hook, "\sim"] & & \M(f_{\overline{A}\overline{B}}) \arrow [>->, dr, "h"] \arrow [dd, hook, near end, "\sim"] \\
                & \M(f_{\overline{A}\overline{C}}) \arrow [>->, rr, crossing over, near start, "h"] \arrow [to=uu, crossing over, near end, "\sim"] & & \D(f_{\overline{A}\overline{B}},f_{\overline{A}\overline{C}}) \arrow [equals, dd, "\sim"] \\
                \overline{A}\times\left(0,2\right) \arrow [dr, hook] \arrow [rr, hook] & & \M^{[0,2)}(f_{\overline{A}\overline{B}}) \arrow [dr, hook] \\
                & \M^{[0,2)}(f_{\overline{A}\overline{C}}) \arrow [from=uu, hook, crossing over, near end, "\sim"] \arrow [rr, hook] & & \D(f_{\overline{A}\overline{B}},f_{\overline{A}\overline{C}}) \\
            \end{tikzcd}
        \end{center}
        is a weak equivalence since pushouts along h-cofibrations are homotopy pushouts by \cref{prop:poAlongHCofibIsHtpyPo}. 
        The analogous argument shows equivalence of the bottom squares.

        The equivalences of left squares and front squares follow from
        \begin{center}
            \begin{tikzcd} [sep = .2 cm]
                \overline{A}\times\left(0,2\right) \arrow [dr, hook] \arrow [rr, "\sim"] \arrow [dd] & & \overline{A} \arrow [dr] \arrow[dd] \\
                & \M^{[0,2)}(f_{\overline{A}\overline{C}}) \arrow [rr, crossing over, "\sim"] & &[3.25em] \overline{C} \arrow [dd] & \\
                A\times(0,2) \arrow [dr, hook] \arrow [rr, near start, "\sim"] & & A \arrow [dr] \\
                & \M^{[0,2)}(f_{AC}) \arrow [from=uu,crossing over] \arrow [rr, "\sim"] & & C
            \end{tikzcd}
        \end{center}
        \begin{center}
            and
        \end{center}
        \begin{center}
            \begin{tikzcd} [sep = .4 cm]
                \M^{[0,2)}(f_{\overline{A}\overline{C}}) \arrow [dr, "\sim"] \arrow [rr, hook] \arrow [dd] & & \D(f_{\overline{A}\overline{B}},f_{\overline{A}\overline{C}}) \arrow [dr, "\sim"] \arrow[dd, near start, "f"] \\
                & \overline{C} \arrow [rr, crossing over] & &\overline{D} \arrow [dd] & \\
                \M^{[0,2)}(f_{AC}) \arrow [dr, "\sim"] \arrow [rr, hook] & & \D(f_{AB},f_{AC}) \arrow [dr, "\sim"] \\
                & C \arrow [from=uu,crossing over] \arrow [rr] & & D
            \end{tikzcd}
        \end{center}
        respectively. 
        The equivalences of the back squares and front squares follow from the analogous argument.

        Thus by \cref{lem:weOfaPairsIsHtpyPb} the map of triads $f\colon\left(\D(f_{\overline{A}\overline{B}},f_{\overline{A}\overline{C}});\M^{[0,2)}(f_{\overline{A}\overline{B}}),\M^{[0,2)}(f_{\overline{A}\overline{C}})\right)\to\left(\D(f_{AB},f_{AC});\M^{[0,2)}(f_{AB}),\M^{[0,2)}(f_{AC})\right)$ fulfills the assumptions of \cref{prop:descentForPairs} and again by \cref{lem:weOfaPairsIsHtpyPb} we conclude that front and right square are homotopy pullback squares.
    \end{proof}
\end{prop}
\addcontentsline{toc}{subsection}{Descent and Quasifibrations}
\subsection*{Descent and Quasifibrations}
We will now give an alternative proof of descent for homotopy pushouts using the notion of quasifibrations.
\begin{definition}[Quasifibration]
    Let $f\colon X\to Y$ a map in $\Top$.
    Then we call $f$ a \emph{quasifibration} if for every map $y\in Y$ the ordinary pullback square
    \begin{center}
        \begin{tikzcd} [sep = 4em]
            F \arrow[r] \arrow[d] & X \arrow[d, "f"] \\
            * \arrow[r, "y"] & Y\\
        \end{tikzcd}
    \end{center}
    is a homotopy pullback (hence the ordinary fiber is a homotopy fiber).
\end{definition}
\begin{remark}\label{rmk:parallelQuasifibsInPb}
    Let 
    \begin{center}
        \begin{tikzcd} [sep = 4em]
            A \arrow[r] \arrow[d, "q"'] & X \arrow[d, "p"] \\
            B \arrow[r, "f"] & Y\\
        \end{tikzcd}
    \end{center}
    be an ordinary pullback square in $\Top$ with $p$ and $q$ quasifibrations.
    Then this square is a homotopy pullback:
    
    By the pasting law for ordinary pullbacks, the ordinary fiber of $q$ at $b\in B$ is also the ordinary fiber of $p$ at $f(b)\in Y$.
    But since these are ordinary fibers are already homotopy fibers, this means that the homotopy fibers are in particular weakly equivalent.
    By the fiberwise characterization of homotopy pullbacks \cref{prop:fiberwiseCharOfHtpyPb} this implies that the given square is a homotopy pullback.

    However, quasifibrations are not generally stable under pullbacks (see e.g. \cite[Example 2.7.6]{cubical_htpy_theory}).
\end{remark}
The following weak version of the locality of quasifibrations (\cref{cor:locOfQuasifib}) is sufficient to prove descent for pushouts.
\begin{corollary}[Weak Locality of Quasifibrations]\label{cor:locOfQuasifib}
    Let $f\colon X\to Y$ be a map and $U_1, U_2\subset Y$ be an open cover such that $f^{-1}(U_1)\to U_1$, $f^{-1}(U_2)\to U_2$ and $f^{-1}(U_1)\cap f^{-1}(U_2)=f^{-1}(U_1\cap U_2)\to U_1\cap U_2$ are quasifibrations.
    Then $f$ is a quasifibration.
    \begin{proof}
        For $i\in\set{1,2}$ the squares 
        \begin{center}
            \begin{tikzcd} [sep = 4em]
                f^{-1}(U_1)\cap f^{-1}(U_2) \arrow [r] \arrow [d, hook] & U_1\cap U_2 \arrow [d, hook]\\
                f^{-1}(U_i) \arrow [r] & U_i
            \end{tikzcd}
            \begin{tikzcd} [sep = 4em]
                f^{-1}(U_i) \arrow [r] \arrow [d, hook] & U_i \arrow [d, hook]\\
                X \arrow [r] & Y
            \end{tikzcd}
        \end{center} 
        are ordinary pullback squares since they are restrictions of maps along inclusions.
        Using \cref{rmk:parallelQuasifibsInPb} the given assumptions imply that the first square is a homotopy pullback since top and bottom map are quasifibrations.
        Thus $f$ is a map of excisive triads $(X;f^{-1}(U_1),f^{-1}(U_2))\to(Y;U_1,U_2)$ such that $f|_{f^{-1}(U_i)}\colon(f^{-1}(U_i),f^{-1}(U_1)\cap f^{-1}(U_2))\to (U_i,U_1\cap U_2)$ are weak equivalences for $i\in\set{1,2}$ by \cref{lem:weOfaPairsIsHtpyPb}.

        Now \cref{prop:descentForPairs} implies that the second square is also homotopy pullback.
        The proposition now follows since every point $y\in Y$ lies in some $U_i$, hence the ordinary fiber of $f\colon X\to Y$ is the homotopy fiber by the pasting law for ordinary pullbacks.
    \end{proof}
\end{corollary}
The relationship of locality of quasifibrations and descent also goes the other way as the next remark shows.
\begin{remark}\label{rmk:locOfQuasifibDescent}
    Once we have established descent in $\Top$ in full generality (\cref{cor:spacesIsInftyTop}) we obtain a stronger version of \cref{cor:locOfQuasifib} which is called ``Locality of Quasifibrations'':
    
    Let $f\colon X\to Y$ be a map and $(U_i)_{i\in I}$ an open cover of $Y$ such that for every pair of indices $i,j\in I$ and point $b\in U_i\cap U_j$, there is a $k\in I$ such that $b\in U_k\subset U_i\cap U_j$.
    Then if $f^{-1}(U_i)\to U_i$ is a quasifibration for all $i\in I$, $f$ is a quasifibration.

    One can prove this by using that the homotopy colimit of such a cover is the space itself (see \cite[Proposition 4.6 (c)]{hypercovers}):
    Because all squares
    \begin{center}
        \begin{tikzcd} [sep = 4em]
            f^{-1}(U_j) \arrow [r, hook] \arrow [d, "f|_{f^{-1}(U_j)}"'] & f^{-1}(U_i) \arrow [d, "f|_{f^{-1}(U_i)}"]\\
            U_j\arrow [r, hook] & U_i
        \end{tikzcd}
    \end{center}
    are homotopy pullbacks by \cref{rmk:parallelQuasifibsInPb} and $X\simeq\hcolim\limits_{i\in I}(f^{-1}(U_i))$ as well as $Y\simeq\hcolim\limits_{i\in I}(U_i)$, the ordinary pullback squares 
    \begin{center}
        \begin{tikzcd} [sep = 4em]
            f^{-1}(U_i) \arrow [r, hook] \arrow [d, "f|_{f^{-1}(U_i)}"'] & X \arrow [d, "f"]\\
            U_i \arrow [r, hook] & Y
        \end{tikzcd}
    \end{center}
    are homotopy pullbacks by descent.
    Since every point $y\in Y$ is contained in some $U_i$, the ordinary fibers of $f$ are homotopy fibers by the pasting law for homotopy pullbacks. 

    A direct proof of the locality of quasifibrations can be found in \cite[Theorem A.1.2]{aguilar2002algebraic}.
\end{remark}
The next two lemmas will be used to reduce to a situation where \cref{cor:locOfQuasifib} is applicable.
\begin{lemma}\label{lem:reductionStepDescent}
    Suppose that for all cubes in the model category $\Top$
    \begin{center}
        \begin{tikzcd} [sep = .5 cm]
            \overline{A} \arrow [dr] \arrow [rr] \arrow [->>,dd] & & \overline{B} \arrow [dr] \arrow[->>,dd] \\
            & \overline{C} \arrow [rr, crossing over] & & \overline{D} \arrow [dd] & \\
            A \arrow [dr] \arrow [rr] & & B \arrow [dr] \\
            & C \arrow [->>,from=uu,crossing over] \arrow [rr] & & D &
        \end{tikzcd}
    \end{center}
    where
    \begin{itemize}
        \item left and back face are homotopy pullbacks
        \item bottom and top square homotopy pushouts 
        \item the maps $\overline{A}\xtwoheadrightarrow{} A$, $\overline{B}\xtwoheadrightarrow{} B$ and $\overline{C}\xtwoheadrightarrow{} C$ are fibrations
    \end{itemize}
    we have that the the front and right squares are homotopy pullbacks.
    Then $\Top$ has descent for homotopy pushouts.
    \begin{proof}
        We will prove that every cube
        \begin{center}
            \begin{tikzcd} [sep = .5 cm]
                A^{\prime} \arrow [dr] \arrow [rr] \arrow [dd] & & B^{\prime} \arrow [dr] \arrow [dd] \\
                & C^{\prime} \arrow [rr, crossing over] \arrow [dd] & & D^{\prime} \arrow [dd] & \\
                A \arrow [dr] \arrow [rr] & & B \arrow [dr] \\
                & C \arrow [from=uu, crossing over] \arrow [rr] & & D &
            \end{tikzcd}
        \end{center}
        where the back and left square are homotopy pullbacks and top and bottom square are homotopy pushouts, can be reduced to a cube of the required form in the sense that all faces are equivalent to the corresponding faces of the starting cube.

        Starting with the diagram
        \begin{center}
            \begin{tikzcd} [sep = 4em]
                C^{\prime} \arrow[d] & A^{\prime} \arrow[l] \arrow[r] \arrow[d] & B^{\prime} \arrow[d] \\
                C  & A \arrow[r] \arrow[l] & B \\
            \end{tikzcd}
        \end{center}
        we factor the maps $A^{\prime}\to A$ and $C^{\prime}\to C$ into trivial cofibrations followed by fibrations to obtain
        \begin{center}
            \begin{tikzcd} [sep = 4em]
                C^{\prime} \arrow[>->,d,"\sim"] & A^{\prime} \arrow[l] \arrow[r] \arrow[>->,d,"\sim"] & B^{\prime} \arrow[dd] \\
                \overline{C} \arrow[->>,d] & \overline{A} \arrow[->>,d] \arrow[l,dashed] & \\
                C  & A \arrow[r] \arrow[l] & B \\
            \end{tikzcd}
        \end{center}
        where the dashed arrow exists by the lifting properties.

        Next we factor the maps $A^{\prime}\to B^{\prime}$ and $A\to B$ into cofibrations followed by trivial fibrations and obtain 
        \begin{center}
            \begin{tikzcd} [sep = 4em]
                C^{\prime} \arrow[>->,d,"\sim"] & A^{\prime} \arrow[l] \arrow[>->,r] \arrow[>->,d,"\sim"] & \widehat{B^{\prime}} \arrow[dd,dashed] \arrow[->>,r,"\sim"] & B^{\prime} \arrow[dd] \\
                \overline{C} \arrow[->>,d] & \overline{A} \arrow[->>,d] \arrow[l] & \\
                C  & A \arrow[r] \arrow[l] \arrow[>->,r] & \widehat{B} \arrow[->>,r,"\sim"] & B \\
            \end{tikzcd}
        \end{center}
        where the dashed arrow exists again by the lifting properties.

        Lastly, we factor $\widehat{B^{\prime}}\to\widehat{B}$ into a trivial cofibration followed by a fibration and get
        \begin{center}
            \begin{tikzcd} [sep = 4em]
                C^{\prime} \arrow[>->,d,"\sim"] & A^{\prime} \arrow[l] \arrow[>->,r] \arrow[>->,d,"\sim"] & \widehat{B^{\prime}} \arrow[>->,d,"\sim"] \arrow[->>,r,"\sim"] & B^{\prime} \arrow[dd] \\
                \overline{C} \arrow[->>,d] & \overline{A} \arrow[->>,d] \arrow[l] \arrow[r, dashed] & \overline{B} \arrow[->>,d]\\
                C  & A \arrow[r] \arrow[l] \arrow[>->,r] & \widehat{B} \arrow[->>,r,"\sim"] & B \\
            \end{tikzcd}
        \end{center} 
        where yet again the dashed arrow exists by the lifting properties.
        Thus we have obtained a diagram
        \begin{center}
            \begin{tikzcd} [sep = 4em]
                \overline{C} \arrow[->>,d] & \overline{A} \arrow[l] \arrow[>->,r] \arrow[->>,d] & \overline{B} \arrow[->>,d] \\
                C & A \arrow[l] \arrow[>->,r] & \widehat{B} \\
            \end{tikzcd}
        \end{center}
        and we can form the cube 
        \begin{center}
            \begin{tikzcd} [sep = .5 cm]
                \overline{A} \arrow[dr] \arrow [>->,rr] \arrow [->>,dd] & & \overline{B} \arrow [dr] \arrow[->>,dd]\\
                & \overline{C} \arrow [>->,rr, crossing over] & &[-1em] \overline{C}\cup_{\overline{A}}\overline{B} \arrow [dd] & \\
                A \arrow[dr] \arrow [>->,rr] & & \widehat{B} \arrow [dr] \\
                & C \arrow[->>, from=uu, crossing over] \arrow [>->,rr] & &[-1em] C\cup_A\widehat{B} &
            \end{tikzcd}
        \end{center}
        by taking pushouts.
        Since these are pushouts along cofibrations, they are already homotopy pushouts since $\Top$ is left proper.
        Following the construction of this cube, we see that all faces are equivalent to their corresponding faces of the starting cube which proves the lemma.
    \end{proof}
\end{lemma}
\begin{lemma}\label{lem:mapOfCylIsQuasiFib}
    Let 
    \begin{center}
        \begin{tikzcd} [sep = 4em]
            X \arrow[d, "p"'] \arrow[r, "f"] & Y \arrow[d, "q"] \\
            A \arrow[r, "g"] & B \\
        \end{tikzcd}
    \end{center}
    be a homotopy pullback such that $p$ and $q$ are quasifibrations.
    Then the induced map $\Gamma\colon\M(f)\to\M(g)$ between mapping cylinders is also a quasifibration.
    The same holds true for closed and open mapping cylinders.
    \begin{proof}
        For convenience we will only prove the case $I=[0,1]$ since the other cases follow by the analogous argument.
        The proof is essentially \cite[Lemma 5.10.6]{cubical_htpy_theory}.

        Let $b_0\in\M(g)$. 
        Then either $b_0=(a_0,t_0)\in A\times(0,1]\subset\M(g)$ or $b_0\in B\subset\M(g)$.
        In the first case, we have a diagram
        \begin{center}
            \begin{tikzcd} [sep = 4em]
                X\times\set{t_0} \arrow[r, hook] \arrow[d, "p"'] & \M(f) \arrow[r, "\sim"] \arrow[d, "\Gamma"] & Y \arrow[d, "q"] \\
                A\times\set{t_0} \arrow[r, hook] & \M(g) \arrow[r, "\sim"] & B \\
            \end{tikzcd}
        \end{center}
        where the outer square is the starting square and in particular a homotopy pullback, so by the pasting law the left square is a homotopy pullback as well.
        The left square is also an ordinary pullback since $X\times\set{t_0}=\Gamma^{-1}(A\times\set{t_0})$.
        We claim that the ordinary fiber $F$ of $p$ at the point $(a_0,t_0)$ is a homotopy fiber of $\Gamma$:
        
        By the pasting law for ordinary pullbacks, the composite square
        \begin{center}
            \begin{tikzcd} [sep = 4em]
                F \arrow[r] \arrow[d] & X\times\set{t_0} \arrow[r, hook] \arrow[d, "p"] & \M(f) \arrow[d, "\Gamma"] \\
                * \arrow[r, "{(a_0,t_0)}"] & A\times\set{t_0} \arrow[r, hook] & \M(g) \\
            \end{tikzcd}
        \end{center}
        is a pullback. 
        The left square is a homotopy pullback since $p$ is a quasifibration and we have already proven that the right square is a homotopy pullback, hence by the pasting law for homotopy pullbacks the composite square is a homotopy pullback. 
        This proves that the fiber $F$ is a homotopy fiber.

        In the second case we have the diagram
        \begin{center}
            \begin{tikzcd} [sep = 4em]
                F \arrow[r] \arrow[d] & Y \arrow[r, hook, "\sim"] \arrow[d, "q"] & \M(f) \arrow[d, "\Gamma"] \\
                * \arrow[r, "b_0"] & B \arrow[r, hook, "\sim"] & \M(g) \\
            \end{tikzcd}
        \end{center}
        where $F$ again is the ordinary fiber.
        Since the right square is pullback (since $Y=\Gamma^{-1}(B)$) and homotopy pullback and the left side is as well (since $q$ is a quasifibration), this means that the fiber $F$ is already a homotopy fiber which proves the lemma.
    \end{proof}
\end{lemma}
This finishes the preparation and we will now prove descent for pushouts via locality of quasifibrations.
\begin{prop}\label{lem:topDescentPoAlt}
    The model category $\Top$ has descent for homotopy pushouts. 
    \begin{proof}
        By \cref{lem:reductionStepDescent} it is sufficient to prove that for a given cube
        \begin{center}
            \begin{tikzcd} [sep = .5 cm]
                \overline{A} \arrow [dr] \arrow [rr] \arrow [->>,dd] & & \overline{B} \arrow [dr] \arrow[->>]{dd}[near start]{} \\
                & \overline{C} \arrow [rr, crossing over] & & \overline{D} \arrow [dd] & \\
                A \arrow [dr] \arrow [rr] & & B \arrow [dr] \\
                & C \arrow [->>,from=uu,crossing over]{}[near start]{} \arrow [rr] & & D &
            \end{tikzcd}
        \end{center}
        where 
        \begin{itemize}
            \item left and back face are homotopy pullbacks
            \item bottom and top square are homotopy pushouts
            \item the maps $\overline{A}\xtwoheadrightarrow{} A$, $\overline{B}\xtwoheadrightarrow{} B$ and $\overline{C}\xtwoheadrightarrow{} C$ are fibrations
        \end{itemize}
        we have that the front and right squares are also homotopy pullbacks.

        By \cref{lem:mapOfCylIsQuasiFib} we know that the induced maps $\M(f_{\overline{A}\overline{C}})\to\M(f_{AC})$ and $\M(f_{\overline{A}\overline{B}})\to\M(f_{AB})$ are quasifibrations.
        The pushout of closed mapping cylinders in \cref{rmk:doubleMappingCylinderAlt} induces a cube
        \begin{center}
            \begin{tikzcd} [sep = .75 cm]
                \overline{A}\times\set{1} \arrow [>->, dr, "h"] \arrow [>->, rr, "h"] \arrow [->>,dd] & & \M(f_{\overline{A}\overline{B}}) \arrow [>->, dr, "h"] \arrow[dd] \\
                & \M(f_{\overline{A}\overline{C}}) \arrow [>->, rr, crossing over,  "h"] & &[-1em] \D(f_{\overline{A}\overline{B}},f_{\overline{A}\overline{C}}) \arrow [dd, "p"] & \\
                A\times\set{1} \arrow [>->, dr, "h"] \arrow [>->, rr, near start, "h"] & & \M(f_{AB}) \arrow [>->, dr, "h"] \\
                & \M(f_{AC}) \arrow [from=uu,crossing over] \arrow [>->, rr, "h"] & & \D(f_{AB},f_{AC}) &
            \end{tikzcd}
        \end{center}
        whose faces are again equivalent to the corresponding faces of the original cube (this follows from the analogous argument in \cref{prop:topDescentPo}).
        
        Next we prove that the map $p\colon\D(f_{\overline{A}\overline{B}},f_{\overline{A}\overline{C}})\to\D(f_{AB},f_{AC})$ is a quasifibration.
        By \cref{cor:locOfQuasifib} we need to find an open cover $U_1,U_2\subset\D(f_{AB},f_{AC})$ such that the induced maps $p^{-1}(U_1)\to U_1$, $p^{-1}(U_2)\to U_2$ and $p^{-1}(U_1\cap U_2)\to U_1\cap U_2$ are quasifibrations.

        We take $U_1=\M^{[0,2)}(f_{AC})\subset\D(f_{AB},f_{AC})$ and $U_2=\M^{[0,2)}(f_{AB})\subset\D(f_{AB},f_{AC})$ as in \cref{rmk:doubleMappingCylinderAlt}.
        Then $p^{-1}(U_1)=\M^{[0,2)}(f_{\overline{A}\overline{C}})\subset\D(f_{\overline{A}\overline{B}},f_{\overline{A}\overline{C}})$, $p^{-1}(U_2)=\M^{[0,2)}(f_{\overline{A}\overline{B}})\subset\D(f_{\overline{A}\overline{B}},f_{\overline{A}\overline{C}})$ and $p^{-1}(U_1\cap U_2)=\overline{A}\times (0,2)\subset\D(f_{\overline{A}\overline{B}},f_{\overline{A}\overline{C}})$.
        It follows $p^{-1}(U_1)\to U_1$ and $p^{-1}(U_2)\to U_2$ are quasifibrations by \cref{lem:mapOfCylIsQuasiFib} since they are both maps of open mapping cylinders induced by the squares 
        \begin{center}
            \begin{tikzcd} [sep = 4em]
                \overline{A} \arrow[d,->>] \arrow[r, "f_{\overline{A}\overline{C}}"] & \overline{C} \arrow[d,->>] \\
                A \arrow[r, "f_{AC}"] & C \\
            \end{tikzcd} 
            \begin{tikzcd} [sep = 4em]
                \overline{A} \arrow[d,->>] \arrow[r, "f_{\overline{A}\overline{B}}"] & \overline{B} \arrow[d,->>] \\
                A \arrow[r, "f_{AB}"] & B \\
            \end{tikzcd}
        \end{center}
        respectively.
        Since the map $p^{-1}(U_1\cap U_2)\to U_1\cap U_2$ is a product of fibrations $f_{\overline{A}A}\times\id_{(0,2)}$, it is again a fibration and thus a quasifibration.

        Finally, since the squares 
        \begin{center}
            \begin{tikzcd} [sep = 4em]
                \M(f_{\overline{A}\overline{C}}) \arrow[d] \arrow[>->,r, "h"] & \D(f_{\overline{A}\overline{B}},f_{\overline{A}\overline{C}}) \arrow[d,"p"] \\
                \M(f_{AC}) \arrow[>->,r, "h"] & \D(f_{AB},f_{AC}) \\
            \end{tikzcd}
            \begin{tikzcd} [sep = 4em]
                \M(f_{\overline{A}\overline{B}}) \arrow[d] \arrow[>->,r, "h"] & \D(f_{\overline{A}\overline{B}},f_{\overline{A}\overline{C}}) \arrow[d,"p"] \\
                \M(f_{AB}) \arrow[>->,r, "h"] & \D(f_{AB},f_{AC}) \\
            \end{tikzcd}
        \end{center}
        are both ordinary pullbacks since h-cofibrations are inclusions and $p^{-1}\left(\M(f_{AC})\right)=\M(f_{\overline{A}\overline{C}})$ and $p^{-1}\left(\M(f_{AB})\right)=\M(f_{\overline{A}\overline{B}})$ respectively, they are homotopy pullbacks by \cref{rmk:parallelQuasifibsInPb}.
    \end{proof}
\end{prop}
\addcontentsline{toc}{subsection}{Descent for Homotopy Coproducts}
\subsection*{Descent for Homotopy Coproducts}
Lastly, we need to prove descent for homotopy coproducts.
For readability purposes, we will abuse notation a little:
We will always work with seemingly fixed diagrams throughout the proofs without replacing them before computing their homotopy limits/colimits.
This is of course not possible in general, since one cannot e.g. extend a given span to a homotopy pushout without replacing it first.
We will do this implicitly; if one is uncomfortable doing that, one can view the proofs as being in the \inftycat/ presented by the given model category, where the issue disappears.

Our proof of descent for homotopy coproducts will demonstrate that descent is somewhat linked to universality.
We will show that descent for homotopy coproducts is a consequence of having disjoint binary homotopy coproducts and universality for homotopy coproducts.
To avoid confusion between homotopy and ordinary limits/colimits, we will tag homotopy limits/colimits with ``h'' (in particular, the ``h'' here does not refer to the \Strom/ model structure).
\begin{definition}
    Let $M$ be a model category.
    We say that \emph{binary homotopy coproducts are disjoint} in $M$ if every homotopy pushout square
    \begin{center}
        \begin{tikzcd} [sep = 1 cm]
            \emptyset \arrow [r] \arrow [d] & X \arrow [d]\\
            Y \arrow [r] & X\coprod^h Y
        \end{tikzcd}
    \end{center}
    (where $\emptyset$ is the initial object and $X\coprod^h Y$ a homotopy coproduct) is a homotopy pullback.
\end{definition}
\begin{lemma}\label{lem:binCoprodDisjoint}
    Binary homotopy coproducts in the model category $\Top$ are disjoint.
    \begin{proof}
        Since homotopy coproducts are given by ordinary coproducts in $\Top$, we have to check that ordinary pushouts
        \begin{center}
            \begin{tikzcd} [sep = 1 cm]
                \emptyset \arrow [r] \arrow [d] & X \arrow [d]\\
                Y \arrow [r] & X\coprod Y
            \end{tikzcd}
        \end{center}
        are already homotopy pullbacks.
        As in the proof for universality of coproducts in $\Top$ \cref{lem:topUniversalCoproduct}, we note that the map $X\to X\coprod Y$ is a fibration.
        Since the square is an ordinary pullback, this already implies that it is a homotopy pullback.
    \end{proof}
\end{lemma}
\begin{corollary}\label{cor:genCoproductComponentPb}
    Let $M$ be a model category with universal homotopy coproducts and disjoint binary homotopy coproducts.
    Let $\left(X_i\right)_{i\in I}$ be a small family of objects and let $\coprod\limits_{i\in I}^h X_i$ be its homotopy coproduct.
    Then
    \begin{center}
        \begin{tikzcd} [sep = 1 cm]
            X_k \arrow [r] \arrow [d] & X_k \arrow [d]\\
            X_k \arrow [r] & \coprod\limits_{i\in I}^h X_i
        \end{tikzcd}
    \end{center}
    is a homotopy pullback square and for $k\neq j$
    \begin{center}
        \begin{tikzcd} [sep = 1 cm]
            \emptyset \arrow [r] \arrow [d] & X_j \arrow [d]\\
            X_k \arrow [r] & \coprod\limits_{i\in I}^h X_i
        \end{tikzcd}
    \end{center}
    is a homotopy pullback square.
    \begin{proof}
        For ease of notation, set $Y_k=\coprod\limits_{i\in I\setminus{\set{k}}}^hX_i$.

        We first consider the case $k\neq j$.
        We define $P_k$ and $P_{kj}$ to be the homotopy pullbacks 
        \begin{center}
            \begin{tikzcd} [sep = 1 cm]
                P_{kj} \arrow [r] \arrow [d] & X_j \arrow [d]\\
                P_{k} \arrow [d] \arrow [r] & Y_k \arrow [d]\\
                X_k \arrow [r] & Y_k\coprod^h X_k\simeq\coprod\limits_{i\in I}^h X_i
            \end{tikzcd}
        \end{center}
        and by disjointness of binary homotopy coproducts we have $P_k\simeq\emptyset$.

        Since $P_k\simeq\emptyset$, we know $P_k$ is a homotopy coproduct over the empty indexing set. 
        Universality for homotopy coproducts applied to the map $P_{kj}\to P_k\simeq\emptyset$ (considered as a natural transformation in $\Fun(\emptyset^{\rhd},M)=\Fun(\set{*},M)$) then proves that $P_{kj}\simeq\emptyset$.

        Next we prove the case $k=j$.
        We need to show that
        \begin{center}
            \begin{tikzcd} [sep = 1 cm]
                X_k \arrow [r] \arrow [d] & X_k \arrow [d]\\
                X_k \arrow [r] & X_k\coprod^h Y_k
            \end{tikzcd}
        \end{center}
        is a homotopy pullback square.
        By universality of homotopy coproducts we have that $\left(X_k\times_{X_k\coprod^h Y_k}^h X_k\right)\coprod^h\left(X_k\times_{X_k\coprod^h Y_k}^hY_k\right)\simeq X_k$.
        But by disjointness of homotopy coproducts we also have $X_k\times_{X_k\coprod^h Y_k}^hY_k\simeq\emptyset$.
        Since $\left(X_k\times_{X_k\coprod^h Y_k}^h X_k\right)\coprod^h\emptyset\simeq\left(X_k\times_{X_k\coprod^h Y_k}^hX_k\right)$, we get $X_k\times_{X_k\coprod^h Y_k}^hX_k\simeq X_k$ which completes the proof.
    \end{proof}
\end{corollary}
\begin{corollary}\label{cor:disjointImpliesDescent}
    Let $M$ be a model category with universal homotopy coproducts and disjoint binary homotopy coproducts.
    Then $M$ has descent for homotopy coproducts.
    \begin{proof}
        Let $\left(X_i\right)_{i\in I}$ and $\left(Y_i\right)_{i\in I}$ be families of objects and let $\left(f_i\colon X_i\to Y_i\right)_{i\in I}$ be a collection of maps.
        We have to prove that 
        \begin{center}
            \begin{tikzcd} [sep = 1 cm]
                X_k \arrow [r] \arrow [d] & \coprod\limits_{i\in I}^hX_i \arrow [d, "\coprod\limits_{i\in I}^hf_i"]\\
                Y_k \arrow [r] & \coprod\limits_{i\in I}^hY_i
            \end{tikzcd}
        \end{center}
        is a homotopy pullback for all $k\in I$.
        
        Let $P_{kj}$ be the homotopy pullback
        \begin{center}
            \begin{tikzcd} [sep = 1 cm]
                P_{kj} \arrow [r] \arrow [dd] & X_j \arrow [d]\\
                & \coprod\limits_{i\in I}^hX_i \arrow [d]\\
                Y_k \arrow [r] & \coprod\limits_{i\in I}^hY_i
            \end{tikzcd}
        \end{center}
        and since $X_j\to\coprod\limits_{i\in I}^hX_i\to\coprod\limits_{i\in I}^hY_i$ is equal to $X_j\to Y_j\to\coprod\limits_{i\in I}^hY_i$, we can also compute $P_{kj}$ as the successive homotopy pullback
        \begin{center}
            \begin{tikzcd} [sep = 1 cm]
                P_{kj} \arrow [r] \arrow [d] & X_j \arrow [d]\\
                Y_k\times_{\coprod\limits_{i\in I}^hY_i}^h Y_j \arrow [d] \arrow [r] & Y_j \arrow [d]\\
                Y_k \arrow [r] & \coprod\limits_{i\in I}^hY_i
            \end{tikzcd}
        \end{center}
        by the pasting law for homotopy pullbacks.

        By \cref{cor:genCoproductComponentPb} we have that $Y_k\times_{\coprod\limits_{i\in I}^hY_i}^h Y_j\simeq\emptyset$ for $k\neq j$ and $Y_k\times_{\coprod\limits_{i\in I}^hY_i}^h Y_k\simeq Y_k$ for $k=j$.
        This in particular shows that $P_{kj}\simeq\emptyset$ for $j\neq k$ (again by universality of homotopy coproducts for the empty indexing set) and $P_{kk}\simeq X_k$.
        By universality of homotopy coproducts $\coprod\limits_{j\in I}^hP_{kj}\simeq\coprod\limits_{j\in I}^h\left(Y_k\times_{\coprod\limits_{i\in I}^hY_i}^hX_j\right)\simeq Y_k\times_{\coprod\limits_{i\in I}^hY_i}^h\left(\coprod\limits_{j\in I}^hX_j\right)$ is a weak equivalence.
        But since $P_{kj}\simeq\emptyset$ for $j\neq k$ and $P_{kk}\simeq X_k$ we obtain $\coprod\limits_{j\in I}P_{kj}\simeq X_k$ and this proves the corollary.
    \end{proof}
\end{corollary}
\begin{remark}
    One can also conclude disjointness of binary homotopy coproducts from descent for homotopy coproducts.
    Hence a model category with universal homotopy coproducts has descent for homotopy coproducts if and only if it has disjoint binary homotopy coproducts.
\end{remark}
\begin{corollary}\label{cor:topDescentCoproduct}
    The model category $\Top$ has descent for homotopy coproducts.
    \begin{proof}
        By \cref{cor:disjointImpliesDescent} this follows from \cref{lem:binCoprodDisjoint} and \cref{lem:topUniversalCoproduct}.
    \end{proof}
\end{corollary}
\begin{remark}
    One can prove descent for coproducts much more directly in $\Top$:
    
    Since homotopy coproducts can be computed by the ordinary coproducts in $\Top$, in the setting of \cref{cor:disjointImpliesDescent} the square
    \begin{center}
        \begin{tikzcd} [sep = 1 cm]
            X_k \arrow [r] \arrow [d] & \coprod\limits_{i\in I}X_i \arrow [d, "\coprod\limits_{i\in I}f_i"]\\
            Y_k \arrow [r] & \coprod\limits_{i\in I}Y_i
        \end{tikzcd}
    \end{center}
    is a pullback square as can be checked by point set topology and since the map $Y_k\to\coprod\limits_{i\in I}Y_i$ is a fibration, it is already a homotopy pullback which proves the proposition.
\end{remark}
Now all that remains for our proof that $\spaces$ is an \inftytop/ is to assemble the parts. 
\begin{corollary}\label{cor:spacesIsInftyTop}
    The \inftycat/ of spaces $\spaces$ is an \inftytop/.
    \begin{proof}
        We know $\spaces$ is locally presentable from \cref{cor:spacesIsLocPres}.
        \Cref{cor:sufficientToProveInModCat} together with \cref{lem:topUniversalPo} and \cref{lem:topUniversalCoproduct} prove that $\spaces$ has universal colimits, and \cref{prop:topDescentPo}/\cref{lem:topDescentPoAlt} together with \cref{cor:topDescentCoproduct} then prove that it has descent.
    \end{proof}
\end{corollary}